


\ssec{}

The category $\hTop$ has several flaws from the perspective of category theory. This has driven 



: even finite colimits/limits do not always exist. Even when they do exist, they are not the desired ones.

\begin{exam}
	\label{exam-pushpull-in-hTop}
	Consider the following diagrams in $\hTop$.
	\begin{itemize}
		\item[(1)]
			The pushout of $* \gets S^1 \to D^1$ is equivalent to $*$ rather than $S^2$.
		\item[(2)]
			The pushout of $* \gets S^1 \to S^1$ does not exist, where the rightward map has index $2$.
		\item[(3)]
			The pullback of $* \to S^1 \gets \mBR^1$ is equivalent to $\mBR^1$ rather than $\mBZ$. 
		\item[(4)]
			The pullback of $* \to \mBC\mBP^2 \gets \mBR\mBP^2$ does not exist.
	\end{itemize}
	For proofs of (2) and (4), see \href{https://mathoverflow.net/q/120572}{here}.
\end{exam}

\ssec{}

These examples suggest that a homotopy invariant definition of colimits/limits of topological spaces \emph{cannot} be extracted from the category $\hTop$. Rather, one needs to work in $\Top$ to get the correct definition.

\begin{defn}
	The \textbf{homotopy pushout} of a diagram $X \gets Y \to Z$ in $\Top$ is defined to be
	\[
		X \po_Y^\mbh Z := X \po_{Y\mt \{0\}} (Y\mt [0,1]) \po_{Y\mt \{1\}} Z.
	\]
	The \textbf{homotopy pullback} of a diagram $X \to Y \gets Z$ in $\Top$ is defined to be
	\[
		X \mt_Y^\mbh Z := X \mt_{Y^{\{0\}}} (Y^{[0,1]}) \mt_{Y^{\{1\}}} Z.
	\]
\end{defn}

\begin{exe}
	Check the above definitions are homotopy invariant, i.e., only depend on the images of the diagrams under $\mbh:\Top \to \hTop$.
\end{exe}

\begin{exe}
	Calculate the \emph{homotopy} pushouts/pullbacks in Example \ref{exam-pushpull-in-hTop}, and convince yourself why the results are ``correct''.
\end{exe}


\ssec{}
\label{ssec-homotopy-pullback}

Let $X \xrightarrow{p} Y \xleftarrow{q} Z$ be a diagram in $\Top$. We want to emphasize the following comparison:
\begin{itemize}
	\item[(1)] 
		Knowing a morphism $S\to X\mt_Y Z$ is equivalent to knowing continuous maps $f:S \to X$ and $g:X \to Z$ \emph{such that} $p\circ f = q\circ g$.
	\item[(2)]
		Knowing a morphism $S\to X\mt^\mbh_Y Z$ is equivalent to knowing continuous maps $f:S \to X$ and $g:X \to Z$ \emph{equipped with} a homotopy from $p\circ f$ to $q\circ g$.
	\item[(3)]
		Knowing a morphism $S \to \mbh X\mt_{\mbh Y} \mbh Z$, when the target exists, is equivalent to knowing homotopy classes of continuous maps $f:S \to X$ and $g:X \to Z$ \emph{such that} $p\circ f$ is homotopic to $q\circ g$.
\end{itemize}

Note that
\begin{itemize}
	\item
		The data in (1) is \emph{strict} and \emph{not homotopy invariant}.
	\item
		The data in (2) consist of $f$, $g$ equipped with a \emph{structure}.
	\item
		The data in (3) consist of $f$, $g$ satisfying a \emph{property}.
\end{itemize}

\begin{slog}
	It is not enough to know two maps are homotopic, rather, we need to know \emph{how} they are homotopic.
\end{slog}

\ssec{}

The reader may entertain themselves with the following exercise.

\begin{exe}
	Let 
	\[
		\xymatrix{
			& X \ar[ld] \ar[rd] \ar[d] \\
			Y_1 \ar[r] & Y & Y_2 \ar[l] \\
			& Z \ar[lu] \ar[ru] \ar[u]
		}
	\]
	be a commutative diagram in $\Top$. Give a definition of the homotopy limit of this diagram, and intepret it as in Sect. \ref{ssec-homotopy-pullback}.
\end{exe}

\begin{slog}
	It is not enough to record homotopies between maps, rather, we need to record all the higher homotopies.
\end{slog}








\begin{comment}

\ssec{}

Heuristically, one may say \emph{plain} category theory is not adaptive to homotopy theory because the former favors \emph{sets}, while the latter is not \emph{discrete}.

\ssec{}

One may be satisfied by the above definition. Then try the following challenge:

\begin{exe}
	Let $p_i: X_i \to Y$, $i\in I$ be morphisms in $\Top$ labelled by a finite set $I$ of order $n$. Find a definition of the \emph{homotopy fiber product} of all the $X_i$'s relative to $Y$. How canonical is your definition?
\end{exe}

\ssec{}

One possible solution is as follows.
\begin{itemize}
	\item[(1)]
		For any total order $\sigma$ on $I$, we define an iterated homotopy pullback $Z_\sigma$ according to this order.
	\item[(2)]
		For any two total orders $\sigma_1$, $\sigma_2$, we construct a homotopy equivalence $\phi_{\sigma_1,\sigma_2}: Z_{\sigma_1} \to Z_{\sigma_2}$.
	\item[(3)]
		For any three total orders $\sigma_1$, $\sigma_2$, $\sigma_3$, we check that $\phi_{\sigma_2,\sigma_3}\circ \phi_{\sigma_1,\sigma_2}$ is homotopic to $\phi_{\sigma_1,\sigma_3}$.
	\item[(4)]
		We conclude that there exists an object $Z\in \hTop$, \emph{unique up to unique isomorphism}, equipped with isomorphisms $\varphi_\sigma:Z \to \mbh Z_\sigma$, such that $\mbh\phi_{\sigma_1,\sigma_2}\circ \varphi_{\sigma_1} = \varphi_{\sigma_2}$.
\end{itemize}
In retrospect,
\begin{itemize}
	\item 
		Step (1) can be fairly viewed as \emph{natural}.
	\item
		Step (2) depends on auxiliary choices of homeomorphisms $[0,1]\simeq [0,1]\po_*\cdots \po_* [0,1]$. But one can say there is only one choice respecting \emph{égalité}.
	\item
		However, any homotopy in step (3) would break this illusion. But one can say they only need the \emph{existence} of a homotopy, rather than choosing some.
\end{itemize}



\ssec{}

Honestly speaking, the above exercise is techinically annoying but heuristically simple. It reflects the following feature of categorical homotopy theory.
\begin{itemize}
	\item[] \emph{To give a homotopy invariant definition, one often needs to make auxiliary choices, and provides homotopies between results coming out of these choices,
	}
\end{itemize}

\end{comment}