\input{headerbeamer.tex}


\title{Lecture 1}

\date{Sep. 10, 2024}


\begin{document}

\frame{\titlepage}

\begin{frame}
	\frametitle{A slogan}

	\[
		\infty\mbox{-category theory }=\mbox{ category theory }+\mbox{ homotopy theory}.
	\]
	
\end{frame}

\begin{frame}
	\frametitle{Classical category theory}

	
\begin{defn}
	A \alert{category} $\mCC$ consists of the following data:\pause
	\begin{itemize}
		\item 
			\alert{Objects}: $a\in \Ob(\mCC)$. \pause
		\item
			\alert{Morphisms}: $f:a\to b$, $f\in \Hom(a,b)$. \pause
		\item
			\alert{Composition of morphisms}: $f$, $g$ $\Rightarrow$ $g\circ f$. \pause
			
	\end{itemize}
	Axioms:
	\begin{itemize}
		\item 
			\alert{Associativity}: 
			\[
				h\circ (g\circ f) = (h\circ g) \circ f. \pause
			\] 
		\item
			\alert{Identity}: $f:a\to b$, 
			\[
				\id_b\circ f = f = f\circ \id_a. \pause
			\]
	\end{itemize}
\end{defn}

	
	
\end{frame}

\begin{frame}
	\frametitle{Classical category theory}

	\begin{itemize}
		\item
			\emph{In order to study a collection of objects one should also consider suitably defined morphisms between such objects.}\pause
		\item
			Examples: $\alert{\Set}$, $\alert{\Grp}$, $\alert{\Ring}$, $\alert{\Top}$... 
	\end{itemize}
	


	
	
\end{frame}

\begin{frame}
	\frametitle{Three doctrines}

	\begin{itemize}
	\item[(1)]
		Morphisms are \alert{discrete}: $f=g$ or $f\neq g$. \pause
	\item[(2)]
		Associativity is \alert{strict}: $h\circ (g\circ f)\alert{=}(h\circ g) \circ f$. \pause
	\item[(3)]
		Composition is \alert{concrete}: there is no ambiguity for $g\circ f$. \pause
\end{itemize}
	
	We will abandon all of them.

\end{frame}

\begin{frame}
	\frametitle{Enrichment}

	\begin{itemize}
		\item
			\st{Doctrine (1): Morphisms are discrete}. \pause
		\item
			Richer structures on $\Hom(-,-)$.\pause
		\item
			\alert{topological spaces} $\Rightarrow$ \alert{topological categories}. \pause
		\item
			\alert{categories} $\Rightarrow$ \alert{strict 2-categories}. \pause
		\item
			\alert{strict $(n-1)$-categories} $\Rightarrow$ \alert{strict $n$-categories}. \pause
		\item
			...

	\end{itemize}
\end{frame}

\begin{frame}
	\frametitle{Strict 2-categories}

	A strict 2-category $\mCC$ consists of the following data:\pause
	\begin{itemize}
		\item
			Objects. \pause
		\item
			Morphisms and their compositions. \pause
		\item
			\alert{2-morphisms}: morphisms between morphisms,\pause and their \alert{vertical} and \alert{horizontal} compositions. \pause

	\end{itemize}
	Axioms: all given by \alert{equalities}. (Doctrine (2))


\end{frame}

\begin{frame}
	\frametitle{Strict 2-categories with one object}

	\begin{itemize}
		\item<1->
			$\mCD$: strict 2-category with $\Ob(\mCD) \simeq \{*\}$.  $\Leftrightarrow$
		\item<2->
			$\mCC:=\Hom(*,*)$: equipped with $-\ot-:\mCC\mt \mCC \to \mCC$ \alert{strictly} associative and unital. 
		\item<3->
			This is \alert{evil}
			\begin{itemize}
				\item[] Principle of isomorphism: \emph{all grammatically correct properties of objects of a fixed category are to be invariant under isomorphism}.\pause
			\end{itemize}
			and \alert{impractical}. \pause



		\item<4->
			\st{Strict} \alert{monoidal category}: supply \alert{associators}:
			\[
				X\ot (Y\ot Z) \xrightarrow{\simeq} (X\ot Y) \ot Z
			\]
			subject to certain coherent conditions.
		\item<3->[]

			\begin{exe}
				In ZFC, $(X\mt Y)\mt Z=X \mt(Y\mt Z)$ or $(X\mt Y)\mt Z \simeq X \mt(Y\mt Z)$?
			\end{exe}

	\end{itemize}

\end{frame}

\begin{frame}
	\frametitle{Weak $n$-categories}

	\begin{itemize}
		\item
			\st{Doctrine (2): associativity is strict} \pause
		\item
			Wishes for weak $n$-categories
			\begin{itemize}
				\item 
					Weak $1$-categories $\Leftrightarrow$ categories. \pause
				\item
					Weak $2$-categories with a single object $\Leftrightarrow$ monoidal categories. \pause
				\item
					$\Hom(a,b)$: weak $(n-1)$-category. \pause
				\item
					\emph{Principle of isomorphism}. \pause
			\end{itemize}
		\item
			For $n\ge 2$, we should never require two morphisms in a weak $n$-category to be \alert{equal}. \pause
		\item
			For $n> k$, we should never require two $k$-morphisms in a weak $n$-category to be \alert{equal}.
	\end{itemize}

\end{frame}

\begin{frame}
	\frametitle{Nightmare of coherence data}

	\begin{itemize}
		\item
			\alert{Associator}: $h\circ (g\circ f) \xrightarrow{\simeq} (h\circ g) \circ f$. \pause
		\item
			Commutative diagram for $2$-morphisms:
			\[
			\xymatrixrowsep{0.2in}
			\xymatrixcolsep{0.2in}
	\xymatrix{
		& & (kh)(gf) \ar[rrd] \\
		k(h(gf)) \ar[rru] \ar[rd]
		& & & & ((kh)g)f \\
		& k((hg)f) \ar[rr]
		& & (k(hg))f. \ar[ru]
	}
		\] \pause
		\item
			$n\ge 3$: \alert{witnessed} by an invertible $3$-morphism. \pause
		\item
			Commutative diagram for $3$-morphisms. \pause
		\item
			$n\ge 4$: \alert{witnessed} by an invertible $4$-morphism. \pause
		\item
			...
	\end{itemize}

\end{frame}

\begin{frame}
	\frametitle{A savior}

	\begin{slog}[Grothendieck's homotopy hypothesis]
		 
		 \[
		 	\infty\mbox{-groupoids }= \mbox{ homotopy types.}\pause
		 \]
	\end{slog}

	\begin{itemize}
		\item<3->
			Coherence data for associativity can be \alert{hidden} in the homotopy theory of spaces.
		\item<2->
			\alert{$\infty$-groupoids}: morphisms and higher morphisms are all invertible.  \pause
		
	\end{itemize}

\end{frame}

\begin{frame}
	\frametitle{A strategy}
	\begin{itemize}
		\item
			\alert{$(\infty,k)$-categories}: $m$-morphisms invertible for $m>k$. \pause
		\item
			Induction on $k$. \pause
		\item
			\alert{$(\infty,0)$-categories}:= $\infty$-groupoids := homotopy types. \pause
		\item
			Induction step: coherence data for associativity is already incorporated into the theory of $\infty$-groupoids!
		
	\end{itemize}

\end{frame}

\begin{frame}
	\frametitle{Models}
	\begin{itemize}
		\item
			Theory of $(\infty,0)$-categories $\Leftrightarrow$ homotopy theory of topological spaces. \pause
		\item
			Theory of $(\infty,1)$-categories $\Leftrightarrow$ homotopy theory of topological categories ? \pause
		\item
			Yes, but... \pause
		\item
			A more developed model: \alert{quasi-categories}. \pause \st{Doctrine (3): composition is concrete}.
		
	\end{itemize}

\end{frame}


\begin{frame}
	\frametitle{Homotopy category}

	\begin{itemize}
		\item 
			The \alert{homotopy category} $\hTop$: \pause
			\begin{itemize}
				\item 
					Objects=\alert{homotopy types}: topological spaces. \pause
				\item
					Morphisms: homotopy classes of continuous maps.  \pause
			\end{itemize}
		\item
			\alert{Homotopy equivalence}: isomorphisms in $\hTop$. \pause
		\item
			$\hTop$: inverting homotopy equivalences in $\Top$. \pause
		\item
			$\hTop_{\le n}$: full subcategory of \alert{homotopy $n$-types}: $\pi_k\simeq 0$, $k>n$.
	\end{itemize}

\end{frame}

\begin{frame}
	\frametitle{fundamental groupoid}

	\begin{itemize}
		\item 
			The \alert{fundamental groupoid} $\pi_{\le 1} X$:\pause
			\begin{itemize}
				\item Objects: points. \pause
				\item Morphisms: \alert{homotopy classes} of pathes. \pause
			\end{itemize}
		\item[]
			\begin{exe}
			Check the axiom of associativity for the above construction. \pause
			\end{exe}
		\item
			$\pi_{\le 1}X$ is indeed a groupoid. \pause
		\item
			$\pi_0(X) \simeq \Ob(\pi_{\le 1}X)/\simeq $. \pause
		\item
			$\pi_1(X,x) \simeq \Hom(x,x)$. \pause
		\item[]
			\begin{exe}
				Show $\pi_{\le 1}: \hTop_{\le 1} \to \Grpd$ is an equivalence.
			\end{exe}
	\end{itemize}
\end{frame}

\begin{frame}
	\frametitle{fundamental $2$-groupoid}

	\begin{itemize}
		\item 
			The \alert{fundamental $2$-groupoid} $\pi_{\le 2} X$:\pause
			\begin{itemize}
				\item Objects: points. \pause
				\item Morphisms: pathes. \pause
				\item 2-morphisms: \alert{homotopy classes} of homotopies between pathes. \pause
			\end{itemize}
		\item 
			Put associativity aside for a moment. \pause
		\item
			$\pi_{\le 2}X$ is indeed a $2$-groupoid. \pause
		\item
			$\pi_0(X) \simeq \Ob(\pi_{\le 2}X)/\simeq $. \pause
		\item
			$\pi_1(X,x) \simeq \Hom(x,x)/\simeq$. \pause
		\item
			$\pi_2(X,x) \simeq \Hom(\id_x,\id_x)$. \pause

	\end{itemize}

	\begin{exe}
		Describe the action of $\pi_1(X,x)$ on $\pi_2(X,x)$ in terms of $\pi_{\le 2}X$. \pause
	\end{exe}

	\begin{exe}
		Can we define morphisms as \alert{homotopy classes} of pathes?
	\end{exe}
\end{frame}

\begin{frame}
	\frametitle{$\pi_{\le 2}$ is not strict}

	\begin{itemize}
		\item 
			Pathes: $f$, $g$, $h$. \pause
			\begin{equation}
			\label{eqn-1}
		\big(h\circ (g\circ f)\big) (t) = 
		\left\{ \begin{array}{rcl}
		f(4t) & \mbox{for} & t\in [0,1/4] \\
		g(4t-1) & \mbox{for} & t\in [1/4,1/2] \\
		h(2t-1) & \mbox{for} & t\in [1/2,1]
			\end{array}\right. \pause
			\end{equation}
			\begin{equation}
	\label{eqn-2}
	\big( (h\circ g)\circ f\big) (t) = 
	\left\{ \begin{array}{rcl}
		f(2t) & \mbox{for} & t\in [0,1/2] \\
		g(4t-2) & \mbox{for} & t\in [1/2,3/4] \\
		h(4t-3) & \mbox{for} & t\in [3/4,1]
	\end{array}\right. \pause
\end{equation}

		\item
			Homotopic but not equal. \pause 
		\item
			Can you find a canonical homotopy? \pause
		\item
			Homotopy class of homotopies (\ref{eqn-1}) $\to$ (\ref{eqn-2}) is canonical. \pause $\Rightarrow$
		\item
			$\pi_{\le 2}X$ is a \alert{weak} $2$-category. \pause
	\end{itemize}

	\begin{exe}
		Challenge: show $\pi_{\le 2}: \hTop_{\le 2} \to 2\mbox{-}\Grpd$ is an equivalence.
	\end{exe}
	
\end{frame}

\begin{frame}
	\frametitle{Nightmare of homotopy coherence data}

	\begin{itemize}
		\item 
			$\pi_{\le n} X$, $n\ge 3$.  \pause
		\item
			2-morphism: \st{homotopy classes of} homotopies between pathes. \pause
		\item
			Make a \alert{non-canonical} choice for homotopy (\ref{eqn-1}) $\to$ (\ref{eqn-2}). \pause
		\item
			Fit the choice into the coherent data of associativity\pause: choose homotopies between homotopies between homotopies. \pause
		\item
			...
	\end{itemize}

\end{frame}


\begin{frame}
	\frametitle{In retrospect}

	\begin{itemize}
		\item 
			Two impossible tasks:\pause
			\begin{itemize}
				\item 
					Give a \alert{combinatorial} definition of weak $n$-groupoids; \pause
				\item
					Supply homotopy coherent data to make $\pi_{\le n}X$ a weak $n$-groupoid.\pause
			\end{itemize}
		\item
			Grothendieck's homotopy hypothesis: same task\pause; do neither. \pause
	\end{itemize}

	\begin{slog}
	\[
		n\mbox{-groupoids } = \mbox{ homotopy } n \mbox{-types}. \pause
	\]
	\[
		\infty\mbox{-groupoids } = \mbox{ homotopy types}.\pause
	\]
	\[
		\infty\mbox{-category theory }=\mbox{ category theory }+\mbox{ homotopy theory}.
	\]
	\end{slog}



\end{frame}


\begin{frame}
	\huge Thank you!


\end{frame}



\end{document}