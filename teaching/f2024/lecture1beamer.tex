\documentclass{beamer}
\usepackage[utf8]{inputenc}
%\usetheme{CambridgeUS}
\setbeamertemplate{theorems}[numbered]

\usepackage{cite}
\usepackage{url}
\usepackage{rotating}
\usepackage{eucal}
\usepackage{tikz-cd,mathtools}
\tikzset{mydescription/.style={anchor=center,fill=white}}
\usepackage[all,2cell,color]{xy}
\UseAllTwocells
\UseCrayolaColors
\usepackage{graphicx}
\usepackage{pifont}
\usepackage{comment}
\usepackage{verbatim}
\usepackage{xcolor}
\usepackage{hyperref}
\usepackage{xparse}
\usepackage{upgreek}
\usepackage{MnSymbol}
\usepackage{graphicx}
\usepackage{soul}


\setbeamertemplate{caption}{\raggedright\insertcaption\par}



%%%%%%%%%%%%%%%%%%%%Theorem%%%%%%%%%%%%%%%%%%%%
\newtheorem{ax}{Axiom}
%\renewcommand*{\theax}{ax}
\newtheorem{conje}{Conjecture}
%\renewcommand*{\theconje}{conje}
\newtheorem{thm}{Theorem}
\newtheorem{cor}{Corollary}
\newtheorem{guess}{Guess}
%\renewcommand*{\theguess}{guess}
\newtheorem{claim}{Claim}
%\renewcommand*{\theclaim}{claim}
\newtheorem{ques}{Question}
%\renewcommand*{\theques}{ques}
\newtheorem{mainfacts}{Fact}
\newtheorem{defn}{Definition}
\newtheorem{exe}{Exercise}
\newtheorem{slog}{Slogan}


%%%%%%%%%%%%%%%%%%%%Commands%%%%%%%%%%%%%%%%%%%%

\newcommand{\nc}{\newcommand}
\nc\on{\operatorname}
\nc\renc{\renewcommand}


%%%%%%%%%%%%%%%%%%%%Sections%%%%%%%%%%%%%%%%%%%%

\nc\ssec{\subsection}
\nc\sssec{\subsubsection}

%%%%%%%%%%%%%%%%%%%%Environment%%%%%%%%%%%%%%%%%
\nc\blongeqn{\[ \begin{aligned}}
\nc\elongeqn{\end{aligned} \]}



%%%%%%%%%%%%%%%%%%%%Mathfont%%%%%%%%%%%%%%%%%%%%

\nc\mBA{{\mathbb A}}
\nc\mBB{{\mathbb B}}
\nc\mBC{{\mathbb C}}
\nc\mBD{{\mathbb D}}
\nc\mBE{{\mathbb E}}
\nc\mBF{{\mathbb F}}
\nc\mBG{{\mathbb G}}
\nc\mBH{{\mathbb H}}
\nc\mBI{{\mathbb I}}
\nc\mBJ{{\mathbb J}}
\nc\mBK{{\mathbb K}}
\nc\mBL{{\mathbb L}}
\nc\mBM{{\mathbb M}}
\nc\mBN{{\mathbb N}}
\nc\mBO{{\mathbb O}}
\nc\mBP{{\mathbb P}}
\nc\mBQ{{\mathbb Q}}
\nc\mBR{{\mathbb R}}
\nc\mBS{{\mathbb S}}
\nc\mBT{{\mathbb T}}
\nc\mBU{{\mathbb U}}
\nc\mBV{{\mathbb V}}
\nc\mBW{{\mathbb W}}
\nc\mBX{{\mathbb X}}
\nc\mBY{{\mathbb Y}}
\nc\mBZ{{\mathbb Z}}


\nc\mCA{{\mathcal A}}
\nc\mCB{{\mathcal B}}
\nc\mCC{{\mathcal C}}
\nc\mCD{{\mathcal D}}
\nc\mCE{{\mathcal E}}
\nc\mCF{{\mathcal F}}
\nc\mCG{{\mathcal G}}
\nc\mCH{{\mathcal H}}
\nc\mCI{{\mathcal I}}
\nc\mCJ{{\mathcal J}}
\nc\mCK{{\mathcal K}}
\nc\mCL{{\mathcal L}}
\nc\mCM{{\mathcal M}}
\nc\mCN{{\mathcal N}}
\nc\mCO{{\mathcal O}}
\nc\mCP{{\mathcal P}}
\nc\mCQ{{\mathcal Q}}
\nc\mCR{{\mathcal R}}
\nc\mCS{{\mathcal S}}
\nc\mCT{{\mathcal T}}
\nc\mCU{{\mathcal U}}
\nc\mCV{{\mathcal V}}
\nc\mCW{{\mathcal W}}
\nc\mCX{{\mathcal X}}
\nc\mCY{{\mathcal Y}}
\nc\mCZ{{\mathcal Z}}


\nc\mbA{{\mathsf A}}
\nc\mbB{{\mathsf B}}
\nc\mbC{{\mathsf C}}
\nc\mbD{{\mathsf D}}
\nc\mbE{{\mathsf E}}
\nc\mbF{{\mathsf F}}
\nc\mbG{{\mathsf G}}
\nc\mbH{{\mathsf H}}
\nc\mbI{{\mathsf I}}
\nc\mbJ{{\mathsf J}}
\nc\mbK{{\mathsf K}}
\nc\mbL{{\mathsf L}}
\nc\mbM{{\mathsf M}}
\nc\mbN{{\mathsf N}}
\nc\mbO{{\mathsf O}}
\nc\mbP{{\mathsf P}}
\nc\mbQ{{\mathsf Q}}
\nc\mbR{{\mathsf R}}
\nc\mbS{{\mathsf S}}
\nc\mbT{{\mathsf T}}
\nc\mbU{{\mathsf U}}
\nc\mbV{{\mathsf V}}
\nc\mbW{{\mathsf W}}
\nc\mbX{{\mathsf X}}
\nc\mbY{{\mathsf Y}}
\nc\mbZ{{\mathsf Z}}

\nc\mba{{\mathsf a}}
\nc\mbb{{\mathsf b}}
\nc\mbc{{\mathsf c}}
\nc\mbd{{\mathsf d}}
\nc\mbe{{\mathsf e}}
\nc\mbf{{\mathsf f}}
\nc\mbg{{\mathsf g}}
\nc\mbh{{\mathsf h}}
\nc\mbi{{\mathsf i}}
\nc\mbj{{\mathsf j}}
\nc\mbk{{\mathsf k}}
\nc\mbl{{\mathsf l}}
\nc\mbm{{\mathsf m}}
\nc\mbn{{\mathsf n}}
\nc\mbo{{\mathsf o}}
\nc\mbp{{\mathsf p}}
\nc\mbq{{\mathsf q}}
\nc\mbr{{\mathsf r}}
\nc\mbs{{\mathsf s}}
\nc\mbt{{\mathsf t}}
\nc\mbu{{\mathsf u}}
\nc\mbv{{\mathsf v}}
\nc\mbw{{\mathsf w}}
\nc\mbx{{\mathsf x}}
\nc\mby{{\mathsf y}}
\nc\mbz{{\mathsf z}}




\nc\mbfA{{\mathbf A}}
\nc\mbfB{{\mathbf B}}
\nc\mbfC{{\mathbf C}}
\nc\mbfD{{\mathbf D}}
\nc\mbfE{{\mathbf E}}
\nc\mbfF{{\mathbf F}}
\nc\mbfG{{\mathbf G}}
\nc\mbfH{{\mathbf H}}
\nc\mbfI{{\mathbf I}}
\nc\mbfJ{{\mathbf J}}
\nc\mbfK{{\mathbf K}}
\nc\mbfL{{\mathbf L}}
\nc\mbfM{{\mathbf M}}
\nc\mbfN{{\mathbf N}}
\nc\mbfO{{\mathbf O}}
\nc\mbfP{{\mathbf P}}
\nc\mbfQ{{\mathbf Q}}
\nc\mbfR{{\mathbf R}}
\nc\mbfS{{\mathbf S}}
\nc\mbfT{{\mathbf T}}
\nc\mbfU{{\mathbf U}}
\nc\mbfV{{\mathbf V}}
\nc\mbfW{{\mathbf W}}
\nc\mbfX{{\mathbf X}}
\nc\mbfY{{\mathbf Y}}
\nc\mbfZ{{\mathbf Z}}

\nc\mbfa{{\mathbf a}}
\nc\mbfb{{\mathbf b}}
\nc\mbfc{{\mathbf c}}
\nc\mbfd{{\mathbf d}}
\nc\mbfe{{\mathbf e}}
\nc\mbff{{\mathbf f}}
\nc\mbfg{{\mathbf g}}
\nc\mbfh{{\mathbf h}}
\nc\mbfi{{\mathbf i}}
\nc\mbfj{{\mathbf j}}
\nc\mbfk{{\mathbf k}}
\nc\mbfl{{\mathbf l}}
\nc\mbfm{{\mathbf m}}
\nc\mbfn{{\mathbf n}}
\nc\mbfo{{\mathbf o}}
\nc\mbfp{{\mathbf p}}
\nc\mbfq{{\mathbf q}}
\nc\mbfr{{\mathbf r}}
\nc\mbfs{{\mathbf s}}
\nc\mbft{{\mathbf t}}
\nc\mbfu{{\mathbf u}}
\nc\mbfv{{\mathbf v}}
\nc\mbfw{{\mathbf w}}
\nc\mbfx{{\mathbf x}}
\nc\mbfy{{\mathbf y}}
\nc\mbfz{{\mathbf z}}

\nc\mfa{{\mathfrak a}}
\nc\mfb{{\mathfrak b}}
\nc\mfc{{\mathfrak c}}
\nc\mfd{{\mathfrak d}}
\nc\mfe{{\mathfrak e}}
\nc\mff{{\mathfrak f}}
\nc\mfg{{\mathfrak g}}
\nc\mfh{{\mathfrak h}}
\nc\mfi{{\mathfrak i}}
\nc\mfj{{\mathfrak j}}
\nc\mfk{{\mathfrak k}}
\nc\mfl{{\mathfrak l}}
\nc\mfm{{\mathfrak m}}
\nc\mfn{{\mathfrak n}}
\nc\mfo{{\mathfrak o}}
\nc\mfp{{\mathfrak p}}
\nc\mfq{{\mathfrak q}}
\nc\mfr{{\mathfrak r}}
\nc\mfs{{\mathfrak s}}
\nc\mft{{\mathfrak t}}
\nc\mfu{{\mathfrak u}}
\nc\mfv{{\mathfrak v}}
\nc\mfw{{\mathfrak w}}
\nc\mfx{{\mathfrak x}}
\nc\mfy{{\mathfrak y}}
\nc\mfz{{\mathfrak z}}

\nc{\one}{{\mathsf{1}}}


\nc\clambda{ {\check{\lambda} }}
\nc\cmu{ {\check{\mu} }}

\nc\bDelta{\mathsf{\Delta}}
\nc\bGamma{\mathsf{\Gamma}}
\nc\bLambda{\mathsf{\Lambda}}


\nc\loccit{\emph{loc.cit.}}



%%%%%%%%%%%%%%%%%%%%Operations-limit%%%%%%%%%%%%%%%%%%%%

\NewDocumentCommand{\ot}{e{_^}}{
  \mathbin{\mathop{\otimes}\displaylimits
    \IfValueT{#1}{_{#1}}
    \IfValueT{#2}{^{#2}}
  }
}
\NewDocumentCommand{\boxt}{e{_^}}{
  \mathbin{\mathop{\boxtimes}\displaylimits
    \IfValueT{#1}{_{#1}}
    \IfValueT{#2}{^{#2}}
  }
}
\NewDocumentCommand{\mt}{e{_^}}{
  \mathbin{\mathop{\times}\displaylimits
    \IfValueT{#1}{_{#1}}
    \IfValueT{#2}{^{#2}}
  }
}
\NewDocumentCommand{\convolve}{e{_^}}{
  \mathbin{\mathop{\star}\displaylimits
    \IfValueT{#1}{_{#1}}
    \IfValueT{#2}{^{#2}}
  }
}
\NewDocumentCommand{\colim}{e{_^}}{
  \mathbin{\mathop{\operatorname{colim}}\displaylimits
    \IfValueT{#1}{_{#1}\,}
    \IfValueT{#2}{^{#2}\,}
  }
}
\NewDocumentCommand{\laxlim}{e{_^}}{
  \mathbin{\mathop{\operatorname{laxlim}}\displaylimits
    \IfValueT{#1}{_{#1}\,}
    \IfValueT{#2}{^{#2}\,}
  }
}
\NewDocumentCommand{\oplaxlim}{e{_^}}{
  \mathbin{\mathop\operatorname{oplax-lim}\displaylimits
    \IfValueT{#1}{_{#1}\,}
    \IfValueT{#2}{^{#2}\,}
  }
}


%%%%%%%%%%%%%%%%%%%%Arrows%%%%%%%%%%%%%%%%%%%%


\makeatletter
\newcommand{\laxto}{\dashedrightarrow}
\newcommand{\xrightleftarrows}[1]{\mathrel{\substack{\xrightarrow{#1} \\[-.9ex] \xleftarrow{#1}}}}
\newcommand{\adj}{\xrightleftarrows{\rule{0.5cm}{0cm}}}

\newcommand*{\da@rightarrow}{\mathchar"0\hexnumber@\symAMSa 4B }
\newcommand*{\da@leftarrow}{\mathchar"0\hexnumber@\symAMSa 4C }
\newcommand*{\xlaxto}[2][]{%
  \mathrel{%
    \mathpalette{\da@xarrow{#1}{#2}{}\da@rightarrow{\,}{}}{}%
  }%
}
\newcommand{\xlaxgets}[2][]{%
  \mathrel{%
    \mathpalette{\da@xarrow{#1}{#2}\da@leftarrow{}{}{\,}}{}%
  }%
}
\newcommand*{\da@xarrow}[7]{%
  % #1: below
  % #2: above
  % #3: arrow left
  % #4: arrow right
  % #5: space left 
  % #6: space right
  % #7: math style 
  \sbox0{$\ifx#7\scriptstyle\scriptscriptstyle\else\scriptstyle\fi#5#1#6\m@th$}%
  \sbox2{$\ifx#7\scriptstyle\scriptscriptstyle\else\scriptstyle\fi#5#2#6\m@th$}%
  \sbox4{$#7\dabar@\m@th$}%
  \dimen@=\wd0 %
  \ifdim\wd2 >\dimen@
    \dimen@=\wd2 %   
  \fi
  \count@=2 %
  \def\da@bars{\dabar@\dabar@}%
  \@whiledim\count@\wd4<\dimen@\do{%
    \advance\count@\@ne
    \expandafter\def\expandafter\da@bars\expandafter{%
      \da@bars
      \dabar@ 
    }%
  }%  
  \mathrel{#3}%
  \mathrel{%   
    \mathop{\da@bars}\limits
    \ifx\\#1\\%
    \else
      _{\copy0}%
    \fi
    \ifx\\#2\\%
    \else
      ^{\copy2}%
    \fi
  }%   
  \mathrel{#4}%
}
\makeatother

%%%%%%%%%%%%%%%%%%%%Decorations%%%%%%%%%%%%%%%%%%%%


%%%%%%%%%%%%%%%%%%%%All%%%%%%%%%%%%%%%%%%%%

\nc{\Ob}{\mathsf{Ob}}
\nc{\Hom}{\mathsf{Hom}}
\nc{\id}{\mathsf{id}}
\nc{\Set}{\mathsf{Set}}
\nc{\Grp}{\mathsf{Grp}}
\nc{\Grpd}{\mathsf{Grpd}}
\nc{\Ring}{\mathsf{Ring}}
\nc{\Top}{\mathsf{Top}}
\nc{\hTop}{\mathsf{hTop}}



\title{Lecture 1}

\date{Sep. 10, 2024}


\begin{document}

\frame{\titlepage}

\begin{frame}
	\frametitle{A slogan}

	\[
		\infty\mbox{-category theory }=\mbox{ category theory }+\mbox{ homotopy theory}.
	\]
	
\end{frame}

\begin{frame}
	\frametitle{Classical category theory}

	
\begin{defn}
	A \alert{category} $\mCC$ consists of the following data:\pause
	\begin{itemize}
		\item 
			\alert{Objects}: $a\in \Ob(\mCC)$. \pause
		\item
			\alert{Morphisms}: $f:a\to b$, $f\in \Hom(a,b)$. \pause
		\item
			\alert{Composition of morphisms}: $f$, $g$ $\Rightarrow$ $g\circ f$. \pause
			
	\end{itemize}
	Axioms:
	\begin{itemize}
		\item 
			\alert{Associativity}: 
			\[
				h\circ (g\circ f) = (h\circ g) \circ f. \pause
			\] 
		\item
			\alert{Identity}: $f:a\to b$, 
			\[
				\id_b\circ f = f = f\circ \id_a. \pause
			\]
	\end{itemize}
\end{defn}

	
	
\end{frame}

\begin{frame}
	\frametitle{Classical category theory}

	\begin{itemize}
		\item
			\emph{In order to study a collection of objects one should also consider suitably defined morphisms between such objects.}\pause
		\item
			Examples: $\alert{\Set}$, $\alert{\Grp}$, $\alert{\Ring}$, $\alert{\Top}$... 
	\end{itemize}
	


	
	
\end{frame}

\begin{frame}
	\frametitle{Three doctrines}

	\begin{itemize}
	\item[(1)]
		Morphisms are \alert{discrete}: $f=g$ or $f\neq g$. \pause
	\item[(2)]
		Associativity is \alert{strict}: $h\circ (g\circ f)\alert{=}(h\circ g) \circ f$. \pause
	\item[(3)]
		Composition is \alert{concrete}: there is no ambiguity for $g\circ f$. \pause
\end{itemize}
	
	We will abandon all of them.

\end{frame}

\begin{frame}
	\frametitle{Enrichment}

	\begin{itemize}
		\item
			\st{Doctrine (1): Morphisms are discrete}. \pause
		\item
			Richer structures on $\Hom(-,-)$.\pause
		\item
			\alert{topological spaces} $\Rightarrow$ \alert{topological categories}. \pause
		\item
			\alert{categories} $\Rightarrow$ \alert{strict 2-categories}. \pause
		\item
			\alert{strict $(n-1)$-categories} $\Rightarrow$ \alert{strict $n$-categories}. \pause
		\item
			...

	\end{itemize}
\end{frame}

\begin{frame}
	\frametitle{Strict 2-categories}

	A strict 2-category $\mCC$ consists of the following data:\pause
	\begin{itemize}
		\item
			Objects. \pause
		\item
			Morphisms and their compositions. \pause
		\item
			\alert{2-morphisms}: morphisms between morphisms,\pause and their \alert{vertical} and \alert{horizontal} compositions. \pause

	\end{itemize}
	Axioms: all given by \alert{equalities}. (Doctrine (2))


\end{frame}

\begin{frame}
	\frametitle{Strict 2-categories with one object}

	\begin{itemize}
		\item<1->
			$\mCD$: strict 2-category with $\Ob(\mCD) \simeq \{*\}$.  $\Leftrightarrow$
		\item<2->
			$\mCC:=\Hom(*,*)$: equipped with $-\ot-:\mCC\mt \mCC \to \mCC$ \alert{strictly} associative and unital. 
		\item<3->
			This is \alert{evil}
			\begin{itemize}
				\item[] Principle of isomorphism: \emph{all grammatically correct properties of objects of a fixed category are to be invariant under isomorphism}.\pause
			\end{itemize}
			and \alert{impractical}. \pause



		\item<4->
			\st{Strict} \alert{monoidal category}: supply \alert{associators}:
			\[
				X\ot (Y\ot Z) \xrightarrow{\simeq} (X\ot Y) \ot Z
			\]
			subject to certain coherent conditions.
		\item<3->[]

			\begin{exe}
				In ZFC, $(X\mt Y)\mt Z=X \mt(Y\mt Z)$ or $(X\mt Y)\mt Z \simeq X \mt(Y\mt Z)$?
			\end{exe}

	\end{itemize}

\end{frame}

\begin{frame}
	\frametitle{Weak $n$-categories}

	\begin{itemize}
		\item
			\st{Doctrine (2): associativity is strict} \pause
		\item
			Wishes for weak $n$-categories
			\begin{itemize}
				\item 
					Weak $1$-categories $\Leftrightarrow$ categories. \pause
				\item
					Weak $2$-categories with a single object $\Leftrightarrow$ monoidal categories. \pause
				\item
					$\Hom(a,b)$: weak $(n-1)$-category. \pause
				\item
					\emph{Principle of isomorphism}. \pause
			\end{itemize}
		\item
			For $n\ge 2$, we should never require two morphisms in a weak $n$-category to be \alert{equal}. \pause
		\item
			For $n> k$, we should never require two $k$-morphisms in a weak $n$-category to be \alert{equal}.
	\end{itemize}

\end{frame}

\begin{frame}
	\frametitle{Nightmare of coherence data}

	\begin{itemize}
		\item
			\alert{Associator}: $h\circ (g\circ f) \xrightarrow{\simeq} (h\circ g) \circ f$. \pause
		\item
			Commutative diagram for $2$-morphisms:
			\[
			\xymatrixrowsep{0.2in}
			\xymatrixcolsep{0.2in}
	\xymatrix{
		& & (kh)(gf) \ar[rrd] \\
		k(h(gf)) \ar[rru] \ar[rd]
		& & & & ((kh)g)f \\
		& k((hg)f) \ar[rr]
		& & (k(hg))f. \ar[ru]
	}
		\] \pause
		\item
			$n\ge 3$: \alert{witnessed} by an invertible $3$-morphism. \pause
		\item
			Commutative diagram for $3$-morphisms. \pause
		\item
			$n\ge 4$: \alert{witnessed} by an invertible $4$-morphism. \pause
		\item
			...
	\end{itemize}

\end{frame}

\begin{frame}
	\frametitle{A savior}

	\begin{slog}[Grothendieck's homotopy hypothesis]
		 
		 \[
		 	\infty\mbox{-groupoids }= \mbox{ homotopy types.}\pause
		 \]
	\end{slog}

	\begin{itemize}
		\item<3->
			Coherence data for associativity can be \alert{hidden} in the homotopy theory of spaces.
		\item<2->
			\alert{$\infty$-groupoids}: morphisms and higher morphisms are all invertible.  \pause
		
	\end{itemize}

\end{frame}

\begin{frame}
	\frametitle{A strategy}
	\begin{itemize}
		\item
			\alert{$(\infty,k)$-categories}: $m$-morphisms invertible for $m>k$. \pause
		\item
			Induction on $k$. \pause
		\item
			\alert{$(\infty,0)$-categories}:= $\infty$-groupoids := homotopy types. \pause
		\item
			Induction step: coherence data for associativity is already incorporated into the theory of $\infty$-groupoids!
		
	\end{itemize}

\end{frame}

\begin{frame}
	\frametitle{Models}
	\begin{itemize}
		\item
			Theory of $(\infty,0)$-categories $\Leftrightarrow$ homotopy theory of topological spaces. \pause
		\item
			Theory of $(\infty,1)$-categories $\Leftrightarrow$ homotopy theory of topological categories ? \pause
		\item
			Yes, but... \pause
		\item
			A more developed model: \alert{quasi-categories}. \pause \st{Doctrine (3): composition is concrete}.
		
	\end{itemize}

\end{frame}


\begin{frame}
	\frametitle{Homotopy category}

	\begin{itemize}
		\item 
			The \alert{homotopy category} $\hTop$: \pause
			\begin{itemize}
				\item 
					Objects=\alert{homotopy types}: topological spaces. \pause
				\item
					Morphisms: homotopy classes of continuous maps.  \pause
			\end{itemize}
		\item
			\alert{Homotopy equivalence}: isomorphisms in $\hTop$. \pause
		\item
			$\hTop$: inverting homotopy equivalences in $\Top$. \pause
		\item
			$\hTop_{\le n}$: full subcategory of \alert{homotopy $n$-types}: $\pi_k\simeq 0$, $k>n$.
	\end{itemize}

\end{frame}

\begin{frame}
	\frametitle{fundamental groupoid}

	\begin{itemize}
		\item 
			The \alert{fundamental groupoid} $\pi_{\le 1} X$:\pause
			\begin{itemize}
				\item Objects: points. \pause
				\item Morphisms: \alert{homotopy classes} of pathes. \pause
			\end{itemize}
		\item[]
			\begin{exe}
			Check the axiom of associativity for the above construction. \pause
			\end{exe}
		\item
			$\pi_{\le 1}X$ is indeed a groupoid. \pause
		\item
			$\pi_0(X) \simeq \Ob(\pi_{\le 1}X)/\simeq $. \pause
		\item
			$\pi_1(X,x) \simeq \Hom(x,x)$. \pause
		\item[]
			\begin{exe}
				Show $\pi_{\le 1}: \hTop_{\le 1} \to \Grpd$ is an equivalence.
			\end{exe}
	\end{itemize}
\end{frame}

\begin{frame}
	\frametitle{fundamental $2$-groupoid}

	\begin{itemize}
		\item 
			The \alert{fundamental $2$-groupoid} $\pi_{\le 2} X$:\pause
			\begin{itemize}
				\item Objects: points. \pause
				\item Morphisms: pathes. \pause
				\item 2-morphisms: \alert{homotopy classes} of homotopies between pathes. \pause
			\end{itemize}
		\item 
			Put associativity aside for a moment. \pause
		\item
			$\pi_{\le 2}X$ is indeed a $2$-groupoid. \pause
		\item
			$\pi_0(X) \simeq \Ob(\pi_{\le 2}X)/\simeq $. \pause
		\item
			$\pi_1(X,x) \simeq \Hom(x,x)/\simeq$. \pause
		\item
			$\pi_2(X,x) \simeq \Hom(\id_x,\id_x)$. \pause

	\end{itemize}

	\begin{exe}
		Describe the action of $\pi_1(X,x)$ on $\pi_2(X,x)$ in terms of $\pi_{\le 2}X$. \pause
	\end{exe}

	\begin{exe}
		Can we define morphisms as \alert{homotopy classes} of pathes?
	\end{exe}
\end{frame}

\begin{frame}
	\frametitle{$\pi_{\le 2}$ is not strict}

	\begin{itemize}
		\item 
			Pathes: $f$, $g$, $h$. \pause
			\begin{equation}
			\label{eqn-1}
		\big(h\circ (g\circ f)\big) (t) = 
		\left\{ \begin{array}{rcl}
		f(4t) & \mbox{for} & t\in [0,1/4] \\
		g(4t-1) & \mbox{for} & t\in [1/4,1/2] \\
		h(2t-1) & \mbox{for} & t\in [1/2,1]
			\end{array}\right. \pause
			\end{equation}
			\begin{equation}
	\label{eqn-2}
	\big( (h\circ g)\circ f\big) (t) = 
	\left\{ \begin{array}{rcl}
		f(2t) & \mbox{for} & t\in [0,1/2] \\
		g(4t-2) & \mbox{for} & t\in [1/2,3/4] \\
		h(4t-3) & \mbox{for} & t\in [3/4,1]
	\end{array}\right. \pause
\end{equation}

		\item
			Homotopic but not equal. \pause 
		\item
			Can you find a canonical homotopy? \pause
		\item
			Homotopy class of homotopies (\ref{eqn-1}) $\to$ (\ref{eqn-2}) is canonical. \pause $\Rightarrow$
		\item
			$\pi_{\le 2}X$ is a \alert{weak} $2$-category. \pause
	\end{itemize}

	\begin{exe}
		Challenge: show $\pi_{\le 2}: \hTop_{\le 2} \to 2\mbox{-}\Grpd$ is an equivalence.
	\end{exe}
	
\end{frame}

\begin{frame}
	\frametitle{Nightmare of homotopy coherence data}

	\begin{itemize}
		\item 
			$\pi_{\le n} X$, $n\ge 3$.  \pause
		\item
			2-morphism: \st{homotopy classes of} homotopies between pathes. \pause
		\item
			Make a \alert{non-canonical} choice for homotopy (\ref{eqn-1}) $\to$ (\ref{eqn-2}). \pause
		\item
			Fit the choice into the coherent data of associativity\pause: choose homotopies between homotopies between homotopies. \pause
		\item
			...
	\end{itemize}

\end{frame}


\begin{frame}
	\frametitle{In retrospect}

	\begin{itemize}
		\item 
			Two impossible tasks:\pause
			\begin{itemize}
				\item 
					Give a \alert{combinatorial} definition of weak $n$-groupoids; \pause
				\item
					Supply homotopy coherent data to make $\pi_{\le n}X$ a weak $n$-groupoid.\pause
			\end{itemize}
		\item
			Grothendieck's homotopy hypothesis: same task\pause; do neither. \pause
	\end{itemize}

	\begin{slog}
	\[
		n\mbox{-groupoids } = \mbox{ homotopy } n \mbox{-types}. \pause
	\]
	\[
		\infty\mbox{-groupoids } = \mbox{ homotopy types}.\pause
	\]
	\[
		\infty\mbox{-category theory }=\mbox{ category theory }+\mbox{ homotopy theory}.
	\]
	\end{slog}



\end{frame}


\begin{frame}
	\huge Thank you!


\end{frame}



\end{document}