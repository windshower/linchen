
%!TEX root = main.tex
\documentclass{amsart}
\textwidth=14.5cm \oddsidemargin=1cm
\evensidemargin=1cm
\usepackage{amsmath}
\usepackage{amsxtra}
\usepackage{amscd}
\usepackage{amsthm}
\usepackage{amsfonts}
\usepackage{amssymb}
\usepackage[foot]{amsaddr}
\usepackage{cite}
\usepackage{url}
\usepackage{rotating}
\usepackage{eucal}
\usepackage{tikz-cd}
\usepackage[all,2cell,color]{xy}
\UseAllTwocells
\UseCrayolaColors
\usepackage{graphicx}
\usepackage{pifont}
\usepackage{comment}
\usepackage{verbatim}
\usepackage{xcolor}
\usepackage{hyperref}
\usepackage{xparse}
\usepackage{upgreek}
\usepackage{MnSymbol}
\sloppy


%%%%%%%%%%%%%%%%%%%%Theorem%%%%%%%%%%%%%%%%%%%%
\newcounter{theorem}
\setcounter{theorem}{0}

\newtheorem{cor}[subsection]{Corollary}
\newtheorem{lem}[subsection]{Lemma}
\newtheorem{goal}[subsection]{Goal}
\newtheorem{lemdefn}[subsection]{Lemma-Definition}
\newtheorem{prop}[subsection]{Proposition}
\newtheorem{propdefn}[subsection]{Proposition-Definition}
\newtheorem{cordefn}[subsection]{Corollary-Definition}
\newtheorem{variant}[subsection]{Variant}
\newtheorem{warn}[subsection]{Warning}
\newtheorem{sugg}[subsection]{Suggestion}
\newtheorem{facts}[subsection]{Fact}
\newtheorem{ques}{Question}
\newtheorem{guess}{Guess}
\newtheorem{claim}{Claim}
\newtheorem{propconstr}[subsection]{Proposition-Construction}
\newtheorem{lemconstr}[subsection]{Lemma-Construction}
\newtheorem{ax}{Axiom}
\newtheorem{conje}[subsection]{Conjecture}
\newtheorem{mainthm}[subsection]{Main-Theorem}
\newtheorem{summ}[subsection]{Summary}
\newtheorem{thm}[subsection]{Theorem}
\newtheorem{thmdefn}[subsection]{Theorem-Definition}
\newtheorem{notn}[subsection]{Notation}
\newtheorem{convn}[subsection]{Convention}
\newtheorem{constr}[subsection]{Construction}


\theoremstyle{definition}

\newtheorem{defn}[subsection]{Definition}
\newtheorem{exam}[subsection]{Example}
\newtheorem{assum}[subsection]{Assumption}

\theoremstyle{remark}
\newtheorem{rem}[subsection]{Remark}
\newtheorem{exe}[subsection]{Exercise}


\numberwithin{equation}{section}


%%%%%%%%%%%%%%%%%%%%Commands%%%%%%%%%%%%%%%%%%%%

\newcommand{\nc}{\newcommand}
\nc\on{\operatorname}
\nc\renc{\renewcommand}


%%%%%%%%%%%%%%%%%%%%Sections%%%%%%%%%%%%%%%%%%%%

\nc\ssec{\subsection}
\nc\sssec{\subsubsection}

%%%%%%%%%%%%%%%%%%%%Environment%%%%%%%%%%%%%%%%%
\nc\blongeqn{\[ \begin{aligned}}
\nc\elongeqn{\end{aligned} \]}



%%%%%%%%%%%%%%%%%%%%Mathfont%%%%%%%%%%%%%%%%%%%%

\nc\mBA{{\mathbb A}}
\nc\mBB{{\mathbb B}}
\nc\mBC{{\mathbb C}}
\nc\mBD{{\mathbb D}}
\nc\mBE{{\mathbb E}}
\nc\mBF{{\mathbb F}}
\nc\mBG{{\mathbb G}}
\nc\mBH{{\mathbb H}}
\nc\mBI{{\mathbb I}}
\nc\mBJ{{\mathbb J}}
\nc\mBK{{\mathbb K}}
\nc\mBL{{\mathbb L}}
\nc\mBM{{\mathbb M}}
\nc\mBN{{\mathbb N}}
\nc\mBO{{\mathbb O}}
\nc\mBP{{\mathbb P}}
\nc\mBQ{{\mathbb Q}}
\nc\mBR{{\mathbb R}}
\nc\mBS{{\mathbb S}}
\nc\mBT{{\mathbb T}}
\nc\mBU{{\mathbb U}}
\nc\mBV{{\mathbb V}}
\nc\mBW{{\mathbb W}}
\nc\mBX{{\mathbb X}}
\nc\mBY{{\mathbb Y}}
\nc\mBZ{{\mathbb Z}}


\nc\mCA{{\mathcal A}}
\nc\mCB{{\mathcal B}}
\nc\mCC{{\mathcal C}}
\nc\mCD{{\mathcal D}}
\nc\mCE{{\mathcal E}}
\nc\mCF{{\mathcal F}}
\nc\mCG{{\mathcal G}}
\nc\mCH{{\mathcal H}}
\nc\mCI{{\mathcal I}}
\nc\mCJ{{\mathcal J}}
\nc\mCK{{\mathcal K}}
\nc\mCL{{\mathcal L}}
\nc\mCM{{\mathcal M}}
\nc\mCN{{\mathcal N}}
\nc\mCO{{\mathcal O}}
\nc\mCP{{\mathcal P}}
\nc\mCQ{{\mathcal Q}}
\nc\mCR{{\mathcal R}}
\nc\mCS{{\mathcal S}}
\nc\mCT{{\mathcal T}}
\nc\mCU{{\mathcal U}}
\nc\mCV{{\mathcal V}}
\nc\mCW{{\mathcal W}}
\nc\mCX{{\mathcal X}}
\nc\mCY{{\mathcal Y}}
\nc\mCZ{{\mathcal Z}}


\nc\mbA{{\mathsf A}}
\nc\mbB{{\mathsf B}}
\nc\mbC{{\mathsf C}}
\nc\mbD{{\mathsf D}}
\nc\mbE{{\mathsf E}}
\nc\mbF{{\mathsf F}}
\nc\mbG{{\mathsf G}}
\nc\mbH{{\mathsf H}}
\nc\mbI{{\mathsf I}}
\nc\mbJ{{\mathsf J}}
\nc\mbK{{\mathsf K}}
\nc\mbL{{\mathsf L}}
\nc\mbM{{\mathsf M}}
\nc\mbN{{\mathsf N}}
\nc\mbO{{\mathsf O}}
\nc\mbP{{\mathsf P}}
\nc\mbQ{{\mathsf Q}}
\nc\mbR{{\mathsf R}}
\nc\mbS{{\mathsf S}}
\nc\mbT{{\mathsf T}}
\nc\mbU{{\mathsf U}}
\nc\mbV{{\mathsf V}}
\nc\mbW{{\mathsf W}}
\nc\mbX{{\mathsf X}}
\nc\mbY{{\mathsf Y}}
\nc\mbZ{{\mathsf Z}}

\nc\mba{{\mathsf a}}
\nc\mbb{{\mathsf b}}
\nc\mbc{{\mathsf c}}
\nc\mbd{{\mathsf d}}
\nc\mbe{{\mathsf e}}
\nc\mbf{{\mathsf f}}
\nc\mbg{{\mathsf g}}
\nc\mbh{{\mathsf h}}
\nc\mbi{{\mathsf i}}
\nc\mbj{{\mathsf j}}
\nc\mbk{{\mathsf k}}
\nc\mbl{{\mathsf l}}
\nc\mbm{{\mathsf m}}
\nc\mbn{{\mathsf n}}
\nc\mbo{{\mathsf o}}
\nc\mbp{{\mathsf p}}
\nc\mbq{{\mathsf q}}
\nc\mbr{{\mathsf r}}
\nc\mbs{{\mathsf s}}
\nc\mbt{{\mathsf t}}
\nc\mbu{{\mathsf u}}
\nc\mbv{{\mathsf v}}
\nc\mbw{{\mathsf w}}
\nc\mbx{{\mathsf x}}
\nc\mby{{\mathsf y}}
\nc\mbz{{\mathsf z}}




\nc\mbfA{{\mathbf A}}
\nc\mbfB{{\mathbf B}}
\nc\mbfC{{\mathbf C}}
\nc\mbfD{{\mathbf D}}
\nc\mbfE{{\mathbf E}}
\nc\mbfF{{\mathbf F}}
\nc\mbfG{{\mathbf G}}
\nc\mbfH{{\mathbf H}}
\nc\mbfI{{\mathbf I}}
\nc\mbfJ{{\mathbf J}}
\nc\mbfK{{\mathbf K}}
\nc\mbfL{{\mathbf L}}
\nc\mbfM{{\mathbf M}}
\nc\mbfN{{\mathbf N}}
\nc\mbfO{{\mathbf O}}
\nc\mbfP{{\mathbf P}}
\nc\mbfQ{{\mathbf Q}}
\nc\mbfR{{\mathbf R}}
\nc\mbfS{{\mathbf S}}
\nc\mbfT{{\mathbf T}}
\nc\mbfU{{\mathbf U}}
\nc\mbfV{{\mathbf V}}
\nc\mbfW{{\mathbf W}}
\nc\mbfX{{\mathbf X}}
\nc\mbfY{{\mathbf Y}}
\nc\mbfZ{{\mathbf Z}}

\nc\mbfa{{\mathbf a}}
\nc\mbfb{{\mathbf b}}
\nc\mbfc{{\mathbf c}}
\nc\mbfd{{\mathbf d}}
\nc\mbfe{{\mathbf e}}
\nc\mbff{{\mathbf f}}
\nc\mbfg{{\mathbf g}}
\nc\mbfh{{\mathbf h}}
\nc\mbfi{{\mathbf i}}
\nc\mbfj{{\mathbf j}}
\nc\mbfk{{\mathbf k}}
\nc\mbfl{{\mathbf l}}
\nc\mbfm{{\mathbf m}}
\nc\mbfn{{\mathbf n}}
\nc\mbfo{{\mathbf o}}
\nc\mbfp{{\mathbf p}}
\nc\mbfq{{\mathbf q}}
\nc\mbfr{{\mathbf r}}
\nc\mbfs{{\mathbf s}}
\nc\mbft{{\mathbf t}}
\nc\mbfu{{\mathbf u}}
\nc\mbfv{{\mathbf v}}
\nc\mbfw{{\mathbf w}}
\nc\mbfx{{\mathbf x}}
\nc\mbfy{{\mathbf y}}
\nc\mbfz{{\mathbf z}}

\nc\mfa{{\mathfrak a}}
\nc\mfb{{\mathfrak b}}
\nc\mfc{{\mathfrak c}}
\nc\mfd{{\mathfrak d}}
\nc\mfe{{\mathfrak e}}
\nc\mff{{\mathfrak f}}
\nc\mfg{{\mathfrak g}}
\nc\mfh{{\mathfrak h}}
\nc\mfi{{\mathfrak i}}
\nc\mfj{{\mathfrak j}}
\nc\mfk{{\mathfrak k}}
\nc\mfl{{\mathfrak l}}
\nc\mfm{{\mathfrak m}}
\nc\mfn{{\mathfrak n}}
\nc\mfo{{\mathfrak o}}
\nc\mfp{{\mathfrak p}}
\nc\mfq{{\mathfrak q}}
\nc\mfr{{\mathfrak r}}
\nc\mfs{{\mathfrak s}}
\nc\mft{{\mathfrak t}}
\nc\mfu{{\mathfrak u}}
\nc\mfv{{\mathfrak v}}
\nc\mfw{{\mathfrak w}}
\nc\mfx{{\mathfrak x}}
\nc\mfy{{\mathfrak y}}
\nc\mfz{{\mathfrak z}}

\nc{\one}{{\mathsf{1}}}


\nc\clambda{ {\check{\lambda} }}
\nc\cmu{ {\check{\mu} }}

\nc\bDelta{\mathsf{\Delta}}
\nc\bGamma{\mathsf{\Gamma}}
\nc\bLambda{\mathsf{\Lambda}}


\nc\loccit{\emph{loc.cit.}}



%%%%%%%%%%%%%%%%%%%%Operations-limit%%%%%%%%%%%%%%%%%%%%

\NewDocumentCommand{\ot}{e{_^}}{
  \mathbin{\mathop{\otimes}\displaylimits
    \IfValueT{#1}{_{#1}}
    \IfValueT{#2}{^{#2}}
  }
}
\NewDocumentCommand{\boxt}{e{_^}}{
  \mathbin{\mathop{\boxtimes}\displaylimits
    \IfValueT{#1}{_{#1}}
    \IfValueT{#2}{^{#2}}
  }
}
\NewDocumentCommand{\mt}{e{_^}}{
  \mathbin{\mathop{\times}\displaylimits
    \IfValueT{#1}{_{#1}}
    \IfValueT{#2}{^{#2}}
  }
}
\NewDocumentCommand{\convolve}{e{_^}}{
  \mathbin{\mathop{\star}\displaylimits
    \IfValueT{#1}{_{#1}}
    \IfValueT{#2}{^{#2}}
  }
}
\NewDocumentCommand{\colim}{e{_^}}{
  \mathbin{\mathop{\operatorname{colim}}\displaylimits
    \IfValueT{#1}{_{#1}\,}
    \IfValueT{#2}{^{#2}\,}
  }
}
\NewDocumentCommand{\laxlim}{e{_^}}{
  \mathbin{\mathop{\operatorname{laxlim}}\displaylimits
    \IfValueT{#1}{_{#1}\,}
    \IfValueT{#2}{^{#2}\,}
  }
}
\NewDocumentCommand{\oplaxlim}{e{_^}}{
  \mathbin{\mathop\operatorname{oplax-lim}\displaylimits
    \IfValueT{#1}{_{#1}\,}
    \IfValueT{#2}{^{#2}\,}
  }
}


%%%%%%%%%%%%%%%%%%%%Arrows%%%%%%%%%%%%%%%%%%%%


\makeatletter
\newcommand{\laxto}{\dashedrightarrow}
\newcommand{\xrightleftarrows}[1]{\mathrel{\substack{\xrightarrow{#1} \\[-.9ex] \xleftarrow{#1}}}}
\newcommand{\adj}{\xrightleftarrows{\rule{0.5cm}{0cm}}}

\newcommand*{\da@rightarrow}{\mathchar"0\hexnumber@\symAMSa 4B }
\newcommand*{\da@leftarrow}{\mathchar"0\hexnumber@\symAMSa 4C }
\newcommand*{\xlaxto}[2][]{%
  \mathrel{%
    \mathpalette{\da@xarrow{#1}{#2}{}\da@rightarrow{\,}{}}{}%
  }%
}
\newcommand{\xlaxgets}[2][]{%
  \mathrel{%
    \mathpalette{\da@xarrow{#1}{#2}\da@leftarrow{}{}{\,}}{}%
  }%
}
\newcommand*{\da@xarrow}[7]{%
  % #1: below
  % #2: above
  % #3: arrow left
  % #4: arrow right
  % #5: space left 
  % #6: space right
  % #7: math style 
  \sbox0{$\ifx#7\scriptstyle\scriptscriptstyle\else\scriptstyle\fi#5#1#6\m@th$}%
  \sbox2{$\ifx#7\scriptstyle\scriptscriptstyle\else\scriptstyle\fi#5#2#6\m@th$}%
  \sbox4{$#7\dabar@\m@th$}%
  \dimen@=\wd0 %
  \ifdim\wd2 >\dimen@
    \dimen@=\wd2 %   
  \fi
  \count@=2 %
  \def\da@bars{\dabar@\dabar@}%
  \@whiledim\count@\wd4<\dimen@\do{%
    \advance\count@\@ne
    \expandafter\def\expandafter\da@bars\expandafter{%
      \da@bars
      \dabar@ 
    }%
  }%  
  \mathrel{#3}%
  \mathrel{%   
    \mathop{\da@bars}\limits
    \ifx\\#1\\%
    \else
      _{\copy0}%
    \fi
    \ifx\\#2\\%
    \else
      ^{\copy2}%
    \fi
  }%   
  \mathrel{#4}%
}
\makeatother

%%%%%%%%%%%%%%%%%%%%Decorations%%%%%%%%%%%%%%%%%%%%
\nc{\wt}{\widetilde}
\nc{\ol}{\overline}

\nc{\red}{\textcolor{red}}
\nc{\blue}{\textcolor{blue}}
\nc{\purple}{\textcolor{violet}}

\nc{\simorlax}{{\red\simeq/\blue\lax}}

%%%%%%%%%%%%%%%%%%%%All%%%%%%%%%%%%%%%%%%%%

\nc{\Id}{\mathsf{Id}}
\nc{\gl}{\mathfrak{gl}}
\renc{\sl}{\mathfrak{sl}}
\nc{\GL}{\mathsf{GL}}
\nc{\SL}{\mathsf{SL}}
\nc{\PGL}{\mathsf{PGL}}
\nc{\hmod}{\mathsf{-mod}}
\nc{\Vect}{\mathsf{Vect}}
\nc{\tr}{\mathsf{tr}}
\nc{\Kil}{\mathsf{Kil}}
\nc{\ad}{{\mathsf{ad}}}
\nc{\Ad}{\mathsf{Ad}}
\nc{\oblv}{\mathsf{oblv}}
\nc{\gr}{\mathsf{gr}}
\nc{\Sym}{\mathsf{Sym}}
\nc{\QCoh}{\mathsf{QCoh}}
\nc{\ind}{\mathsf{ind}}
\nc{\Spec}{\mathsf{Spec}}
\nc{\Hom}{\mathsf{Hom}}
\nc{\Ext}{\mathsf{Ext}}
\nc{\Grp}{\mathsf{Grp}}
\nc{\pt}{\mathsf{pt}}
\nc{\Lie}{\mathsf{Lie}}
\nc{\CAlg}{\mathsf{CAlg}}
\nc{\Der}{\mathsf{Der}}
\nc{\Rep}{\mathsf{Rep}}
\renc{\sc}{{\mathsf{sc}}}
\nc{\Fl}{\mathsf{Fl}}
\nc{\Fun}{\mathsf{Fun}}
\nc{\ev}{\mathsf{ev}}
\nc{\surj}{\twoheadrightarrow}
\nc{\inj}{\hookrightarrow}
\nc{\HC}{\mathsf{HC}}
\nc{\cl}{\mathsf{cl}}
\renc{\Im}{\mathsf{Im}}
\renc{\ker}{\mathsf{ker}}
\nc{\coker}{\mathsf{coker}}
\nc{\Tor}{\mathsf{Tor}}
\nc{\op}{\mathsf{op}}
\nc{\length}{\mathsf{length}}
\nc{\fd}{{\mathsf{fd}}}
\nc{\weight}{\mathsf{wt}}
\nc{\semis}{{\mathsf{ss}}}
\nc{\qc}{{\mathsf{qc}}}
\nc{\pr}{\mathsf{pr}}
\nc{\act}{\mathsf{act}}
\nc{\dR}{{\mathsf{dR}}}
\nc{\hol}{{\mathsf{hol}}}
\nc{\Pic}{{\mathsf{Pic}}}
\nc{\Loc}{\mathsf{Loc}}
\nc{\IC}{\mathsf{IC}}

\begin{document}


\title{Lecture 1}

\date{Sep. 10, 2024}

\maketitle

The goal of this course is to give a non-technical introduction to the theory of $\infty$-categories, or in general, homotopy coherent mathematics. This course focuses on ideas and motivations, and hopefully serves as a guide to the foundational references in this field, including \cite{HTT} and \cite{HA}.

\medskip

Throughout this course, (topological) spaces mean \emph{CW complexes}.

\section{A slogan}

\begin{slog}
	\[
		\infty\textrm{-category theory }=\textrm{ category theory }+\textrm{ homotopy theory}.
	\]
\end{slog}

\section{Classical category theory}

\ssec{}

Categories were introduced by Eilenberg and Maclane in 1945 among their works on algebraic topology and \emph{homological} algebra. 

\begin{defn}
	A \textbf{category} $\mCC$ consists of the following data:
	\begin{itemize}
		\item 
			A \emph{class} $\Ob(\mCC)$, whose elements are called \textbf{objects}.
		\item
			For any two objects $a$ and $b$, a \emph{class} $\Hom(a,b)$, whose elements are called \textbf{morphisms} from $a$ to $b$, denoted by $f:a\to b$.
		\item
			A binary operation $\circ$, called \textbf{composition of morphisms}, such that for any three objects $a$, $b$ and $c$, we have a map
			\[
				\circ: \Hom(a,b) \mt \Hom(b,c) \to \Hom(a,c).
			\]
	\end{itemize}
	The above data should satisfy two axioms:
	\begin{itemize}
		\item 
			Associativity: for morphisms $f:a\to b$, $g: b\to c$ and $h:c\to d$,
			\[
				h\circ (g\circ f) = (h\circ g) \circ f.
			\]
		\item
			Identity: for any object $x$, there exists a morphism $\id_x: x\to x$, called the \textbf{identity morphism} for $x$, such that for any morphism $f:a\to b$, we have
			\[
				\id_b\circ f = f = f\circ \id_a.
			\]
	\end{itemize}
\end{defn}

\ssec{}

The power of category theory is reflected in the following principle: 
\begin{itemize}
	\item[] \emph{In order to study a collection of objects one should also consider suitably defined morphisms between such objects}.
\end{itemize}
People assemble their favorite mathematical entities, which are often \emph{structured sets} into a category, and declare the morphisms to be functions that preserve these structures. Examples include $\Set$, $\Grp$, $\Ring$, $\Top$... 

\ssec{}
\label{ssec-doctrines}

We want to highlight the following doctrines in (classical) category theory:
\begin{itemize}
	\item[(1)]
		Morphisms are \emph{discrete}: for two morphisms $f$ and $g$, one can say $f=g$ or $f\neq g$, and this is the only comparison that one can make.
	\item[(2)]
		Associativity is \emph{strict}: $h\circ (g\circ f)$ and $(h\circ g) \circ f$ are required to be equal, rather than equivalent in a weaker sense.
	\item[(3)]
		Composition is \emph{concrete}: there is no ambiguity for $g\circ f$.
\end{itemize}
In this course, we will abandon all of them.

\section{Towards higher categories}

\ssec{}

We will abandon Doctrine (1) and endow $\Hom(a,b)$ with richer structures: they can be topological spaces or even categories themselves. The latter defines \textbf{strict $2$-categories}, which have objects and morphisms, as well as morphisms between morphisms, known as \textbf{$2$-morphisms}. There are two types of compositions of 2-morphisms: the \textbf{vertical} one and the \textbf{horizontal} one.
\[
	\begin{tikzcd}
	\bullet 
		\ar[r, bend  left=60, ""{name=f,mydescription}]
		\ar[r, bend  left=60]
		\ar[r, ""{name=g, mydescription}]
		\ar[r]
  		\ar[r, bend right=60, ""{name=h, mydescription}]
  		\ar[r, bend right=60]
    & \bullet 
    	\ar[Rightarrow, from=f, to=g, ""]
    	\ar[Rightarrow, from=g, to=h, ""]
    & \bullet
    	\ar[r, bend  left=40, ""{name=a,mydescription}]
		\ar[r, bend  left=40]
		\ar[r, bend right=40, ""{name=b, mydescription}]
  		\ar[r, bend right=40]
  	& \bullet
  		\ar[Rightarrow, from=a, to=b, ""]
  		\ar[r, bend  left=40, ""{name=c,mydescription}]
		\ar[r, bend  left=40]
		\ar[r, bend right=40, ""{name=e, mydescription}]
  		\ar[r, bend right=40]
  	& \bullet
  		\ar[Rightarrow, from=c, to=e, ""]
	\end{tikzcd}
\]
These compositions should satisfy a list of axioms of associativity and identity, which are all described via \emph{equalities}. 

\medskip
By induction, one obtains the notion of strict $n$-categories. In the language of classical category theory, we have:

\begin{defn}
	A \textbf{strict $n$-category} is a category enriched in strict $(n-1)$-categories.
\end{defn}

\ssec{}

However, there are very few interesting examples of strict $n$-categories:

\begin{exam}
	A strict $2$-category with a single object $*$ amounts to the data of a category $\mCC:=\Hom(*,*)$ equipped with a multiplication functor $-\ot-:\mCC\mt \mCC \to \mCC$ which is \emph{strictly} associative and unital. This definition is \emph{evil}\footnote{Principle of isomorphism: \emph{all grammatically correct properties of objects of a fixed category are to be invariant under isomorphism}.} and impractical\footnote{Even for sets, $(X\mt Y)\mt Z=X \mt(Y\mt Z)$ does not make sense in ZF.}. The correct notion is that of a \textbf{monoidal category}, where instead we \emph{supply} natural isomorphisms
	\[
		X\ot (Y\ot Z) \xrightarrow{\simeq} (X\ot Y) \ot Z,\;  \mathbb{1} \ot X \xrightarrow{\simeq} X \xleftarrow{\simeq} X \ot \mathbb{1}
	\]
	subject to certain coherent conditions.
\end{exam}

\ssec{}

The above example suggests we should also abandon Doctrine (2) and allow associativity to hold in a weaker sense. This leads to the concept, but not a definition, of \textbf{weak $n$-categories} or even \textbf{weak $\omega$-categories} when $n=\infty$.

\medskip

We have at least the following wishes in a definition of weak $n$-categories:
\begin{itemize}
	\item
		Weak $1$-categories are just categories.
	\item
		Weak $2$-category with a single object amounts to the data of a monoidal category.
	\item
		For any two objects $a$ and $b$ in a weak $n$-category, $\Hom(a,b)$ should be a weak $(n-1)$-category.
	\item
		Its definition should satisfy the \emph{principle of isomorphism}. 
\end{itemize}
Combining the last two wishes, we obtain:
\begin{itemize}
	\item[]
		\emph{For $n\ge 2$, we should never require two morphisms in a weak $n$-category to be equal}.
\end{itemize}
Then by induction,
\begin{itemize}
	\item[]
		\emph{For $n> k$, we should never require two $k$-morphisms in a weak $n$-category to be equal}.
\end{itemize}
These innocuous wishes would lead to a combinatorial nightmare.

\ssec{}
Let $f$, $g$ and $h$ be composable morphisms in a weak $n$-category. By previous discussion, in the axioms of associativities, the compositions $h\circ (g\circ f)$ and $(h\circ g) \circ f$ should be \emph{equivalent} rather than \emph{equal}. 

\medskip

However, as suggested by the definition of monoidal categories, this equivalence must be viewed as a \emph{structure} rather than a \emph{property}: we need to \emph{supply} an invertible $2$-morphism from $h\circ (g\circ f)$ to $(h\circ g) \circ f$, called the \textbf{associator}, such that the following diagram commutes:
\[
	\xymatrix{
		& & (kh)(gf) \ar[rrd] \\
		k(h(gf)) \ar[rru] \ar[rd]
		& & & & ((kh)g)f \\
		& k((hg)f) \ar[rr]
		& & (k(hg))f. \ar[ru]
	}
\]

\medskip

However, when $n\ge 3$, even this commutativity of $2$-morphisms should be understood as equivalence rather than equality, and should be \emph{witnessed} by an invertible $3$-morphism, say, from the clockwise arch to the counterclockwise one. These $3$-morphisms themselves should make a certain diagram commute, which is again witnessed by an invertible $4$-morphisms if $n\ge 4$...

\medskip

Even worse, we also need to treat axioms of associativity for composition of higher morphisms, and there are various types of them. Reminder: $2$-morphisms can be composed both \textbf{vertically} and \textbf{horizontally}, and the latter interacts with composition of $1$-morphisms.

\medskip

This endless list of associativity, known as the \textbf{coherence data}, soon become impossible to write down and difficult to work with.

\ssec{}
	
This painful pursuit of defining weak $n$-categories combinatorially was started by Bénabou in 1967 and probably terminated around early 2000s.


\ssec{}

What saves higher-categorists (and this course) is the following \textbf{homotopy hypothesis} proposed in Grothendieck's \emph{pursuing stacks}, written around 1983:

\begin{slog}
	The theory of $\infty$-groupoids should be the same as the homotopy theory of spaces.
\end{slog}

\ssec{}

Here \textbf{$\infty$-groupoids} mean weak $\omega$-categories whose morphisms and higher morphisms are all invertible. In general, \textbf{$(n,k)$-categories} mean weak $n$-categories whose $m$-morphisms are invertible for $m>k$. When $n=\infty$, weak $\infty$-categories in above mean weak $\omega$-categories\footnote{
	$(\infty,1)$-categories are often just called $\infty$-categories.
}.

\medskip

The insight is: the coherence data for associativity, which is combinatorially formidable, can be \emph{hidden away} in the homotopy theory of spaces.

\medskip
This gives one approach to develop the theory of higher categories: we start with \emph{declaring} $\infty$-groupoids, or $(\infty,0)$-categories, to be homotopy types of spaces, and inductively define $(\infty,k)$-categories. In this induction step, the coherence data for associativity, which is about \emph{invertible} higher morphisms, has already been tamed by the theory of $\infty$-groupoids.

\ssec{}
One may ask: if the theory of $(\infty,0)$-categories is the same as homotopy theory of topological spaces, should the theory of $(\infty,1)$-categories be the same as homotopy theory of \textbf{topological categories}, i.e., categories enriched in topological spaces?

\medskip
The answer is: yes, but we have to first understand the meaning of the latter. This requires Quillen's \emph{abstract homotopy theory}, known as \emph{model categories}, which will be the content of the next lecture.

\medskip
Nevertheless, topological categories are not the most convenient model of $(\infty,1)$-categories, at least in certain interesting applications of the latter. The current most developed model, thanks to Lurie's books, is \textbf{quasi-categories}, wihch, as we alluded, abandon Doctrine (3).


\section{Homotopy hypothesis}

\ssec{}
	
Homotopy theory dates back to the works of Poincaré on fundamental groups starting from 1895.


\begin{defn}
	Let $X$, $Y$ be topological spaces, and $f$, $g:X\to Y$ be continuous functions. A \textbf{homotopy} between $f$ and $g$ is a continuous function $H: X\mt [0,1] \to Y$ such that $H(-,0)=f$ and $H(-,1)=g$. We say $f$ and $g$ are \textbf{homotopic} if there exists a homotopy between them.

	Let $X$, $Y$ be topological spaces. A \textbf{homotopy equivalence} between $X$ and $Y$ is a pair of continuous functions $p:X\to Y$ and $q:Y \to X$ such that $q\circ p$ is homotopic to $\id_X$ and $p\circ q$ is homotopic to $\id_Y$.
\end{defn}

\ssec{}

Classical homotopy theory focuses on information about topological spaces that are invariant under homotopy equivalence. Such information can be encoded into a category.

\begin{constr}
	The \textbf{homotopy category} $\hTop$ is defined as follows:
	\begin{itemize}
		\item 
			Objects are topological spaces.
		\item
			Morphisms are homotopy classes of continuous maps, with composition of morphisms induced by composition of continuous maps.
	\end{itemize}
\end{constr}

\begin{variant}
	Alternatively, we can consider \emph{pointed} spaces, i.e. spaces equipped with a base point, we obtain a category denoted by $\hTop_*$.
\end{variant}

\ssec{}

By design, homotopy equivalences are exactly equivalences in $\hTop$. In fact, $\hTop$ can be obtained from $\Top$ by inverting all homotopy equivalences.

\begin{defn}
	Let $X$ be a pointed space. We define
	\[
		\pi_n(X):= [S^n,X]:= \Hom_{\hTop_*}(S^n,X) 
	\]
\end{defn}

\ssec{}

Recall for $n\ge 1$, $\pi_n(X)$ has a natural group structure given by concatenation of spheres, known as the $n$-th \textbf{homotopy groups} of $X$. When $n\ge 2$, $\pi_n(X)$ is abelian.

\medskip
Also recall that $\pi_0(X)$ does not depend on the base point.

\begin{defn}
	A \textbf{homotopy type} is an object $X$ in $\hTop$. We say $X$ is a \textbf{homotopy $n$-type} if for any base point, $\pi_k(X,x)\simeq 0$ for $k>n$. 

	\medskip
	Let $\hTop_{\le n} \subset \hTop$ be the full subcategory of homotopy $n$-types.
\end{defn}


\ssec{}

It turns out the information of $\pi_0(X)$ and $\pi_1(X)$ can be captured by a category associated to $X$.

\begin{constr}
	Let $X$ be a topological space. The \textbf{fundamental groupoid} $\pi_{\le 1} X$ of $X$ is a category defined as follows:
	\begin{itemize}
		\item 
			Objects are points of $X$;
		\item 
			Morphisms are \emph{homotopy classes} of pathes in $X$, with induced by concatenation of intervals.
	\end{itemize}
\end{constr}

\begin{exe}
	Check the axiom of associativity for the above construction.
\end{exe}

\ssec{}

It is easy to see:
\begin{itemize}
	\item
		The fundamental groupoid $\pi_{\le 1} X$ is indeed a groupoid, i.e., all the morphisms are invertible.
	\item 
		The set $\pi_0(X)$ can be identified with the set of isomorphism classes of objects in $\pi_{\le 1}X$.
	\item
		There is a natural group homomorphism $\pi_1(X,x) \simeq \Hom(x,x)$, where $x$ in the RHS is viewed as an object in $\pi_{\le 1}X$.
\end{itemize}

\medskip

\begin{exe}
	Show that $\pi_{\le 1}$ defines an equivalence from $\hTop_{\le 1}$ to the category $\Grpd$ of (small) groupoids, where morphisms are given by \emph{equivalence classes} of functors. Hint: Eilenberg–MacLane spaces.
\end{exe}

\ssec{}

Encouraged by the above, one may attempt to construct a $2$-category $\pi_{\le 2}X$ as follows:
\begin{itemize}
	\item 
		Objects are points of $X$;
	\item
		Morphisms are pathes in $X$, with composition induced by concatenation of intervals.
	\item
		$2$-morphisms are homotopy classes of homotopies between pathes in $X$, with composition induced by concatenation of squares.
\end{itemize}
If this definition is possible, note that
\begin{itemize}
	\item 
		All the morphisms and 2-morphisms are invertible.
	\item
		The set $\pi_0(X)$ can be identified with the set of isomorphism classes of objects in $\pi_{\le 2}X$.
	\item
		The group $\pi_1(X,x)$ can be identified with the set of isomorphism classes of objects in $\Hom(x,x)$.
	\item
		There is a natural group homomorphism $\pi_2(X,x) \simeq \Hom(\id_x,\id_x)$, where $\id_x$ in the RHS is viewed as an object in $\Hom(x,x)$.

\end{itemize}

\begin{exe}
	Describe the action of $\pi_1(X,x)$ on $\pi_2(X,x)$ in terms of $\pi_{\le 2}X$.
\end{exe}

\begin{exe}
	\label{exe-why-not-modulo-homotopy?}
	In the above definition of $\pi_{\le 2}X$, can we still define morphisms as \emph{homotopy classes} of pathes in $X$? Convince yourself that then $2$-morphisms will not be well-defined.

	If you get stuck, try the following: in the definition of $\pi_{\le 1}X$, can we define objects as \emph{homotopy classes} of points, a.k.a. connected components of $X$?
\end{exe}



\ssec{}
	
Note that $\pi_{\le 2}X$ cannot be strict: for composable pathes $f$, $g$, $h:[0,1]\to X$, the compositions
\begin{equation}
	\label{eqn-1}
	\big(h\circ (g\circ f)\big) (t) = 
	\left\{ \begin{array}{rcl}
		f(4t) & \mbox{for} & t\in [0,1/4] \\
		g(4t-1) & \mbox{for} & t\in [1/4,1/2] \\
		h(2t-1) & \mbox{for} & t\in [1/2,1]
	\end{array}\right.
\end{equation}
\begin{equation}
	\label{eqn-2}
	\big( (h\circ g)\circ f\big) (t) = 
	\left\{ \begin{array}{rcl}
		f(2t) & \mbox{for} & t\in [0,1/2] \\
		g(4t-2) & \mbox{for} & t\in [1/2,3/4] \\
		h(4t-3) & \mbox{for} & t\in [3/4,1]
	\end{array}\right.
\end{equation}
are homotopic but not equal. Moreover, it is hard to find a \emph{natural} homotopy between them, although all such homotopies are homotopic to each other, as long as their constructions work for any $X$.

\medskip

The last statement provides the coherent data for associativity in $\pi_{\le 2}X$. One can check that $\pi_{\le 2}X$ is indeed a weak $2$-category.

\begin{cha}
	Show that $\pi_{\le 2}$ defines an equivalence from $\hTop_{\le 2}$ to the category $2\mbox{-}\Grpd$ of (small) weak $2$-groupoids, where morphisms are given by \emph{equivalence classes} of functors.
\end{cha}

\ssec{}

For $n\ge 3$, to construct $\pi_{\le n} X$, according to Exercise \ref{exe-why-not-modulo-homotopy?}, 2-morphism would be homotopies between pathes rather than the homotopy classes of such homotopies. Hence we have to make a \emph{choice} of homotopy from (\ref{eqn-1}) to (\ref{eqn-2}), as long as the theory of weak $n$-categories is developed combinatorially. Such choice is unnatural, and we have to fit them into the coherent data of associativity in the definition of weak $n$-categories. As $n$ in $\pi_{\le n}X$ grows, such \textbf{homotopy coherent data} become formidable.

\ssec{}

Heuristically, we have two impossible tasks:
\begin{itemize}
	\item 
		To give a combinatorial definition of (weak) $n$-groupoids;
	\item
		To verify, or rather, provide homotopy coherent data to make $\pi_{\le n}X$ a weak $n$-groupoid.
\end{itemize}
Grothendieck's homotopy hypothesis says these are actually the same task, and we should do neither.

\begin{slog}
	\[
		n\mbox{-groupoids } = \mbox{ homotopy } n \mbox{-types};
	\]
	\[
		\infty\mbox{-groupoids } = \mbox{ homotopy types}.
	\]
	
\end{slog}

\ssec{}

This reunion of category theory and homotopy theory, which can even date back to Kan's work in the 1950s, is the guiding philosophy of this course.




\bibliography{mybiblio.bib}{}
\bibliographystyle{alpha}






\end{document} 


