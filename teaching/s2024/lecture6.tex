
%!TEX root = main.tex
\documentclass{amsart}
\textwidth=14.5cm \oddsidemargin=1cm
\evensidemargin=1cm
\usepackage{amsmath}
\usepackage{amsxtra}
\usepackage{amscd}
\usepackage{amsthm}
\usepackage{amsfonts}
\usepackage{amssymb}
\usepackage[foot]{amsaddr}
\usepackage{cite}
\usepackage{url}
\usepackage{rotating}
\usepackage{eucal}
\usepackage{tikz-cd}
\usepackage[all,2cell,color]{xy}
\UseAllTwocells
\UseCrayolaColors
\usepackage{graphicx}
\usepackage{pifont}
\usepackage{comment}
\usepackage{verbatim}
\usepackage{xcolor}
\usepackage{hyperref}
\usepackage{xparse}
\usepackage{upgreek}
\usepackage{MnSymbol}
\sloppy


%%%%%%%%%%%%%%%%%%%%Theorem%%%%%%%%%%%%%%%%%%%%
\newcounter{theorem}
\setcounter{theorem}{0}

\newtheorem{cor}[subsection]{Corollary}
\newtheorem{lem}[subsection]{Lemma}
\newtheorem{goal}[subsection]{Goal}
\newtheorem{lemdefn}[subsection]{Lemma-Definition}
\newtheorem{prop}[subsection]{Proposition}
\newtheorem{propdefn}[subsection]{Proposition-Definition}
\newtheorem{cordefn}[subsection]{Corollary-Definition}
\newtheorem{variant}[subsection]{Variant}
\newtheorem{warn}[subsection]{Warning}
\newtheorem{sugg}[subsection]{Suggestion}
\newtheorem{facts}[subsection]{Fact}
\newtheorem{ques}{Question}
\newtheorem{guess}{Guess}
\newtheorem{claim}{Claim}
\newtheorem{propconstr}[subsection]{Proposition-Construction}
\newtheorem{lemconstr}[subsection]{Lemma-Construction}
\newtheorem{ax}{Axiom}
\newtheorem{conje}[subsection]{Conjecture}
\newtheorem{mainthm}[subsection]{Main-Theorem}
\newtheorem{summ}[subsection]{Summary}
\newtheorem{thm}[subsection]{Theorem}
\newtheorem{thmdefn}[subsection]{Theorem-Definition}
\newtheorem{notn}[subsection]{Notation}
\newtheorem{convn}[subsection]{Convention}
\newtheorem{constr}[subsection]{Construction}


\theoremstyle{definition}

\newtheorem{defn}[subsection]{Definition}
\newtheorem{exam}[subsection]{Example}
\newtheorem{assum}[subsection]{Assumption}

\theoremstyle{remark}
\newtheorem{rem}[subsection]{Remark}
\newtheorem{exe}[subsection]{Exercise}


\numberwithin{equation}{section}


%%%%%%%%%%%%%%%%%%%%Commands%%%%%%%%%%%%%%%%%%%%

\newcommand{\nc}{\newcommand}
\nc\on{\operatorname}
\nc\renc{\renewcommand}


%%%%%%%%%%%%%%%%%%%%Sections%%%%%%%%%%%%%%%%%%%%

\nc\ssec{\subsection}
\nc\sssec{\subsubsection}

%%%%%%%%%%%%%%%%%%%%Environment%%%%%%%%%%%%%%%%%
\nc\blongeqn{\[ \begin{aligned}}
\nc\elongeqn{\end{aligned} \]}



%%%%%%%%%%%%%%%%%%%%Mathfont%%%%%%%%%%%%%%%%%%%%

\nc\mBA{{\mathbb A}}
\nc\mBB{{\mathbb B}}
\nc\mBC{{\mathbb C}}
\nc\mBD{{\mathbb D}}
\nc\mBE{{\mathbb E}}
\nc\mBF{{\mathbb F}}
\nc\mBG{{\mathbb G}}
\nc\mBH{{\mathbb H}}
\nc\mBI{{\mathbb I}}
\nc\mBJ{{\mathbb J}}
\nc\mBK{{\mathbb K}}
\nc\mBL{{\mathbb L}}
\nc\mBM{{\mathbb M}}
\nc\mBN{{\mathbb N}}
\nc\mBO{{\mathbb O}}
\nc\mBP{{\mathbb P}}
\nc\mBQ{{\mathbb Q}}
\nc\mBR{{\mathbb R}}
\nc\mBS{{\mathbb S}}
\nc\mBT{{\mathbb T}}
\nc\mBU{{\mathbb U}}
\nc\mBV{{\mathbb V}}
\nc\mBW{{\mathbb W}}
\nc\mBX{{\mathbb X}}
\nc\mBY{{\mathbb Y}}
\nc\mBZ{{\mathbb Z}}


\nc\mCA{{\mathcal A}}
\nc\mCB{{\mathcal B}}
\nc\mCC{{\mathcal C}}
\nc\mCD{{\mathcal D}}
\nc\mCE{{\mathcal E}}
\nc\mCF{{\mathcal F}}
\nc\mCG{{\mathcal G}}
\nc\mCH{{\mathcal H}}
\nc\mCI{{\mathcal I}}
\nc\mCJ{{\mathcal J}}
\nc\mCK{{\mathcal K}}
\nc\mCL{{\mathcal L}}
\nc\mCM{{\mathcal M}}
\nc\mCN{{\mathcal N}}
\nc\mCO{{\mathcal O}}
\nc\mCP{{\mathcal P}}
\nc\mCQ{{\mathcal Q}}
\nc\mCR{{\mathcal R}}
\nc\mCS{{\mathcal S}}
\nc\mCT{{\mathcal T}}
\nc\mCU{{\mathcal U}}
\nc\mCV{{\mathcal V}}
\nc\mCW{{\mathcal W}}
\nc\mCX{{\mathcal X}}
\nc\mCY{{\mathcal Y}}
\nc\mCZ{{\mathcal Z}}


\nc\mbA{{\mathsf A}}
\nc\mbB{{\mathsf B}}
\nc\mbC{{\mathsf C}}
\nc\mbD{{\mathsf D}}
\nc\mbE{{\mathsf E}}
\nc\mbF{{\mathsf F}}
\nc\mbG{{\mathsf G}}
\nc\mbH{{\mathsf H}}
\nc\mbI{{\mathsf I}}
\nc\mbJ{{\mathsf J}}
\nc\mbK{{\mathsf K}}
\nc\mbL{{\mathsf L}}
\nc\mbM{{\mathsf M}}
\nc\mbN{{\mathsf N}}
\nc\mbO{{\mathsf O}}
\nc\mbP{{\mathsf P}}
\nc\mbQ{{\mathsf Q}}
\nc\mbR{{\mathsf R}}
\nc\mbS{{\mathsf S}}
\nc\mbT{{\mathsf T}}
\nc\mbU{{\mathsf U}}
\nc\mbV{{\mathsf V}}
\nc\mbW{{\mathsf W}}
\nc\mbX{{\mathsf X}}
\nc\mbY{{\mathsf Y}}
\nc\mbZ{{\mathsf Z}}

\nc\mba{{\mathsf a}}
\nc\mbb{{\mathsf b}}
\nc\mbc{{\mathsf c}}
\nc\mbd{{\mathsf d}}
\nc\mbe{{\mathsf e}}
\nc\mbf{{\mathsf f}}
\nc\mbg{{\mathsf g}}
\nc\mbh{{\mathsf h}}
\nc\mbi{{\mathsf i}}
\nc\mbj{{\mathsf j}}
\nc\mbk{{\mathsf k}}
\nc\mbl{{\mathsf l}}
\nc\mbm{{\mathsf m}}
\nc\mbn{{\mathsf n}}
\nc\mbo{{\mathsf o}}
\nc\mbp{{\mathsf p}}
\nc\mbq{{\mathsf q}}
\nc\mbr{{\mathsf r}}
\nc\mbs{{\mathsf s}}
\nc\mbt{{\mathsf t}}
\nc\mbu{{\mathsf u}}
\nc\mbv{{\mathsf v}}
\nc\mbw{{\mathsf w}}
\nc\mbx{{\mathsf x}}
\nc\mby{{\mathsf y}}
\nc\mbz{{\mathsf z}}




\nc\mbfA{{\mathbf A}}
\nc\mbfB{{\mathbf B}}
\nc\mbfC{{\mathbf C}}
\nc\mbfD{{\mathbf D}}
\nc\mbfE{{\mathbf E}}
\nc\mbfF{{\mathbf F}}
\nc\mbfG{{\mathbf G}}
\nc\mbfH{{\mathbf H}}
\nc\mbfI{{\mathbf I}}
\nc\mbfJ{{\mathbf J}}
\nc\mbfK{{\mathbf K}}
\nc\mbfL{{\mathbf L}}
\nc\mbfM{{\mathbf M}}
\nc\mbfN{{\mathbf N}}
\nc\mbfO{{\mathbf O}}
\nc\mbfP{{\mathbf P}}
\nc\mbfQ{{\mathbf Q}}
\nc\mbfR{{\mathbf R}}
\nc\mbfS{{\mathbf S}}
\nc\mbfT{{\mathbf T}}
\nc\mbfU{{\mathbf U}}
\nc\mbfV{{\mathbf V}}
\nc\mbfW{{\mathbf W}}
\nc\mbfX{{\mathbf X}}
\nc\mbfY{{\mathbf Y}}
\nc\mbfZ{{\mathbf Z}}

\nc\mbfa{{\mathbf a}}
\nc\mbfb{{\mathbf b}}
\nc\mbfc{{\mathbf c}}
\nc\mbfd{{\mathbf d}}
\nc\mbfe{{\mathbf e}}
\nc\mbff{{\mathbf f}}
\nc\mbfg{{\mathbf g}}
\nc\mbfh{{\mathbf h}}
\nc\mbfi{{\mathbf i}}
\nc\mbfj{{\mathbf j}}
\nc\mbfk{{\mathbf k}}
\nc\mbfl{{\mathbf l}}
\nc\mbfm{{\mathbf m}}
\nc\mbfn{{\mathbf n}}
\nc\mbfo{{\mathbf o}}
\nc\mbfp{{\mathbf p}}
\nc\mbfq{{\mathbf q}}
\nc\mbfr{{\mathbf r}}
\nc\mbfs{{\mathbf s}}
\nc\mbft{{\mathbf t}}
\nc\mbfu{{\mathbf u}}
\nc\mbfv{{\mathbf v}}
\nc\mbfw{{\mathbf w}}
\nc\mbfx{{\mathbf x}}
\nc\mbfy{{\mathbf y}}
\nc\mbfz{{\mathbf z}}

\nc\mfa{{\mathfrak a}}
\nc\mfb{{\mathfrak b}}
\nc\mfc{{\mathfrak c}}
\nc\mfd{{\mathfrak d}}
\nc\mfe{{\mathfrak e}}
\nc\mff{{\mathfrak f}}
\nc\mfg{{\mathfrak g}}
\nc\mfh{{\mathfrak h}}
\nc\mfi{{\mathfrak i}}
\nc\mfj{{\mathfrak j}}
\nc\mfk{{\mathfrak k}}
\nc\mfl{{\mathfrak l}}
\nc\mfm{{\mathfrak m}}
\nc\mfn{{\mathfrak n}}
\nc\mfo{{\mathfrak o}}
\nc\mfp{{\mathfrak p}}
\nc\mfq{{\mathfrak q}}
\nc\mfr{{\mathfrak r}}
\nc\mfs{{\mathfrak s}}
\nc\mft{{\mathfrak t}}
\nc\mfu{{\mathfrak u}}
\nc\mfv{{\mathfrak v}}
\nc\mfw{{\mathfrak w}}
\nc\mfx{{\mathfrak x}}
\nc\mfy{{\mathfrak y}}
\nc\mfz{{\mathfrak z}}

\nc{\one}{{\mathsf{1}}}


\nc\clambda{ {\check{\lambda} }}
\nc\cmu{ {\check{\mu} }}

\nc\bDelta{\mathsf{\Delta}}
\nc\bGamma{\mathsf{\Gamma}}
\nc\bLambda{\mathsf{\Lambda}}


\nc\loccit{\emph{loc.cit.}}



%%%%%%%%%%%%%%%%%%%%Operations-limit%%%%%%%%%%%%%%%%%%%%

\NewDocumentCommand{\ot}{e{_^}}{
  \mathbin{\mathop{\otimes}\displaylimits
    \IfValueT{#1}{_{#1}}
    \IfValueT{#2}{^{#2}}
  }
}
\NewDocumentCommand{\boxt}{e{_^}}{
  \mathbin{\mathop{\boxtimes}\displaylimits
    \IfValueT{#1}{_{#1}}
    \IfValueT{#2}{^{#2}}
  }
}
\NewDocumentCommand{\mt}{e{_^}}{
  \mathbin{\mathop{\times}\displaylimits
    \IfValueT{#1}{_{#1}}
    \IfValueT{#2}{^{#2}}
  }
}
\NewDocumentCommand{\convolve}{e{_^}}{
  \mathbin{\mathop{\star}\displaylimits
    \IfValueT{#1}{_{#1}}
    \IfValueT{#2}{^{#2}}
  }
}
\NewDocumentCommand{\colim}{e{_^}}{
  \mathbin{\mathop{\operatorname{colim}}\displaylimits
    \IfValueT{#1}{_{#1}\,}
    \IfValueT{#2}{^{#2}\,}
  }
}
\NewDocumentCommand{\laxlim}{e{_^}}{
  \mathbin{\mathop{\operatorname{laxlim}}\displaylimits
    \IfValueT{#1}{_{#1}\,}
    \IfValueT{#2}{^{#2}\,}
  }
}
\NewDocumentCommand{\oplaxlim}{e{_^}}{
  \mathbin{\mathop\operatorname{oplax-lim}\displaylimits
    \IfValueT{#1}{_{#1}\,}
    \IfValueT{#2}{^{#2}\,}
  }
}


%%%%%%%%%%%%%%%%%%%%Arrows%%%%%%%%%%%%%%%%%%%%


\makeatletter
\newcommand{\laxto}{\dashedrightarrow}
\newcommand{\xrightleftarrows}[1]{\mathrel{\substack{\xrightarrow{#1} \\[-.9ex] \xleftarrow{#1}}}}
\newcommand{\adj}{\xrightleftarrows{\rule{0.5cm}{0cm}}}

\newcommand*{\da@rightarrow}{\mathchar"0\hexnumber@\symAMSa 4B }
\newcommand*{\da@leftarrow}{\mathchar"0\hexnumber@\symAMSa 4C }
\newcommand*{\xlaxto}[2][]{%
  \mathrel{%
    \mathpalette{\da@xarrow{#1}{#2}{}\da@rightarrow{\,}{}}{}%
  }%
}
\newcommand{\xlaxgets}[2][]{%
  \mathrel{%
    \mathpalette{\da@xarrow{#1}{#2}\da@leftarrow{}{}{\,}}{}%
  }%
}
\newcommand*{\da@xarrow}[7]{%
  % #1: below
  % #2: above
  % #3: arrow left
  % #4: arrow right
  % #5: space left 
  % #6: space right
  % #7: math style 
  \sbox0{$\ifx#7\scriptstyle\scriptscriptstyle\else\scriptstyle\fi#5#1#6\m@th$}%
  \sbox2{$\ifx#7\scriptstyle\scriptscriptstyle\else\scriptstyle\fi#5#2#6\m@th$}%
  \sbox4{$#7\dabar@\m@th$}%
  \dimen@=\wd0 %
  \ifdim\wd2 >\dimen@
    \dimen@=\wd2 %   
  \fi
  \count@=2 %
  \def\da@bars{\dabar@\dabar@}%
  \@whiledim\count@\wd4<\dimen@\do{%
    \advance\count@\@ne
    \expandafter\def\expandafter\da@bars\expandafter{%
      \da@bars
      \dabar@ 
    }%
  }%  
  \mathrel{#3}%
  \mathrel{%   
    \mathop{\da@bars}\limits
    \ifx\\#1\\%
    \else
      _{\copy0}%
    \fi
    \ifx\\#2\\%
    \else
      ^{\copy2}%
    \fi
  }%   
  \mathrel{#4}%
}
\makeatother

%%%%%%%%%%%%%%%%%%%%Decorations%%%%%%%%%%%%%%%%%%%%
\nc{\wt}{\widetilde}
\nc{\ol}{\overline}

\nc{\red}{\textcolor{red}}
\nc{\blue}{\textcolor{blue}}
\nc{\purple}{\textcolor{violet}}

\nc{\simorlax}{{\red\simeq/\blue\lax}}

%%%%%%%%%%%%%%%%%%%%All%%%%%%%%%%%%%%%%%%%%

\nc{\Id}{\mathsf{Id}}
\nc{\gl}{\mathfrak{gl}}
\renc{\sl}{\mathfrak{sl}}
\nc{\GL}{\mathsf{GL}}
\nc{\SL}{\mathsf{SL}}
\nc{\PGL}{\mathsf{PGL}}
\nc{\hmod}{\mathsf{-mod}}
\nc{\Vect}{\mathsf{Vect}}
\nc{\tr}{\mathsf{tr}}
\nc{\Kil}{\mathsf{Kil}}
\nc{\ad}{{\mathsf{ad}}}
\nc{\Ad}{\mathsf{Ad}}
\nc{\oblv}{\mathsf{oblv}}
\nc{\gr}{\mathsf{gr}}
\nc{\Sym}{\mathsf{Sym}}
\nc{\QCoh}{\mathsf{QCoh}}
\nc{\ind}{\mathsf{ind}}
\nc{\Spec}{\mathsf{Spec}}
\nc{\Hom}{\mathsf{Hom}}
\nc{\Ext}{\mathsf{Ext}}
\nc{\Grp}{\mathsf{Grp}}
\nc{\pt}{\mathsf{pt}}
\nc{\Lie}{\mathsf{Lie}}
\nc{\CAlg}{\mathsf{CAlg}}
\nc{\Der}{\mathsf{Der}}
\nc{\Rep}{\mathsf{Rep}}
\renc{\sc}{{\mathsf{sc}}}
\nc{\Fl}{\mathsf{Fl}}
\nc{\Fun}{\mathsf{Fun}}
\nc{\ev}{\mathsf{ev}}
\nc{\surj}{\twoheadrightarrow}
\nc{\inj}{\hookrightarrow}
\nc{\HC}{\mathsf{HC}}
\nc{\cl}{\mathsf{cl}}
\renc{\Im}{\mathsf{Im}}
\renc{\ker}{\mathsf{ker}}
\nc{\coker}{\mathsf{coker}}
\nc{\Tor}{\mathsf{Tor}}
\nc{\op}{\mathsf{op}}
\nc{\length}{\mathsf{length}}
\nc{\fd}{{\mathsf{fd}}}
\nc{\weight}{\mathsf{wt}}
\nc{\semis}{{\mathsf{ss}}}
\nc{\qc}{{\mathsf{qc}}}
\nc{\pr}{\mathsf{pr}}
\nc{\act}{\mathsf{act}}
\nc{\dR}{{\mathsf{dR}}}
\nc{\hol}{{\mathsf{hol}}}
\nc{\Pic}{{\mathsf{Pic}}}
\nc{\Loc}{\mathsf{Loc}}
\nc{\IC}{\mathsf{IC}}

\begin{document}


\title{Lecture 6}

\date{Apr 1, 2024}

\maketitle

	Last time we proved the Harish-Chandra isomorphism
	\[
		\phi_\HC: Z(\mfg)  \xrightarrow{\sim} \Sym(\mft)^{W_\bullet}.
	\]
	We obtain isomorphisms between $k$-schemes:
	\[
		\Spec(Z(\mfg)) \simeq \mft^*/\!/ W_\bullet\simeq \mft^*/\!/ W \simeq \mft/\!/W \simeq \mfg/\!/G,
	\]
	where
	\begin{itemize}
		\item
			The 1st isomorphism is due to the Harish-Chandra isomorphism;
		\item 
			The 2nd isomorphism is given by the tranlation $\mft^*\to \mft^*$, $\lambda \mapsto \lambda +\rho$;
		\item 
			The 3rd isomorphism is given by the $W$-equivariant isomorphism $\mft^*\simeq \mft$ provided by the Killing form;
		\item
			The 4th isomorphism is due to Chevalley's restriction theorem.
	\end{itemize}
	This time, we first give more algebro-geometric results about the above schemes. Then we study the structure of $\mCO$.


\section{More on \texorpdfstring{$\Spec(Z(\mfg))$}{Spec(Z(g)}}


	\begin{prop}[\!\!{\cite[Sect. 1.10]{H}}]
		\label{prop-good-GIT}
		The morphism $\varpi:\mft \to \mft/\!/W_\bullet \simeq \Spec(Z(\mfg))$ is surjective, and $W_\bullet$ acts transitively on the fiber at each closed point.
	\end{prop}

	\begin{rem}
		In other words, any character $\chi:Z(\mfg) \to k$ is the central character $\chi_\lambda$ for some Verma module $V_\lambda$, and $\chi_\lambda =\chi_\mu$ iff $\mu=w\cdot \lambda$ for some $w\in W$.
	\end{rem}

	\begin{rem}
		Similar claim is true for any finite group action on an affine scheme. See \cite[Exp. V, Prop. 1.1]{SGA1}.
	\end{rem}

	\begin{thm}[Chevalley--Shephard--Todd]
		\label{thm-CST}
		Consider the homomorphism $\Sym(\mft)^W\to \Sym(\mft)$. We have:
		\begin{itemize}
			\item[(1)]
				There exists homogeneous elements $c_1,\cdots,c_r\in \Sym(\mft)^W$, $r=\dim(\mft)$ such that $\Sym(\mft)^W$ is isomorphic to $k[c_1,\cdots,c_r]$ as graded algebras. In other words, $\Sym(\mft)^W$ is a \emph{graded polynomial algebra} of transcendence degree $\dim(\mft)$.
			\item[(2)]
				There exists homogeneous elements $a_w \in \Sym(\mft)$, $w\in W$ such that $\Sym(\mft)$ is isomorphic to the free $\Sym(\mft)^W$-module generated by them. In other words, $\Sym(\mft)$ is a \emph{graded free} $\Sym(\mft)^W$-module of rank $\# W$.		
		\end{itemize}
	\end{thm}

	\begin{cor}
		\label{cor-CST}
		The scheme $ \mft/\!/W $ is an affine space whose dimension is equal to $\dim(\mft)$. Also, the map $\mft \to  \mft/\!/W$ is finite and flat.
	\end{cor}

	\begin{rem}
		Similar claim is true for any linear action of a finite group $H$ on a $k$-vector space $V$ as long as:
		\begin{itemize}
			\item
				The order $\# H$ of the group is relatively prime to $\mathsf{char}(k)$ (which is true by our assumption);
			\item
				The group $H$ admits a set of generators consisting of elements $w$ that act as pseudoreflections\footnote{This means $\Id - w$ is of rank $1$.} on $V$.
		\end{itemize}
		For a proof, see \cite[Sect. 5]{Bo}.
	\end{rem}

	\begin{rem}
		There is no canonical choice for the generators $c_1,\cdots,c_r$ of the polynomial algebra $\Sym(\mft)^W$. However, the degrees $d_1,\cdots,d_r \in \mBZ^{\ge 0}$ of them are determined up to order. Also, we have $\# W = \prod_i d_i$.

		Similarly, there is no canonical choice for the generators of $\Sym(\mft)$ as a free $\Sym(\mft)^W$-module. However, the degrees of these generators are determined up to order.
	\end{rem}

	\begin{exam}
		\label{exam-sln-Chevalley}
		For $\mfg = \sl_n$, recall $W\simeq \Sigma_n$ acts on $\mft \simeq \ker( k^{\oplus n} \xrightarrow{\Sigma} k )$ in the standard way. It follows that $\Sym(\mft^*) \simeq \Fun(\mft) \simeq k[x_1,\cdots,x_n]/(\sigma_1)$ and $\Sym(\mft^*)^W \simeq k[\sigma_1,\cdots,\sigma_n]/(\sigma_1)$. Here $\sigma_i$ is the basic symmetric polynomial of degree $i$, i.e., $\prod_{j=1}^n (x+x_j) = x^n + \sigma_1 x^{m-1} + \cdots \sigma_n$.
	\end{exam}

	Our next goal is to prove the following:

	\begin{thm}[Kostant]
		As a $Z(\mfg)$-module, $U(\mfg)$ is free.
	\end{thm}

	\begin{rem}
		This classical result was first proved by Kostant in the 60's (\!\!\cite{K}) and used to be considered a hard theorem. The simplified proof below was due to Bernstein--Lunts (\!\!\cite{BL}), after more than 30 years\footnote{Gaitsgory also sketched a proof in \cite[Cor. 7.3]{G}, but I think there is a gap in the second paragaph. It is not clear what is the logic behind \emph{``... it is enough to show that $\Sym(\mfg/\mfn)$ is free as a $\Sym(\mfh)^W$-module.''} In fact, this reduction is the main point of \cite{BL}.}.
	\end{rem}

	\begin{constr}
		We equip $U(\mfg)$ with a $U(\mfn^-)\ot Z(\mfg)\ot U(\mfn)^{\op}$-module structure by the formula
		\[
			(u^-\ot z\ot u)(x):= u^- x z u.
		\]
		Here $u^-\ot z\ot u \in U(\mfn^-)\ot Z(\mfg)\ot U(\mfn)^{\op}$ and $x\in U(\mfg)$.
	\end{constr}

	We will prove the following stronger result.

	\begin{thm}
		\label{thm-Kostant-stronger}
		Let $\{a_w\}$ be a homogeneous free basis of $\Sym(\mft)$ over $\Sym(\mft)^W$. We view $a_w$ as elements in $U(\mfg)$ via the embedding $\Sym(\mft) \inj U(\mfg)$. Then $\{a_w\}$ is also a free basis of $U(\mfg)$ over $U(\mfn^-)\ot Z(\mfg)\ot U(\mfn)^{\op}$. 
	\end{thm}

	\begin{constr}
		Consider the PBW filtrations on $U(\mfg)$ and $U(\mfn^-)\ot Z(\mfg)\ot U(\mfn)^{\op}$. Taking associated graded spaces, we obtain a graded $\Sym(\mfn^-) \ot \Sym(\mfg)^\mfg \ot \Sym(\mfn)$-module structure on $\Sym(\mfg)$. By definition, this module structure comes from the obvious graded homomorphism
		\begin{equation}
			\label{eqn1}
			\Sym(\mfn^-) \ot \Sym(\mfg)^\mfg \ot \Sym(\mfn) \xrightarrow{\mathsf{mult}} \Sym(\mfg).
		\end{equation}
	\end{constr}

	This reduces Theorem \ref{thm-Kostant-stronger} to the following result:

	\begin{prop}
		\label{prop-Kostant-stronger-PBW}
		Let $\{a_w\}$ be a homogeneous free basis of $\Sym(\mft)$ over $\Sym(\mft)^W$. We view $a_w$ as homogeneous elements in $\Sym(\mfg)$ via the embedding $\Sym(\mft) \inj \Sym(\mfg)$. Then $\{a_w\}$ is also a free basis of $\Sym(\mfg)$ over $\Sym(\mfn^-) \ot \Sym(\mfg)^\mfg \ot \Sym(\mfn)$.
	\end{prop}

	We also record the following corollary of the proposition.
	\begin{cor}
		The map $\mfg \to \mfn \mt \mfg/\!/G \mt \mfn^- $ is flat. In particular, the map $\mfg \to \mfg/\!/G$ is flat.
	\end{cor}
	

	\begin{exam}
		For $\mfg=\sl_2$ and the standard basis $e,f,h$. Recall $\Sym(\mft)^W = k[h^2]$ and $\Sym(\mfg)^{\mfg} = k[\Omega]$, where $\Omega = h^2 + 4ef$ is the image of the Casimir element. Then the proposition says $k[e,f,h]$ is free over $k[e,f,h^2+4ef]$ and any homogeneous basis of $k[h]$ over $k[h^2]$ is also a free basis of $k[e,f,h]$ over $k[e,f,h^2+4ef]$.
	\end{exam}

	
	\begin{warn}
		Proposition \ref{prop-Kostant-stronger-PBW} does \emph{not} follow obviously from Theorem \ref{thm-CST}. Namely, we have two homomorphisms
		\begin{eqnarray*}
			\Sym(\mfn^-) \ot \Sym(\mfg)^\mfg \ot \Sym(\mfn) &\to& \Sym(\mfg) \\
			\Sym(\mfn^-) \ot \Sym(\mft)^W \ot \Sym(\mfn) &\to& \Sym(\mfg).
		\end{eqnarray*}
		Theorem \ref{thm-CST} implies $\Sym(\mfg)$ is a free module over the source of the second map, but the images of these two homomorphisms are in general not the same. 

		To see this, it is better to pass to dualities. Via the isomorphisms $\mfg \simeq \mfg^*$, $\mfn \simeq (\mfn^-)^*$ and $\mft\simeq \mft^*$ provided by the Killing form, the above two homomorphisms are given by
		\begin{eqnarray}
			\label{eqn2}
			\Fun(\mfn) \ot \Fun(\mfg)^\mfg \ot \Fun(\mfn^-) &\to& \Fun(\mfg) \\
			\label{eqn3}
			\Fun(\mfn) \ot \Fun(\mft)^W \ot \Fun(\mfn^-) &\to& \Fun(\mfg),
		\end{eqnarray}
		which are induced by the decomposition $\mfg = \mfn^- \oplus \mft \oplus \mfn$. Now the following exercise says the two images are not the same even for $\mfg=\sl_3$, although they happen to be the same for $\sl_2$.
	\end{warn}

	\begin{exe}
		This is \red{Homework 3, Problem 1}. Let $\mfg=\sl_n$ and $\sigma_i \in \Fun(\mft)^W$ be as in Example \ref{exam-sln-Chevalley}.
		\begin{itemize}
			\item[(1)]
				For each $1<i\le n$, find the unique element $\wt\sigma_i \in \Fun(\mfg)^\mfg$ corresponding to $\sigma_i$ via the Chevalley isomorphim $\Fun(\mfg)^\mfg \xrightarrow{\simeq}  \Fun(\mft)^W$. In other words, find the unique extension of $\sigma_i$ to an adjoint invariant polynomial function on $\mfg$.
			\item[(2)]
				For $\mfg=\sl_2$, prove that \eqref{eqn2} and \eqref{eqn3} are both injective and have the same image.
			\item[(3)]
				For $\mfg=\sl_3$, prove that $\wt\sigma_3$ is contained in the image of \eqref{eqn2} but not in the image of \eqref{eqn3}.
		\end{itemize}
	\end{exe}


	Let us return to Bernstein--Lunts's proof. We need the following general result, which is an easy exercise in linear algebra.

	\begin{lem}
		\label{lem-fil-generator}
		Let $A = \bigcup_{n\ge 0} \Phi^{\le n} A$ be a filtered algebra and $M =\bigcup_{n\ge 0} \Phi^{\le n} M$ be a filtered $A$-module. Let $\{b_i\}$ be a family of elements of $M$ such that their symbols\footnote{For $b\in M$, let $n\ge 0$ be the smallest index such that $b\in \Phi^{\le n} M$. Then the \textbf{symbol} $\sigma_\Phi(b)$ is the image of $b$ under the projection $\Phi^{\le n} M \to \gr_\Phi^n M$. Note that $\sigma_\Phi$ is \emph{not} a linear map!} $\{ \sigma_\Phi(b_i) \}$ form a free basis of the $\gr_\Phi^\bullet A$-module $\gr_\Phi^\bullet M$. Then $\{b_i\}$ is a free basis of the $A$-module $M$.
	\end{lem}

	\proof[Proof of Proposition \ref{prop-Kostant-stronger-PBW}]
		The strategy is as follows. We will construct compatible new filtrations on the source and the target of the homomorphism \eqref{eqn1}: 
		\[
			\Sym(\mfn^-\oplus \mfn) \ot \Sym(\mfg)^\mfg  \to \Sym(\mfg).
		\]
		Hence we can view $\Sym(\mfg)$ as a filtered module over $\Sym(\mfn^-\oplus \mfn) \ot \Sym(\mfg)^\mfg$ with respect to these new filtrations. Then we apply Lemma \ref{lem-fil-generator}.

		The desired filtration on $\Sym(\mfg)$ can be summarized in a sentence: we ignore the indeterminates contained in $\mfn^-\oplus \mfn$ and only count the degree for those contained in $\mft$. In other words, we define
		\[
			\Phi^{\le i} \Sym(\mfg) := \Sym(\mfn^-\oplus \mfn) \cdot \Sym^{\le i}(\mft),
		\]
		where $\Sym^{\le \bullet}(\mft)$ is the standard filtration on $\Sym(\mft)$ given by degrees. For $f\in \Sym(\mfg)$, let $\deg_\mft(f)$ be the minimal index $i$ such that $f\in \Phi^{\le i} \Sym(\mfg)$. In other words, it is the $\mft$-degree of the polynomial $f$.

		Note that this filtraiton is compatible with the multiplication. Also note that
		\begin{equation}
			\label{eqn-grPhiSymg}
			\gr_\Phi^\bullet \Sym(\mfg) \simeq  \Sym(\mfn \oplus \mfn^-) \ot  \Sym^\bullet(\mft)
		\end{equation}
		as graded commutative algebras, where the grading on the RHS is the standard grading on $\Sym^\bullet(\mft)$. 

		Consider the injective homomorphisms
		\[
			\Sym(\mfg)^\mfg \to \Sym(\mfg) \gets \Sym(\mft)^W.
		\]
		We equip the sources with the induced $\Phi$-filtrations. Note that the $\Phi$-filtration on $\Sym(\mft)^W$ is the standard one. By definition, the following diagram commutes\footnote{Warning: the right vercial map is not the identity map. It abandons all non-highest degree terms in a polynomial.}
		\[
			\xymatrix{
				\Sym(\mfg)^\mfg \ar[r]^-\subset \ar[d]^-{\sigma_\Phi} 
				& \Sym(\mfg) \ar[d]^-{\sigma_\Phi} 
				& \Sym(\mft)^W \ar[l]_-\supset \ar[d]^-{\sigma_\Phi} \\
				\gr_\Phi^\bullet(\Sym(\mfg)^\mfg) \ar[r]^-\subset
				& \gr_\Phi^\bullet \Sym(\mfg) 
				& \Sym^\bullet(\mft)^W. \ar[l]_-\supset
			}
		\]
		We have the following key observation:
		\begin{lem}
			\label{lem-key}
			The injective homomorphisms
			\[
				\gr_\Phi^\bullet(\Sym(\mfg)^\mfg) \to \gr_\Phi^\bullet \Sym(\mfg)  \gets \Sym^\bullet(\mft)^W
			\]
			have the same image.
		\end{lem}

		Let us first finish the proof of the proposition assuming this lemma. We can tensor the sources with the factor $\Sym(\mfn\oplus \mfn^-)$ and consider the homomorphisms
		\begin{eqnarray*}
			\Sym(\mfn\oplus \mfn^-) \ot \Sym(\mfg)^\mfg  &\xrightarrow{m_1}& \Sym(\mfg), \\
			\Sym(\mfn\oplus \mfn^-) \ot \Sym(\mft)^W  &\xrightarrow{m_2}& \Sym(\mfg).
		\end{eqnarray*}
		Via the isomorphism \eqref{eqn-grPhiSymg}, the graded homomorphism $\gr_\Phi^\bullet(m_2)$ is given by the obvious embedding
		\[
			\Sym(\mfn\oplus \mfn^-) \ot \Sym^\bullet(\mft)^W \to \Sym(\mfn\oplus \mfn^-) \ot \Sym^\bullet(\mft).
		\]
		Hence the above lemma implies $\gr_\Phi^\bullet(m_1)$ is also an embedding with the same image. By Theorem \ref{thm-CST}, the graded homomorphism $\gr_\Phi^\bullet(m_2)$ exhibits its target as a graded free module over its source, and the images of $\{a_w\}$ form a basis. Hence the same is true for the graded homomorphism $\gr_\Phi^\bullet(m_1)$. Then we win by Lemma \ref{lem-fil-generator}.

	\qed[Proposition \ref{prop-Kostant-stronger-PBW}]

	It remains to prove Lemma \ref{lem-key}. We first prove the following elementary result.

		\begin{lem}
			\label{lem-Chev-via-symbol}
			For any $i\ge 0$, we have:
			\begin{itemize}
				\item[(1)]
					For any nonzero $f\in \Sym^i(\mfg)^\mfg$, we have $\deg_\mft(f) = i$.
				\item[(2)]
					By (2), taking $\Phi$-symbols gives a map
					\[
						\sigma_\Phi^i: \Sym^i(\mfg)^\mfg \to \gr_\Phi^i(\Sym(\mfg)^\mfg).
					\]
					We claim the following diagram commutes:
					\begin{equation}
						\label{eqn-key}
						\xymatrix{
							\Sym^i(\mfg)^\mfg \ar[d]^-{\sigma_\Phi^i} \ar[rr]^-{\phi_\cl}_-\simeq
							& & \Sym^i(\mft)^W  \ar@{=}[d] \\
							\gr_\Phi^i(\Sym(\mfg)^\mfg) \ar[r]^-\subset
							& \gr_\Phi^i \Sym(\mfg) 
							& \Sym^i(\mft)^W. \ar[l]_-\supset
						}
					\end{equation}
				\item[(3)]
					The map $\sigma_\Phi^i: \Sym^i(\mfg)^\mfg \to \gr_\Phi^i(\Sym(\mfg)^\mfg)$  is bijective.
			\end{itemize}
		\end{lem}

		\proof
			Let $f\in \Sym^i(\mfg)^\mfg$ be any nonzero element. By definition, we have
			\[
				f\in \bigoplus_{0\le j\le i} \Sym^{i-j}(\mfn^-\oplus \mfn) \cdot \Sym^{j}(\mft).
			\] 
			Let $f_j\in \Sym^{i-j}(\mfn^-\oplus \mfn) \cdot \Sym^{j}(\mft)$ be the $j$-th entry of $f$ with respect to the above direct sum decomposition, i.e., the part of $f$ whose $\mft$-degree is $j$.

			By the construction of $\phi_\cl$\footnote{Recall it kills any factor in $\Sym^{\ge 1}(\mfn^-\oplus \mfn)$.}, we have
			\[
				\phi_\cl(f) = f_i \in \Sym^{i}(\mft)
			\]
			Since $\phi_\cl$ is an isomorphism, we obtain $f_i\neq 0$. In particular, $\deg_\mft(f) = i$. This proves (1).

			By definition, we also have $\sigma_\Phi^i(f) = f_i$ because this is the sum of the highest $\mft$-degree terms. This proves (2).

			The commutative diagram in (2) implies $\sigma_\Phi^i$ is injective. It remains to show it is surjective. Let $\overline{h} \in \gr_\Phi^i(\Sym(\mfg)^\mfg) $ be any nonzero element and $h\in \Phi^{\le i}(\Sym(\mfg)^\mfg)$ be a lifting of it, i.e., $ \overline{h}=\sigma_\Phi(h)  $. Note that $\deg_\mft(h) = i$. Write $h = h_0+h_1+\cdots+ h_d$ such that $h_j \in \Sym^j(\mfg)^\mfg$ and $h_d\neq 0$. By (1) and (2), either $h_j = 0$ or $\deg_\mft(h_j) = j$. It follows that we must have $d=i$ and $\deg_t(h-h_i)<i$. This implies $\overline{h}=\sigma_\Phi(h)  = \sigma_\Phi^i(h_i)$ with $h_i \in \Sym^i(\mfg)^\mfg$. In other words, the given map is surjective as desired.

		\qed

	\begin{warn}
			The digram \eqref{eqn-key} would \emph{not} commute if we dropped the superscripts $i$ from $\Sym^i(-)$. In other words, the following diagram does not commute
			\[
			\xymatrix{
				\Sym(\mfg)^\mfg \ar[d]^-{\sigma_\Phi} \ar[rr]^-{\phi_\cl}_-\simeq
				& 
				& \Sym(\mft)^W  \ar[d]^-{\sigma_\Phi} \\
				\gr_\Phi^\bullet(\Sym(\mfg)^\mfg) \ar[r]^-\subset
				& \gr_\Phi^\bullet \Sym(\mfg) 
				& \Sym^\bullet(\mft)^W. \ar[l]_-\supset
			}
			\]
	\end{warn}

	\proof[Proof of Lemma \ref{lem-key}]

		This follows from Lemma \ref{lem-Chev-via-symbol} by a diagram chasing.

	\qed[Lemma \ref{lem-key}]



\section{Category \texorpdfstring{$\mCO$}{O} is Artinian and Noetherian}

	The next few lectures are devoted to the algebraic study of $\mCO$. In this section, we prove any object in $\mCO$ has finite length. This was promised in the second lecture.

	We first recall the following corollary of Proposition \ref{prop-good-GIT}.

	\begin{prop}[Linkage principle]
		Verma modules $M_\lambda$ and $M_\mu$ belong to the same block iff $\mu = w\cdot \lambda$ for some $w\in W$.
	\end{prop}

	We also recall the following corollary of [Thm. 25, Prop. 31, Lect. 2]:

	\begin{prop}
		Each Verma module $M_\lambda$ has a unique irreducible quotient $L_\lambda$, whose highest weight is equal to $\lambda$. Any irreducible object in $\mCO$ is of such form.
	\end{prop}

	\begin{cor}
		\label{cor-irresubq-of-Verma}
		If the irreducible module $L_\mu$ is isomorphic to a subquotient of the Verma module $M_\lambda$, then $\mu = w\cdot \lambda$ for some $w\in W$. In particular, $M_\lambda$ has only finitely many non-isomorphic irreducible subquotients.
	\end{cor}

	\begin{thm}
		Each object $M\in \mCO$ is both Artinian and Noetherian.
	\end{thm}

	\begin{rem}
		By Jordan--Hölder, $M$ is both Artinian and Noetherian iff there exists a finite filtration $0=M_0 \subset M_1 \subset\cdots\subset M_n=M$ such that each $M_i/M_{i-1}$ is a (nonzero) irreducible object. Moreover, for each irreducible object $L_\lambda \in \mCO$, its multiplicity in the collection of such quotients does not depend on the choice of the filtration, and is denoted by $[M: L_\lambda]$. This implies $n$ does not depend on the choice of the filtration, and is called the \textbf{length} of $M$, i.e., $\length(M)$. The numbers $[M: L_\lambda]$ and $\length(M)$ are basic objects in the study of $\mCO$ (and any representation theory).

	\end{rem}

	\proof
		Since $M$ can be written as a quotient of a successive extension of Verma modules ([Prop. 31, Lect. 2]), we only need to prove the theorem for Verma modules $M=M_\lambda$. By Corollary \ref{cor-irresubq-of-Verma}, we only need to prove for \emph{any} finite filtration of $M_\lambda$, the multiplicity of each $L_\mu$ ($\mu=w\cdot \lambda$, $w\in W$) that appears in the graded pieces is uniformly bounded. Recall $L_\mu$ is a weight module with highest weight $\mu$. Since any module in $\mCO$ is a weight module (with respect to the $\mft$-action), the aforementioned multiplicity is bounded by the dimension of the $\mu$-weight subspace of $M_\lambda$. But this is finite by ([Cor. 33, Lect. 2]).

	\qed

	\begin{prop}
		For any $M, N\in \mCO$, the vector space $\Hom_\mCO(M,N)$ is finite dimensional.
	\end{prop}

	\proof
		By dévissage, we only need to show $\Hom_\mCO(L_\lambda,L_\mu)$ is finite dimensional. This is a subspace of $\Hom_\mCO(M_\lambda,L_\mu)$. And the latter is a subspace of the $\lambda$-weight subspace of $L_\mu$, which is finite-dimensional.

	\qed

	Note that the above argument actually shows

	\begin{lem}
		For any $\lambda \in \mft^*$, $\Hom_\mCO(L_\lambda,L_\lambda) = k\cdot \Id$ is 1-dimensional.
	\end{lem}

\section{The \texorpdfstring{$\sl_3$}{sl3} case}

	The following exercises are about the \textbf{principle block} $\mCO_{\varpi(0)}$ for $\mfg=\sl_3$.

	\begin{exe}
		\label{exe-H3P2}
		This is \red{Homework 3, Problem 2}. Consider $\mfg = \sl_3$ and its standard Borel $\mfb$ and Cartan subalgebras $\mft$. Let $\alpha_1$ and $\alpha_2$ be the two simple positive roots.
		\begin{itemize}
			\item[(1)]
				Prove: the elements in $W\cdot 0$ is given by
				\begin{center}
					\begin{tabular}{|c |c| c| c| c| c|}
					\hline 
					0 & $s_1\cdot 0$ & $s_2\cdot 0$ & $s_1s_2 \cdot 0$ & $s_2 s_1 \cdot 0$ & $w_0 \cdot 0$ \\
				\hline
						0 & $-\alpha_1$ & $-\alpha_2$ & $-2\alpha_1-\alpha_2$ & $-\alpha_1-2\alpha_2$ & $-2\alpha_1-2\alpha_2$. \\
			\hline
				\end{tabular}
			\end{center}
			Here $w_0 = s_1 s_2 s_1 = s_2 s_1 s_2$ is the \textbf{longest element} in $W$.
			\item[(2)]
				Prove: $M_{-2\alpha_1-2\alpha_2}$ is irreducible.
			\item[(3)]
				Prove: $M_{-\alpha_1-2\alpha_2}$ contains $M_{-2\alpha_1-2\alpha_2}$ as a submodule\footnote{Hint: [Lem. 4, Lect. 5].} and the quotient is irreducible\footnote{Hint: count the dimension of the $(-2\alpha_1-2\alpha_2)$-weight subspace of $M_{-\alpha_1-2\alpha_2}$.}. Deduce $\length(M_{-\alpha_1-2\alpha_2})=2$ and $[M_{-\alpha_1-2\alpha_2}: L_{-\alpha_1-2\alpha_2}] = [M_{-\alpha_1-2\alpha_2}: L_{-2\alpha_1-2\alpha_2}] = 1$.
		\end{itemize}
	\end{exe}

	\begin{rem}
		By symmetry, $M_{-2\alpha_1-\alpha_2}$ has length 2, and contains $M_{-2\alpha_1-2\alpha_2}$ as a submodule.
	\end{rem}

	\begin{exe}
		This is \red{Homework 3, Problem 3}. We continue with the case $\mfg = \sl_3$.
		\begin{itemize}
			\item[(1)]
				Prove: $M_0$ contains $M_{-\alpha_1}$ and $M_{-\alpha_2}$ as submodules and $[M_0: L_{-\alpha_1}]=[M_0: L_{-\alpha_2}]=1$.
			\item[(2)]
				Prove: $M_{-\alpha_2}$ contains $M_{-2\alpha_1-\alpha_2}$ as a submodule and $[M_{-\alpha_2}: L_{-2\alpha_1-\alpha_2}] = 1$.
			\item[(3)]
				Prove: there exists a (unique) dotted arrow making the following diagram commutes\footnote{Hint: show $f_1^2 f_2 \cdot v_0 = u \cdot v_{-\alpha_1}$ for some $u$. Here $v_0$ is the highest weight of $M_0$, $v_{-\alpha_1} = f_1 \cdot v_0$ is the highest weight of $M_{-\alpha_1}$, and $f_i\in \mfn^-$ is the root vector corresponding to $\alpha_i$.}:
				\[
					\xymatrix{
						& M_{-\alpha_1} \ar[r]^-\subset
						& M_0 \\
						M_{-2\alpha_1-\alpha_2} \ar[rrr]^-\subset \ar@{.>}[ru]
						& & & M_{-\alpha_2}. \ar[lu]^-\subset
					}
				\]
			
		\end{itemize}
	\end{exe}

	\begin{exe}
		This is \red{Homework 3, Problem 4}. We continue with the case $\mfg = \sl_3$. Prove: for $\lambda,\mu \in W\cdot 0$, $[M_\lambda: L_\mu] \neq 0$ iff $\lambda \succeq \mu$.
	\end{exe}

	\begin{rem}
		You may want to play with the above example for more time. E.g., can you find upper or lower bounds, or even the exact value, of $\length(M_0)$?
	\end{rem}

	Let us draw the set $W\cdot 0$ for $\sl_3$ in a directed graph:
	\[
		\xymatrix{
			& s_1\cdot 0 \ar[rr]
			& &  0 \\
			s_1s_2\cdot 0 \ar[ru] \ar[rrrr]
			& & & & s_2\cdot 0 \ar[lu] \\
			& w_0\cdot 0 \ar[rr] \ar[lu]
			& & s_2s_1\cdot 0. \ar[ru] \ar[lluu]
		}
	\]
	Here an edge between two vertices means there \emph{exists} an injective map between the corresponding Verma modules. This defines a partial order $\preceq_\subset$ on $W\cdot 0$\footnote{Question: why there is no nontrivial loop?} such that $\lambda\preceq_\subset \mu$ iff $M_{\lambda} \subset M_\mu$. The above exercises claim this partial order is exactly the partial order $\preceq$ defined using positive roots. 

	Note that
	\begin{itemize}
		\item $M_0$ contains any other Verma module in this block as a submodule;
		\item $M_{w_0\cdot 0}$ is contained in any other Verma module in this block and is irreducible.
		\item All the generating relation is of the form $w' \cdot 0 \to w\cdot 0$ with $w'=s_\alpha w$, $\alpha\in \Phi^+$ and $l(w')> l(w)$. Here $l(w)$ is the \textbf{length} of an element $w\in W$, defined as the minimal length of any expression of $w$ as products of \emph{simple} reflections. Note that $w_0$ is actually $s_{\alpha_1+\alpha_2}$.
	\end{itemize}
	All these results remain true for general $\mfg$, and even for any other block $\mCO_\chi$ after suitable modifications. These are the main contents of the following lectures.

	
\begin{thebibliography}{Yau}
	
	\bibitem[BL]{BL} Bernstein, Joseph, and Valery Lunts. A simple proof of Kostant's theorem that U(g) is free over its center, American Journal of Mathematics 118, no. 5 (1996): 979-987.

	\bibitem[B]{Bo} Bourbaki, N. Lie Groups and Lie Algebras.

	\bibitem[G]{G} Gaitsgory, Dennis. Course Notes for Geometric Representation Theory, 2005, available at \url{https://people.mpim-bonn.mpg.de/gaitsgde/267y/catO.pdf}.

	\bibitem[K]{K} Kostant, B. Lie group representations on polynomial rings, Amer. J. Math. 85 (1963), 327-402.

	\bibitem[H]{H} Humphreys, James E. Representations of Semisimple Lie Algebras in the BGG Category $\mathcal{O} $. Vol. 94. American Mathematical Soc., 2008.

	\bibitem[SGA1]{SGA1} Grothendieck, Alexandre. Revêtement étales et groupe fondamental (SGA1). Lecture Notes in Math. 288 (1971).


\end{thebibliography}

\end{document} 


