
%!TEX root = main.tex
\documentclass{amsart}
\textwidth=14.5cm \oddsidemargin=1cm
\evensidemargin=1cm
\usepackage{amsmath}
\usepackage{amsxtra}
\usepackage{amscd}
\usepackage{amsthm}
\usepackage{amsfonts}
\usepackage{amssymb}
\usepackage[foot]{amsaddr}
\usepackage{cite}
\usepackage{url}
\usepackage{rotating}
\usepackage{eucal}
\usepackage{tikz-cd}
\usepackage[all,2cell,color]{xy}
\UseAllTwocells
\UseCrayolaColors
\usepackage{graphicx}
\usepackage{pifont}
\usepackage{comment}
\usepackage{verbatim}
\usepackage{xcolor}
\usepackage{hyperref}
\usepackage{xparse}
\usepackage{upgreek}
\usepackage{MnSymbol}
\sloppy


%%%%%%%%%%%%%%%%%%%%Theorem%%%%%%%%%%%%%%%%%%%%
\newcounter{theorem}
\setcounter{theorem}{0}

\newtheorem{cor}[subsection]{Corollary}
\newtheorem{lem}[subsection]{Lemma}
\newtheorem{goal}[subsection]{Goal}
\newtheorem{lemdefn}[subsection]{Lemma-Definition}
\newtheorem{prop}[subsection]{Proposition}
\newtheorem{propdefn}[subsection]{Proposition-Definition}
\newtheorem{cordefn}[subsection]{Corollary-Definition}
\newtheorem{variant}[subsection]{Variant}
\newtheorem{warn}[subsection]{Warning}
\newtheorem{sugg}[subsection]{Suggestion}
\newtheorem{facts}[subsection]{Fact}
\newtheorem{ques}{Question}
\newtheorem{guess}{Guess}
\newtheorem{claim}{Claim}
\newtheorem{propconstr}[subsection]{Proposition-Construction}
\newtheorem{lemconstr}[subsection]{Lemma-Construction}
\newtheorem{ax}{Axiom}
\newtheorem{conje}[subsection]{Conjecture}
\newtheorem{mainthm}[subsection]{Main-Theorem}
\newtheorem{summ}[subsection]{Summary}
\newtheorem{thm}[subsection]{Theorem}
\newtheorem{thmdefn}[subsection]{Theorem-Definition}
\newtheorem{notn}[subsection]{Notation}
\newtheorem{convn}[subsection]{Convention}
\newtheorem{constr}[subsection]{Construction}


\theoremstyle{definition}

\newtheorem{defn}[subsection]{Definition}
\newtheorem{exam}[subsection]{Example}
\newtheorem{assum}[subsection]{Assumption}

\theoremstyle{remark}
\newtheorem{rem}[subsection]{Remark}
\newtheorem{exe}[subsection]{Exercise}


\numberwithin{equation}{section}


%%%%%%%%%%%%%%%%%%%%Commands%%%%%%%%%%%%%%%%%%%%

\newcommand{\nc}{\newcommand}
\nc\on{\operatorname}
\nc\renc{\renewcommand}


%%%%%%%%%%%%%%%%%%%%Sections%%%%%%%%%%%%%%%%%%%%

\nc\ssec{\subsection}
\nc\sssec{\subsubsection}

%%%%%%%%%%%%%%%%%%%%Environment%%%%%%%%%%%%%%%%%
\nc\blongeqn{\[ \begin{aligned}}
\nc\elongeqn{\end{aligned} \]}



%%%%%%%%%%%%%%%%%%%%Mathfont%%%%%%%%%%%%%%%%%%%%

\nc\mBA{{\mathbb A}}
\nc\mBB{{\mathbb B}}
\nc\mBC{{\mathbb C}}
\nc\mBD{{\mathbb D}}
\nc\mBE{{\mathbb E}}
\nc\mBF{{\mathbb F}}
\nc\mBG{{\mathbb G}}
\nc\mBH{{\mathbb H}}
\nc\mBI{{\mathbb I}}
\nc\mBJ{{\mathbb J}}
\nc\mBK{{\mathbb K}}
\nc\mBL{{\mathbb L}}
\nc\mBM{{\mathbb M}}
\nc\mBN{{\mathbb N}}
\nc\mBO{{\mathbb O}}
\nc\mBP{{\mathbb P}}
\nc\mBQ{{\mathbb Q}}
\nc\mBR{{\mathbb R}}
\nc\mBS{{\mathbb S}}
\nc\mBT{{\mathbb T}}
\nc\mBU{{\mathbb U}}
\nc\mBV{{\mathbb V}}
\nc\mBW{{\mathbb W}}
\nc\mBX{{\mathbb X}}
\nc\mBY{{\mathbb Y}}
\nc\mBZ{{\mathbb Z}}


\nc\mCA{{\mathcal A}}
\nc\mCB{{\mathcal B}}
\nc\mCC{{\mathcal C}}
\nc\mCD{{\mathcal D}}
\nc\mCE{{\mathcal E}}
\nc\mCF{{\mathcal F}}
\nc\mCG{{\mathcal G}}
\nc\mCH{{\mathcal H}}
\nc\mCI{{\mathcal I}}
\nc\mCJ{{\mathcal J}}
\nc\mCK{{\mathcal K}}
\nc\mCL{{\mathcal L}}
\nc\mCM{{\mathcal M}}
\nc\mCN{{\mathcal N}}
\nc\mCO{{\mathcal O}}
\nc\mCP{{\mathcal P}}
\nc\mCQ{{\mathcal Q}}
\nc\mCR{{\mathcal R}}
\nc\mCS{{\mathcal S}}
\nc\mCT{{\mathcal T}}
\nc\mCU{{\mathcal U}}
\nc\mCV{{\mathcal V}}
\nc\mCW{{\mathcal W}}
\nc\mCX{{\mathcal X}}
\nc\mCY{{\mathcal Y}}
\nc\mCZ{{\mathcal Z}}


\nc\mbA{{\mathsf A}}
\nc\mbB{{\mathsf B}}
\nc\mbC{{\mathsf C}}
\nc\mbD{{\mathsf D}}
\nc\mbE{{\mathsf E}}
\nc\mbF{{\mathsf F}}
\nc\mbG{{\mathsf G}}
\nc\mbH{{\mathsf H}}
\nc\mbI{{\mathsf I}}
\nc\mbJ{{\mathsf J}}
\nc\mbK{{\mathsf K}}
\nc\mbL{{\mathsf L}}
\nc\mbM{{\mathsf M}}
\nc\mbN{{\mathsf N}}
\nc\mbO{{\mathsf O}}
\nc\mbP{{\mathsf P}}
\nc\mbQ{{\mathsf Q}}
\nc\mbR{{\mathsf R}}
\nc\mbS{{\mathsf S}}
\nc\mbT{{\mathsf T}}
\nc\mbU{{\mathsf U}}
\nc\mbV{{\mathsf V}}
\nc\mbW{{\mathsf W}}
\nc\mbX{{\mathsf X}}
\nc\mbY{{\mathsf Y}}
\nc\mbZ{{\mathsf Z}}

\nc\mba{{\mathsf a}}
\nc\mbb{{\mathsf b}}
\nc\mbc{{\mathsf c}}
\nc\mbd{{\mathsf d}}
\nc\mbe{{\mathsf e}}
\nc\mbf{{\mathsf f}}
\nc\mbg{{\mathsf g}}
\nc\mbh{{\mathsf h}}
\nc\mbi{{\mathsf i}}
\nc\mbj{{\mathsf j}}
\nc\mbk{{\mathsf k}}
\nc\mbl{{\mathsf l}}
\nc\mbm{{\mathsf m}}
\nc\mbn{{\mathsf n}}
\nc\mbo{{\mathsf o}}
\nc\mbp{{\mathsf p}}
\nc\mbq{{\mathsf q}}
\nc\mbr{{\mathsf r}}
\nc\mbs{{\mathsf s}}
\nc\mbt{{\mathsf t}}
\nc\mbu{{\mathsf u}}
\nc\mbv{{\mathsf v}}
\nc\mbw{{\mathsf w}}
\nc\mbx{{\mathsf x}}
\nc\mby{{\mathsf y}}
\nc\mbz{{\mathsf z}}




\nc\mbfA{{\mathbf A}}
\nc\mbfB{{\mathbf B}}
\nc\mbfC{{\mathbf C}}
\nc\mbfD{{\mathbf D}}
\nc\mbfE{{\mathbf E}}
\nc\mbfF{{\mathbf F}}
\nc\mbfG{{\mathbf G}}
\nc\mbfH{{\mathbf H}}
\nc\mbfI{{\mathbf I}}
\nc\mbfJ{{\mathbf J}}
\nc\mbfK{{\mathbf K}}
\nc\mbfL{{\mathbf L}}
\nc\mbfM{{\mathbf M}}
\nc\mbfN{{\mathbf N}}
\nc\mbfO{{\mathbf O}}
\nc\mbfP{{\mathbf P}}
\nc\mbfQ{{\mathbf Q}}
\nc\mbfR{{\mathbf R}}
\nc\mbfS{{\mathbf S}}
\nc\mbfT{{\mathbf T}}
\nc\mbfU{{\mathbf U}}
\nc\mbfV{{\mathbf V}}
\nc\mbfW{{\mathbf W}}
\nc\mbfX{{\mathbf X}}
\nc\mbfY{{\mathbf Y}}
\nc\mbfZ{{\mathbf Z}}

\nc\mbfa{{\mathbf a}}
\nc\mbfb{{\mathbf b}}
\nc\mbfc{{\mathbf c}}
\nc\mbfd{{\mathbf d}}
\nc\mbfe{{\mathbf e}}
\nc\mbff{{\mathbf f}}
\nc\mbfg{{\mathbf g}}
\nc\mbfh{{\mathbf h}}
\nc\mbfi{{\mathbf i}}
\nc\mbfj{{\mathbf j}}
\nc\mbfk{{\mathbf k}}
\nc\mbfl{{\mathbf l}}
\nc\mbfm{{\mathbf m}}
\nc\mbfn{{\mathbf n}}
\nc\mbfo{{\mathbf o}}
\nc\mbfp{{\mathbf p}}
\nc\mbfq{{\mathbf q}}
\nc\mbfr{{\mathbf r}}
\nc\mbfs{{\mathbf s}}
\nc\mbft{{\mathbf t}}
\nc\mbfu{{\mathbf u}}
\nc\mbfv{{\mathbf v}}
\nc\mbfw{{\mathbf w}}
\nc\mbfx{{\mathbf x}}
\nc\mbfy{{\mathbf y}}
\nc\mbfz{{\mathbf z}}

\nc\mfa{{\mathfrak a}}
\nc\mfb{{\mathfrak b}}
\nc\mfc{{\mathfrak c}}
\nc\mfd{{\mathfrak d}}
\nc\mfe{{\mathfrak e}}
\nc\mff{{\mathfrak f}}
\nc\mfg{{\mathfrak g}}
\nc\mfh{{\mathfrak h}}
\nc\mfi{{\mathfrak i}}
\nc\mfj{{\mathfrak j}}
\nc\mfk{{\mathfrak k}}
\nc\mfl{{\mathfrak l}}
\nc\mfm{{\mathfrak m}}
\nc\mfn{{\mathfrak n}}
\nc\mfo{{\mathfrak o}}
\nc\mfp{{\mathfrak p}}
\nc\mfq{{\mathfrak q}}
\nc\mfr{{\mathfrak r}}
\nc\mfs{{\mathfrak s}}
\nc\mft{{\mathfrak t}}
\nc\mfu{{\mathfrak u}}
\nc\mfv{{\mathfrak v}}
\nc\mfw{{\mathfrak w}}
\nc\mfx{{\mathfrak x}}
\nc\mfy{{\mathfrak y}}
\nc\mfz{{\mathfrak z}}

\nc{\one}{{\mathsf{1}}}


\nc\clambda{ {\check{\lambda} }}
\nc\cmu{ {\check{\mu} }}

\nc\bDelta{\mathsf{\Delta}}
\nc\bGamma{\mathsf{\Gamma}}
\nc\bLambda{\mathsf{\Lambda}}


\nc\loccit{\emph{loc.cit.}}



%%%%%%%%%%%%%%%%%%%%Operations-limit%%%%%%%%%%%%%%%%%%%%

\NewDocumentCommand{\ot}{e{_^}}{
  \mathbin{\mathop{\otimes}\displaylimits
    \IfValueT{#1}{_{#1}}
    \IfValueT{#2}{^{#2}}
  }
}
\NewDocumentCommand{\boxt}{e{_^}}{
  \mathbin{\mathop{\boxtimes}\displaylimits
    \IfValueT{#1}{_{#1}}
    \IfValueT{#2}{^{#2}}
  }
}
\NewDocumentCommand{\mt}{e{_^}}{
  \mathbin{\mathop{\times}\displaylimits
    \IfValueT{#1}{_{#1}}
    \IfValueT{#2}{^{#2}}
  }
}
\NewDocumentCommand{\convolve}{e{_^}}{
  \mathbin{\mathop{\star}\displaylimits
    \IfValueT{#1}{_{#1}}
    \IfValueT{#2}{^{#2}}
  }
}
\NewDocumentCommand{\colim}{e{_^}}{
  \mathbin{\mathop{\operatorname{colim}}\displaylimits
    \IfValueT{#1}{_{#1}\,}
    \IfValueT{#2}{^{#2}\,}
  }
}
\NewDocumentCommand{\laxlim}{e{_^}}{
  \mathbin{\mathop{\operatorname{laxlim}}\displaylimits
    \IfValueT{#1}{_{#1}\,}
    \IfValueT{#2}{^{#2}\,}
  }
}
\NewDocumentCommand{\oplaxlim}{e{_^}}{
  \mathbin{\mathop\operatorname{oplax-lim}\displaylimits
    \IfValueT{#1}{_{#1}\,}
    \IfValueT{#2}{^{#2}\,}
  }
}


%%%%%%%%%%%%%%%%%%%%Arrows%%%%%%%%%%%%%%%%%%%%


\makeatletter
\newcommand{\laxto}{\dashedrightarrow}
\newcommand{\xrightleftarrows}[1]{\mathrel{\substack{\xrightarrow{#1} \\[-.9ex] \xleftarrow{#1}}}}
\newcommand{\adj}{\xrightleftarrows{\rule{0.5cm}{0cm}}}

\newcommand*{\da@rightarrow}{\mathchar"0\hexnumber@\symAMSa 4B }
\newcommand*{\da@leftarrow}{\mathchar"0\hexnumber@\symAMSa 4C }
\newcommand*{\xlaxto}[2][]{%
  \mathrel{%
    \mathpalette{\da@xarrow{#1}{#2}{}\da@rightarrow{\,}{}}{}%
  }%
}
\newcommand{\xlaxgets}[2][]{%
  \mathrel{%
    \mathpalette{\da@xarrow{#1}{#2}\da@leftarrow{}{}{\,}}{}%
  }%
}
\newcommand*{\da@xarrow}[7]{%
  % #1: below
  % #2: above
  % #3: arrow left
  % #4: arrow right
  % #5: space left 
  % #6: space right
  % #7: math style 
  \sbox0{$\ifx#7\scriptstyle\scriptscriptstyle\else\scriptstyle\fi#5#1#6\m@th$}%
  \sbox2{$\ifx#7\scriptstyle\scriptscriptstyle\else\scriptstyle\fi#5#2#6\m@th$}%
  \sbox4{$#7\dabar@\m@th$}%
  \dimen@=\wd0 %
  \ifdim\wd2 >\dimen@
    \dimen@=\wd2 %   
  \fi
  \count@=2 %
  \def\da@bars{\dabar@\dabar@}%
  \@whiledim\count@\wd4<\dimen@\do{%
    \advance\count@\@ne
    \expandafter\def\expandafter\da@bars\expandafter{%
      \da@bars
      \dabar@ 
    }%
  }%  
  \mathrel{#3}%
  \mathrel{%   
    \mathop{\da@bars}\limits
    \ifx\\#1\\%
    \else
      _{\copy0}%
    \fi
    \ifx\\#2\\%
    \else
      ^{\copy2}%
    \fi
  }%   
  \mathrel{#4}%
}
\makeatother

%%%%%%%%%%%%%%%%%%%%Decorations%%%%%%%%%%%%%%%%%%%%
\nc{\wt}{\widetilde}
\nc{\ol}{\overline}

\nc{\red}{\textcolor{red}}
\nc{\blue}{\textcolor{blue}}
\nc{\purple}{\textcolor{violet}}

\nc{\simorlax}{{\red\simeq/\blue\lax}}

%%%%%%%%%%%%%%%%%%%%All%%%%%%%%%%%%%%%%%%%%

\nc{\Id}{\mathsf{Id}}
\nc{\gl}{\mathfrak{gl}}
\renc{\sl}{\mathfrak{sl}}
\nc{\GL}{\mathsf{GL}}
\nc{\SL}{\mathsf{SL}}
\nc{\PGL}{\mathsf{PGL}}
\nc{\hmod}{\mathsf{-mod}}
\nc{\Vect}{\mathsf{Vect}}
\nc{\tr}{\mathsf{tr}}
\nc{\Kil}{\mathsf{Kil}}
\nc{\ad}{{\mathsf{ad}}}
\nc{\Ad}{\mathsf{Ad}}
\nc{\oblv}{\mathsf{oblv}}
\nc{\gr}{\mathsf{gr}}
\nc{\Sym}{\mathsf{Sym}}
\nc{\QCoh}{\mathsf{QCoh}}
\nc{\ind}{\mathsf{ind}}
\nc{\Spec}{\mathsf{Spec}}
\nc{\Hom}{\mathsf{Hom}}
\nc{\Ext}{\mathsf{Ext}}
\nc{\Grp}{\mathsf{Grp}}
\nc{\pt}{\mathsf{pt}}
\nc{\Lie}{\mathsf{Lie}}
\nc{\CAlg}{\mathsf{CAlg}}
\nc{\Der}{\mathsf{Der}}
\nc{\Rep}{\mathsf{Rep}}
\renc{\sc}{{\mathsf{sc}}}
\nc{\Fl}{\mathsf{Fl}}
\nc{\Fun}{\mathsf{Fun}}
\nc{\ev}{\mathsf{ev}}
\nc{\surj}{\twoheadrightarrow}
\nc{\inj}{\hookrightarrow}
\nc{\HC}{\mathsf{HC}}
\nc{\cl}{\mathsf{cl}}
\renc{\Im}{\mathsf{Im}}
\renc{\ker}{\mathsf{ker}}
\nc{\coker}{\mathsf{coker}}
\nc{\Tor}{\mathsf{Tor}}
\nc{\op}{\mathsf{op}}
\nc{\length}{\mathsf{length}}
\nc{\fd}{{\mathsf{fd}}}
\nc{\weight}{\mathsf{wt}}
\nc{\semis}{{\mathsf{ss}}}
\nc{\qc}{{\mathsf{qc}}}
\nc{\pr}{\mathsf{pr}}
\nc{\act}{\mathsf{act}}
\nc{\dR}{{\mathsf{dR}}}
\nc{\hol}{{\mathsf{hol}}}
\nc{\Pic}{{\mathsf{Pic}}}
\nc{\Loc}{\mathsf{Loc}}
\nc{\IC}{\mathsf{IC}}

\begin{document}


\title{Lecture 10}

\date{Apr 29, 2024}

\maketitle

	From this lecture on, we turn to the geometric side of the localization theory. The main player will be (algebraic) D-modules on the flag variety $G/B$. In this lecture, we define D-modules. There are many good references for this theory. For example, \cite{HTT} is a thorough textbook, while \cite{B} is a short notes.

\section{Recollection: (co)tangent sheaves}

	Recall the following definitions in algebraic geometry.

	\begin{defn}
		Let $A$ be a $k$-algebra and $M$ be an $A$-module. A \textbf{$k$-derivation} of $A$ into $M$ is a $k$-linear map $D:A\to M$ satisfying the \textbf{Lebniz rule}
		\[
			D(f\cdot g) = f\cdot D (g) + g\cdot D (f).
		\]
		Let $\Der(A,M)$ be the set of such $k$-derivations. This is naturally an $A$-module.
	\end{defn}

	\begin{defn}
		Let $(X,\mCO_X)$ be any $k$-ringed space and $\mCM$ be an $\mCO_X$-module. A \textbf{$k$-derivation} of $\mCO_X$ into $\mCM$ is a $k$-linear morphism $D:\mCO_X \to \mCM$ such that for any open subscheme $U\subset X$, $D(U): \mCO(U) \to \mCM(U)$ is a $k$-derivation. Let $\Der(\mCO_X,\mCM)$ be the space of $k$-derivations.
	\end{defn}

	\begin{propdefn}
		Let $X$ be a $k$-scheme. The functor $\mCO_X\hmod \to \Vect$, $\mCM \mapsto \Der(\mCO_X,\mCM)$ is represented by a quasi-coherent $\mCO_X$-module $\Omega_{X}^1$, i.e. 
		\[
			\Hom_{\mCO_X}(\Omega_{X}^1, \mCM) \simeq \Der(\mCO_X,\mCM).
		\]
		We call $\Omega_{X}^1$ the \textbf{sheaf of differentials}, or the \textbf{cotangent sheaf}, of $X$ over $k$.

		The identity map on $\Omega_{X}^1$ corresponds to a $k$-derivation 
		\[
			d:\mCO_X \to \Omega_{X}^1,
		\]
		which is called the \textbf{universal $k$-derivation} of $\mCO_X$.

		When $X=\Spec(A)$ is affine, let $\Omega_{A}^1\in A\hmod$ be such that $\Omega_{X}^1 \simeq \wt{\Omega_{A}^1}$. We call $\Omega_{A}^1$ the \textbf{module of differentials} of $A$ over $k$.
	\end{propdefn}

	\begin{exam}
		\label{exam-cotangent-affine-space}
		If $A=\Sym(V)$ for a $k$-vector space $V\in \Vect$, then $\Omega_{A}^1 \simeq A\ot_k V$ and $d: A\to \Omega_{A}^1$ sends $v\in V \subset \Sym(V)$ to $dv=1\ot v \in A\ot_k V $.
	\end{exam}
	
	\begin{constr}
		Let $f:X\to Y$ be a morphism between $k$-schemes. For any $\mCO_X$-module $\mCM$, consider the composition
		\[
			\Der(\mCO_X,\mCM) \to \Der( f_* \mCO_X, f_* \mCM ) \to \Der( \mCO_Y, f_* \mCM )
		\]
		that sends a $k$-derivation $D: \mCO_X \to \mCM$ to $\mCO_Y \to f_* \mCO_X \xrightarrow{f_*(D)} f_* \mCM$. By definition, we obtain maps
		\[
			\Hom_{\mCO_X}(\Omega_{X}^1, \mCM) \to \Hom_{\mCO_Y}(\Omega_{Y}^1, f_*\mCM) \simeq  \Hom_{\mCO_X}(f^*\Omega_{Y}^1, \mCM)
		\] 
		that are functorial in $\mCM$. This gives an $\mCO_X$-linear morphism
		\begin{equation}
			\label{eqn-functorial-cotangent}
			f^*\Omega_{Y}^1\to \Omega_{X}^1.
		\end{equation}
	\end{constr}

	\begin{lem}
		If $f: X\to Y$ is an open embedding, or more generally an étale morphism, then $f^*\Omega_{Y}^1\to \Omega_{X}^1$ is an isomorphism.
	\end{lem}

	\begin{lem}
		Let $f:X\to Y$ be a closed embedding corresponding to the ideal sheaf $\mCI \subset \mCO_Y$. Then we have 
		\[
			\mCI/\mCI^2 \xrightarrow{d} f^*\Omega_{Y}^1\to \Omega_{X}^1 \to 0,
		\]
		where the map $\mCI/\mCI^2 \to f^*\Omega_Y^1 \simeq \Omega_Y^1/\mCI\Omega_Y^1$ is induced by $\mCI\inj \mCO_Y \xrightarrow{d}\Omega_Y^1$.
	\end{lem}

	\begin{rem}
		Since any affine scheme is a closed subscheme of $\Spec(\Sym(V))$ for some $V\in \Vect$, the above lemmas, together with Example \ref{exam-cotangent-affine-space}, allow us to calculate $\Omega_{X}^1$ for any $k$-scheme $X$.
	\end{rem}

	\begin{cor}
		If $X$ is (locally) of finite type over $k$, then $\Omega^1_{X}$ is (locally) coherent.
	\end{cor}

	\begin{cor}
		If $X$ is a smooth $k$-scheme of dimension $n$, then $\Omega^1_{X}$ is locally free of rank $n$.
	\end{cor}

	From now on, we always assume $X$ is a smooth $k$-scheme.

	\begin{constr}
		Let $X$ be a smooth $k$-scheme. Define $\Omega_X^n:=\bigwedge_{\mCO_X}^n (\Omega_X^1)$, i.e., the anti-symmetric quotient of $\Omega_X^1 \ot_{\mCO_X} \cdots \ot_{\mCO_X} \Omega_X^1$. As in the study of differential geometry, there is a unique complex
		\[
			\mCO_X \xrightarrow{d} \Omega_X^1 \xrightarrow{d} \Omega_X^2 \xrightarrow{d} \cdots
		\]
		such that $d(\alpha\wedge \beta) = d\alpha\wedge \beta + (-1)^m (\alpha\wedge d\beta)$, where $\alpha\in \Omega_X^m(U)$ is a $m$-form. This is the \textbf{de Rham complex} of $X$.

	\end{constr}

	\begin{constr}
		Let $X$ be a smooth $k$-scheme. We define the \textbf{tangent sheaf} $\mCT_{X}$ of $X$ over $k$ to be the dual of $\Omega^1_{X}$, i.e., 
		\[
			\mCT_{X} := \mathcal{H}om_{\mCO_X}( \Omega^1_{X},\mCO_X ).
		\]
		Note that $\mCT_{X}$ is quasi-coherent, and for any open subset $U$, we have
		\[
			\mCT(U):= \Der( \mCO_U, \mCO_U).
		\]
		By definition, for $\partial\in \mCT(U)$ and $f\in \mCO(U)$, we have $\partial(f) = \langle \partial, df\rangle$.
	\end{constr}

	\begin{cor}
		\label{cor-der-tensor}
		We have equivalences between functors $\mCO_X\hmod \to \Vect$:
		\[
			\Der_k(\mCO_X,-) \simeq \Gamma(X, \mCT_X \ot_{\mCO_X} - ).
		\]
	\end{cor}

	\begin{rem}
		The \textbf{tangent space} $T_{X,x}$ introduced in [Section 3, Lecture 3] can be identified with the stalk of $\mCT_{X}$ at $x$.
	\end{rem}

	\begin{cor}
		If $X$ is a smooth $k$-scheme of dimension $n$, then $\mCT_{X}$ is locally free of rank $n$.
	\end{cor}

	

	\begin{constr}
		Let $f:X\to Y$ be a morphism between smooth $k$-schemes. The morphism \eqref{eqn-functorial-cotangent} induces an $\mCO_X$-linear morphism
		\[
			df: \mCT_{X}\to f^*\mCT_{Y}.
		\]

	\end{constr}

	\begin{lem}
		Let $X = X_1 \mt X_2$ be a smooth $k$-scheme. Then the $\mCO_X$-linear morphism
		\[
			(d\pr_1, d\pr_2): \mCT_{X}\to \pr_1^*\mCT_{X_1} \oplus \pr_2^*\mCT_{X_2}
		\]
		is an isomorphism.
	\end{lem}

	\begin{constr}
		Let $X$ be a smooth $k$-scheme. For any open subset $U$ of $X$, $[D_1,D_2]:= D_1\circ D_2 - D_2\circ D_1$ defines a Lie bracket on $\Der( \mCO_U, \mCO_U)$. By defition, $\mCO_U$ is a representation of the obtained Lie algebra.

		It follows that $\mCT_{X}$ is a \emph{sheaf of Lie algebras} on $X$, and $\mCO_X$ is a $\mCT_{X}$-module.
	\end{constr}

	\begin{warn}
		The category $\mCO_X\hmod$ has a natural symmetric monoidal structure but $\mCT_{X}$ is \emph{not} a Lie algebra object in this symmetric monoidal category. In other words, the Lie bracket $[-,-]: \mCT_{X} \ot_k \mCT_{X} \to \mCT_{X}$ does not factor through $\mCT_{X} \ot_{\mCO_X} \mCT_{X} $.
	\end{warn}

	\begin{constr}
		As in the study of differential geometry, for any local section $\partial\in \mCT(U)$, we can define the \textbf{contraction} operator
		\[
			i_\partial: \Omega^{n+1}_X(U) \to \Omega^{n}_X(U)
		\]
		and the \textbf{Lie derivative} 
		\[
			\mCL_\partial:  \Omega^{n}_X(U)\to  \Omega^{n}_X(U).
		\]
		They satisfy all the identities in differential geometry. In particular, we have the \textbf{Cartan's magic formula}:
		\[
			\mCL_\partial(\omega) = i_\partial(d\omega) + d( i_\partial\omega ).
		\]

	\end{constr}

	We have the following useful result, which allows us to apply the techniques in differential geometry to the study of $\mCT_X$ and $\Omega_X$. For a proof, see \cite[Theorem A.5.1]{HTT}.

	\begin{propdefn}
		Let $X$ be any $n$-dimensional smooth $k$-scheme and $p\in X$ be a closed point. Then there exists an affine open neighborhood $U$ of $p$ and functions $x_i\in \mCO(U)$, $i=1,\cdots,n$, such that
		\begin{itemize}
		 	\item[(i)] 
		 		$\{dx_i\}$ is a free basis of $\Omega^1(U)$ as an $\mCO(U)$-module;
		 	\item[(ii)] 
		 		Let $\{\partial_i\}$ be the dual basis of $\mCT(U)$, then $[\partial_i,\partial_j] = 0$ and $\partial_i(x_j) = \delta_{ij}$.
		 	\item[(iii)] 
		 		The images of $x_i$ in the local ring $\mCO_{X,p}$ generate the maximal ideal $\mfm_{X,p}$.
		 \end{itemize} 
		 We call such a system $\{x_i\}$ an \textbf{étale coordinate system} of $X$ near $p$.
	\end{propdefn}

	\begin{rem}
		(ii) actually follows from (i).
	\end{rem}

	\begin{rem}
		The functions $x_i$ define an étale map $X\to \mBA^n$ sending $p\in X$ to the origin $0\in \mBA^n$.
	\end{rem}




\section{Tangent sheaf vs. Lie algebra}
	
	\begin{constr}
		Let $X$ be a finite type $k$-scheme equipped with a \emph{right} action of an algebraic group $G$. Let $\mfg:=\Lie(G)$ be the Lie algebra of $G$. We construct a Lie algebra homomorphism
		\[
			a: \mfg\to \mCT(X)
		\]
		as follows. Consider the action map $\act: X\mt G \to X$ and the $\mCO_{X\mt G}$-linear morphisms
		\[
			\mCO_X \ot \mCT_{G} \simeq \pr_2^*\mCT_{G} \inj \mCT_{X\mt G} \xrightarrow{d\act} \act^*\mCT_{X}.
		\]
		By restricting along $X \xrightarrow{\Id\mt e} X\mt G$, we otain $\mCO_X$-linear morphisms
		\[
			\mCO_X \ot \mfg  \to \mCT_{X},
		\]
		which induces the desired $k$-linear map $a: \mfg \to \mCT(X)$.
	\end{constr}

	\begin{rem}
		Using the language of differential geometry, for $x\in \mfg$, $a(x)$ is the vector field of the flow $\act(-,\exp(tx)): X \to X$. 
	\end{rem}

	\begin{lem}
		The above map $a: \mfg \to \mCT(X)$ is a Lie algebra homomorphism.
	\end{lem}

	\proof
		Unwinding the definitions, the corresponding map $\Der(\mCO_G,k_e ) \to \Der(\mCO_X,\mCO_X )$ sends a $k$-derivation $\partial: \mCO_G \to k_e$ to the composition\footnote{For non-affine $G$, we use the morphism between $\mCO_X$-modules: $\mCO_X \to \act_*(\mCO_G \ot \mCO_X) \xrightarrow{\partial\ot \Id} \act_*(k_e \ot \mCO_X) \simeq \mCO_X$.}
		\[
			a(\partial_1):\mCO_X \to \mCO_X \ot \mCO_G \xrightarrow{\Id \ot \partial} \mCO_X \ot k_e \simeq \mCO_X.
		\]
		Using the axioms of group actions, we can identify $a(\partial_1)\circ a(\partial_2)$ with
		\[
			\mCO_X \to \mCO_X \ot \mCO_G \xrightarrow{ \Id \ot \Delta } \mCO_X \ot \mCO_G \ot \mCO_G \xrightarrow{ \Id \ot \partial_1 \ot \partial_2 } \mCO_X \ot k_e \ot k_e \simeq \mCO_X.
		\]
		This implies 
		\[
			[a(\partial_1),a(\partial_2)] = a(\partial_1)\circ a(\partial_2) - a(\partial_2)\circ a(\partial_1) = a( [\partial_1, \partial_2] ),
		\]
		where the last equation is due to [Remark 4.5, Lecture 3].

	\qed

	\begin{rem}
		Let $X$ be a finite type $k$-scheme equipped with a \emph{left} action of an algebraic group $G$. We can obtain a right $G$-action by precomposing with $g\mapsto g^{-1}$. It follows that we also have a Lie algebra homomorphism $a: \mfg\to \mCT(X)$. Using the language of differential geometry, for $x\in \mfg$, $a(x)$ is the vector field of the flow $\act(\exp(-tx),-): X \to X$. 

	\end{rem}


	\begin{exam}
		Consider the left and right multplication actions of $G$ on itself. We obtain Lie algebra homomorphisms
		\[
			\mfg \xrightarrow{a_l} \mCT(G) \xleftarrow{a_r} \mfg.
		\]
		By construction, the image of $a_r$ consists of \emph{left} invariant vector fields on $G$, while the image of $a_l$ consists of \emph{right} invariant ones. 

		It is easy to see that the images of $a_l: \mfg \to \mCT(G)$ and $a_r: \mfg \to \mCT(G)$ commute with respect to the Lie bracket, i.e., $[a_l(x), a_r(y)] = 0$ for $x,y\in \mfg$. Indeed, this follows by considering the $(G\mt G)$-action on $G$ given by $(g_1,g_2)\cdot x := g_1^{-1}xg_2$.

		It is easy to show the obtained $\mCO_G$-linear maps, which we denote by the same symbols,
		\[
			\mCO_G \ot \mfg \xrightarrow{a_l} \mCT_G \xleftarrow{a_r} \mCO_G \ot \mfg
		\]
		are isomorphisms. Note that the stalks of the above maps at $e\in G$ are given by
		\[
			\mfg \xrightarrow{ -\Id } \mfg \xleftarrow{\Id} \mfg.
		\]
	\end{exam}


	The following terminology is not standard\footnote{The standard terminology would be: $\mCM$ is a module for the \emph{Lie algebroid $\mCT_{X}$}.}.

	\begin{defn}
		Let $X$ be a smooth $k$-scheme. A \textbf{quasi-coherent $\mCT_{X}$-module} is a quasi-coherent $\mCO_X$-module $\mCM$ equipped with a $\mCT_{X}$-module structure such that 
		\begin{itemize}
			\item[(i)]
				The map $\mCT_X\ot_k \mCM \to \mCM$ is $\mCO_X$-linear, where $\mCT_X\ot_k \mCM$ is viewed as an $\mCO_X$-module via the first factor.
			\item[(ii)]
				The map $\mCO_X \ot_k \mCM \to \mCM$ is $\mCT_{X}$-linear, where $\mCO_X \ot_k \mCM $ is viewed as an $\mCT_X$-module via the diagonal action.
		\end{itemize}
		Let $\mCT_X\hmod_\qc$ be the category of quasi-coherent $\mCT_{X}$-module, where morphisms are defined in the obvious way.
	\end{defn}

	\begin{rem}
		Unwinding the definitions, (ii) means the action of $\mCT_X$ on $\mCM$ satisfies the \textbf{Lebniz rule}: for any local sections $\partial\in \mCT(U)$, $m\in \mCM(U)$ and $f\in \mCO(U)$, we have
		\[
			\partial( f\cdot m ) = f\cdot \partial(m)+\partial(f)\cdot m.
		\]
	\end{rem}

	\begin{exam}
		The structure sheaf $\mCO_X$, equipped with the standard $\mCT_X$-action, is a quasi-coherent $\mCT_X$-module.
	\end{exam}

	\begin{rem}
		It is easy to show $\mCT_X\hmod_\qc$ is an abelian category and the forgetful functor $\mCT_X\hmod_\qc \to \mCO_X\hmod_\qc$ is exact.
	\end{rem}


	\begin{constr}
		Let $X$ be a finite type $k$-scheme equipped with an action of an algebraic group $G$. We have a functor
		\[
			\Gamma: \mCT_X\hmod_\qc \to \mfg\hmod,\; \mCM \mapsto \mCM(X),
		\]
		where $\mCM(X)$ is viewed as a $\mfg$-module via the Lie algebra homomorphism $a:\mfg \to \mCT(X)$.
	\end{constr}

	\begin{rem}
		One version of the localizaiton theory says for $X=G/B$, the above functor induces an equivalence
		\[
			\Gamma: \mCT_X\hmod_\qc \xrightarrow{\simeq} \mfg\hmod_{\chi_0},
		\]
		where $\chi_0=\varpi(0)$ is the central character of $M_0$.
	\end{rem}

\section{Differential operators}

	Just like the associative algebra $U(\mfg)$ plays a significant role in the study of $\mfg$-modules, there is a sheaf of associative algebras, known as the sheaf of differential operators $\mCD_X$, that plays a similar role in the study of $\mCT_X$-modules\footnote{The sheaf $\mCD_X$ is so useful that $\mCD_X$-modules were introduced much earlier than $\mCT_X$-modules. The latter point of view was ignored until the study of general Lie algebroids.}.

	\begin{defn}
		\label{defn-diff}
		Let $X$ be an affine smooth $k$-scheme. We define the notion of differential operator on $X$ inductively. 

		A $k$-linear map $D: \mCO(X) \to \mCO(X)$ is a differential operator of order $-n$ ($n>0$) iff $D=0$.

		A $k$-linear map $D: \mCO(X) \to \mCO(X)$ is a \textbf{differential operator of order $n$} ($n\ge 0$) if for any function $f\in \mCO(X)$, the $k$-linear map $[D,f]:=D\circ f- f\circ D$, i.e.,
		\[
			g \mapsto D(fg) - fD(g)
		\]
		is a differential operator of order $n-1$. 

		Let $\mbF^{\le n}\mCD(X)$ be the space of differential operators of order $n$ on $X$, and $\mCD(X):= \cup_n \mbF^{\le n}\mCD(X)$ be the space of all differential operators on $X$.
	\end{defn}

	\begin{exam}
		Multplication by any $f\in \mCO(X)$ is a differential operator of order $0$ and the map $\mCO(X) \to \mbF^{\le 0}\mCD(X) \simeq \gr^0 \mCD(X)$ is an isomorphism.
	\end{exam}

	\begin{exe}
		\label{exe-diff-2}
		This is \red{Homework 5, Problem 1}. Let $X$ be an affine smooth $k$-scheme. Prove: any $k$-derivation $\mCO(X) \to \mCO(X)$ is a differential operator of order $1$, and the obtained map $\mCO(X) \oplus \mCT(X) \to \mbF^{\le 1} \mCD(X)$ is an isomorphism.
	\end{exe}



	\begin{lem}
		Let $X$ be an affine smooth $k$-scheme, and $D_1\in \mbF^{\le m}\mCD(X)$, $D_2\in \mbF^{\le n}\mCD(X)$ be differential operators. Prove:
		\begin{itemize}
			\item[(1)]
				The composition $D_1\circ D_2$ is a differential operator of order $m+n$.
			\item[(2)]
				The commutator $[D_1,D_2]:=D_1\circ D_2 - D_2 \circ D_1$ is a differential operator of order $m+n-1$.		
		\end{itemize}
		In other words, $\mbF^\bullet\mCD(X)$ is a filtered associative algebra such that $\gr^\bullet \mCD(X)$ is commutative.
	\end{lem}

	\proof
		Follow from the above exercise and induction.

	\qed

	\begin{rem}
		\label{rem-check-diff}
		For \emph{any} $k$-linear map $D: \mCO(X) \to \mCO(X)$ and functions $u,v\in \mCO(X)$, we have
		\begin{equation}
			\label{eqn-lebniz-for-commutator}
			[D, uv] = [D,u]v + u[D,v].
		\end{equation}
		It follows that in Definition \ref{defn-diff}, we can replace ``for any function $f\in \mCO(X)$'' by ``for generators $f$ of $\mCO(X)$''.

		Note that we also have
		\begin{equation}
			\label{eqn-lebniz-for-operator}
			D(uv) = [D,u](v) + uD(v).
		\end{equation}

	\end{rem}

	\begin{rem}
		\label{rem-diff-disj}
		For a disjoint union $V_1\sqcup V_2$ of affine smooth $k$-schemes, we have $\mbF^\bullet \mCD(V_1\sqcup V_2) \simeq \mbF^\bullet \mCD(V_1) \oplus  \mbF^\bullet \mCD(V_2)$. 
	\end{rem}



	\begin{warn}
		Definition \ref{defn-diff} makes sense even when $X$ is singular, but the obtained algebra $\mCD(X)$ is ill-behaved and is not the correct algebra to consider. In modern point of view, for singular $X$, the ``correct'' $\mCD(X)$ should be a DG algebra.
	\end{warn}

\section{Sheaf of differential operators}

	In this section, for any smooth $k$-scheme, we construct a sheaf $\mCD_X$ of filtered associative algebras such that for any affine open subscheme $U$, we have $\mCD_X(U) \simeq \mCD(U)$. The proofs in this section are technical and can be treated as blackboxes.

	Throughout this section, let $V$ be a \emph{connected} affine smooth $k$-scheme and $f\in \mCO(V)$ be a nonzero function. Let $U\subset V$ be the affine open subscheme where $f\neq 0$, i.e. $\mCO(U) \simeq \mCO(V)_f$. Note that $U$ is also connected. We identify $\mCO(V)$ as a subalgebra of $\mCO(U)$.

	\begin{lem}
		\label{lem-extend-diff-op-unique}
		Any differential operator $D\in \mCD(U)$ is determined by its restriction $D|_{\mCO(V)}: \mCO(V) \to \mCO(U)$.
	\end{lem}

	\proof
		Let $D\in \mbF^{\le n} \mCD(U) $. We prove by induction in $n$. When $n<0$, there is nothing to prove. For $n\ge 0$, for any $g\in \mCO(V)$ and $m\ge 0$, we have
		\begin{equation}
			\label{eqn-D-local-proof-1}
			D(g) = D(f^m \cdot \frac{g}{f^m}) = f^m \cdot D(\frac{g}{f^m}) + [D,f^m](  \frac{g}{f^m}).
		\end{equation}
		It follows that
		\begin{equation}
			\label{eqn-D-local-proof-2}
			D(\frac{g}{f^m}) = \frac{D(g)}{f^m} - \frac{[D,f^m](  \frac{g}{f^m})}{f^m}.
		\end{equation}
		Note that $[D,f^m]\in \mbF^{\le n-1} \mCD(U)$ hence it is determined by $[D,f^m]|_{\mCO(V)} = [D|_{\mCO(V)}, f^m]$, which is determined by $D|_{\mCO(V)}$. Now \eqref{eqn-D-local-proof-2} implies $D$ is also determined by $D|_{\mCO(V)}$.

	\qed

	\begin{lem}
		\label{lem-extend-diff-op-exist}
		For any differential operator $D_0 \in \mbF^{\le n}\mCD(V)$, there is a unique differential operator $D\in \mbF^{\le n}\mCD(U)$ making the following diagram commute:
		\[
			\xymatrix{
				\mCO(V) \ar[d] \ar[r]^-{D_0} & \mCO(V) \ar[d]  \\
				\mCO(U) \ar@{.>}[r]^-D & \mCO(U).
			}
		\]
	\end{lem}

	\proof
		The uniqueness follows from Lemma \ref{lem-extend-diff-op-unique}. It remains to show the existence. We prove by induction in $n$. When $n<0$, there is nothing to prove. For $n\ge 0$ and any $m\ge 0$, by induction hypothesis, there is a unique differential operator $[D_0,f^m]^\sharp\in  \mbF^{\le n-1}\mCD(U)$ that extends $[D_0,f^m]\in \mbF^{\le n-1}\mCD(V)$. Motivated by \eqref{eqn-D-local-proof-2}, for any $g\in \mCO(V)$, we define
		\[
			D(\frac{g}{f^m}) := \frac{D_0(g)}{f^m} - \frac{[D_0,f^m]^\sharp(  \frac{g}{f^m})}{f^m}.
		\]

		We need to show the map $D$ is well-defined, i.e., $D(\frac{g}{f^m}) = D(\frac{f^lg}{f^{l+m}})$ for any $l\ge 0$. Note that we have 
		\[
			D_0(f^l g) = f^l D_0(g) + [D_0,f^l](g).
		\]
		We also have $[D_0, f^{l+m}] = [D_0,f^l] f^m + f^m [D_0, f^l]$. This implies 
		\[
			[D_0, f^{l+m}]^\sharp = [D_0,f^l]^\sharp f^m + f^m [D_0, f^l]^\sharp.
		\] 
		Combining the above two equations, a direct calculation shows $D(\frac{g}{f^m}) = D(\frac{f^lg}{f^{l+m}})$ as desired.

		It remains to show $D$ is a differential operator of order $n$. By Remark \ref{rem-check-diff}, we only need to show $[D,f^{-1}]$ and $[D,h]$, $h\in \mCO(V)$ are differential operators of order $n-1$. By \eqref{eqn-lebniz-for-commutator}, we have $0 = [D, f f^{-1}] = f [D, f^{-1}] + [D,f] f^{-1}$. Hence $[D, f^{-1}] = -f^{-1} [D,f] f^{-1}$. Therefore we only need to show $D':=[D,h]\in \mbF^{\le n-1} \mCD(U)$. Write $D'_0:=[D_0,h] \in  \mbF^{\le n-1} \mCD(V)$. A direct calculations shows $D'$ can be obtained from $D_0'$ using the same formula that defines $D$ from $D_0$, i.e.,
		\[
			D'(\frac{g}{f^m}) := \frac{D_0'(g)}{f^m} - \frac{[D_0',f^m]^\sharp(  \frac{g}{f^m})}{f^m}.
		\]
		Hence we win by induction in $n$ (again).

	\qed

	

	\begin{lem}
		\label{lem-uniform-bound-integral}
		For any differential operator $D \in \mbF^{\le n}\mCD(U)$, there exists an integer $N\ge 0$ such that $f^ND$ and $Df^N$ send $\mCO(V)$ into $\mCO(V)$.
	\end{lem}

	\proof
		We prove by induction in $n$. When $n<0$, there is nothing to prove. For $n\ge 0$,let $g_i\in \mCO(V)$, $i\in I$ be a \emph{finite} set of generators. By induction hypothesis, there exists an integer $N\ge 0$ such that $f^N[D,g_i]$ and $[D,g_i]f^N$ preserve $\mCO(V)$ for any $i\in I$. By \eqref{eqn-lebniz-for-commutator}, $f^N[D,g]$ and $[D,g]f^N$ preserve $\mCO(V)$ for any $g\in \mCO(V)$. By enlarging $N$, we can assume $f^N D(1) \in \mCO(V)$. Hence by \eqref{eqn-lebniz-for-operator}, $f^N D(g) = f^N[D,g](1)+ f^N g D(1) \in \mCO(V)$ and $D(f^{2N}g) = [D,f^N](f^N g) + f^{N} D(f^N g)\in \mCO(V)$.

	\qed

	\begin{constr}
		\label{constr-qc-diff}
		Let $\mbF^\bullet \mCD(V) \to \mbF^\bullet \mCD(U)$ be the map defined by Lemma \ref{lem-extend-diff-op-exist}. Recall $\mCD(U)$ is an $\mCO(U)$-bimodule and $\mCD(V)$ is an $\mCO(V)$-bimodule. It is easy to see the above map is compatible with the homomorphism $\mCO(V) \to \mCO(U)$. It follows that we have maps
		\begin{equation}
			\label{eqn-D-local}
			\mCO(U) \ot_{\mCO(V)} \mbF^\bullet\mCD(V) \to \mbF^\bullet\mCD(U) \gets \mbF^\bullet \mCD(V) \ot_{\mCO(V)} \mCO(U),
		\end{equation}
		such that the rightward map is left $\mCO(U)$-linear while the leftward map is right $\mCO(U)$-linear.
	\end{constr}

	\begin{lem}
		\label{lem-diff-Zar-loc}
		The maps \eqref{eqn-D-local} are isomorphisms.
	\end{lem}

	\proof
		We first show the maps are injective. Note that any element in $\mCO(U) \ot_{\mCO(V)} \mbF^\bullet\mCD(V) $ is a pure tensor $f^{-m} \ot D_0$ for some $m\ge 0$ and $D_0 \in \mbF^n\mCD(V)$. The rightward map sends it to $f^{-m} D_0\in  \mbF^n\mCD(U)$. If $f^{-m} D_0=0$, then $D_0=0$ because $\mCO(V)$ is integral. This proves the rightward map is injective. A similar argument shows the leftward map is injective.

		To prove the maps are surjective, let $D\in  \mbF^{\le n}\mCD(U)$. By Lemma \ref{lem-uniform-bound-integral}, there exists $N\ge 0$ such that $D_0:=f^N D$ and $D_0':=D f^N$ preserve $\mCO(V)$. We view $D_0$ and $D_0'$ as elements in $\mbF^{\le n}\mCD(V)$. By Lemma \ref{lem-extend-diff-op-unique}, the rightward (resp. leftward) map sends $f^{-N}\ot D_0$ (resp. $D_0'\ot f^{-N}$) to $D$.

	\qed

	\begin{constr}
		Let $X$ be a smooth $k$-scheme. By Lemma \ref{lem-diff-Zar-loc} and Remark \ref{rem-diff-disj}, for any $n$, there exists a unique sheaf $\mbF^{\le n} \mCD_X$ such that:
		\begin{itemize}
			\item 
				For any affine open subscheme $U\subset X$,
				\[
					\mbF^{\le n} \mCD_X(U) := \mbF^{\le n} \mCD(U);
				\]
			\item
				For $U\subset V\subset X$ as before, the connecting map $\mbF^{\le n} \mCD_X(V) \to \mbF^{\le n} \mCD_X(U)$ is given by the map $\mbF^{\le n}  \mCD(U)\to \mbF^{\le n}  \mCD(V)$ in Construction \ref{constr-qc-diff}.
		\end{itemize}
		We call $\mbF^{\le n} \mCD_X$ the \textbf{sheaf of differential operators of order $n$} on $X$.

		Define the \textbf{sheaf of differential operators} on $X$ to be
		\[
			\mCD_X:= \cup_n \mbF^{\le n} \mCD_X.
		\]
		Note that $\mbF^\bullet\mCD_X$ is naturally a sheaf of filtered associtive $k$-algebras such that $\gr^\bullet \mCD_X$ is commutative.
	\end{constr}

	\begin{rem}
		Unwinding the definitions, for any open subscheme $V\subset X$, $\mCD_X(V)$ can be identified with the algebra of $k$-linear morphisms $\mCO_V \to \mCO_V$ such that for any affine open subscheme $U\subset V$, the obtained map $\mCO(U)\to \mCO(U)$ is a differential operator.

	\end{rem}

	\begin{cor}
		\label{cor-DX-gr0-gr1}
		Let $X$ be a smooth $k$-scheme. We have natural isomorphisms $\mCO_X \simeq \mbF^{\le 0} \mCD_X$ and $\mCO_X\oplus \mCT_X \simeq \mbF^{\le 1} \mCD_X$.
	\end{cor}

	\begin{cor}
		Let $X$ be a smooth $k$-scheme. Then $\mCD_X$ is naturally an $\mCO_X$-bimodule, and it is quasi-coherent for both the left and right $\mCO_X$-module structures.
	\end{cor}

\section{The PBW theorem for \texorpdfstring{$\mCD_X$}{DX}}

	
	\begin{constr}
		Let $X$ be a smooth $k$-scheme. By Corollary \ref{cor-DX-gr0-gr1}, we have a homomorphism $\mCO_X \to \gr^0 \mCD_X$ and a morphism $\mCT_X \to \gr^1 \mCD_X$ between their modules. By the universal property of symmetric algebras, we obtain a graded homomorphism
		\begin{equation}
			\label{eqn-PBW-diff}
			\Sym^\bullet_{\mCO_X} (\mCT_X) \to \gr^\bullet \mCD_X.
		\end{equation}
	\end{constr}
	
	The proof of the following theorem can be treated as a blackbox\footnote{In fact, for the purpose of \emph{this} course, we can define $\mCD_X$ to be the subsheaf of $\mCH om_k(\mCO_X,\mCO_X)$ generated by the images $\mCO_X$ and $\mCT_X$ under compositions. Then the PBW theorem becomes obvious. However, it is not obvious to show this definition coincides with ours.}.

	\begin{thm}[{PBW for $\mCD_X$}]
		The above graded homomorphism is an isomorphism.
	\end{thm}

	\proof

		We can assume $X$ is affine and connected. Hence we only need to show 
		\[
			\Sym^n_{\mCO(X)} (\mCT(X)) \to \gr^n \mCD(X)
		\]
		is an isomorphism. We prove by induction in $n$. For $n< 0$, there is nothing to prove. For $n\ge 0$ and $D\in \mbF^{\le n} \mCD(X)$, consider the composition
		\[
			\mCO(X) \xrightarrow{[D,-]} \mbF^{\le n-1} \mCD(X) \surj \gr^{n-1} \mCD(X).
		\]
		By \eqref{eqn-lebniz-for-commutator}, this is a $k$-derivation of $\mCO(X)$ into $\gr^{n-1} \mCD(X)$, where the latter is viewed as an $\mCO(X)$-module via the homomorphism $\mCO(X) \to \gr^0\mCD(X)$. Moreover, this $k$-derivation only depends on the image of $D$ in $\gr^n \mCD(X)$. In other words, we have a $k$-linear map
		\[
			\gr^n \mCD(X) \to \Der_k( \mCO(X), \gr^{n-1} \mCD(X) ),\; D \mapsto [D,-].
		\]
		Note that $[gD,f]=g[D,f]$ for any $f,g\in \mCO(X)$. It follows that the above map is $\mCO(X)$-linear. By induction hypothesis and Corollary \ref{cor-der-tensor}, we have
		\[
			\Der_k( \mCO(X), \gr^{n-1} \mCD(X) ) \simeq  \mCT(X) \ot_{\mCO(X)}\gr^{n-1} \mCD(X)  \simeq \mCT(X) \ot_{\mCO(X)} \Sym_{\mCO(X)}^{n-1} (\mCT(X)).
		\]
		Composing with the multplication map
		\[
			\mCT(X) \ot_{\mCO(X)} \Sym_{\mCO(X)}^{n-1} (\mCT(X)) \to \Sym_{\mCO(X)}^{n} (\mCT(X)),
		\]
		we obtain an $\mCO(X)$-linear map
		\[
			\gr^n \mCD(X) \to   \Sym_{\mCO(X)}^{n} (\mCT(X)).
		\]
		By considering affine open subschemes, we obtain an $\mCO_X$-linear morphism 
		\[
			\gr^n \mCD_X \to \Sym_{\mCO_X}^m (\mCT_X).
		\]
		We claim it is inverse to \eqref{eqn-PBW-diff} up to multplication by $n$.

		A direct calculation shows 
		\[
			\Sym_{\mCO(X)}^{n} (\mCT(X))\to \gr^n \mCD(X)\to   \Sym_{\mCO(X)}^{n} (\mCT(X))
		\]
		is given by multplication by $n$. Indeed, for $\partial_1,\cdots ,\partial_n\in \mCT(X)$, we have
		\[
			[\partial_1\cdots \partial_n ,f] \equiv \sum_{i=1}^n \partial_i(f) \partial_1\cdots \widehat{\partial_i} \cdots \partial_n  \;\;\in \;\; \gr^{\le n-1} \mCD(X).
		\]
		Hence $\partial_1\cdots \partial_n$ is sent to the element
		\[
			\sum_{i=1}^n \partial_i \otimes \partial_1\cdots \widehat{\partial_i} \cdots \partial_n \in \mCT(X) \ot_{\mCO(X)} \Sym_{\mCO(X)}^{n-1} (\mCT(X)),
		\]
		which is sent to $n \partial_1\cdots \partial_n \in \Sym_{\mCO(X)}^{n} (\mCT(X))$ as desired.

		It remains to show
		\begin{equation}
			\label{eqn-proof-PBW-1}
			 \gr^n \mCD_X \to \Sym_{\mCO_X}^m (\mCT_X) \to \gr^n \mCD_X 
		\end{equation}
		is given by multplication by $n$. Note that the problem is Zariski local in $X$, hence we can assume $X$ admits an étale coordinate system $x_1,\cdots,x_d$. Now let $D \in \mbF^{\le n}\mCD_X$ and 
		\[
			\sum_{i=1}^d \partial_i \ot D_i \in  \mCT(X) \ot_{\mCO(X)}\gr^{n-1} \mCD(X) 
		\]
		be its image. By definition, for any function $f\in \mCO(X)$, we have
		\[
			[D,f] \equiv \sum_{i=1}^d \partial_i(f) D_i  \;\;\in \;\; \gr^{n-1} \mCD(X)
		\]
		Taking $f=x_i$, we obtain $D_i \equiv [D,x_i]$. Therefore for any $f\in \mCO(X)$,
		\begin{equation}
			\label{eqn-proof-PBW-2}
			[D,f] \equiv \sum_{i=1}^d \partial_i(f)  [D,x_i]  \;\;\in \;\; \gr^{n-1} \mCD(X)
		\end{equation}
		and the composition \eqref{eqn-proof-PBW-1} sends $D$ to $\sum_{i=1}^d \partial_i[D,x_i]$. It remains to show 
		\[
			nD \equiv \sum_{i=1}^d \partial_i[D,x_i] \;\; \in \;\; \gr^{n} \mCD(X)
		\]
		To prove this, we use induction in $n$ (again). For $n\le 0$, the claim is obvious. For $n>0$, we only need to show 
		\begin{equation}
			\label{eqn-proof-PBW-3}
			n[D,f] \equiv \sum_{i=1}^d [\partial_i[D,x_i],f] \;\; \in \;\; \gr^{n-1} \mCD(X)
		\end{equation}
		for any $f\in \mCO(X)$. 

		A direct calculation shows
		\begin{equation}
			\label{eqn-proof-PBW-4}
			[\partial_i[D,x_i],f] = \partial_i[[D,f],x_i] + \partial_i(f)[D,x_i].
		\end{equation}
		By induction hypothesis, we have
		\begin{equation}
			\label{eqn-proof-PBW-5}
			(n-1)[D,f] \equiv \sum_{i=1}^d \partial_i[[D,f],x_i] \;\; \in \;\; \gr^{n-1} \mCD(X).
		\end{equation}
		Now \eqref{eqn-proof-PBW-2}$+$\eqref{eqn-proof-PBW-5} implies \eqref{eqn-proof-PBW-3} by \eqref{eqn-proof-PBW-4}.
		
	\qed

	\begin{cor}
		Let $X$ be a affine smooth $k$-scheme, then the associative algebra $\mCD(X)$ is generated by the images of $\mCO(X)$ and $\mCT(X)$ subject to the following relations:
		\[
			f_1 \star f_2 = f_1 f_2,\; \partial_1 \star\partial_2 - \partial_2 \star\partial_1 = [\partial_1,\partial_2],\; f \star \partial = f\partial,\; \partial\star f - f\star\partial = \partial(f)
		\]
		for $f,f_1,f_2\in \mCO(X)$ and $\partial,\partial_1,\partial_2\in \mCT(X)$. Here $\star$ (temporarily) denotes the multplication in $\mCD(X)$.

	\end{cor}

	\begin{exam}
		For $X=\mBA^d$, we have $\mCD(X) = k[x_1,\cdots,x_d, \partial_1,\cdots,\partial_d]$ such that $[x_i,x_j]=[\partial_i,\partial_j] =0$ and $[\partial_i,x_j]=\delta_{i,j}$. This is known as the \textbf{Weyl algebra}.
	\end{exam}

	

\section{Definition of D-modules}

	\begin{defn}
		Let $X$ be a smooth $k$-scheme. A \textbf{left (resp. right) $\mCD_X$-module} is a sheaf $\mCM$ of $k$-vector spaces equipped with a left (resp. right) action by $\mCD_X$.

		Let $\mCD_X\hmod^l$ (resp. $\mCD_X\hmod^r$) be the category of left (resp. right) $\mCD_X$-modules, where morphisms are defined in the obvious way.
	\end{defn}

	\begin{constr}
		Let $X$ be a smooth $k$-scheme. Restricting along the homomorphism $\mCO_X \to \mCD_X$, we obtain forgetful functors
		\[
			\oblv^l: \mCD_X\hmod^l\to \mCO_X\hmod,\; \oblv^r: \mCD_X\hmod^r\to \mCO_X\hmod.
		\]
		
	\end{constr}

	\begin{defn}
		We say a left (resp. right) $\mCD_X$-module $\mCM$ is \textbf{quasi-coherent} if the underlying $\mCO_X$-module is quasi-coherent.

		When $X$ is quasi-compact, we say a quasi-coherent left (resp. right) $\mCD_X$-module $\mCM$ is \textbf{coherent} if it is locally finitely generated.

		Let $\mCD_X\hmod_{\mathsf{(q)c}}^l$ (resp. $\mCD_X\hmod_{\mathsf{(q)c}}^r$) be the category of left (resp. right) (quasi-)coherent $\mCD_X$-modules.

	\end{defn}

	\begin{rem}
		It is easy to see the above categories are abelian categories and the forgetful functors are exact.
	\end{rem}

	\begin{warn}
		In general, a coherent $\mCD_X$-module is not coherent as an $\mCO_X$-module. When there is danger of ambiguity, we use the terminologies ``$\mCD_X$-coherent'' vs ``$\mCO_X$-coherent''.
	\end{warn}

	The following result follows from the PBW theorem:

	\begin{prop}
		Restricting along $\mCT_X \to \mCD_X$ defines an equivalence
		\[
			\mCD_X\hmod^l_\qc \simeq \mCT_X\hmod_\qc.
		\]
	\end{prop}

	\begin{rem}
		In fact, we could have defined a notion of \emph{quasi-coherent right $\mCT_X$-modules} such that they form an abelian category equivalent to $\mCD_X\hmod^r_\qc$. We did not introduce this notion because people are less familiar with right modules of Lie algebras.
	\end{rem}

\section{Examples of D-modules}

	\begin{exam}
		The sheaf $\mCD_X$ itself is a left and a right $\mCD_X$-module.
	\end{exam}

	\begin{exam}
		The sheaf $\mCO_X$ is a left $\mCD_X$-module with the action given by $D\cdot f:=D(f)$.
	\end{exam}

	\begin{exe}
		This is \red{Homework 5, Problem 2}. Let $X$ be a smooth $k$-scheme of dimension $n$. Prove: there is a unique right $\mCD_X$-module structure on $\Omega_X^n$ such that for local sections $f\in \mCO(U)$, $\partial\in \mCT(U)$ and $\omega\in \Omega^n(U)$, the right action is given by
		\[
			\omega \cdot f = f\omega,\; \omega \cdot \partial = -\mCL_\partial(\omega)
		\]
	\end{exe}

	\begin{exam}
		For $X=\mBA^1=\Spec (k[x])$, we define a left D-module $\mCM_{e^x}$ whose underlying $\mCO_X$-module is isomorphic to $\mCO_X$, with a generator denoted by ``$e^x$''. The $\mCT_X$-action is determined by the formular $\partial_x \cdot e^x = e^x$. It is easy to see $\mCM_{e^x} \simeq \mCD_X/\mCD_X\cdot(\partial_x-1)$, where $\mCD_X\cdot(\partial_x-1)$ is the left ideal of $\mCD_X$ generated by the section $\partial_x-1$.
	\end{exam}

	\begin{exam}
		\label{exam-monodromy}
		For $X=\mBA^1-0=\Spec (k[x^\pm])$ and $\lambda\in k$, we define a left D-module $\mCM_{x^\lambda}$ whose underlying $\mCO_X$-module is isomorphic to $\mCO_X$, with a generator denoted by ``$x^\lambda$''. The $\mCT_X$-action is determined by the formular $\partial_x \cdot x^\lambda = \lambda x^{-1}\cdot x^{\lambda}$. It is easy to see $\mCM_{x^\lambda} \simeq \mCD_X/\mCD_X\cdot(\partial_x-\lambda x^{-1})$, where $\mCD_X\cdot(\partial_x-\lambda x^{-1})$ is the left ideal of $\mCD_X$ generated by the section $\partial_x-\lambda x^{-1}$.
	\end{exam}

	\begin{exe}
		This is \red{Homework 5, Problem 3}. In Example \ref{exam-monodromy}, prove $\mCM_{x^\lambda}$ is isomorphic to $\mCO_X$ as left $\mCD_X$-modules iff $\lambda\in \mBZ$.
	\end{exe}
	

\section{\texorpdfstring{$\mCD_X$}{DX} vs. \texorpdfstring{$U(\mfg)$}{U(g)}}
	
	\begin{constr}
		Let $X$ be a smooth $k$-scheme equipped with an action by an algebraic group $G$. Consider the Lie algebra homomorphisms
		\[
			\mfg \xrightarrow{a} \mCT(X) \to \mCD(X)
		\]
		where $\mCD(X)$ is viewed as a Lie algebra via the forgetful functor $\mathsf{Alg} \to \Lie$. By the universal property of $U(\mfg)$, we obtain a homomorphism
		\[
			U(\mfg) \xrightarrow{a}  \mCD(X).
		\]
		By construction, this homomorphism is compatible with the PBW filtrations on both sides.

		This induces a functor
		\[
			\Gamma: \mCD_X\hmod^l_\qc \to U(\mfg)\hmod, \; \mCM \mapsto \mCM(X).
		\]
		By construction, the following diagram commutes:
		\[
			\xymatrix{
				\mCD_X\hmod^l_\qc \ar[r]^-\simeq \ar[d]^-\Gamma&
				\mCT_X\hmod_\qc \ar[d]^-\Gamma \\
				U(\mfg)\hmod \ar[r]^-\simeq & \mfg\hmod.
			}
		\]
	\end{constr}

	\begin{rem}
		The localizaiton theory says for $X=G/B$, the homomorphism $U(\mfg) \xrightarrow{a}  \mCD(X)$ induces an isomorphism
		\[
			a: U(\mfg)_{\chi_0}:= U(\mfg)\ot_{Z(\mfg)} k_{\chi_0} \xrightarrow{\simeq}\mCD(X).
		\]
		and the functor $\Gamma$ induces an equivalence
		\[
			\Gamma: \mCD_X\hmod^l_\qc \xrightarrow{\simeq} U(\mfg)_{\chi_0}\hmod.
		\]
	\end{rem}

	
	




\begin{thebibliography}{Yau}

	\bibitem[B]{B} Bernstein, Joseph. Algebraic theory of D-modules, 1984, abailable at \url{https://gauss.math.yale.edu/~il282/Bernstein_D_mod.pdf}.

	\bibitem[G]{G} Gaitsgory, Dennis. Course Notes for Geometric Representation Theory, 2005, available at \url{https://people.mpim-bonn.mpg.de/gaitsgde/267y/catO.pdf}.

	\bibitem[H]{H} Humphreys, James E. Representations of Semisimple Lie Algebras in the BGG Category $\mathcal{O} $. Vol. 94. American Mathematical Soc., 2008.

	\bibitem[HTT]{HTT} Hotta, Ryoshi, and Toshiyuki Tanisaki. D-modules, perverse sheaves, and representation theory. Vol. 236. Springer Science \& Business Media, 2007.
\end{thebibliography}


\end{document} 


