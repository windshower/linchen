
%!TEX root = main.tex
\documentclass{amsart}
\textwidth=14.5cm \oddsidemargin=1cm
\evensidemargin=1cm
\usepackage{amsmath}
\usepackage{amsxtra}
\usepackage{amscd}
\usepackage{amsthm}
\usepackage{amsfonts}
\usepackage{amssymb}
\usepackage[foot]{amsaddr}
\usepackage{cite}
\usepackage{url}
\usepackage{rotating}
\usepackage{eucal}
\usepackage{tikz-cd}
\usepackage[all,2cell,color]{xy}
\UseAllTwocells
\UseCrayolaColors
\usepackage{graphicx}
\usepackage{pifont}
\usepackage{comment}
\usepackage{verbatim}
\usepackage{xcolor}
\usepackage{hyperref}
\usepackage{xparse}
\usepackage{upgreek}
\usepackage{MnSymbol}
\sloppy


%%%%%%%%%%%%%%%%%%%%Theorem%%%%%%%%%%%%%%%%%%%%
\newcounter{theorem}
\setcounter{theorem}{0}

\newtheorem{cor}[subsection]{Corollary}
\newtheorem{lem}[subsection]{Lemma}
\newtheorem{goal}[subsection]{Goal}
\newtheorem{lemdefn}[subsection]{Lemma-Definition}
\newtheorem{prop}[subsection]{Proposition}
\newtheorem{propdefn}[subsection]{Proposition-Definition}
\newtheorem{cordefn}[subsection]{Corollary-Definition}
\newtheorem{variant}[subsection]{Variant}
\newtheorem{warn}[subsection]{Warning}
\newtheorem{sugg}[subsection]{Suggestion}
\newtheorem{facts}[subsection]{Fact}
\newtheorem{ques}{Question}
\newtheorem{guess}{Guess}
\newtheorem{claim}{Claim}
\newtheorem{propconstr}[subsection]{Proposition-Construction}
\newtheorem{lemconstr}[subsection]{Lemma-Construction}
\newtheorem{ax}{Axiom}
\newtheorem{conje}[subsection]{Conjecture}
\newtheorem{mainthm}[subsection]{Main-Theorem}
\newtheorem{summ}[subsection]{Summary}
\newtheorem{thm}[subsection]{Theorem}
\newtheorem{thmdefn}[subsection]{Theorem-Definition}
\newtheorem{notn}[subsection]{Notation}
\newtheorem{convn}[subsection]{Convention}
\newtheorem{constr}[subsection]{Construction}


\theoremstyle{definition}

\newtheorem{defn}[subsection]{Definition}
\newtheorem{exam}[subsection]{Example}
\newtheorem{assum}[subsection]{Assumption}

\theoremstyle{remark}
\newtheorem{rem}[subsection]{Remark}
\newtheorem{exe}[subsection]{Exercise}


\numberwithin{equation}{section}


%%%%%%%%%%%%%%%%%%%%Commands%%%%%%%%%%%%%%%%%%%%

\newcommand{\nc}{\newcommand}
\nc\on{\operatorname}
\nc\renc{\renewcommand}


%%%%%%%%%%%%%%%%%%%%Sections%%%%%%%%%%%%%%%%%%%%

\nc\ssec{\subsection}
\nc\sssec{\subsubsection}

%%%%%%%%%%%%%%%%%%%%Environment%%%%%%%%%%%%%%%%%
\nc\blongeqn{\[ \begin{aligned}}
\nc\elongeqn{\end{aligned} \]}



%%%%%%%%%%%%%%%%%%%%Mathfont%%%%%%%%%%%%%%%%%%%%

\nc\mBA{{\mathbb A}}
\nc\mBB{{\mathbb B}}
\nc\mBC{{\mathbb C}}
\nc\mBD{{\mathbb D}}
\nc\mBE{{\mathbb E}}
\nc\mBF{{\mathbb F}}
\nc\mBG{{\mathbb G}}
\nc\mBH{{\mathbb H}}
\nc\mBI{{\mathbb I}}
\nc\mBJ{{\mathbb J}}
\nc\mBK{{\mathbb K}}
\nc\mBL{{\mathbb L}}
\nc\mBM{{\mathbb M}}
\nc\mBN{{\mathbb N}}
\nc\mBO{{\mathbb O}}
\nc\mBP{{\mathbb P}}
\nc\mBQ{{\mathbb Q}}
\nc\mBR{{\mathbb R}}
\nc\mBS{{\mathbb S}}
\nc\mBT{{\mathbb T}}
\nc\mBU{{\mathbb U}}
\nc\mBV{{\mathbb V}}
\nc\mBW{{\mathbb W}}
\nc\mBX{{\mathbb X}}
\nc\mBY{{\mathbb Y}}
\nc\mBZ{{\mathbb Z}}


\nc\mCA{{\mathcal A}}
\nc\mCB{{\mathcal B}}
\nc\mCC{{\mathcal C}}
\nc\mCD{{\mathcal D}}
\nc\mCE{{\mathcal E}}
\nc\mCF{{\mathcal F}}
\nc\mCG{{\mathcal G}}
\nc\mCH{{\mathcal H}}
\nc\mCI{{\mathcal I}}
\nc\mCJ{{\mathcal J}}
\nc\mCK{{\mathcal K}}
\nc\mCL{{\mathcal L}}
\nc\mCM{{\mathcal M}}
\nc\mCN{{\mathcal N}}
\nc\mCO{{\mathcal O}}
\nc\mCP{{\mathcal P}}
\nc\mCQ{{\mathcal Q}}
\nc\mCR{{\mathcal R}}
\nc\mCS{{\mathcal S}}
\nc\mCT{{\mathcal T}}
\nc\mCU{{\mathcal U}}
\nc\mCV{{\mathcal V}}
\nc\mCW{{\mathcal W}}
\nc\mCX{{\mathcal X}}
\nc\mCY{{\mathcal Y}}
\nc\mCZ{{\mathcal Z}}


\nc\mbA{{\mathsf A}}
\nc\mbB{{\mathsf B}}
\nc\mbC{{\mathsf C}}
\nc\mbD{{\mathsf D}}
\nc\mbE{{\mathsf E}}
\nc\mbF{{\mathsf F}}
\nc\mbG{{\mathsf G}}
\nc\mbH{{\mathsf H}}
\nc\mbI{{\mathsf I}}
\nc\mbJ{{\mathsf J}}
\nc\mbK{{\mathsf K}}
\nc\mbL{{\mathsf L}}
\nc\mbM{{\mathsf M}}
\nc\mbN{{\mathsf N}}
\nc\mbO{{\mathsf O}}
\nc\mbP{{\mathsf P}}
\nc\mbQ{{\mathsf Q}}
\nc\mbR{{\mathsf R}}
\nc\mbS{{\mathsf S}}
\nc\mbT{{\mathsf T}}
\nc\mbU{{\mathsf U}}
\nc\mbV{{\mathsf V}}
\nc\mbW{{\mathsf W}}
\nc\mbX{{\mathsf X}}
\nc\mbY{{\mathsf Y}}
\nc\mbZ{{\mathsf Z}}

\nc\mba{{\mathsf a}}
\nc\mbb{{\mathsf b}}
\nc\mbc{{\mathsf c}}
\nc\mbd{{\mathsf d}}
\nc\mbe{{\mathsf e}}
\nc\mbf{{\mathsf f}}
\nc\mbg{{\mathsf g}}
\nc\mbh{{\mathsf h}}
\nc\mbi{{\mathsf i}}
\nc\mbj{{\mathsf j}}
\nc\mbk{{\mathsf k}}
\nc\mbl{{\mathsf l}}
\nc\mbm{{\mathsf m}}
\nc\mbn{{\mathsf n}}
\nc\mbo{{\mathsf o}}
\nc\mbp{{\mathsf p}}
\nc\mbq{{\mathsf q}}
\nc\mbr{{\mathsf r}}
\nc\mbs{{\mathsf s}}
\nc\mbt{{\mathsf t}}
\nc\mbu{{\mathsf u}}
\nc\mbv{{\mathsf v}}
\nc\mbw{{\mathsf w}}
\nc\mbx{{\mathsf x}}
\nc\mby{{\mathsf y}}
\nc\mbz{{\mathsf z}}




\nc\mbfA{{\mathbf A}}
\nc\mbfB{{\mathbf B}}
\nc\mbfC{{\mathbf C}}
\nc\mbfD{{\mathbf D}}
\nc\mbfE{{\mathbf E}}
\nc\mbfF{{\mathbf F}}
\nc\mbfG{{\mathbf G}}
\nc\mbfH{{\mathbf H}}
\nc\mbfI{{\mathbf I}}
\nc\mbfJ{{\mathbf J}}
\nc\mbfK{{\mathbf K}}
\nc\mbfL{{\mathbf L}}
\nc\mbfM{{\mathbf M}}
\nc\mbfN{{\mathbf N}}
\nc\mbfO{{\mathbf O}}
\nc\mbfP{{\mathbf P}}
\nc\mbfQ{{\mathbf Q}}
\nc\mbfR{{\mathbf R}}
\nc\mbfS{{\mathbf S}}
\nc\mbfT{{\mathbf T}}
\nc\mbfU{{\mathbf U}}
\nc\mbfV{{\mathbf V}}
\nc\mbfW{{\mathbf W}}
\nc\mbfX{{\mathbf X}}
\nc\mbfY{{\mathbf Y}}
\nc\mbfZ{{\mathbf Z}}

\nc\mbfa{{\mathbf a}}
\nc\mbfb{{\mathbf b}}
\nc\mbfc{{\mathbf c}}
\nc\mbfd{{\mathbf d}}
\nc\mbfe{{\mathbf e}}
\nc\mbff{{\mathbf f}}
\nc\mbfg{{\mathbf g}}
\nc\mbfh{{\mathbf h}}
\nc\mbfi{{\mathbf i}}
\nc\mbfj{{\mathbf j}}
\nc\mbfk{{\mathbf k}}
\nc\mbfl{{\mathbf l}}
\nc\mbfm{{\mathbf m}}
\nc\mbfn{{\mathbf n}}
\nc\mbfo{{\mathbf o}}
\nc\mbfp{{\mathbf p}}
\nc\mbfq{{\mathbf q}}
\nc\mbfr{{\mathbf r}}
\nc\mbfs{{\mathbf s}}
\nc\mbft{{\mathbf t}}
\nc\mbfu{{\mathbf u}}
\nc\mbfv{{\mathbf v}}
\nc\mbfw{{\mathbf w}}
\nc\mbfx{{\mathbf x}}
\nc\mbfy{{\mathbf y}}
\nc\mbfz{{\mathbf z}}

\nc\mfa{{\mathfrak a}}
\nc\mfb{{\mathfrak b}}
\nc\mfc{{\mathfrak c}}
\nc\mfd{{\mathfrak d}}
\nc\mfe{{\mathfrak e}}
\nc\mff{{\mathfrak f}}
\nc\mfg{{\mathfrak g}}
\nc\mfh{{\mathfrak h}}
\nc\mfi{{\mathfrak i}}
\nc\mfj{{\mathfrak j}}
\nc\mfk{{\mathfrak k}}
\nc\mfl{{\mathfrak l}}
\nc\mfm{{\mathfrak m}}
\nc\mfn{{\mathfrak n}}
\nc\mfo{{\mathfrak o}}
\nc\mfp{{\mathfrak p}}
\nc\mfq{{\mathfrak q}}
\nc\mfr{{\mathfrak r}}
\nc\mfs{{\mathfrak s}}
\nc\mft{{\mathfrak t}}
\nc\mfu{{\mathfrak u}}
\nc\mfv{{\mathfrak v}}
\nc\mfw{{\mathfrak w}}
\nc\mfx{{\mathfrak x}}
\nc\mfy{{\mathfrak y}}
\nc\mfz{{\mathfrak z}}

\nc{\one}{{\mathsf{1}}}


\nc\clambda{ {\check{\lambda} }}
\nc\cmu{ {\check{\mu} }}

\nc\bDelta{\mathsf{\Delta}}
\nc\bGamma{\mathsf{\Gamma}}
\nc\bLambda{\mathsf{\Lambda}}


\nc\loccit{\emph{loc.cit.}}



%%%%%%%%%%%%%%%%%%%%Operations-limit%%%%%%%%%%%%%%%%%%%%

\NewDocumentCommand{\ot}{e{_^}}{
  \mathbin{\mathop{\otimes}\displaylimits
    \IfValueT{#1}{_{#1}}
    \IfValueT{#2}{^{#2}}
  }
}
\NewDocumentCommand{\boxt}{e{_^}}{
  \mathbin{\mathop{\boxtimes}\displaylimits
    \IfValueT{#1}{_{#1}}
    \IfValueT{#2}{^{#2}}
  }
}
\NewDocumentCommand{\mt}{e{_^}}{
  \mathbin{\mathop{\times}\displaylimits
    \IfValueT{#1}{_{#1}}
    \IfValueT{#2}{^{#2}}
  }
}
\NewDocumentCommand{\convolve}{e{_^}}{
  \mathbin{\mathop{\star}\displaylimits
    \IfValueT{#1}{_{#1}}
    \IfValueT{#2}{^{#2}}
  }
}
\NewDocumentCommand{\colim}{e{_^}}{
  \mathbin{\mathop{\operatorname{colim}}\displaylimits
    \IfValueT{#1}{_{#1}\,}
    \IfValueT{#2}{^{#2}\,}
  }
}
\NewDocumentCommand{\laxlim}{e{_^}}{
  \mathbin{\mathop{\operatorname{laxlim}}\displaylimits
    \IfValueT{#1}{_{#1}\,}
    \IfValueT{#2}{^{#2}\,}
  }
}
\NewDocumentCommand{\oplaxlim}{e{_^}}{
  \mathbin{\mathop\operatorname{oplax-lim}\displaylimits
    \IfValueT{#1}{_{#1}\,}
    \IfValueT{#2}{^{#2}\,}
  }
}


%%%%%%%%%%%%%%%%%%%%Arrows%%%%%%%%%%%%%%%%%%%%


\makeatletter
\newcommand{\laxto}{\dashedrightarrow}
\newcommand{\xrightleftarrows}[1]{\mathrel{\substack{\xrightarrow{#1} \\[-.9ex] \xleftarrow{#1}}}}
\newcommand{\adj}{\xrightleftarrows{\rule{0.5cm}{0cm}}}

\newcommand*{\da@rightarrow}{\mathchar"0\hexnumber@\symAMSa 4B }
\newcommand*{\da@leftarrow}{\mathchar"0\hexnumber@\symAMSa 4C }
\newcommand*{\xlaxto}[2][]{%
  \mathrel{%
    \mathpalette{\da@xarrow{#1}{#2}{}\da@rightarrow{\,}{}}{}%
  }%
}
\newcommand{\xlaxgets}[2][]{%
  \mathrel{%
    \mathpalette{\da@xarrow{#1}{#2}\da@leftarrow{}{}{\,}}{}%
  }%
}
\newcommand*{\da@xarrow}[7]{%
  % #1: below
  % #2: above
  % #3: arrow left
  % #4: arrow right
  % #5: space left 
  % #6: space right
  % #7: math style 
  \sbox0{$\ifx#7\scriptstyle\scriptscriptstyle\else\scriptstyle\fi#5#1#6\m@th$}%
  \sbox2{$\ifx#7\scriptstyle\scriptscriptstyle\else\scriptstyle\fi#5#2#6\m@th$}%
  \sbox4{$#7\dabar@\m@th$}%
  \dimen@=\wd0 %
  \ifdim\wd2 >\dimen@
    \dimen@=\wd2 %   
  \fi
  \count@=2 %
  \def\da@bars{\dabar@\dabar@}%
  \@whiledim\count@\wd4<\dimen@\do{%
    \advance\count@\@ne
    \expandafter\def\expandafter\da@bars\expandafter{%
      \da@bars
      \dabar@ 
    }%
  }%  
  \mathrel{#3}%
  \mathrel{%   
    \mathop{\da@bars}\limits
    \ifx\\#1\\%
    \else
      _{\copy0}%
    \fi
    \ifx\\#2\\%
    \else
      ^{\copy2}%
    \fi
  }%   
  \mathrel{#4}%
}
\makeatother

%%%%%%%%%%%%%%%%%%%%Decorations%%%%%%%%%%%%%%%%%%%%
\nc{\wt}{\widetilde}
\nc{\ol}{\overline}

\nc{\red}{\textcolor{red}}
\nc{\blue}{\textcolor{blue}}
\nc{\purple}{\textcolor{violet}}

\nc{\simorlax}{{\red\simeq/\blue\lax}}

%%%%%%%%%%%%%%%%%%%%All%%%%%%%%%%%%%%%%%%%%

\nc{\Id}{\mathsf{Id}}
\nc{\gl}{\mathfrak{gl}}
\renc{\sl}{\mathfrak{sl}}
\nc{\GL}{\mathsf{GL}}
\nc{\SL}{\mathsf{SL}}
\nc{\PGL}{\mathsf{PGL}}
\nc{\hmod}{\mathsf{-mod}}
\nc{\Vect}{\mathsf{Vect}}
\nc{\tr}{\mathsf{tr}}
\nc{\Kil}{\mathsf{Kil}}
\nc{\ad}{{\mathsf{ad}}}
\nc{\Ad}{\mathsf{Ad}}
\nc{\oblv}{\mathsf{oblv}}
\nc{\gr}{\mathsf{gr}}
\nc{\Sym}{\mathsf{Sym}}
\nc{\QCoh}{\mathsf{QCoh}}
\nc{\ind}{\mathsf{ind}}
\nc{\Spec}{\mathsf{Spec}}
\nc{\Hom}{\mathsf{Hom}}
\nc{\Ext}{\mathsf{Ext}}
\nc{\Grp}{\mathsf{Grp}}
\nc{\pt}{\mathsf{pt}}
\nc{\Lie}{\mathsf{Lie}}
\nc{\CAlg}{\mathsf{CAlg}}
\nc{\Der}{\mathsf{Der}}
\nc{\Rep}{\mathsf{Rep}}
\renc{\sc}{{\mathsf{sc}}}
\nc{\Fl}{\mathsf{Fl}}
\nc{\Fun}{\mathsf{Fun}}
\nc{\ev}{\mathsf{ev}}
\nc{\surj}{\twoheadrightarrow}
\nc{\inj}{\hookrightarrow}
\nc{\HC}{\mathsf{HC}}
\nc{\cl}{\mathsf{cl}}
\renc{\Im}{\mathsf{Im}}
\renc{\ker}{\mathsf{ker}}
\nc{\coker}{\mathsf{coker}}
\nc{\Tor}{\mathsf{Tor}}
\nc{\op}{\mathsf{op}}
\nc{\length}{\mathsf{length}}
\nc{\fd}{{\mathsf{fd}}}
\nc{\weight}{\mathsf{wt}}
\nc{\semis}{{\mathsf{ss}}}
\nc{\qc}{{\mathsf{qc}}}
\nc{\pr}{\mathsf{pr}}
\nc{\act}{\mathsf{act}}
\nc{\dR}{{\mathsf{dR}}}
\nc{\hol}{{\mathsf{hol}}}
\nc{\Pic}{{\mathsf{Pic}}}
\nc{\Loc}{\mathsf{Loc}}
\nc{\IC}{\mathsf{IC}}

\begin{document}


\title{Lecture 11}

\date{May 6, 2024}

\maketitle

	Last time, for a smooth $k$-scheme $X$, we introduced the sheaf of differential operators $\mCD_X$ on $X$ and defined quasi-coherent $\mCD_X$-modules, known as \emph{$\mCD$-module on $X$}. In this lecture, we introduce operations on $\mCD$-modules. We will mainly focus on the formal aspect of this thoery, known as the \emph{six functors formalism for $\mCD$-modules}. For more details, see \cite{B} and \cite{HTT}.

\section{Conventions on derived categories}

	For the purpose of this course, we do not need the full power of the derived categories of $\mCD$-modules. However, these categories are useful in other topics of geometric representation theory, hence I choose to include the results about them in this lecture. 

	When talking about derived categories of $\mCD$-modules, we always assume $X$ is quasi-projective. In this case, the abelian category $\mCD_X\hmod_\qc^{l/r}$ has enough injective and locally projective objects, and any object admits a resolution by locally projective $\mCD_X$-modules with length $\le 2d_X$, where $d_X=\dim(X)$ is the dimension (function) of $X$. The latter implies $\Ext^i(-,-)\simeq 0$ for $i>2d_X$.

	We have the following triangulated categoies equipped with natural t-strcutures:
	\begin{itemize}
		\item 
			$D(\mCD_X\hmod_\qc^{l/r})$, the derived category of quasi-coherent $\mCD_X$-modules. This can be identified with the full subcategory of the derived category $D(\mCD_X\hmod^{l/r})$ of \emph{all} $\mCD_X$-modules containing those complices whose cohomologies are quasi-coherent.
		\item
			 $D^b(\mCD_X\hmod_\mbc^{l/r})$, the bounded derived category of coherent $\mCD_X$-modules. This can be identified with the full subcategory of the bounded derived category $D^b(\mCD_X\hmod^{l/r})$ of \emph{all} $\mCD_X$-modules containing those complices whose cohomologies are coherent.
	\end{itemize}

	When talking about functors between derived categories, even if such functors are left/right derived functors, we drop the decorations ``L/R'' from the notations. For example, $-\ot - $ in derived categories would mean $-\ot^L -$ in classical literatures. We choose to do so because we will enconter functors that are not derived functors.

	\begin{rem}
		The (essential) image of the fully faithful functor 
		\[
			D^b(\mCD_X\hmod_\mbc^{l/r}) \to D(\mCD_X\hmod_\qc^{l/r})
		\]
		contains exactly the \emph{compact} objects in the target, i.e., those objects $\mCM$ such that $\Hom(\mCM,-)$ commutes with filtered (homotopy) colimits.


	\end{rem}

\section{Forget and induce}

	\begin{constr}
		We have adjoint functors
		\[
			 \ind^{l/r}: \mCO_X\hmod_\qc \adj \mCD_X \hmod_\qc^{l/r} : \oblv^{l/r} 
 		\]
 		such that
 		\[
 			\ind^l( \mCF ) := \mCD_X\ot_{\mCO_X} \mCF,\; \ind^r(\mCD):= \mCF \ot_{\mCO_X} \mCD_X.
 		\]
 		Both functors are exact.
	\end{constr}

\section{Tensor and Hom}
	
	\begin{constr}
		\label{constr-tensor-Hom}
		Let $\mCM, \mCN \in \mCD_X\hmod_\qc^l$ and  $\mCM', \mCN' \in \mCD_X\hmod_\qc^r$. Then there are natural objects defined using the signed Lebniz rules: 
		\begin{itemize}
			\item 
				$\mCM \ot_{\mCO_X} \mCN \in \mCD_X\hmod_\qc^l$ defined by $\partial\cdot (m\ot n) = (\partial\cdot m) \ot n + m \ot (\partial\cdot n)$;
			\item 
				$\mCM' \ot_{\mCO_X} \mCN \in \mCD_X\hmod_\qc^r$ defined by $ (m'\ot n)\cdot \partial = (m' \cdot \partial)\ot n  - m'\ot(\partial \cdot n) $;
			\item
				$\mCH om_{\mCO_X}(\mCM,\mCN) \in \mCD_X\hmod_\qc^l$ defined by $(\partial\cdot \phi)(m) = \partial \cdot \phi(m) - \phi( \partial \cdot m )$;
			\item
				$\mCH om_{\mCO_X}(\mCM',\mCN') \in \mCD_X\hmod_\qc^l$ defined by $(\partial\cdot \phi)(m') = -\phi(m')\cdot \partial + \phi(  m \cdot \partial )$;
			\item
				$\mCH om_{\mCO_X}(\mCM,\mCN') \in \mCD_X\hmod_\qc^r$ defined by $(\phi\cdot \partial)(m) = \phi(m)\cdot \partial + \phi(  m \cdot \partial )$.
		\end{itemize}
	\end{constr}

	\begin{rem}
		One way to memorize the above rules for left vs. right is using the known objects $\mCO_X \in \mCD_X\hmod_\qc^l$ and $\omega_X:=\Omega_X^n \in \mCD_X\hmod_\qc^r$. For example, $\mCH om_{\mCO_X}(\omega_X,\omega_X)\simeq \mCO_X$ has a left $\mCD$-module structure, while $\mCH om_{\mCO_X}(\omega_X,\mCO_X)\simeq \omega_X^{-1}$ in general has no $\mCD$-module structures.
	\end{rem}

	\begin{rem}
		One way to memorize the signed Lebniz rules: (i) put a minus sign when acting on a section of the source object in $\mCH om_{\mCO_X}(-,-)$; (ii) put a minus sign when moving $\partial$ from the one side of $\cdot$ to the other side.
	\end{rem}

	\begin{rem}
		The tensor operations make $\mCD_X\hmod_\qc^l$ a symmetric monoidal category such that the forgetful functor $\mCD_X\hmod_\qc^l \to \mCO_X\hmod_\qc^l$ is naturally symmetric monoidal. The category $\mCD_X\hmod_\qc^r$ is a module category of it.
	\end{rem}

	The following result follows by unwinding the definitions:

	\begin{lem}
		Let $\mCM\in \mCD_X\hmod_\qc^l$ and  $\mCM' \in \mCD_X\hmod_\qc^r$. We have adjoint functors
		\begin{eqnarray*}
			\mCM \ot_{\mCO_X} - : \mCD_X\hmod_\qc^{l/r} &\adj& \mCD_X\hmod_\qc^{l/r} : \mCH om_{\mCO_X}(\mCM,-) \\
			\mCM' \ot_{\mCO_X} - : \mCD_X\hmod_\qc^{l} &\adj& \mCD_X\hmod_\qc^{r} : \mCH om_{\mCO_X}(\mCM',-)
		\end{eqnarray*}
		compatible with the similar adjunction between $\mCO_X$-modules.
	\end{lem}

	Recall $\omega_X$ is a line bundle. It follows formally that:

	\begin{cor}
		The following functors are inverse to each other:
		\[
			\omega_X \ot_{\mCO_X} - : \mCD_X\hmod_\qc^{l} \adj \mCD_X\hmod_\qc^{r} : \mCH om_{\mCO_X}(\omega_X,-).
		\]
	\end{cor}

	\begin{rem}
		In particular, for any right $\mCD$-module $\mCN'$, we obtain a left $\mCD$-module structure on $\omega_X^{-1} \ot \mCN'$.
	\end{rem}


	\begin{rem}
		We also have similar results for the derived category of $\mCD$-modules and the corresponding derived functors. The left derived functor $-\ot_{\mCO_X}-$ has cohomological amplitude $[-d_X,0]$ while the right derived functor $\mCH om_{\mCO_X}(-,-)$ has cohomological amplitude $[0,d_X]$.
	\end{rem}


	\begin{rem}
		When identifying the \emph{derived} categories of left and right $\mCD$-modules, it is more convenient to use the complex $\omega_X[d_X]$, which is also known as the \emph{dualizing complex}\footnote{This notion makes sense even when $X$ is singular.} on $X$. In other words, we use the equivalences
		\[
			\omega_X[d_X] \ot_{\mCO_X} - : D(\mCD_X\hmod_\qc^{l}) \adj D(\mCD_X\hmod_\qc^{r}) : \mCH om_{\mCO_X}(\omega_X[d_X],-).
		\]		
	\end{rem}


	\begin{constr}
		We define a symmetric monoidal structure $-\ot^! - $ on $D(\mCD_X\hmod_\qc^{r})$ by translating the symmetric monoidal structure $-\ot_{\mCO_X} - $ of $D(\mCD_X\hmod_\qc^{l})$ via the above equivalences. In other words, as $\mCO_X$-modules, we have
		\[
			\mCM' \ot^! \mCN' \simeq \mCM' \ot_{\mCO_X} \mCN' \ot_{\mCO_X} \omega_X^{-1}[-d_X].
		\]

	\end{constr}

\section{External tensor}
	\begin{constr}
		\label{constr-external}
		Let $\mCM_i \in \mCD_{X_i}\hmod_\qc^{l/r}$. Then there are natural objects
		\[
			\mCM_1 \boxt \mCM_2:=\mCM_1 \ot_{k} \mCM_2 \in \mCD_{X_1\mt X_2}\hmod_\qc^{l/r}
		\]
		induced by the isomorphism $\mCD_{X_1}\ot_k \mCD_{X_2} \simeq \mCD_{X_1\mt X_2}$. This is called the \textbf{external tensor product} of $\mCD$-modules.
		
	\end{constr}
	

\section{Pullbacks}

	Let $\phi: Y\to X$ be a map between smooth $k$-schemes.
	
	\begin{constr}
		We will construct a commutative diagram
		\[
			\xymatrix{
				\mCD_X\hmod^l_\qc \ar@{.>}[r]^-{\phi^*} \ar[d]^-{\oblv^l} &
				\mCD_Y\hmod^l_\qc \ar[d]^-{\oblv^l} \\
				\mCO_X\hmod_\qc \ar[r]^-{\phi^*} & \mCO_Y\hmod_\qc.
			}
		\]
		The functor 
		\[
			\phi^*: \mCD_X\hmod^l_\qc \to \mCD_Y\hmod^l_\qc
		\]
		is called the \textbf{($*$-)pullback of left $\mCD$-modules}.

		The construction is as follows. For $\mCM\in \mCO_X\hmod_\qc$, recall $\phi^*(\mCM):= \mCO_Y \ot_{\phi^{-1}\mCO_X} \phi^{-1}\mCM$. Suppose $\mCM$ is equipped with a left $\mCD_X$-module structure, then there is a left $\mCD_Y$-module structure on $\phi^*(\mCM)$ defined by the Lebniz rule:
		\[
			\partial \cdot ( f\ot  s) :=  \partial(f) \ot s + f \overline{\partial} \cdot s,
		\]
		where
		\begin{itemize}
			\item 
				$\partial$ is a local section of $\mCT_Y$ and $\overline{\partial}$ is the image of it under the map $\mCT_Y \to \phi^*\mCT_X = \mCO_Y \ot_{\phi^{-1}\mCO_X} \phi^{-1}\mCT_X  $;
			\item
				$f$ is a local section of $\mCO_Y$;
			\item
				$s$ is a local section of $\phi^{-1}\mCM$, and $\overline{\partial}\cdot s$ is defined using the action of $\phi^{-1}\mCT_X$ on $\phi^{-1}\mCM$.
		\end{itemize}
	\end{constr}

	\begin{rem}
		One can show the pullback of left $\mCD$-modules are compatible with composition of maps between smooth $k$-schemes.
	\end{rem}

	\begin{exam}
		The pullback of the object $\mCO_X\in \mCD_X\hmod^l_\qc$ is $\mCO_Y\in \mCD_Y\hmod^l_\qc$.
	\end{exam}

	\begin{constr}
		We write:
		\[
			\mCD_{Y\to X}:= \phi^*\mCD_X \simeq \mCO_Y \ot_{\phi^{-1}\mCO_X} \phi^{-1}\mCD_X  
		\]
		and call it the \textbf{transfer module}.

		The above construction gives a left $\mCD_Y$-module structure on $\mCD_{Y\to X}$. On the other hand, there is an obvious \emph{right} $\phi^{-1}\mCD_X$-module structure on $\mCD_{Y\to X}$. One can show there two actions commute. In other words, $\mCD_{Y\to X}$ is a $(\mCD_Y, \phi^{-1}\mCD_X)$-bimodule.

		Note that for $\mCM \in  \mCD_X\hmod^l_\qc$, we have
		\[
			\phi^* \mCM \simeq \mCD_{Y\to X}\ot_{ \phi^{-1}\mCD_X } \phi^{-1}\mCM.
		\]
	\end{constr}

	\begin{constr}
		Note that the functor $\phi^*: \mCD_X\hmod^l_\qc\to \mCD_Y\hmod^l_\qc$ is right exact. We abuse notation and let
		\[
			\phi^*: D(\mCD_X\hmod^l_\qc) \to D(\mCD_Y\hmod^l_\qc)
		\]
		be the left derived functor of it. Note that it is compatible with the left derived functor $\phi^*: D(\mCO_X\hmod_\qc) \to D(\mCO_Y\hmod_\qc)$ and the forgetful functors.
	\end{constr}

	\begin{rem}
		We have:
		\begin{itemize}
			\item If $\phi$ is flat, then $\phi^*$ is t-exact, i.e., preserves the heart.
			\item If $\phi$ is a closed embedding (which is automatically regular), then $\phi^*$ has cohomological amplitude $[-d_X+d_Y,0]$.
		\end{itemize}
	\end{rem}

	\begin{exam}
		Let $\phi$ be a closed embedding. Since $X$ and $Y$ are smooth, $\phi$ is a regular immersion. For any closed point $p\in Y$, we can find an étale coordinate system $x_1,\cdots,x_m$ of $X$ near $p$ such that $Y$ is locally cut out by the ideal $(x_{n+1},\cdots,x_m)$ ($m=\dim(\mCO_{X,y})$ and $n=\dim(\mCO_{Y,y})$). Let $y_1,\cdots,y_n$ be the restriction of $x_1,\cdots,x_n$ on $Y$. They form an étale coordinate system of $Y$ near $p$. Then near the point $p\in Y$, we have
		\[
			\mCD_{Y\to X} \simeq \mCD_Y\ot_k k[ \partial_{n+1},\cdots,\partial_m ]
		\]
		as left $\mCD_Y$-modules. In particular, $\mCD_{Y\to X}$ is a locally free left $\mCD_Y$-module.

		
	\end{exam}

	\begin{constr}
		Let $\phi^!$ be the unique functor that makes the following diagram commute
		\[
			\xymatrix{
				D(\mCD_X\hmod^l_\qc) \ar[r]^-{\phi^*} \ar[d]_-{\omega_X[d_X]}^-\simeq &
				D(\mCD_Y\hmod^l_\qc) \ar[d]^-{\omega_Y[d_Y]}_-\simeq \\
				D(\mCD_X\hmod^r_\qc) \ar@{.>}[r]^-{\phi^!} &
				D(\mCD_Y\hmod^r_\qc),
			}
		\]
	 	The obtained functor
		\[
			\phi^!: D(\mCD_X\hmod^r_\qc) \to D(\mCD_Y\hmod^r_\qc)
		\]
		is called the \textbf{!-pullback of (complices of) right $\mCD$-modules}. It has cohomological amplitude $[-d_Y, d_X-d_Y]$ and in general is not a derived functor.
	\end{constr}

	\begin{rem}
		We have:
		\begin{itemize}
			\item If $\phi$ is flat, then $\phi^!$ is t-exact up to a shift.
			\item If $\phi$ is a closed embedding, then $\phi^!$ has cohomological amplitude $[0,d_X-d_Y]$.
		\end{itemize}
	\end{rem}

	\begin{exam}
		By definition, $\phi^!(\omega_X[d_X])\simeq \omega_Y[d_Y]$.
	\end{exam}

	\begin{exam}
		If $j:U \to X$ is an open embedding, then $j^!$ is t-exact and the corresponding functor $\mCD_X\hmod^r_\qc\to \mCD_U\hmod^r_\qc$ is the restriction functor. Indeed, this follows from $\omega_X|_{U} \simeq \omega_U$. In this case, we write $j^!=j^*$.
	\end{exam}

	\begin{warn}
		In general, for a right $\mCD_X$-module $\mCM$, there is \emph{no} $\mCD$-module structure on its $\mCO$-module pullback $\phi^*\mCM$.
	\end{warn}

	\begin{facts}
		If $\phi:Y \to X$ is a closed embedding, then $\phi^!$ is equivalent to the right derived functor of a functor
		\[
			\phi^!: \mCD_X\hmod^r_\qc \to \mCD_Y\hmod^r_\qc.
		\]
	\end{facts}

	\begin{constr}
		Let $\phi:Y \to X$ be a closed embedding. The functor $\phi^!: \mCD_X\hmod^r_\qc \to \mCD_Y\hmod^r_\qc$ can be described as follows. 

		Recall we have adjoint functors
		\[
			\phi_*: \mCO_Y\hmod_\qc\adj \mCO_X\hmod_\qc : \phi^! 
		\]
		where for a quasi-coherent $\mCO_X$-module $\mCM$ and any open subset $U\subset X$, a section $m$ of $\phi^!(\mCM)$ on $U\cap Y$ corresponds to a section $\wt{m}$ of $\mCM$ on $U$ annihilated by the ideal $\mCI_Y:=\ker(\mCO_X \to \phi_*\mCO_Y)$. Suppose $\mCM$ is equipped with a right $\mCD_X$-module structure. For any local section $\partial$ of $\mCT_Y$, we can extend it to a local section $\wt{\partial}$ of $\mCT_X$. Now for a local section $m$ of $\phi^!(\mCM)$, we define $m\cdot \partial$ such that
		\[
			\wt{m\cdot \partial} = \wt{m} \cdot \wt{\partial}.
		\]
		One can show the local section $m\cdot \partial$ is well-defined and does not depend on the choice of $\wt\partial$. Moreover, this defines a right $\mCD_Y$-module structure on $\phi^!(\mCM)$. $\phi^!(\mCM)$.

	\end{constr}

	\begin{rem}
		For any map $\phi:Y\to X$ between finite type $k$-schemes, one can define a functor
		\[
			\phi^!: D(\mCO_X\hmod_\qc) \to D(\mCO_Y\hmod_\qc)
		\]
		as follows.

		If $\phi$ is an open embedding, take $\phi^!:=\phi^*$. If $\phi$ is proper, take $\phi^!$ to be the right adjoint of (the right derived functor) $\phi_*$. For the general case, choose a Nagata compactification $Y\xrightarrow{j}\overline{Y} \xrightarrow{\overline{\phi}} X$ such that $j$ is an open embedding and $\overline{\phi}$ is proper, and take $\phi^!:= j^! \circ \overline{\phi}^!$. One can show the functor $\phi^!$ does not depend on the choice of the compactification, and these functors are compatible with compositions of maps. In fact, the construction $\phi \mapsto \phi^!$ can be uniquely characterized by these properties (if stated properly).

		When $X$ and $Y$ are smooth, the $!$-pullback functors of $\mCO$-modules and right $\mCD$-modules are compatible via the forgeful functors. In other words, we have a commutative diagram
		\begin{equation}
			\label{eqn-pull-vs-oblv}
			\xymatrix{
				D(\mCO_Y\hmod_\qc) &
				D(\mCO_X\hmod_\qc)   \ar[l]^-{\phi^!} \\
				D(\mCD_Y\hmod^r_\qc)   \ar[u]_-{\oblv^r} &
				D(\mCD_X\hmod^r_\qc) \ar[l]^-{\phi^!} \ar[u]_-{\oblv^r}
			}
		\end{equation}
	\end{rem}

	\begin{facts}
		In the (derived) setting of Construction \ref{constr-tensor-Hom}, we have
		\begin{eqnarray*}
			\phi^*( \mCM\ot_{\mCO_X} \mCN ) &\simeq& \phi^*(\mCM) \ot_{\mCO_Y} \phi^*(\mCN) ,\\
			\phi^!( \mCM'\ot_{\mCO_X} \mCN ) &\simeq& \phi^!(\mCM') \ot_{\mCO_Y} \phi^*(\mCN) ,\\
			\phi^* \mCH om_{\mCO_X}(\mCM,\mCN ) &\simeq& \mCH om_{\mCO_Y}(\phi^*\mCM,\phi^*\mCN ), \\
			\phi^* \mCH om_{\mCO_X}(\mCM',\mCN' ) &\simeq& \mCH om_{\mCO_Y}(\phi^!\mCM',\phi^!\mCN' ),\\
			\phi^! \mCH om_{\mCO_X}(\mCM,\mCN' ) &\simeq& \mCH om_{\mCO_Y}(\phi^*\mCM,\phi^!\mCN' ).
		\end{eqnarray*}
	\end{facts}

	\begin{facts}
		For $\pr_i: X_1\mt X_2 \to X_i$, we have
		\begin{eqnarray*}
			\mCM_1 \boxt \mCM_2 &\simeq& \pr_1^*(\mCM_1) \ot_{\mCO_{X_1\mt X_2}} \pr_2^*(\mCM_2) \\
			\mCM_1' \boxt \mCM_2' &\simeq& \pr_1^!(\mCM_1) \ot^! \pr_2^!(\mCM_2).
		\end{eqnarray*}

	\end{facts}

\section{Pushforwards}

	

	\begin{constr}
		Let $\phi: Y\to X$ be a map between smooth $k$-schemes. Recall the transfer module
		\[
			\mCD_{Y\to X}:= \phi^*\mCD_X \simeq \mCO_Y \ot_{\phi^{-1}\mCO_X} \phi^{-1}\mCD_X  
		\]
		is a bimodule for $(\mCD_Y,\phi^{-1}\mCD_X)$. We define a functor
		\[
			\phi_{*,\dR}: D( \mCD_Y\hmod_\qc^r ) \to  D( \mCD_X\hmod_\qc^r ),\; \mCN \mapsto \phi_*(\mCN\ot_{\mCD_Y} \mCD_{Y\to X}),
		\]
		where
		\begin{itemize}
			\item 
				The (left derived) tensor product functor $-\ot_{\mCD_Y} \mCD_{Y\to X}$ sends a complex of right $\mCD_Y$-modules to a complex of right $\phi^{-1}\mCD_X$-modules. 
			\item
				The (right derived) functor $\phi_*$ sends a complex of right $\phi^{-1}\mCD_X$-modules to a complex of right $\mCD_X$-modules via the homomorphism $\mCD_X \to \phi_*(\phi^{-1}\mCD_X)$.
		\end{itemize}
		We call $\phi_{*,\dR}$ the \textbf{direct image functor}, or \textbf{de Rham pushforward functor}, of (complices) of right $\mCD$-modules. 
	\end{constr}

	\begin{rem}
		One can show the direct image functors of right $\mCD$-modules are compatible with composition of maps between smooth $k$-schemes.

	\end{rem}

	\begin{rem}
		The functor $\phi_{*,\dR}$ is called the \emph{de Rham} pushforward functor because for $\pi: X\to \pt$, $\pi_{*,\dR}(\omega_X[-d_X])$ can be identified with the de Rham complex of $X$. For this reason, we also write 
		\[
			\Gamma_{\dR}(X,-):= \pi_{*,\dR}(-).
		\]

		You are strongly encouraged to look at its proof in \cite[Sect. 5.17]{G}.
	\end{rem}

	\begin{rem}
		Some authors use the notation $\phi_\star$ for $\phi_{*,\dR}$.
	\end{rem}

	\begin{rem}
		The cohomological amplitude of $\phi_{*,\dR}$ is $[-d_Y,d_Y]$. Better estimation exist in the following cases:
		\begin{itemize}
			\item If $\phi$ is affine, then the bounds can be $[-d_Y,0]$.
			\item If $\phi$ is smooth, then the bounds can be $[-d_Y+d_X,d_Y]$.
			\item If $\phi$ is a closed embedding, then the functor is t-exact.
		\end{itemize}
	\end{rem}

	\begin{warn}
		One can define a functor between the abelian categories using the same formula. However, that functor would not be $\mCH^0(\phi_{*,\dR})$ and is of less interests.
	\end{warn}

	\begin{exam}
		If $j:U\to X$ is an open embedding, then $\mCD_{U\to X} \simeq j^*\mCD_X \simeq \mCD_U$. It follows that $j_{*,\dR}\mCM \simeq j_*\mCM$. In other words, there is a right $\mCD_X$-module structure on the $\mCO$-module direct image of $\mCM$. In this case, we write $j_{*,\dR}=j_*$.
	\end{exam}

	\begin{exe}
		This is \red{Homework 5, Problem 4}. Let $x:\pt \to X$ be a closed point of $X$. We write $\delta_x:= x_{*,\dR}(k)$. Prove:
		\begin{itemize}
			\item[(1)]
				$\delta_x\simeq k_x \ot_{\mCO_X}\mCD_X$ as a right $\mCD_X$-module.
			\item[(2)]
				$\delta_x$ is set-theoretically supported on at $x$, i.e., for the complement open $U:=X-x$, we have $\delta_x|_U = 0$.
			\item[(3)]
				There exists a unique section $\mathsf{Dirac}_x$ of $\delta_x$ such that $\mathsf{Dirac}_x\cdot f = f(x)\mathsf{Dirac}_x$ for any local section $f$ of $\mCO_X$ defined near $x$.
			\item[(4)]
				$\delta_x$ is generated by $\mathsf{Dirac}_x$ as a right $\mCD_X$-module.
		\end{itemize}
	\end{exe}

	\begin{rem}
		The section $\mathsf{Dirac}_x$ should be viewed as the incarnation of the Dirac function in the theory of $\mCD$-modules.
	\end{rem}

	

	\begin{lem}
		The following diagram commutes:
		\begin{equation}
			\label{eqn-push-vs-ind}
			\xymatrix{
				D(\mCO_Y\hmod_\qc) \ar[r]^-{\phi_*} \ar[d]_-{\ind^r} &
				D(\mCO_X\hmod_\qc) \ar[d]_-{\ind^r} \\
				D(\mCD_Y\hmod^r_\qc) \ar[r]^-{\phi_{*,\dR}}  &
				D(\mCD_X\hmod^r_\qc)
			}
		\end{equation}
	\end{lem}

	\proof[Sketch]
		For $\mCF\in D(\mCO_Y\hmod_\qc)$, we have
		\[
			\phi_{*,\dR}\circ \ind^r(\mCF) \simeq \phi_*(\mCF \ot_{\mCO_Y} \mCD_Y \ot_{\mCD_Y} \mCD_{Y\to X} ) \simeq \phi_*(\mCF \ot_{\mCO_Y} \phi^*\mCD_X ) \simeq \phi_*\mCF \ot_{\mCO_X} \mCD_X \simeq \ind^r\circ \phi_*(\mCF)
		\]
		where the second last isomorphism is the (derived) projection formula.

	\qed

	We state the following results without proof.

	\begin{prop}
		If $\phi:Y \to X$ is proper, then we have adjoint functors
		\[
			\phi_{*,\dR}: D( \mCD_Y\hmod_\qc^r ) \adj  D( \mCD_X\hmod_\qc^r ): \phi^!.
		\]
	\end{prop}

	\begin{rem}
		If $\phi:Y \to X$ is proper, then the square (\ref{eqn-pull-vs-oblv}) can be obtained from (\ref{eqn-push-vs-ind}) by passing to right adjoints.
	\end{rem}

	\begin{prop}
		If $\phi:Y \to X$ is smooth, then we have adjoint functors
		\[
			\phi^![-2 d_Y+2d_X]: D( \mCD_Y\hmod_\qc^r ) \adj  D( \mCD_X\hmod_\qc^r ): \phi_{*,\dR}.
		\]
	\end{prop}

	\begin{exam}
		\label{exam-push-open-embedding}
		If $j: U\to X$ is a closed embedding, then $j^!\simeq j^*$ is left adjoint to $j_{*,\dR}=j_*$. Note that the right adjoint functor is fully faithful.
	\end{exam}

	

	

	\begin{constr}
		As in the case of pullback functors, we can define the \textbf{direct image functor} of left $\mCD$-modules:
		\[
			\xymatrix{
				D(\mCD_Y\hmod^l_\qc) \ar@{.>}[r]^-{\phi_{*,\dR}} \ar[d]_-{\omega_Y[d_Y]}^-\simeq &
				D(\mCD_X\hmod^l_\qc) \ar[d]^-{\omega_X[d_X]}_-\simeq \\
				D(\mCD_Y\hmod^r_\qc) \ar[r]^-{\phi_{*,\dR}} &
				D(\mCD_X\hmod^r_\qc).
			}
		\]
	\end{constr}



\section{Kashiwara's lemma}

	If $\phi:Y\to X$ is a closed embedding, then the tensor product functor $-\ot_{\mCD_Y} \mCD_{Y\to X}$ is t-exact because $\mCD_{Y\to X}$ is locally free as a $\mCD_Y$-module. On the other hand, the functor $\phi_*$ is also t-exact because $\phi$ is affine. Therefore the functor $\phi_{*,\dR} $ is t-exact. 

	\begin{thm}[Kashiwara's lemma]
		Let $\phi:Y \to X$ be a closed embedding between smooth $k$-schemes, then the exact functor
		\[
			\phi_{*,\dR}:  \mCD_Y\hmod_\qc^r  \to  \mCD_X\hmod_\qc^r
		\]
		is fully faithful and its essential image contains exactly right $\mCD_X$-modules that are set-theoretically supported on $Y$.
	\end{thm}


	

	\begin{rem}
		Using Kashiwara's lemma, we can define $\mCD_Y\hmod_\qc^r$ even for finite type singular $k$-scheme $Y$. Namely, if $Y$ is affine, we can embed $Y$ into a smooth ambidient $k$-scheme $X$ and define a right $\mCD$-module on $Y$ to be a right $\mCD$-module on $X$ that is set-theoretically supported on the image of $Y$. One can show the obtained abelian category does not depend on the choice of the embedding. When $Y$ is not affine, we can define the category by gluing.

		Moreover, all the previous constructions about right $\mCD$-modules can be generalized to the singular case.

		A more canonical construction of $\mCD_Y\hmod_\qc^r$ or even $\mCD_Y\hmod_\qc^l$ for singular $k$-schemes is to use the theory of (Grothendieck's) crystals.
	\end{rem}

	Another application of Kashiwara's lemma is the following result. See \cite[Sect. 5.12]{G} for a proof.

	\begin{cor}
		Let $X$ be a smooth $k$-scheme, then any $\mCO_X$-coherent $\mCD_X$-module is locally free as an $\mCO_X$-module.
	\end{cor}

\section{Base-change isomorphism and projection formula}
	\begin{facts}
		Let
		\[
			\xymatrix{
				Y' \ar[r]^-{\phi'} \ar[d]^-f & X' \ar[d]^-g \\
				Y \ar[r]^-\phi & X
			}
		\]
		be a Cartesian square of finite type $k$-schemes. Then we have equivalences
		\[
			g^!\circ \phi_{*,\dR} \simeq \phi'_{*,\dR}\circ f^!
		\]
		between functors $D( \mCD_Y\hmod_\qc^r ) \to D( \mCD_{X'}\hmod_\qc^r ). $
	\end{facts}

	\begin{facts}
		Let $\phi: Y \to X$ be any morphism between finite type $k$-schemes. Then we have
		\[
			\phi_{*,\dR}( - \ot^! \phi^!(\bullet) ) \simeq \phi_{*,\dR}( -) \ot^! \bullet.
		\]
	\end{facts}

	\begin{exe}
		This is \red{Homework 5, Problem 5}. Let $x:\pt \to X$ be a closed point of $X$. Prove\footnote{A formal proof exists, but you are encouraged to do some direct calculations to see $\mCH^i(\delta_x\ot_{\mCO_X} \delta_x)=0$ unless $i=-d_X$ and  $\mCH^{-d}(\delta_x\ot_{\mCO_X} \delta_x)=\delta_x$.} $\delta_x \ot^! \delta_x \simeq \delta_x$. 

	\end{exe}

	\begin{cor}
		Let $U \xrightarrow{j} X \xleftarrow{i} Y$ be finite type $k$-schemes such that $i$ is a closed embedding and $j$ is its complementary open embedding. Then for any $\mCM\in D( \mCD_X\hmod_\qc^r )$, we have a distinguished triangle
		\[
			i_{*,\dR}\circ i^! \mCM \to \mCM \to j_{*,\dR} \circ j^! \mCM \xrightarrow{+1}.
		\]
	\end{cor}

	\proof[Sketch]
		Consider the cone $\mCN$ of the morphism $\mCM \to j_{*,\dR} \circ j^* \mCM$ provided by the adjoint pair $(j^!,j_{*,\dR})$. Note that $j^!\mCN$ is isomorphic the cone of $j^!\mCM \to j^!\circ j_{*,\dR} \circ j^* \mCM$. And the latter morphism is an isomorphism because $j_{*,\dR}=j_*$ is fully faithful (see Example \ref{exam-push-open-embedding}). Hence $\mCN$ is a complex of right $\mCD_X$-modules that are set-theoretically supported on $Y$. By Kashiwara's lemma, we have $\mCN \simeq i_{*,\dR}\circ i^!(\mCN)$. Note that $i_{*,\dR}\circ i^!\mCN$ is isomorphic to the cone of $i_{*,\dR}\circ i^!\mCM \to i_{*,\dR}\circ i^!\circ  j_{*,\dR} \circ j^!\mCM$, and the target is isomorphic to $0$ by the base-change isomorphism. It follows that $i_{*,\dR}\circ i^!\mCN \simeq i_{*,\dR}\circ i^!\mCM [1]$ as desired.
 

	\qed
	
	
\section{Duality}

	The duality functor is only defined on \emph{coherent} $\mCD$-modules.

	\begin{facts}
		For any $\mCM\in D^b(\mCD_X\hmod_\mbc^r)$, there exists a unique object $\mBD \mCM \in D^b(\mCD_X\hmod_\mbc^r) $ such that
		\[
			\Gamma_\dR( X, \mCM \ot^! - ) \simeq \Hom( \mBD\mCM,- )
		\]
		as functors $D(\mCD_X\hmod_\qc^r) \to \Vect$. The obtained functor
		\[
			\mBD: D^b(\mCD_X\hmod_\mbc^r)^\op \to D^b(\mCD_X\hmod_\mbc^r)
		\]
		is an anti-involution, i.e., $\mBD\circ \mBD\simeq \Id$.

	\end{facts}

	\begin{rem}
		The construction of $\mBD\mCM$ can be treated as a blackbox. For completeness,
		\[
			\mBD\mCM \simeq \mCH om_{\mCD_X^r}( \mCM, \omega_X\ot_{\mCO_X}\mCD_X[d] ),
		\]
		where
		\begin{itemize}
			\item 
				$\omega_X\ot_{\mCO_X}\mCD_X$ has two right $\mCD_X$-module structures;
			\item
				The first one comes from the right multiplication of $\mCD_X$ on itself. We use this right $\mCD_X^r$-structure to define the inner $\mCH om_{\mCD_X^r}$ object.
			\item
				The second one comes from the right $\mCD_X$-module structure on the tensor product of a right $\mCD_X$-module (i.e. $\omega_X$) and a left $\mCD_X$-module (i.e. $\mCD_X$).
			\item
				The second right $\mCD_X$-module structure survives after taking the inner $\mCH om_{\mCD_X^r}$, and the RHS is viewed as a right $\mCD_X$-module using this structure.
		\end{itemize}
		

	\end{rem}

	\begin{exam}
		We have $\mBD(\omega_X) \simeq \omega_X$.
	\end{exam}
	
	\begin{constr}
		Let $\phi:Y \to X$ be a map between finite type $k$-schemes. The standard functors $\phi^!$ and $\phi_{*,\dR}$ in general do not preserve coherent complices. Hence we only have \emph{partially defined} functors
		\[
			\phi_!:= \mBD \circ \phi_{*,\dR}\circ \mBD,\; \phi_\dR^{*}:=\mBD \circ \phi^!\circ \mBD.
		\]
		They are called the \textbf{$!$-direct image functor} and the \textbf{de Rham pullback functor}.
	\end{constr}

	\begin{facts}
		Let $\phi:Y \to X$ be a map between finite type $k$-schemes. Then $\phi_!$ is equivalent to the partially defined left adjoint of $\phi^!$. More precisely, we have
		\[
			\Hom(\phi_!\mCM,-) \simeq \Hom( \mCM,\phi^!(-) )
		\]
		whenever $\phi_!\mCM$ is well-defined. Similarly, $\phi_\dR^*$ is equivalent to the partially defined left adjoint of $\phi_{*,\dR}$.
	\end{facts}

	\begin{rem}
		If $\phi$ is proper, then $\phi_!\simeq \phi_{*,\dR}$. If $\phi$ is smooth, then $\phi_\dR^* \simeq \phi^![-2d_Y + 2d_X]$.

	\end{rem}

\section{Holonomic D-modules}

	We do not give the standard definition of holonomic D-modules. Instead, we characterize them as follows:

	\begin{facts}
		Let $\mCM\in \mCD_X\hmod_\mbc^r$, then $\mCM$ is \textbf{holonomic} iff $\mBD \mCM \in  \mCD_X\hmod_\mbc^r$ (rather than just in the derived category).
	\end{facts}

	\begin{facts}
		Let $\mCM\in D^b(\mCD_X\hmod_\mbc^r)$, then $\mCM$ has \textbf{holonomic cohomologies}, i.e., $\mCH^\bullet(\mCM)$ are holonomic, iff for any closed point $i:x\to X$, the complex $i^!\mCM \in  D^b(\mCD_\pt\hmod_\mbc^r)\simeq D^b(\Vect)$ has finite dimensional cohomologies.
	\end{facts}

	\begin{notn}
		Let $\mCD_X\hmod_\hol^r$ be the abelian category of holonomic right $\mCD_X$-modules and $D^b(\mCD_X\hmod_\hol^r)$ be the bounded derived category.
	\end{notn}

	\begin{facts}
		$D^b(\mCD_X\hmod_\hol^r)$ is equivalent to the full subcategory of $D^b(\mCD_X\hmod_\mbc^r)$ containing complices with holonomic cohomologies.
	\end{facts}

	\begin{facts}
		All the functors defined so far preserve bounded holonomic complices.
	\end{facts}
	

	

\begin{thebibliography}{Yau}

	\bibitem[B]{B} Bernstein, Joseph. Algebraic theory of D-modules, 1984, abailable at \url{https://gauss.math.yale.edu/~il282/Bernstein_D_mod.pdf}.
	

	\bibitem[G]{G} Gaitsgory, Dennis. Course Notes for Geometric Representation Theory, 2005, available at \url{https://people.mpim-bonn.mpg.de/gaitsgde/267y/catO.pdf}.

	\bibitem[HTT]{HTT} Hotta, Ryoshi, and Toshiyuki Tanisaki. $\mCD$-modules, perverse sheaves, and representation theory. Vol. 236. Springer Science \& Business Media, 2007.
\end{thebibliography}


\end{document} 


