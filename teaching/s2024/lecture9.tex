
%!TEX root = main.tex
\documentclass{amsart}
\textwidth=14.5cm \oddsidemargin=1cm
\evensidemargin=1cm
\usepackage{amsmath}
\usepackage{amsxtra}
\usepackage{amscd}
\usepackage{amsthm}
\usepackage{amsfonts}
\usepackage{amssymb}
\usepackage[foot]{amsaddr}
\usepackage{cite}
\usepackage{url}
\usepackage{rotating}
\usepackage{eucal}
\usepackage{tikz-cd}
\usepackage[all,2cell,color]{xy}
\UseAllTwocells
\UseCrayolaColors
\usepackage{graphicx}
\usepackage{pifont}
\usepackage{comment}
\usepackage{verbatim}
\usepackage{xcolor}
\usepackage{hyperref}
\usepackage{xparse}
\usepackage{upgreek}
\usepackage{MnSymbol}
\sloppy


%%%%%%%%%%%%%%%%%%%%Theorem%%%%%%%%%%%%%%%%%%%%
\newcounter{theorem}
\setcounter{theorem}{0}

\newtheorem{cor}[subsection]{Corollary}
\newtheorem{lem}[subsection]{Lemma}
\newtheorem{goal}[subsection]{Goal}
\newtheorem{lemdefn}[subsection]{Lemma-Definition}
\newtheorem{prop}[subsection]{Proposition}
\newtheorem{propdefn}[subsection]{Proposition-Definition}
\newtheorem{cordefn}[subsection]{Corollary-Definition}
\newtheorem{variant}[subsection]{Variant}
\newtheorem{warn}[subsection]{Warning}
\newtheorem{sugg}[subsection]{Suggestion}
\newtheorem{facts}[subsection]{Fact}
\newtheorem{ques}{Question}
\newtheorem{guess}{Guess}
\newtheorem{claim}{Claim}
\newtheorem{propconstr}[subsection]{Proposition-Construction}
\newtheorem{lemconstr}[subsection]{Lemma-Construction}
\newtheorem{ax}{Axiom}
\newtheorem{conje}[subsection]{Conjecture}
\newtheorem{mainthm}[subsection]{Main-Theorem}
\newtheorem{summ}[subsection]{Summary}
\newtheorem{thm}[subsection]{Theorem}
\newtheorem{thmdefn}[subsection]{Theorem-Definition}
\newtheorem{notn}[subsection]{Notation}
\newtheorem{convn}[subsection]{Convention}
\newtheorem{constr}[subsection]{Construction}


\theoremstyle{definition}

\newtheorem{defn}[subsection]{Definition}
\newtheorem{exam}[subsection]{Example}
\newtheorem{assum}[subsection]{Assumption}

\theoremstyle{remark}
\newtheorem{rem}[subsection]{Remark}
\newtheorem{exe}[subsection]{Exercise}


\numberwithin{equation}{section}


%%%%%%%%%%%%%%%%%%%%Commands%%%%%%%%%%%%%%%%%%%%

\newcommand{\nc}{\newcommand}
\nc\on{\operatorname}
\nc\renc{\renewcommand}


%%%%%%%%%%%%%%%%%%%%Sections%%%%%%%%%%%%%%%%%%%%

\nc\ssec{\subsection}
\nc\sssec{\subsubsection}

%%%%%%%%%%%%%%%%%%%%Environment%%%%%%%%%%%%%%%%%
\nc\blongeqn{\[ \begin{aligned}}
\nc\elongeqn{\end{aligned} \]}



%%%%%%%%%%%%%%%%%%%%Mathfont%%%%%%%%%%%%%%%%%%%%

\nc\mBA{{\mathbb A}}
\nc\mBB{{\mathbb B}}
\nc\mBC{{\mathbb C}}
\nc\mBD{{\mathbb D}}
\nc\mBE{{\mathbb E}}
\nc\mBF{{\mathbb F}}
\nc\mBG{{\mathbb G}}
\nc\mBH{{\mathbb H}}
\nc\mBI{{\mathbb I}}
\nc\mBJ{{\mathbb J}}
\nc\mBK{{\mathbb K}}
\nc\mBL{{\mathbb L}}
\nc\mBM{{\mathbb M}}
\nc\mBN{{\mathbb N}}
\nc\mBO{{\mathbb O}}
\nc\mBP{{\mathbb P}}
\nc\mBQ{{\mathbb Q}}
\nc\mBR{{\mathbb R}}
\nc\mBS{{\mathbb S}}
\nc\mBT{{\mathbb T}}
\nc\mBU{{\mathbb U}}
\nc\mBV{{\mathbb V}}
\nc\mBW{{\mathbb W}}
\nc\mBX{{\mathbb X}}
\nc\mBY{{\mathbb Y}}
\nc\mBZ{{\mathbb Z}}


\nc\mCA{{\mathcal A}}
\nc\mCB{{\mathcal B}}
\nc\mCC{{\mathcal C}}
\nc\mCD{{\mathcal D}}
\nc\mCE{{\mathcal E}}
\nc\mCF{{\mathcal F}}
\nc\mCG{{\mathcal G}}
\nc\mCH{{\mathcal H}}
\nc\mCI{{\mathcal I}}
\nc\mCJ{{\mathcal J}}
\nc\mCK{{\mathcal K}}
\nc\mCL{{\mathcal L}}
\nc\mCM{{\mathcal M}}
\nc\mCN{{\mathcal N}}
\nc\mCO{{\mathcal O}}
\nc\mCP{{\mathcal P}}
\nc\mCQ{{\mathcal Q}}
\nc\mCR{{\mathcal R}}
\nc\mCS{{\mathcal S}}
\nc\mCT{{\mathcal T}}
\nc\mCU{{\mathcal U}}
\nc\mCV{{\mathcal V}}
\nc\mCW{{\mathcal W}}
\nc\mCX{{\mathcal X}}
\nc\mCY{{\mathcal Y}}
\nc\mCZ{{\mathcal Z}}


\nc\mbA{{\mathsf A}}
\nc\mbB{{\mathsf B}}
\nc\mbC{{\mathsf C}}
\nc\mbD{{\mathsf D}}
\nc\mbE{{\mathsf E}}
\nc\mbF{{\mathsf F}}
\nc\mbG{{\mathsf G}}
\nc\mbH{{\mathsf H}}
\nc\mbI{{\mathsf I}}
\nc\mbJ{{\mathsf J}}
\nc\mbK{{\mathsf K}}
\nc\mbL{{\mathsf L}}
\nc\mbM{{\mathsf M}}
\nc\mbN{{\mathsf N}}
\nc\mbO{{\mathsf O}}
\nc\mbP{{\mathsf P}}
\nc\mbQ{{\mathsf Q}}
\nc\mbR{{\mathsf R}}
\nc\mbS{{\mathsf S}}
\nc\mbT{{\mathsf T}}
\nc\mbU{{\mathsf U}}
\nc\mbV{{\mathsf V}}
\nc\mbW{{\mathsf W}}
\nc\mbX{{\mathsf X}}
\nc\mbY{{\mathsf Y}}
\nc\mbZ{{\mathsf Z}}

\nc\mba{{\mathsf a}}
\nc\mbb{{\mathsf b}}
\nc\mbc{{\mathsf c}}
\nc\mbd{{\mathsf d}}
\nc\mbe{{\mathsf e}}
\nc\mbf{{\mathsf f}}
\nc\mbg{{\mathsf g}}
\nc\mbh{{\mathsf h}}
\nc\mbi{{\mathsf i}}
\nc\mbj{{\mathsf j}}
\nc\mbk{{\mathsf k}}
\nc\mbl{{\mathsf l}}
\nc\mbm{{\mathsf m}}
\nc\mbn{{\mathsf n}}
\nc\mbo{{\mathsf o}}
\nc\mbp{{\mathsf p}}
\nc\mbq{{\mathsf q}}
\nc\mbr{{\mathsf r}}
\nc\mbs{{\mathsf s}}
\nc\mbt{{\mathsf t}}
\nc\mbu{{\mathsf u}}
\nc\mbv{{\mathsf v}}
\nc\mbw{{\mathsf w}}
\nc\mbx{{\mathsf x}}
\nc\mby{{\mathsf y}}
\nc\mbz{{\mathsf z}}




\nc\mbfA{{\mathbf A}}
\nc\mbfB{{\mathbf B}}
\nc\mbfC{{\mathbf C}}
\nc\mbfD{{\mathbf D}}
\nc\mbfE{{\mathbf E}}
\nc\mbfF{{\mathbf F}}
\nc\mbfG{{\mathbf G}}
\nc\mbfH{{\mathbf H}}
\nc\mbfI{{\mathbf I}}
\nc\mbfJ{{\mathbf J}}
\nc\mbfK{{\mathbf K}}
\nc\mbfL{{\mathbf L}}
\nc\mbfM{{\mathbf M}}
\nc\mbfN{{\mathbf N}}
\nc\mbfO{{\mathbf O}}
\nc\mbfP{{\mathbf P}}
\nc\mbfQ{{\mathbf Q}}
\nc\mbfR{{\mathbf R}}
\nc\mbfS{{\mathbf S}}
\nc\mbfT{{\mathbf T}}
\nc\mbfU{{\mathbf U}}
\nc\mbfV{{\mathbf V}}
\nc\mbfW{{\mathbf W}}
\nc\mbfX{{\mathbf X}}
\nc\mbfY{{\mathbf Y}}
\nc\mbfZ{{\mathbf Z}}

\nc\mbfa{{\mathbf a}}
\nc\mbfb{{\mathbf b}}
\nc\mbfc{{\mathbf c}}
\nc\mbfd{{\mathbf d}}
\nc\mbfe{{\mathbf e}}
\nc\mbff{{\mathbf f}}
\nc\mbfg{{\mathbf g}}
\nc\mbfh{{\mathbf h}}
\nc\mbfi{{\mathbf i}}
\nc\mbfj{{\mathbf j}}
\nc\mbfk{{\mathbf k}}
\nc\mbfl{{\mathbf l}}
\nc\mbfm{{\mathbf m}}
\nc\mbfn{{\mathbf n}}
\nc\mbfo{{\mathbf o}}
\nc\mbfp{{\mathbf p}}
\nc\mbfq{{\mathbf q}}
\nc\mbfr{{\mathbf r}}
\nc\mbfs{{\mathbf s}}
\nc\mbft{{\mathbf t}}
\nc\mbfu{{\mathbf u}}
\nc\mbfv{{\mathbf v}}
\nc\mbfw{{\mathbf w}}
\nc\mbfx{{\mathbf x}}
\nc\mbfy{{\mathbf y}}
\nc\mbfz{{\mathbf z}}

\nc\mfa{{\mathfrak a}}
\nc\mfb{{\mathfrak b}}
\nc\mfc{{\mathfrak c}}
\nc\mfd{{\mathfrak d}}
\nc\mfe{{\mathfrak e}}
\nc\mff{{\mathfrak f}}
\nc\mfg{{\mathfrak g}}
\nc\mfh{{\mathfrak h}}
\nc\mfi{{\mathfrak i}}
\nc\mfj{{\mathfrak j}}
\nc\mfk{{\mathfrak k}}
\nc\mfl{{\mathfrak l}}
\nc\mfm{{\mathfrak m}}
\nc\mfn{{\mathfrak n}}
\nc\mfo{{\mathfrak o}}
\nc\mfp{{\mathfrak p}}
\nc\mfq{{\mathfrak q}}
\nc\mfr{{\mathfrak r}}
\nc\mfs{{\mathfrak s}}
\nc\mft{{\mathfrak t}}
\nc\mfu{{\mathfrak u}}
\nc\mfv{{\mathfrak v}}
\nc\mfw{{\mathfrak w}}
\nc\mfx{{\mathfrak x}}
\nc\mfy{{\mathfrak y}}
\nc\mfz{{\mathfrak z}}

\nc{\one}{{\mathsf{1}}}


\nc\clambda{ {\check{\lambda} }}
\nc\cmu{ {\check{\mu} }}

\nc\bDelta{\mathsf{\Delta}}
\nc\bGamma{\mathsf{\Gamma}}
\nc\bLambda{\mathsf{\Lambda}}


\nc\loccit{\emph{loc.cit.}}



%%%%%%%%%%%%%%%%%%%%Operations-limit%%%%%%%%%%%%%%%%%%%%

\NewDocumentCommand{\ot}{e{_^}}{
  \mathbin{\mathop{\otimes}\displaylimits
    \IfValueT{#1}{_{#1}}
    \IfValueT{#2}{^{#2}}
  }
}
\NewDocumentCommand{\boxt}{e{_^}}{
  \mathbin{\mathop{\boxtimes}\displaylimits
    \IfValueT{#1}{_{#1}}
    \IfValueT{#2}{^{#2}}
  }
}
\NewDocumentCommand{\mt}{e{_^}}{
  \mathbin{\mathop{\times}\displaylimits
    \IfValueT{#1}{_{#1}}
    \IfValueT{#2}{^{#2}}
  }
}
\NewDocumentCommand{\convolve}{e{_^}}{
  \mathbin{\mathop{\star}\displaylimits
    \IfValueT{#1}{_{#1}}
    \IfValueT{#2}{^{#2}}
  }
}
\NewDocumentCommand{\colim}{e{_^}}{
  \mathbin{\mathop{\operatorname{colim}}\displaylimits
    \IfValueT{#1}{_{#1}\,}
    \IfValueT{#2}{^{#2}\,}
  }
}
\NewDocumentCommand{\laxlim}{e{_^}}{
  \mathbin{\mathop{\operatorname{laxlim}}\displaylimits
    \IfValueT{#1}{_{#1}\,}
    \IfValueT{#2}{^{#2}\,}
  }
}
\NewDocumentCommand{\oplaxlim}{e{_^}}{
  \mathbin{\mathop\operatorname{oplax-lim}\displaylimits
    \IfValueT{#1}{_{#1}\,}
    \IfValueT{#2}{^{#2}\,}
  }
}


%%%%%%%%%%%%%%%%%%%%Arrows%%%%%%%%%%%%%%%%%%%%


\makeatletter
\newcommand{\laxto}{\dashedrightarrow}
\newcommand{\xrightleftarrows}[1]{\mathrel{\substack{\xrightarrow{#1} \\[-.9ex] \xleftarrow{#1}}}}
\newcommand{\adj}{\xrightleftarrows{\rule{0.5cm}{0cm}}}

\newcommand*{\da@rightarrow}{\mathchar"0\hexnumber@\symAMSa 4B }
\newcommand*{\da@leftarrow}{\mathchar"0\hexnumber@\symAMSa 4C }
\newcommand*{\xlaxto}[2][]{%
  \mathrel{%
    \mathpalette{\da@xarrow{#1}{#2}{}\da@rightarrow{\,}{}}{}%
  }%
}
\newcommand{\xlaxgets}[2][]{%
  \mathrel{%
    \mathpalette{\da@xarrow{#1}{#2}\da@leftarrow{}{}{\,}}{}%
  }%
}
\newcommand*{\da@xarrow}[7]{%
  % #1: below
  % #2: above
  % #3: arrow left
  % #4: arrow right
  % #5: space left 
  % #6: space right
  % #7: math style 
  \sbox0{$\ifx#7\scriptstyle\scriptscriptstyle\else\scriptstyle\fi#5#1#6\m@th$}%
  \sbox2{$\ifx#7\scriptstyle\scriptscriptstyle\else\scriptstyle\fi#5#2#6\m@th$}%
  \sbox4{$#7\dabar@\m@th$}%
  \dimen@=\wd0 %
  \ifdim\wd2 >\dimen@
    \dimen@=\wd2 %   
  \fi
  \count@=2 %
  \def\da@bars{\dabar@\dabar@}%
  \@whiledim\count@\wd4<\dimen@\do{%
    \advance\count@\@ne
    \expandafter\def\expandafter\da@bars\expandafter{%
      \da@bars
      \dabar@ 
    }%
  }%  
  \mathrel{#3}%
  \mathrel{%   
    \mathop{\da@bars}\limits
    \ifx\\#1\\%
    \else
      _{\copy0}%
    \fi
    \ifx\\#2\\%
    \else
      ^{\copy2}%
    \fi
  }%   
  \mathrel{#4}%
}
\makeatother

%%%%%%%%%%%%%%%%%%%%Decorations%%%%%%%%%%%%%%%%%%%%
\nc{\wt}{\widetilde}
\nc{\ol}{\overline}

\nc{\red}{\textcolor{red}}
\nc{\blue}{\textcolor{blue}}
\nc{\purple}{\textcolor{violet}}

\nc{\simorlax}{{\red\simeq/\blue\lax}}

%%%%%%%%%%%%%%%%%%%%All%%%%%%%%%%%%%%%%%%%%

\nc{\Id}{\mathsf{Id}}
\nc{\gl}{\mathfrak{gl}}
\renc{\sl}{\mathfrak{sl}}
\nc{\GL}{\mathsf{GL}}
\nc{\SL}{\mathsf{SL}}
\nc{\PGL}{\mathsf{PGL}}
\nc{\hmod}{\mathsf{-mod}}
\nc{\Vect}{\mathsf{Vect}}
\nc{\tr}{\mathsf{tr}}
\nc{\Kil}{\mathsf{Kil}}
\nc{\ad}{{\mathsf{ad}}}
\nc{\Ad}{\mathsf{Ad}}
\nc{\oblv}{\mathsf{oblv}}
\nc{\gr}{\mathsf{gr}}
\nc{\Sym}{\mathsf{Sym}}
\nc{\QCoh}{\mathsf{QCoh}}
\nc{\ind}{\mathsf{ind}}
\nc{\Spec}{\mathsf{Spec}}
\nc{\Hom}{\mathsf{Hom}}
\nc{\Ext}{\mathsf{Ext}}
\nc{\Grp}{\mathsf{Grp}}
\nc{\pt}{\mathsf{pt}}
\nc{\Lie}{\mathsf{Lie}}
\nc{\CAlg}{\mathsf{CAlg}}
\nc{\Der}{\mathsf{Der}}
\nc{\Rep}{\mathsf{Rep}}
\renc{\sc}{{\mathsf{sc}}}
\nc{\Fl}{\mathsf{Fl}}
\nc{\Fun}{\mathsf{Fun}}
\nc{\ev}{\mathsf{ev}}
\nc{\surj}{\twoheadrightarrow}
\nc{\inj}{\hookrightarrow}
\nc{\HC}{\mathsf{HC}}
\nc{\cl}{\mathsf{cl}}
\renc{\Im}{\mathsf{Im}}
\renc{\ker}{\mathsf{ker}}
\nc{\coker}{\mathsf{coker}}
\nc{\Tor}{\mathsf{Tor}}
\nc{\op}{\mathsf{op}}
\nc{\length}{\mathsf{length}}
\nc{\fd}{{\mathsf{fd}}}
\nc{\weight}{\mathsf{wt}}
\nc{\semis}{{\mathsf{ss}}}
\nc{\qc}{{\mathsf{qc}}}
\nc{\pr}{\mathsf{pr}}
\nc{\act}{\mathsf{act}}
\nc{\dR}{{\mathsf{dR}}}
\nc{\hol}{{\mathsf{hol}}}
\nc{\Pic}{{\mathsf{Pic}}}
\nc{\Loc}{\mathsf{Loc}}
\nc{\IC}{\mathsf{IC}}

\begin{document}


\title{Lecture 9}

\date{Apr 22, 2024}

\maketitle



\section{Standard filtrations}

	Last time we proved $\mCO$ has enough projectives and injectives. Hence for $M,N\in \mCO$, the vector space $\Ext_\mCO^i(M,N), i\ge 0$ is well-defined. It can be calculated in either of the following methods:
	\begin{itemize}
		\item 
			Choose a projective resolution $P^\bullet \to M$ and consider the chain complex $\Hom_\mCO(P^{-\bullet},N)$. Then $\Ext_\mCO^i(M,N)$ is the $i$-th cohomology of this complex.
		\item
			Choose an injective resolution $N\to I^\bullet$ and consider the chain complex $\Hom_\mCO(M,I^\bullet)$. Then $\Ext_\mCO^i(M,N)$ is the $i$-th cohomology of this complex.
	\end{itemize}

	We will prove the following result:

	\begin{thmdefn}
		\label{thmdefn-standard-fil}
		For an object $M \in \mCO$, the following conditions are equivalent:
		\begin{itemize}
			\item[(a)]
				The object $M$ admits a finite filtration such that the subquotients are isomorphic to Verma modules.
			\item[(b)]
				For any weight $\mu$ and $i>0$, $\Ext_\mCO^i(M,M_\mu^\vee) = 0$.
			\item[(c)]
				For any weight $\mu$, $\Ext_\mCO^1(M,M_\mu^\vee) = 0$.
		\end{itemize}
		A filtration as in (a) is called a \textbf{standard filtration} of $M$. Let $\mCO^\Delta\subset \mCO$ be the full subcategory of objects admitting standard filtrations.
	\end{thmdefn}
	
	Via contragradient duality, the above result is equivalent to:

	\begin{thmdefn}
		For an object $M \in \mCO$, the following conditions are equivalent:
		\begin{itemize}
			\item[(a)]
				The object $M$ admits a finite filtration such that the subquotients are isomorphic to dual Verma modules.
			\item[(b)]
				For any weight $\lambda$ and $i>0$, $\Ext_\mCO^i(M_\lambda,M) = 0$.
			\item[(c)]
				For any weight $\lambda$, $\Ext_\mCO^1(M_\lambda,M) = 0$.
		\end{itemize}
		A filtration as in (a) is called a \textbf{costandard filtration} of $M$. Let $\mCO^\nabla\subset \mCO$ be the full subcategory of objects admitting costandard filtrations.
	\end{thmdefn}

	As a particular case, we obtain

	\begin{cor}
		\label{cor-orthogonal-Ext}
		For weights $\lambda,\mu$ and $i>0$, we have $\Ext_\mCO^i(M_\lambda,M_\mu^\vee) = 0$.
	\end{cor}

	We will also prove the following result.

	\begin{thm}
		\label{thm-proj-stadard-fil}
		Every projective object of $\mCO$ admits a standard filtration.
	\end{thm}

	To prove these results, we need some preparations.

	\begin{lem}
		\label{lem-gen-Verma-standard-fil}
		The \emph{generalized Verma modules}\footnote{See [Proof of Lemma 4.10, Lecture 8].} $M_{\lambda,n}$ admit standard filtrations.
	\end{lem}

	\proof
		Recall $M_{\lambda,n}\simeq \ind_\mfb^\mfg( U(\mfb)/ I_{\lambda,n}' )$, where $I_{\lambda,n}' $ is the left ideal of $U(\mfb)$ generated by the following elements:
		\begin{itemize}
			\item The element $t-\lambda(t)$ for any $t\in \mft$;
			\item The element $x_1x_2\cdots x_n$ for $x_i \in \mfn^+$.
		\end{itemize}
		By considering the weights, the finite-dimensional $\mfb$-module $U(\mfb)/ I_{\lambda,n}'$ has a filtration such that the subquotients are 1-dimensional. Since the functor $\ind_\mfb^\mbg$ is exact, we obtain a standard filtration of $M_{\lambda,n}$.

	\qed

	\begin{lem}
		\label{lem-standard-fil-sub-Verma}
		 Let $M\in \mCO^\Delta$ be an object admitting a standard filtration $\mbF^{\le \bullet}M$ with length $m$. Let $v\in M$ be a nonzero highest weight vector with weight $\lambda$, and $M_\lambda \to M$ be the unique $\mfg$-linear map sending $v_\lambda$ to $v$. Let $i$ be the smallest index such that the image $\Im(f)$ is contained in $\mbF^{\le i}M$. Then:
		\begin{itemize}
			\item[(1)]
				The composition $M_\lambda \to M \to \gr^i M$ is an isomorphism. In particular, $M_\lambda \to M$ is injective.
			\item[(2)]
				The quotient $M/M_\lambda$ admits a standard filtration with length $m-1$.
		\end{itemize}
	\end{lem}

	\proof
		By assumption $\gr^i M$ is a Verma module $M_\mu$ and the composition $M_\lambda \to M \to \gr^i M$ is nonzero. This implies $\lambda\preceq \mu$. Since $\lambda$ is a highest weight of $M$. We get $\lambda= \mu$. This implies $M_\lambda \to M \to \gr^i M$ is an isomorphism because it preserves the nonzero highest weight vectors. This proves (1).

		For (2), we have a short exact sequence $0 \to \mbF^{\le i-1} M \to M/M_\lambda \to M/\mbF^{\le i}M \to 0$. By assumption, $\mbF^{\le i-1} M$ has a standard filtration of length $i-1$, and $M/\mbF^{\le i}M $ has one of length $m-i$. It follows that $M/M_\lambda$ admits a standard filtration with length $(i-1)+(m-i)=m-1$.

	\qed

	\begin{lem}
		\label{lem-direct-summand-standard-fil}
		For direct sum decomposition $M=M_1\oplus M_2$ in $\mCO$, if $M$ admits a standard filtration, so do $M_1$ and $M_2$.
	\end{lem}

	\proof
		We use induction on the length of the standard filtration of $M$. When the length is $0$, the claim is trivial. If the length is $m>0$, then $M\neq 0$. Without lose of generality, we can assume $M_1$ contains a nonzero highest weight vector $v$ of $M$ with weight $\lambda$. By Lemma \ref{lem-standard-fil-sub-Verma}, we have injections $M_\lambda \inj M_1 \inj M$ such that $M/M_\lambda$ admits a standard filtration of length $m-1$. Since $M/M_\lambda \simeq M_1/M_\lambda \oplus M_2$, by induction hypothesis, we have $M_1/M_\lambda, M_2\in \mCO^\Delta$. It follows that we also have $M_1\in \mCO^\Delta$.

	\qed



	\proof[Proof of Theorem \ref{thm-proj-stadard-fil}]
		In the proof of [Theorem 4.3, Lecture 8], we proved any object in $\mCO$ is a quotient of a direct sum of some $M_{\lambda,n}$. In general, if a projective object is a quotient of another object, then it is a direct summand of the latter. It follows that any projective object is a direct summand of a direct sum of some $M_{\lambda,n}$. Then the claim follows from Lemma \ref{lem-gen-Verma-standard-fil} and Lemma \ref{lem-direct-summand-standard-fil}.

	\qed[Theorem \ref{thm-proj-stadard-fil}]

	\begin{lem}
		\label{lem-kernel-standard-fil}
		Let $0 \to K \to M \to N \to 0$ be a short exact sequence such that $N\in \mCO^\Delta$. Then $M\in \mCO^\Delta$ iff $K\in \mCO^\Delta$.
	\end{lem}

	\begin{warn}
		$K,M\in \mCO^\Delta$ does not imply $N\in \mCO^\Delta$. This can be seen from the $\sl_2$-case and the short exact sequence $0\to M_{-l-2} \to M_l \to L_l \to 0$.
	\end{warn}

	\proof
		The ``if'' part is obvious. For the ``only if'' part, let $M\in \mCO^\Delta$. Using induction, it is easy to reduce to the case when $N$ is a Verma module $M_\mu$. Let $M_\lambda \to M$ be as in Lemma \ref{lem-standard-fil-sub-Verma}. Then $M/M_\lambda\in \mCO^\Delta$. Note that the composition $M_\lambda \to M \to M_\mu$ is either $0$ or an isomorphism by considering the weights. If this is the zero map, then $M/M_\lambda \to M_\mu$ is still a surjection and we can finish the proof by using induction. Otherwise $M\simeq K\oplus M_\mu$ and the claim follows from Lemma \ref{lem-direct-summand-standard-fil}.

	\qed

	\proof[Proof of Theorem-Definition \ref{thmdefn-standard-fil}]
		We first prove $(a)\Rightarrow (b)$. We prove by induction on $i$. For any fixed $i$, using the long exact sequences
		\[
			\cdots \to \Ext_\mCO^i(\mbF^{\le k-1} M,M_\mu^\vee) \to \Ext_\mCO^i(\mbF^{\le k} M,M_\mu^\vee) \to \Ext_\mCO^i(\gr^k M,M_\mu^\vee)  \to \cdots,
		\]
		we only need to show $ \Ext_\mCO^i(\gr^k M,M_\mu^\vee)  = 0$ for any $k$. By assumption, $\gr^k M\simeq M_\lambda$ for some weight $\lambda$. The case $i=1$ is just [Lemma 3.16, Lecture 8]. For $i>1$, let $0 \to N \to P\to  M_\lambda\to 0$ be a short exact sequence such that $P$ is projective. We have a long exact sequence
		\[
			\cdots \to \Ext_\mCO^{i-1}(P,M_\mu^\vee) \to \Ext_\mCO^{i-1}(N,M_\mu^\vee) \to \Ext_\mCO^i(M_\lambda,M_\mu^\vee) \to \Ext_\mCO^{i}(P,M_\mu^\vee) \to \cdots.
		\]
		Since $P$ is projective, we have $\Ext_\mCO^{i-1}(N,M_\mu^\vee) \simeq \Ext_\mCO^i(M_\lambda,M_\mu^\vee)$. By Theorem \ref{thm-proj-stadard-fil} and Lemma \ref{lem-kernel-standard-fil}, $N\in \mCO^\Delta$. Hence by induction hypothesis $\Ext_\mCO^{i-1}(N,M_\mu^\vee) \simeq 0$ as desired.

		$(b)\Rightarrow (c)$ is obvious.

		It remains to show $(c)\Rightarrow (a)$. 

		We prove by induction on $\length(M)$. The case $\length(M)=0$ is obvious. If $\length(M)>0$, let $\lambda$ be a highest weight vector of $M$ and $n:=\dim(M^{\weight = \lambda})$. Then there is a nonzero $\mfg$-linear map $ M_\lambda ^{\oplus n} \to M $. Let\footnote{The following language can be rewritten in the language of spectral sequences.} $N_1$ and $N_2$ be the kernel and cokernel of this map, and $M'\neq 0$ be the image of it. We have short exact sequences
		\[
			\begin{aligned}
				0 \to N_1 \to M_\lambda ^{\oplus n} \to M' \to 0 ,\\
				0 \to M' \to M \to N_2 \to 0.
			\end{aligned}
		\]
		They induce long exact sequences
		\[
			\begin{aligned}
				0 \to \Hom_\mCO(M', M_\mu^\vee  ) \to  \Hom_\mCO( M_\lambda ^{\oplus n}, M_\mu^\vee  ) \to  \Hom_\mCO( N_1, M_\mu^\vee  ) \to \Ext_\mCO^1(M', M_\mu^\vee  ) \to\cdots \\
				\cdots  \to \Hom_\mCO( M , M_\mu^\vee )  \to \Hom_\mCO( M' , M_\mu^\vee ) \to \Ext_\mCO^1( N_2 , M_\mu^\vee ) \to \Ext_\mCO^1( M , M_\mu^\vee ) \to \\
				\to \Ext_\mCO^1( M' , M_\mu^\vee ) \to \Ext_\mCO^2( N_2 , M_\mu^\vee ) \to \cdots
			\end{aligned}
		\]
		By assumption, $\Ext_\mCO^1( M , M_\mu^\vee )=0$. We claim $\Ext_\mCO^1( N_2 , M_\mu^\vee )=0$. Indeed:
		\begin{itemize}
			\item 
				If $\lambda\neq \mu$, then $ \Hom_\mCO( M_\lambda ^{\oplus n}, M_\mu^\vee  )=0$ ([Lemma 3.14, Lecture 8]). By the first sequence, $ \Hom_\mCO(M', M_\mu^\vee  ) =0$. By the second sequence, $\Ext_\mCO^1( N_2 , M_\mu^\vee )=0$. 
			\item
				If $\lambda = \mu$, note that $\mu$ is a highest weight of both $M'$ and $M$ and by construction, $(M')^{\weight=\mu} \simeq M^{\weight = \mu}$. Hence we have
				\[
					\Hom_\mCO(M', M_\mu^\vee  ) \simeq ((M')^{\weight=\mu})^* \simeq (M^{\weight = \mu})^* \simeq \Hom_\mCO(M, M_\mu^\vee  ) .
				\]
				By the second sequence, $\Ext_\mCO^1( N_2 , M_\mu^\vee )=0$. 
		\end{itemize}

		Note that $\length(N_2)<\length(M)$. By the induction hypothesis, $N_2\in \mCO^\Delta$. Using $(a)\Rightarrow(b)$, we get $\Ext_\mCO^2( N_2 , M_\mu^\vee )=0$. By the second seqeunce, $\Ext_\mCO^1( M' , M_\mu^\vee )=0$. Now we have two cases:
		\begin{itemize}
			\item 
				If $N_2\neq 0$, then $\length(M')<\length(M)$. By the induction hypothesis, $M'\in \mCO^\Delta$. Then $M\in \mCO^\length$ because it is an extension of $N_2$ by $M'$.
			\item
				If $N_2 = 0$, then $M\simeq M'$. An argument similar to that in the last paragraph implies $ \Hom_\mCO( N_1, M_\mu^\vee  ) = 0$. Note that $\mu$ can be any weight. This forces $N_1 =0$ and therefore $M\simeq M_\lambda^{\oplus n}$. Then it is clear $M\in \mCO^\Delta$.
		\end{itemize}


	\qed[Theorem-Definition \ref{thmdefn-standard-fil}]

	\begin{lem}
		Let $M\in \mCO_\chi$ and $M'\in \mCO_{\chi'}$ such that $\chi\neq \chi'$. Then $\Ext^i(M,N)=0$ for any $i\ge 0$.
	\end{lem}

	\proof
		Let $P^\bullet \to M$ be a projective resolution of $M$. We can replace each $P^{-n}$ by its image under the functor $\mCO \to \mCO_{\chi}$ and obtain a projective resolution of $M$ contained in $\mCO_\chi$. Now the claim follows from the case $i=0$.

	\qed

	\begin{exe}
		This is \red{Homework 4, Problem 4}. Let $\lambda$ be a weight. Prove:
		\begin{itemize}
			\item[(1)]
				If $M\in \mCO$ such that $\weight(M)\cap \{\mu\,\vert\, \mu\succeq \lambda\}=\emptyset$, then $\Ext^i_\mCO(M,M_\lambda^\vee) =0$ for $i\ge 0$\footnote{Hint: Step 1: reduce to the case $M=L_\mu$ with $\varpi(\mu) = \varpi(\lambda)$ and $\mu\nsucceq \lambda$. Step 2: consider $0\to K \to M_\mu \to L_\mu \to 0$ and note that $\weight(K)\prec \mu$.}.
			\item[(2)]
				$\Ext^i_\mCO(L_\lambda,M_\lambda^\vee) =0$ for $i>0$.
			\item[(3)]
				Combining (1) and (2), deduce $\Ext^i_\mCO(M_\lambda,L_\mu) =0$ and $\Ext^i_\mCO(L_\mu,M_\lambda^\vee) =0$ for $i>0$ and $\mu\nsucc \lambda$.
		\end{itemize}

	\end{exe}
		
\section{BGG reciprocity}

	The following result follows from Theorem-Definition \ref{thmdefn-standard-fil} by dévissage.

	\begin{propdefn}
		\label{propdefn-mult-standard-fil}
		Let $M\in \mCO^\Delta$, then for any standard filtration of $M$, the multiplicity of $M_\lambda$ in the subquotients does not depend on the filtration. We denote this number by $(M: M_\lambda)$. Moreover, we have
		\[
			(M: M_\lambda) \simeq \dim( \Hom_\mCO(M, M_\lambda^\vee) ).
		\]
	\end{propdefn}

	\begin{thm}[BGG reciprocity]
		\label{thm-BGG-rec}
		For weights $\lambda,\mu\in \mft^*$, we have
		\[
			(P_\mu: M_\lambda) = [M_\lambda^\vee: L_\mu] = [M_\lambda:L_\mu].
		\]
	\end{thm}

	\proof
		The last identity follows from $L_\mu^\vee \simeq L_\mu$. The first one follows from Proposition-Definition \ref{propdefn-mult-standard-fil} and [Corollary 4.9, Lecture 8].

	\qed

	\begin{rem}
		The previous discussions on standard filtrations and BGG reciprocity can be axiomized using the language of \textbf{highest weight categories}. See \cite{CPS}.
	\end{rem}

	\begin{rem}
		In $\mCO$, or any highest weight category, if an object admits both a standard filtration and a costandard filtration, then it is called a \textbf{tilting object}. One can show the set of indecomposable tilting objects is bijective to the set of irreducible objects. Tilting objects play important roles in representation theory. For a geometry-oriented\footnote{You should ask experts in representation theory for other good references.} introduction, see \cite{BBM}.
	\end{rem}

\section{Translation functors}
	
	\begin{constr}
		Let $V$ be a finite-dimensional $\mfg$-module. Consider the functor
		\[
			\mfg\hmod \to \mfg\hmod,\; M\mapsto V\ot M,
		\]
		where (recall) the $\mfg$-module structure on $V\ot M$ is given by the Lebniz rule. It is easy to see this functor preserves $\mCO$. We denote the obtained functor by $T_V:\mCO \to \mCO$. Note that $T_V$ is exact.
	\end{constr}

	\begin{lem}
		The functor $T_V:\mCO \to \mCO$ commutes with contragradient duality.
	\end{lem}

	\proof
		For any finite dimensional $\mfg$-module, we have $V^\vee \simeq V$ because $L_\lambda^\vee \simeq L_\lambda$.

	\qed

	\begin{lem}
		The functor $T_{V^*}$ is both left and right adjoint to $T_V$. In particular, $T_V$ preserves both projectives and injectives.
	\end{lem}

	\proof
		We have the unit and pairing maps $k \to V\ot V^*$ and $V^* \ot V \to k$, which induce natural transformations $T_{k} \to T_{V\ot V^*}$ and $T_{V^*\ot V} \to T_k$. Note that $T_k \simeq \Id$ and $T_{V\ot V'} \simeq T_V \circ T_{V'}$. Hence we obtain natural transformations $\Id \to T_V \circ T_{V^*}$ and $T_{V^*} \circ T_{V} \to \Id$. Now the duality data between $V$ and $V^*$ are translated exactly to the djunction data between $T_V$ and $T_{V^*}$.


	\qed

	\begin{rem}
		It follows formally that $T_{V^*}$ and $T_V$ are adjoint in the derived sense, i.e. $\Ext^i( T_V(M),N )\simeq \Ext^i( M, T_{V^*}(N) )$.
	\end{rem}

	\begin{lem}
		\label{lem-st-fil-trans}
		For any weight $\lambda$, the module $T_V(M_\lambda)$ admits a standard filtration $\mbF^{\le k}(T_V(M_\lambda))$ such that the highest weights of $\gr^k(T_V(M_\lambda))$ is (non-strictly) decreasing in $k$. Moreover,
		\[
			(T_V(M_\lambda): M_\mu) = \dim V^{\weight = \mu-\lambda}.
		\]
	\end{lem}

	\proof
		Follows from the \emph{projection formula} 
		\[
			\ind_\mfb^\mfg(V_1) \ot V_2 \simeq \ind_\mfb^\mfg( V_1 \ot V_2 ),  \; V_1\in \mfb\hmod, V_2\in \mfg\hmod.
		\]
		Indeed, any finite-dimensional weight $\mfb$-module, such as $V\otimes k_\lambda$, admits a finite filtration whose subquotients are 1-dimensional modules with decreasing weights.

	\qed

	\begin{constr}
		Let $\chi_1,\chi_2$ be two central characters. For any finite-dimensional $\mfg$-module $V$, consider the composition
		\[
			T_{\chi_1,V,\chi_2}: \mCO_{\chi_1} \to \mCO \xrightarrow{T_V} \mCO \to \mCO_{\chi_2}.
		\]
		We call such functors the \textbf{translation functors}. It follows that $T_{\chi_1,V,\chi_2}$ is exact, and is both left and right adjoint to $T_{\chi_2,V^*,\chi_1}$.
	\end{constr}

	\begin{constr}
		Let $\lambda$ and $\mu$ be weights such that $\mu - \lambda$ is integral. Then the $W$-orbit (for the linear action) $W(\mu-\lambda)$ contains a unique dominant integral weigth $\nu$. Consider the finite-dimensional $\mfg$-module $L_\nu$. We write
		\[
			T_\lambda^\mu:= T_{\varpi(\lambda), L_\nu, \varpi(\mu)} : \mCO_{\varpi(\lambda)} \to \mCO_{\varpi(\mu)}.
		\] 
		Note that $w_0(-\nu)$ is also dominant and integral. By definition, we have
		\[
			T_\mu^\lambda = T_{\varpi(\mu), L_{-w_0(\nu)}, \varpi(\lambda)},
		\]
		which is both left and right adjoint to $T_\lambda^\mu$ because $L_{-w_0(\nu)} \simeq L_\nu^*$\footnote{Here $L_\nu^*$ is the usual linear dual of $L_\mu$. It is not the contragradient dual. Note that $w_0(\nu)$ is indeed the lowest weight of $L_\mu$.}
	\end{constr}

	Recall the following definition:

	\begin{defn}
		Let $(E,\Phi)$ be the root system of $\mfg$. For $\lambda,\mu \in E$, we say they belong to the same \textbf{dot-Weyl facet} if the signs of $\langle \lambda+\rho , \check\alpha \rangle$ and $\langle \mu+\rho , \check\alpha \rangle$ are the same for any $\alpha\in \Phi$. 

		For $\lambda\in E$, let $F_\lambda$ be the dot-Weyl facet containing it.
	\end{defn}

	

	\begin{defn}
		For a dot-Weyl facet $F_\lambda$, its \textbf{upper closure} $F_\lambda^+$ is the subset of $\mu\in E$ such that
		\[
			\langle \mu+\rho,\check \alpha \rangle \left\{ 
			\begin{array}{rcl}  
						> 0	&	\textrm{if } \langle \lambda+\rho,\check \alpha \rangle>0,\\
						=0	&	\textrm{if } \langle \lambda+\rho,\check \alpha \rangle=0,\\
						\le 0	&	\textrm{if } \langle \lambda+\rho,\check \alpha \rangle<0
			\end{array}\right.
		\]
	\end{defn}

	\begin{rem}
		When equipped with the standard topology, $F_\lambda$ is a locally closed subset of $E$, and both the closure $\overline{F}_\lambda$ and the upper closure $F_\lambda^+$ are unions of dot-Weyl facets. 

		Note that each dot-Weyl facet is contained in the upper closure of a unique \textbf{dot-Weyl chamber}, i.e., open dot-Weyl facet. Also, $\lambda$ is dot-regular iff $F_\lambda$ is a dot-Weyl chamber.
	\end{rem}

	\begin{exam}
		For $\sl_2$ and the coordinate $l:=\langle \lambda,\check\alpha\rangle$, there are three facets: $(-\infty,-1)$, $\{-1\}$ and $(-1,\infty)$. Note that $\{-1\}$ is contained in the upper closure of $(-\infty,-1)$.

	\end{exam}

	We have the following theorem. For complete proofs and its generalization to non-integral weights, see \cite[Sect. 7]{H}\footnote{See \cite[Sect. 4.23]{G} for a simplified proof in a special case.}.

	\begin{thm}
		\label{thm-tranlation-st-irre}
		Let $\lambda$ and $\mu$ be dot-antidominant integral weights such that $F_\mu\subset \overline{F_\lambda}$. Then for any $w\in W$
		\[
			T_\lambda^\mu( M_{w\cdot \lambda} ) \simeq M_{w\cdot \mu},\; T_\lambda^\mu( M_{w\cdot \lambda}^\vee ) \simeq M_{w\cdot \mu}^\vee
		\]
		and
		\[	 T_\lambda^\mu( L_{w\cdot \lambda} ) \simeq
			\left\{ 
					\begin{array}{rcl}  
						L_{w\cdot \mu}	&	\textrm{if } F_{w\cdot \mu} \subset F_{w\cdot \lambda}^+ ,\\
						0	&	\textrm{otherwise}.
					\end{array}\right.
		\]
	\end{thm}

	\proof[Sketch]
		Let $\nu \in W(\mu-\lambda)$ be the unique dominant integral weight in this orbit. Write $V:=L_\nu$. By Lemma \ref{lem-st-fil-trans}, $T_\lambda^\mu( M_{w\cdot \lambda} )$ admits a standard filtration and the multiplicity 
		\[
			(T_\lambda^\mu( M_{w\cdot \lambda} ):  M_{w'\cdot \mu}) = \dim V^{\weight = w'\cdot \mu -w\cdot \lambda}.
		\]
		The RHS is nonzero unless $w'\cdot \mu -w\cdot \lambda\preceq \nu$. Now a combinatorial argument (see \cite[Lemma 7.5]{H}) shows the latter can happen only if $w'\cdot \mu = w\cdot \mu$, and in this case the multplicity is $1$ because $w\cdot \mu-w\cdot \lambda = w(\mu-\lambda) \in W(\nu)$. This implies $T_\lambda^\mu( M_{w\cdot \lambda} ) \simeq M_{w\cdot \mu}$. 

		The statement for dual Verma modules follows from the contragradient duality.

		Now consider the chain $M_{w\cdot \lambda} \surj  L_{w\cdot \lambda} \inj M_{w\cdot \lambda}^\vee$. Since $T_\lambda^\mu$ is exact, we obtain a chain $ M_{w\cdot \mu}\surj T_\lambda^\mu(L_{w\cdot \lambda}) \inj M_{w\cdot \mu}^\vee$. This forces $T_\lambda^\mu(L_{w\cdot \lambda})\simeq L_{w\cdot \mu}$ or $0$ ([Lemma 3.14, Lecture 8]). 

		It remains to pinpoint these two cases. Since the exact functor $T_\lambda^\mu$ sends $ M_{w\cdot \lambda} $ to $M_{w\cdot \mu}$, there exists a unique composition factor $L_{w'\cdot \lambda}$ of $M_{w\cdot \lambda} $ such that $T_\lambda^\mu(L_{w'\cdot \lambda})\simeq L_{w\cdot \mu}$. By the last paragraph, we must have $w'\cdot \mu = w\cdot \mu$. Now we have two cases:
		\begin{itemize}
			\item 
				If $F_{w\cdot \mu}$ is contained in the upper closure $F_{w\cdot \lambda}^+$, then a combinatorial argument shows $w'\cdot \lambda = w\cdot \lambda$ and therefore $T_\lambda^\mu(L_{w\cdot \lambda})\simeq L_{w\cdot \mu}$ as desired.
			\item
				If $F_{w\cdot \mu}$ is not contained in the upper closure $F_{w\cdot \lambda}^+$, then there exists a reflection $s_\alpha$ such that $s_{\alpha}w\cdot \lambda\prec w\cdot\lambda$ while $s_{\alpha}w\cdot \mu = w\cdot\mu$. By Verma's theorem, there exists a proper embedding $M_{s_{\alpha}w\cdot \lambda} \subset M_{w\cdot\lambda}$ which is sent to an isomorphism by $T_\lambda^\mu$. Hence $L_{w\cdot \lambda}$, which is a quotient of $M_{w\cdot\lambda}/M_{s_{\alpha}w\cdot \lambda} $, is sent to 0 as desired.
		\end{itemize}

	\qed

	The following result is a formal consequence of the above theorem:

	\begin{thm}
		\label{thm-tranlation-equiv}
		Let $\lambda$ and $\mu$ be dot-antidominant integral weights such that $F_\lambda=F_\mu$. Then $T_\lambda^\mu: \mCO_{\varpi(\lambda)} \to  \mCO_{\varpi(\mu)}$ is an equivalence.
	\end{thm}

	\proof
		By the previous theorem, the exact functor $F:=T_\lambda^\mu$ is left adjoint to the exact functor $G:=T_\mu^\lambda$, and they induce a bijection between the sets of irreducible objects. In general, such adjoint functors are inverse to each other. 

		Proof: we only to show the adjunctions $\Id \to G\circ F$ and $F\circ G \to \Id$ are equivalences. We will prove the first equivalence and the second follows similarly. Note that for any object $M$, $M$ and $G\circ F(M)$ have the same composition factors and multiplicities. Hence we only need to show $M \to G\circ F(M)$ is injective. Since $F$ sends nonzero objects to nonzero objects, we only need to show $F(M) \to F\circ G\circ F(M)$ is injective. By the axiom of adjoint functors, this morphism has a left inverse, hence is indeed injective.

	\qed


	\begin{rem}
		The above theorems essentially reduce the study about any integral block $\mCO_\chi$ to the principle block $\mCO_{\varpi(0)}$.
	\end{rem}

	\begin{exam}
		Any dot-regular integral block $\mCO_\chi$ is equivalent to the principle block $\mCO_{\varpi(0)}$. Indeed, this is the special case of the above theorem when $\lambda=0$ and $\mu\in \varpi^{-1}(\chi)$ is dot-antidominant. Note that the equivalence $T_0^\mu$ preserves (dual) Verma modules and irreducible modules. 
	\end{exam}

	\begin{exam}
		For $\mfg=\sl_2$ and the coordinate $l:=\langle \lambda, \check \alpha \rangle$, then $\mCO_{\varpi(0)} \simeq \mCO_{\varpi(l)}$ for $l\in \mBZ^{\ge 0}$ such that the short exact sequence $0\to M_{-2} \to M_0 \to L_0 \to 0$ is sent to $0\to M_{-l-2} \to M_l \to L_l \to 0$. 

		On the other hand, the translation functor $\mCO_{\varpi(0)} \to \mCO_{\varpi(-1)}$ sends this sequence to $0 \to M_{-1} \to M_{-1} \to 0 \to 0$. Note that $L_{-2}$ is sent to $L_{-1}$ and $-1 \in F_{-2}^+$, while $L_{0}$ is sent to $0$ and $-1 \notin F_{0}^+$.
	\end{exam}


	\begin{constr}
		Let $\mu$ be any dot-antidominant integral weight. The functor $T_{-\rho}^\mu:\mCO_{\varpi(-\rho)} \to \mCO_{\varpi(\mu)}$ is \emph{not} covered by the above theorems (although its adjoint $T_\mu^{-\rho}$ is). Recall the most singular block $\mCO_{\varpi(-\rho)}$ is semi-simple and contains a unique irreducible object $L_{-\rho}$, which is both projective and injective ([Example 4.16, Lecture 7]). It follows that 
		\[
			\Xi_{\mu}:= T_{-\rho}^\mu(L_{-\rho})
		\]
		is both projective and injective.
	\end{constr}

	\begin{exe}
		This is \red{Homework 4, Problem 5}. Let $\mu$ be any dot-antidominant integral weight. Prove\footnote{Hint: Lemma \ref{lem-st-fil-trans} for (1); Theorem \ref{thm-BGG-rec} and [Corollary 4.15, Lecture 7] for (2). For (3), first find a surjection $\Xi_\mu \surj P_\mu$ then use (1) and (2).}:
		\begin{itemize}
			\item[(1)]
				For any $w\in W$, $(\Xi_\mu: M_{w\cdot\mu})=1$ and there is a surjection $\Xi_\mu \surj M_\mu$.
			\item[(2)]
				For any $w\in W$, $(P_\mu: M_{w\cdot\mu})\ge 1$ and there is a surjection $P_\mu \surj M_\mu$.
			\item[(3)]
				There exists an isomorphism $\Xi_\mu \simeq P_\mu$ compatible with the surjections to $M_\mu$.
			\item[(4)]
				For any $w\in W$, $[M_{w\cdot \mu}: L_\mu]=1$\footnote{See \cite[Proposition 4.20]{G} for a different proof of this fact.}.
		\end{itemize}

	\end{exe}

	\begin{rem}
		For dot-antidominant weight $\mu$, the projective $P_\mu$ is called the \textbf{big projective module}, which plays an important role in representation theory.
	\end{rem}
		




\begin{thebibliography}{Yau}

	\bibitem[BBM]{BBM} Beilinson, A., R. Bezrukavnikov, and I. Mirkovic. Tilting Exercises. Moscow Mathematical Journal 4, no. 3 (2004): 547-557.

	\bibitem[CPS]{CPS} Scott, L., Brian Parshall, and E. Cline. Finite dimensional algebras and highest weight categories. (1988): 85-99.

	\bibitem[G]{G} Gaitsgory, Dennis. Course Notes for Geometric Representation Theory, 2005, available at \url{https://people.mpim-bonn.mpg.de/gaitsgde/267y/catO.pdf}.



	\bibitem[H]{H} Humphreys, James E. Representations of Semisimple Lie Algebras in the BGG Category $\mathcal{O} $. Vol. 94. American Mathematical Soc., 2008.
\end{thebibliography}


\end{document} 


