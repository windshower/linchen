
%!TEX root = main.tex
\documentclass{amsart}
\textwidth=14.5cm \oddsidemargin=1cm
\evensidemargin=1cm
\usepackage{amsmath}
\usepackage{amsxtra}
\usepackage{amscd}
\usepackage{amsthm}
\usepackage{amsfonts}
\usepackage{amssymb}
\usepackage[foot]{amsaddr}
\usepackage{cite}
\usepackage{url}
\usepackage{rotating}
\usepackage{eucal}
\usepackage{tikz-cd}
\usepackage[all,2cell,color]{xy}
\UseAllTwocells
\UseCrayolaColors
\usepackage{graphicx}
\usepackage{pifont}
\usepackage{comment}
\usepackage{verbatim}
\usepackage{xcolor}
\usepackage{hyperref}
\usepackage{xparse}
\usepackage{upgreek}
\usepackage{MnSymbol}
\sloppy


%%%%%%%%%%%%%%%%%%%%Theorem%%%%%%%%%%%%%%%%%%%%
\newcounter{theorem}
\setcounter{theorem}{0}

\newtheorem{cor}[subsection]{Corollary}
\newtheorem{lem}[subsection]{Lemma}
\newtheorem{goal}[subsection]{Goal}
\newtheorem{lemdefn}[subsection]{Lemma-Definition}
\newtheorem{prop}[subsection]{Proposition}
\newtheorem{propdefn}[subsection]{Proposition-Definition}
\newtheorem{cordefn}[subsection]{Corollary-Definition}
\newtheorem{variant}[subsection]{Variant}
\newtheorem{warn}[subsection]{Warning}
\newtheorem{sugg}[subsection]{Suggestion}
\newtheorem{facts}[subsection]{Fact}
\newtheorem{ques}{Question}
\newtheorem{guess}{Guess}
\newtheorem{claim}{Claim}
\newtheorem{propconstr}[subsection]{Proposition-Construction}
\newtheorem{lemconstr}[subsection]{Lemma-Construction}
\newtheorem{ax}{Axiom}
\newtheorem{conje}[subsection]{Conjecture}
\newtheorem{mainthm}[subsection]{Main-Theorem}
\newtheorem{summ}[subsection]{Summary}
\newtheorem{thm}[subsection]{Theorem}
\newtheorem{thmdefn}[subsection]{Theorem-Definition}
\newtheorem{notn}[subsection]{Notation}
\newtheorem{convn}[subsection]{Convention}
\newtheorem{constr}[subsection]{Construction}


\theoremstyle{definition}

\newtheorem{defn}[subsection]{Definition}
\newtheorem{exam}[subsection]{Example}
\newtheorem{assum}[subsection]{Assumption}

\theoremstyle{remark}
\newtheorem{rem}[subsection]{Remark}
\newtheorem{exe}[subsection]{Exercise}


\numberwithin{equation}{section}


%%%%%%%%%%%%%%%%%%%%Commands%%%%%%%%%%%%%%%%%%%%

\newcommand{\nc}{\newcommand}
\nc\on{\operatorname}
\nc\renc{\renewcommand}


%%%%%%%%%%%%%%%%%%%%Sections%%%%%%%%%%%%%%%%%%%%

\nc\ssec{\subsection}
\nc\sssec{\subsubsection}

%%%%%%%%%%%%%%%%%%%%Environment%%%%%%%%%%%%%%%%%
\nc\blongeqn{\[ \begin{aligned}}
\nc\elongeqn{\end{aligned} \]}



%%%%%%%%%%%%%%%%%%%%Mathfont%%%%%%%%%%%%%%%%%%%%

\nc\mBA{{\mathbb A}}
\nc\mBB{{\mathbb B}}
\nc\mBC{{\mathbb C}}
\nc\mBD{{\mathbb D}}
\nc\mBE{{\mathbb E}}
\nc\mBF{{\mathbb F}}
\nc\mBG{{\mathbb G}}
\nc\mBH{{\mathbb H}}
\nc\mBI{{\mathbb I}}
\nc\mBJ{{\mathbb J}}
\nc\mBK{{\mathbb K}}
\nc\mBL{{\mathbb L}}
\nc\mBM{{\mathbb M}}
\nc\mBN{{\mathbb N}}
\nc\mBO{{\mathbb O}}
\nc\mBP{{\mathbb P}}
\nc\mBQ{{\mathbb Q}}
\nc\mBR{{\mathbb R}}
\nc\mBS{{\mathbb S}}
\nc\mBT{{\mathbb T}}
\nc\mBU{{\mathbb U}}
\nc\mBV{{\mathbb V}}
\nc\mBW{{\mathbb W}}
\nc\mBX{{\mathbb X}}
\nc\mBY{{\mathbb Y}}
\nc\mBZ{{\mathbb Z}}


\nc\mCA{{\mathcal A}}
\nc\mCB{{\mathcal B}}
\nc\mCC{{\mathcal C}}
\nc\mCD{{\mathcal D}}
\nc\mCE{{\mathcal E}}
\nc\mCF{{\mathcal F}}
\nc\mCG{{\mathcal G}}
\nc\mCH{{\mathcal H}}
\nc\mCI{{\mathcal I}}
\nc\mCJ{{\mathcal J}}
\nc\mCK{{\mathcal K}}
\nc\mCL{{\mathcal L}}
\nc\mCM{{\mathcal M}}
\nc\mCN{{\mathcal N}}
\nc\mCO{{\mathcal O}}
\nc\mCP{{\mathcal P}}
\nc\mCQ{{\mathcal Q}}
\nc\mCR{{\mathcal R}}
\nc\mCS{{\mathcal S}}
\nc\mCT{{\mathcal T}}
\nc\mCU{{\mathcal U}}
\nc\mCV{{\mathcal V}}
\nc\mCW{{\mathcal W}}
\nc\mCX{{\mathcal X}}
\nc\mCY{{\mathcal Y}}
\nc\mCZ{{\mathcal Z}}


\nc\mbA{{\mathsf A}}
\nc\mbB{{\mathsf B}}
\nc\mbC{{\mathsf C}}
\nc\mbD{{\mathsf D}}
\nc\mbE{{\mathsf E}}
\nc\mbF{{\mathsf F}}
\nc\mbG{{\mathsf G}}
\nc\mbH{{\mathsf H}}
\nc\mbI{{\mathsf I}}
\nc\mbJ{{\mathsf J}}
\nc\mbK{{\mathsf K}}
\nc\mbL{{\mathsf L}}
\nc\mbM{{\mathsf M}}
\nc\mbN{{\mathsf N}}
\nc\mbO{{\mathsf O}}
\nc\mbP{{\mathsf P}}
\nc\mbQ{{\mathsf Q}}
\nc\mbR{{\mathsf R}}
\nc\mbS{{\mathsf S}}
\nc\mbT{{\mathsf T}}
\nc\mbU{{\mathsf U}}
\nc\mbV{{\mathsf V}}
\nc\mbW{{\mathsf W}}
\nc\mbX{{\mathsf X}}
\nc\mbY{{\mathsf Y}}
\nc\mbZ{{\mathsf Z}}

\nc\mba{{\mathsf a}}
\nc\mbb{{\mathsf b}}
\nc\mbc{{\mathsf c}}
\nc\mbd{{\mathsf d}}
\nc\mbe{{\mathsf e}}
\nc\mbf{{\mathsf f}}
\nc\mbg{{\mathsf g}}
\nc\mbh{{\mathsf h}}
\nc\mbi{{\mathsf i}}
\nc\mbj{{\mathsf j}}
\nc\mbk{{\mathsf k}}
\nc\mbl{{\mathsf l}}
\nc\mbm{{\mathsf m}}
\nc\mbn{{\mathsf n}}
\nc\mbo{{\mathsf o}}
\nc\mbp{{\mathsf p}}
\nc\mbq{{\mathsf q}}
\nc\mbr{{\mathsf r}}
\nc\mbs{{\mathsf s}}
\nc\mbt{{\mathsf t}}
\nc\mbu{{\mathsf u}}
\nc\mbv{{\mathsf v}}
\nc\mbw{{\mathsf w}}
\nc\mbx{{\mathsf x}}
\nc\mby{{\mathsf y}}
\nc\mbz{{\mathsf z}}




\nc\mbfA{{\mathbf A}}
\nc\mbfB{{\mathbf B}}
\nc\mbfC{{\mathbf C}}
\nc\mbfD{{\mathbf D}}
\nc\mbfE{{\mathbf E}}
\nc\mbfF{{\mathbf F}}
\nc\mbfG{{\mathbf G}}
\nc\mbfH{{\mathbf H}}
\nc\mbfI{{\mathbf I}}
\nc\mbfJ{{\mathbf J}}
\nc\mbfK{{\mathbf K}}
\nc\mbfL{{\mathbf L}}
\nc\mbfM{{\mathbf M}}
\nc\mbfN{{\mathbf N}}
\nc\mbfO{{\mathbf O}}
\nc\mbfP{{\mathbf P}}
\nc\mbfQ{{\mathbf Q}}
\nc\mbfR{{\mathbf R}}
\nc\mbfS{{\mathbf S}}
\nc\mbfT{{\mathbf T}}
\nc\mbfU{{\mathbf U}}
\nc\mbfV{{\mathbf V}}
\nc\mbfW{{\mathbf W}}
\nc\mbfX{{\mathbf X}}
\nc\mbfY{{\mathbf Y}}
\nc\mbfZ{{\mathbf Z}}

\nc\mbfa{{\mathbf a}}
\nc\mbfb{{\mathbf b}}
\nc\mbfc{{\mathbf c}}
\nc\mbfd{{\mathbf d}}
\nc\mbfe{{\mathbf e}}
\nc\mbff{{\mathbf f}}
\nc\mbfg{{\mathbf g}}
\nc\mbfh{{\mathbf h}}
\nc\mbfi{{\mathbf i}}
\nc\mbfj{{\mathbf j}}
\nc\mbfk{{\mathbf k}}
\nc\mbfl{{\mathbf l}}
\nc\mbfm{{\mathbf m}}
\nc\mbfn{{\mathbf n}}
\nc\mbfo{{\mathbf o}}
\nc\mbfp{{\mathbf p}}
\nc\mbfq{{\mathbf q}}
\nc\mbfr{{\mathbf r}}
\nc\mbfs{{\mathbf s}}
\nc\mbft{{\mathbf t}}
\nc\mbfu{{\mathbf u}}
\nc\mbfv{{\mathbf v}}
\nc\mbfw{{\mathbf w}}
\nc\mbfx{{\mathbf x}}
\nc\mbfy{{\mathbf y}}
\nc\mbfz{{\mathbf z}}

\nc\mfa{{\mathfrak a}}
\nc\mfb{{\mathfrak b}}
\nc\mfc{{\mathfrak c}}
\nc\mfd{{\mathfrak d}}
\nc\mfe{{\mathfrak e}}
\nc\mff{{\mathfrak f}}
\nc\mfg{{\mathfrak g}}
\nc\mfh{{\mathfrak h}}
\nc\mfi{{\mathfrak i}}
\nc\mfj{{\mathfrak j}}
\nc\mfk{{\mathfrak k}}
\nc\mfl{{\mathfrak l}}
\nc\mfm{{\mathfrak m}}
\nc\mfn{{\mathfrak n}}
\nc\mfo{{\mathfrak o}}
\nc\mfp{{\mathfrak p}}
\nc\mfq{{\mathfrak q}}
\nc\mfr{{\mathfrak r}}
\nc\mfs{{\mathfrak s}}
\nc\mft{{\mathfrak t}}
\nc\mfu{{\mathfrak u}}
\nc\mfv{{\mathfrak v}}
\nc\mfw{{\mathfrak w}}
\nc\mfx{{\mathfrak x}}
\nc\mfy{{\mathfrak y}}
\nc\mfz{{\mathfrak z}}

\nc{\one}{{\mathsf{1}}}


\nc\clambda{ {\check{\lambda} }}
\nc\cmu{ {\check{\mu} }}

\nc\bDelta{\mathsf{\Delta}}
\nc\bGamma{\mathsf{\Gamma}}
\nc\bLambda{\mathsf{\Lambda}}


\nc\loccit{\emph{loc.cit.}}



%%%%%%%%%%%%%%%%%%%%Operations-limit%%%%%%%%%%%%%%%%%%%%

\NewDocumentCommand{\ot}{e{_^}}{
  \mathbin{\mathop{\otimes}\displaylimits
    \IfValueT{#1}{_{#1}}
    \IfValueT{#2}{^{#2}}
  }
}
\NewDocumentCommand{\boxt}{e{_^}}{
  \mathbin{\mathop{\boxtimes}\displaylimits
    \IfValueT{#1}{_{#1}}
    \IfValueT{#2}{^{#2}}
  }
}
\NewDocumentCommand{\mt}{e{_^}}{
  \mathbin{\mathop{\times}\displaylimits
    \IfValueT{#1}{_{#1}}
    \IfValueT{#2}{^{#2}}
  }
}
\NewDocumentCommand{\convolve}{e{_^}}{
  \mathbin{\mathop{\star}\displaylimits
    \IfValueT{#1}{_{#1}}
    \IfValueT{#2}{^{#2}}
  }
}
\NewDocumentCommand{\colim}{e{_^}}{
  \mathbin{\mathop{\operatorname{colim}}\displaylimits
    \IfValueT{#1}{_{#1}\,}
    \IfValueT{#2}{^{#2}\,}
  }
}
\NewDocumentCommand{\laxlim}{e{_^}}{
  \mathbin{\mathop{\operatorname{laxlim}}\displaylimits
    \IfValueT{#1}{_{#1}\,}
    \IfValueT{#2}{^{#2}\,}
  }
}
\NewDocumentCommand{\oplaxlim}{e{_^}}{
  \mathbin{\mathop\operatorname{oplax-lim}\displaylimits
    \IfValueT{#1}{_{#1}\,}
    \IfValueT{#2}{^{#2}\,}
  }
}


%%%%%%%%%%%%%%%%%%%%Arrows%%%%%%%%%%%%%%%%%%%%


\makeatletter
\newcommand{\laxto}{\dashedrightarrow}
\newcommand{\xrightleftarrows}[1]{\mathrel{\substack{\xrightarrow{#1} \\[-.9ex] \xleftarrow{#1}}}}
\newcommand{\adj}{\xrightleftarrows{\rule{0.5cm}{0cm}}}

\newcommand*{\da@rightarrow}{\mathchar"0\hexnumber@\symAMSa 4B }
\newcommand*{\da@leftarrow}{\mathchar"0\hexnumber@\symAMSa 4C }
\newcommand*{\xlaxto}[2][]{%
  \mathrel{%
    \mathpalette{\da@xarrow{#1}{#2}{}\da@rightarrow{\,}{}}{}%
  }%
}
\newcommand{\xlaxgets}[2][]{%
  \mathrel{%
    \mathpalette{\da@xarrow{#1}{#2}\da@leftarrow{}{}{\,}}{}%
  }%
}
\newcommand*{\da@xarrow}[7]{%
  % #1: below
  % #2: above
  % #3: arrow left
  % #4: arrow right
  % #5: space left 
  % #6: space right
  % #7: math style 
  \sbox0{$\ifx#7\scriptstyle\scriptscriptstyle\else\scriptstyle\fi#5#1#6\m@th$}%
  \sbox2{$\ifx#7\scriptstyle\scriptscriptstyle\else\scriptstyle\fi#5#2#6\m@th$}%
  \sbox4{$#7\dabar@\m@th$}%
  \dimen@=\wd0 %
  \ifdim\wd2 >\dimen@
    \dimen@=\wd2 %   
  \fi
  \count@=2 %
  \def\da@bars{\dabar@\dabar@}%
  \@whiledim\count@\wd4<\dimen@\do{%
    \advance\count@\@ne
    \expandafter\def\expandafter\da@bars\expandafter{%
      \da@bars
      \dabar@ 
    }%
  }%  
  \mathrel{#3}%
  \mathrel{%   
    \mathop{\da@bars}\limits
    \ifx\\#1\\%
    \else
      _{\copy0}%
    \fi
    \ifx\\#2\\%
    \else
      ^{\copy2}%
    \fi
  }%   
  \mathrel{#4}%
}
\makeatother

%%%%%%%%%%%%%%%%%%%%Decorations%%%%%%%%%%%%%%%%%%%%
\nc{\wt}{\widetilde}
\nc{\ol}{\overline}

\nc{\red}{\textcolor{red}}
\nc{\blue}{\textcolor{blue}}
\nc{\purple}{\textcolor{violet}}

\nc{\simorlax}{{\red\simeq/\blue\lax}}

%%%%%%%%%%%%%%%%%%%%All%%%%%%%%%%%%%%%%%%%%

\nc{\Id}{\mathsf{Id}}
\nc{\gl}{\mathfrak{gl}}
\renc{\sl}{\mathfrak{sl}}
\nc{\GL}{\mathsf{GL}}
\nc{\SL}{\mathsf{SL}}
\nc{\PGL}{\mathsf{PGL}}
\nc{\hmod}{\mathsf{-mod}}
\nc{\Vect}{\mathsf{Vect}}
\nc{\tr}{\mathsf{tr}}
\nc{\Kil}{\mathsf{Kil}}
\nc{\ad}{{\mathsf{ad}}}
\nc{\Ad}{\mathsf{Ad}}
\nc{\oblv}{\mathsf{oblv}}
\nc{\gr}{\mathsf{gr}}
\nc{\Sym}{\mathsf{Sym}}
\nc{\QCoh}{\mathsf{QCoh}}
\nc{\ind}{\mathsf{ind}}
\nc{\Spec}{\mathsf{Spec}}
\nc{\Hom}{\mathsf{Hom}}
\nc{\Ext}{\mathsf{Ext}}
\nc{\Grp}{\mathsf{Grp}}
\nc{\pt}{\mathsf{pt}}
\nc{\Lie}{\mathsf{Lie}}
\nc{\CAlg}{\mathsf{CAlg}}
\nc{\Der}{\mathsf{Der}}
\nc{\Rep}{\mathsf{Rep}}
\renc{\sc}{{\mathsf{sc}}}
\nc{\Fl}{\mathsf{Fl}}
\nc{\Fun}{\mathsf{Fun}}
\nc{\ev}{\mathsf{ev}}
\nc{\surj}{\twoheadrightarrow}
\nc{\inj}{\hookrightarrow}
\nc{\HC}{\mathsf{HC}}
\nc{\cl}{\mathsf{cl}}
\renc{\Im}{\mathsf{Im}}
\renc{\ker}{\mathsf{ker}}
\nc{\coker}{\mathsf{coker}}
\nc{\Tor}{\mathsf{Tor}}
\nc{\op}{\mathsf{op}}
\nc{\length}{\mathsf{length}}
\nc{\fd}{{\mathsf{fd}}}
\nc{\weight}{\mathsf{wt}}
\nc{\semis}{{\mathsf{ss}}}
\nc{\qc}{{\mathsf{qc}}}
\nc{\pr}{\mathsf{pr}}
\nc{\act}{\mathsf{act}}
\nc{\dR}{{\mathsf{dR}}}
\nc{\hol}{{\mathsf{hol}}}
\nc{\Pic}{{\mathsf{Pic}}}
\nc{\Loc}{\mathsf{Loc}}
\nc{\IC}{\mathsf{IC}}

\begin{document}


\title{Lecture 1}

\date{Feb 26, 2024}

\maketitle

The main goal of this course is to study representations of semisimple Lie algebras via geometric methods. We restrict ourselves to the case when the base field $k$ is algebraically closed and of characteristic $0$, such as the field $\mBC$ of complex numbers.


\section{Semisimple Lie Algebras}
This is just a quick review of the definitions about finite-dimensional semisimple Lie algebras. See \cite[Chapter 0]{Hum} for the abc's and \cite{Ser} for a thorough textbook.

\begin{defn}
	A \textbf{Lie algebra} (over $k$) is a vector space $\mfg$ equipped with a binary operation $[-,-]:\mfg\times \mfg \to \mfg$, called the \textbf{Lie bracket}, such that:
	\begin{itemize}
		\item
			The Lie bracket is \textbf{bilinear}, i.e., factors as $\mfg\times \mfg \to \mfg\otimes\mfg \to \mfg$.
		\item 
			The Lie bracket is \textbf{alternating}: $[x,x]=0$.
		\item
			The \textbf{Jacobi identity} holds: $[x,[y,z]]+[y,[z,x]]+[z,[x,y]]=0$.
	\end{itemize}

	Let $\mfg_1$ and $\mfg_2$ be Lie algebras. A \textbf{Lie algebra homomorphism} between them is a $k$-linear map $f: \mfg_1\to \mfg_2$ commuting with Lie brackets, i.e., $f([x,y]) = [f(x),f(y)]$.

	This defines a category $\mathsf{Lie}_k$ of Lie algebras.
\end{defn}


\begin{exam}
	Any vector space $V$ is equipped with a trivial Lie bracket: $[x,y]=0$. Such Lie algebras are called \textbf{abelian Lie algebras}.
\end{exam}

\begin{exam}
	\label{exam-assoc-to-Lie}
	Let $A$ be an associative algebra. Then the underlying vector space has a natural Lie algebra structure with Lie bracket given by $[x,y]:= xy-yx$. This defines a functor $\mathsf{oblv}:\mathsf{Alg}_k \to \mathsf{Lie}_k$ from the category of associative algebras to that of Lie algebras.
\end{exam}

\begin{exam}
	Let $V$ be a vector space and $\gl(V)$ be the vector space of endomorphisms of $V$. By Example \ref{exam-assoc-to-Lie}, $\gl(V)$ is naturally a Lie algebra with Lie bracket given by $[f,g]=f\circ g-g\circ f$. This is the \textbf{general linear Lie algebra} of $V$. 

	If $V$ is finite-dimensional, let $\sl(V)\subset \gl(V)$ be the subspace of endomorphisms $f$ such that the trace $\tr(f)=0$.

	When $V=k^{\oplus n}$, we write $\gl_n:= \gl(V)$, $\sl_n:=\sl(V)$. Note that $\gl_n$ (resp. $\sl_n$) can be identified with the space of $n\times n$ matrices (resp. whose traces are zero).
\end{exam}

\begin{facts}
	We have $[\gl(V),\gl(V)] = \sl(V)$.
\end{facts}

\begin{defn}
	Let $\mfg$ be a Lie algebra. A \textbf{representation} of $\mfg$, or \textbf{$\mfg$-module}, is a vector space $V$ equipped with a Lie algebra homomorphism $\mfg \to \gl(V)$. In other words, there is a bilinear map $(-\cdot-):\mfg \times V \to V$ such that $[x,y]\cdot v = x\cdot (y\cdot v) - y\cdot(x\cdot v)$.

	Let $V_1$ and $V_2$ be representations of $\mfg$. A $\mfg$-linear map between them is a $k$-linear map $f:V_1\to V_2$ such that the following diagram commutes:
	\[
		\xymatrix{
			\mfg \times V_1 
				\ar[r] \ar[d]_-{\Id\times f}
			& V_1 
				\ar[d]^-f \\
			\mfg \times V_2
				\ar[r] 
			& V_2.
		}
	\]

	This defines a category $\mfg\hmod$ of representations of $\mfg$.
\end{defn}

\begin{facts}
	The category $\mfg\hmod$ is an abelian category. The forgetful functor $\mfg\hmod\to \Vect_k$ is exact.
\end{facts}

\begin{exam}
	Let $\mfg$ be a Lie algebra. The map $\ad: \mfg \to \gl(\mfg)$, $x\mapsto \ad_x:=[x,-]$ defines a $\mfg$-module structure on $\mfg$ itself. This is called the \textbf{adjoint representation}. 
\end{exam}

\begin{defn}
	Let $\mfg$ be a finite dimensional Lie algebra and $x\in \mfg$ be an element. We say $x$ is \textbf{semisimple} (resp. \textbf{nilpotent}) if the corresponding endomorphism $\ad_x: \mfg \to \mfg$ is so.
\end{defn}



\begin{defn}
	Let $\mfg$ be a Lie algebra. An \textbf{ideal} $\mfa \subset \mfg$ is a sub-representation of the adjoint representation. In other words, we require $[\mfg,\mfa] \subset \mfa$.
\end{defn}

\begin{rem}
	Note that an ideal $\mfa$ is also a Lie subalgebra.
\end{rem}

\begin{exam}
	Let $\mfg$ be a Lie algebra, then $[\mfg,\mfg]\subset \mfg$ is an ideal. We call it the \textbf{derived Lie algebra} of $\mfg$.
\end{exam}

\begin{defn}
	Let $\mfg$ be a Lie algebra. We say $\mfg$ is \textbf{simple} if:
	\begin{itemize}
		\item It is not abelian;
		\item The adjoint representation is simple (a.k.a. irreducible), i.e., $\mfg$ has no ideal other than $0$ and itself.
	\end{itemize}
\end{defn}

\begin{exam}
	The Lie algebra $\gl_n$ is not simple because $[\gl_n,\gl_n] = \sl_n$ is a proper ideal of it. The Lie algebra $\sl_n$ is simple for $n\ge 2$.
\end{exam}

\begin{rem}
	Finite-dimensional simple Lie algebras (over $k$) are fully classified. A similar classification for infinite-dimensional simple Lie algebras seems to be hopeless.
\end{rem}

\begin{defn}
	Let $\mfg_i$, $i\in I$ be Lie algebras indexed by a set $I$. The direct sum $\oplus \mfg_i$ of the underlying vector spaces has a natural Lie bracket given by $[(x_i)_{i\in I},(y_i)_{i\in I}]:=([x_i,y_i])_{i\in I}$. The obtained Lie algebra is called the \textbf{direct sum} of the Lie algebras $\mfg_i$.
\end{defn}

\begin{warn}
	The direct sum $\oplus \mfg_i$ is \emph{not} the coproduct in the category $\mathsf{Lie}_k$. Instead, if $I$ is a finite set, then it is the \emph{product} of $\mfg_i$ in this category.
\end{warn}

\begin{rem}
	Representation theory for $\mfg_1\oplus \mfg_2$ can be obtained from those for $\mfg_1$ and $\mfg_2$ in a non-trivial mechanism\footnote{The abelian category $(\mfg_1\oplus \mfg_2)\hmod$ is the \emph{tensor product} of $\mfg_1\hmod$ and $\mfg_2\hmod$.}.
\end{rem}

\begin{defn}
	Let $\mfg$ be a Lie algebra. We say $\mfg$ is \textbf{semisimple} if it is a direct sum of simple Lie algebras.
\end{defn}

\begin{rem}
	The zero Lie algebra $0$ is semisimple but not simple.
\end{rem}

The main goal of this course is to study representations of finite-dimensional semisimple Lie algebras.

\begin{convn}
	From now on, unless otherwise stated, Lie algebras are assumed to be finite-dimensional.
\end{convn}

\begin{exe}
	\red{This is not a homework!}
	\begin{itemize}
		\item[(1)]
			Let $\mfg$ be a finite-dimensional semisimple Lie algebra. Show $[\mfg,\mfg]=\mfg$.
		\item
			The opposite statement is generally \emph{not} true. Below is a counterexample. Let $\mfh$ be a simple Lie algebra and $V$ be a nontrivial simple $\mfh$-module. Define a bracket on the vector space by the formula $\mfh\oplus V$ by $[(x,u),(y,v)]:=([x,y],x\cdot v-y\cdot u)$.
		\item[(2)]
			Show this bracket defines a Lie algebra structure on $\mfh\oplus V$. We denote this Lie algebra by $\mfh\ltimes V$.
		\item[(3)]
			Show $[\mfh\ltimes V, \mfh\ltimes V]=\mfh\ltimes V$ but $\mfh\ltimes V$ is not semisimple.
	\end{itemize}
\end{exe}

\section{Root Space Decomposition}



\begin{convn}
	From now on, unless otherwise stated, $\mfg$ means a finite-dimensional semisimple Lie algebra.
\end{convn}

\begin{defn}
	A \textbf{Cartan subalgebra} $\mft$ of $\mfg$ is a maximal abelian subalgebra of it consisting of semisimple elements.
\end{defn}

\begin{warn}
	A maximal abelian subalgebra of $\mfg$ might not be a Cartan subalgebra. For example, $\sl_2$ contains a maximal abelian subalgebra spanned by $e:=\big(\begin{smallmatrix} 0 & 1\\ 0 & 0
	\end{smallmatrix}\big)$ but $e$ is not semisimple\footnote{
		I made this mistake during the lecture.}.


\end{warn}

\begin{warn}
	Cartan subalgebras for general finite-dimensional Lie algebras are defined in a different way and they are not abelian in general. That definition is equivalent to the above one if $\mfg$ is semisimple.
\end{warn}

\begin{thm}
	Cartan subalgebras of $\mfg$ have a same dimension, which is called the \textbf{(semisimple) rank} of $\mfg$\footnote{
		In fact, Cartan subalgebras are all conjugate to each other by a (non-unique) element of the corresponding Lie group $G$ of $\mfg$.
	}. 
\end{thm}

\begin{exam}
	\label{exam-Cartan-sln}
	The rank of $\sl_n$ is $n-1$. One Cartan subalgebra of it is the subspace of diagonal matrices.
\end{exam}

\begin{notn}
	From now on, we fix a Cartan subalgebra $\mft$ of $\mfg$. Let $\mft^*:=\mathsf{Hom}(\mft,k)$ be the dual vector space of $\mft$. For any $\alpha\in \mft^*$, let $\mfg_\alpha \subset \mfg$ be the $\alpha$-eigenspace for the adjoint $\mft$-action on $\mfg$, i.e.,
	\[
		\mfg_\alpha:= \{ x\in \mfg\; \vert\; [h,x] = \alpha(h)x \textrm{ for any }h\in \mft\}.
	\]
\end{notn}

\begin{rem}
	Note that $\mfg_0 = \mft$ (because $\mft$ is maximal) and $[\mfg_\alpha,\mfg_\beta] \subset \mfg_{\alpha+\beta}$ (because of the Jacobi identity).
\end{rem}

\begin{prop}[Root Space Decomposition\footnote{Some authors, including Humphreys, prefer the name \emph{Cartan decomposition}. But there is a completely different Cartan decomposition in the study of real Lie algerbas.}]
	Let $\mfg$ be a finite-dimensional semisimple Lie algebra with a fixed Cartan subalgebra $\mft$. Then we have a direct sum decomposition
	\[
		\mfg = \mft \oplus \bigoplus_{\alpha\in \Phi} \mfg_\alpha,
	\]
	where $\Phi\subset \mft^*\setminus 0$ is the finite set containing those nonzero $\alpha$ such that $\mfg_\alpha$ is nonempty. Moreover, if $\alpha\in \Phi$, then $-\alpha \in \Phi$ and $\mfg_\alpha$ is 1-dimensional.
\end{prop}

\begin{prop}
	There exists a (non-unique) subset $\Phi^+ \subset \Phi$ such that:
	\begin{itemize}
		\item 
			We have a disjoint decomposition $\Phi = \Phi^+ \sqcup -\Phi^+$;
		\item
			If $\alpha,\beta\in \Phi^+$ and $\alpha+\beta\in \Phi$\footnote{I forgot to mention this condition in the class.
			}, then $\alpha+\beta\in \Phi^+$.
	\end{itemize}
\end{prop}

\begin{notn}
	From now on, we fix such a subset $\Phi^+$. Write $\Phi^- = -\Phi^+$.
\end{notn}

\begin{defn}
	(For above choices), elements in $\Phi$ are called \textbf{roots} of $\mfg$. Elements in $\Phi^+$ (resp. $\Phi^-$) are called \textbf{positive roots} (resp. \textbf{negative roots}). For $\alpha\in \Phi^+$, we say $\alpha$ is a \textbf{(positive) simple root} if it cannot be written as the sum of two positive roots. Let $\mathsf{\Delta} \subset \Phi^+$ be the subset of simple roots.
\end{defn}

\begin{prop}
	The subset $\mathsf{\Delta} \subset \mft^*$ is a basis. In particular, any positive root can be uniquely written as a linear combination of simple roots with non-negative coefficients.
\end{prop}

\begin{defn}
	Define 
	\[
		\mfb:= \mft \oplus \bigoplus_{\alpha\in \Phi^+} \mfg_\alpha,\; \mfn:= \bigoplus_{\alpha\in \Phi^+} \mfg_\alpha
	\]
	which are Lie subalgebras of $\mfg$. We call $\mfb$ the \textbf{Borel subalgebra} of $\mfg$ (that corresponds to the choice of $\Phi^+$) and $\mfn$ the \textbf{nilpotent radical} of $\mfb$.
\end{defn}

\begin{rem}
	In general, a Borel subalgebra $\mfb$ of any Lie algebra $\mfg$ is defined to be a maximal \emph{solvable} subalgebra of it. Here solvable means the sequence $D^1(\mfb):=\mfb$, $D^{n+1}(\mfb):=[D^n(\mfb),D^n(\mfb)]$ satisfies $D^n(\mfb)=0$ for $n>>0$. It is known that all Borel subalgebras are conjugate to each other.

	The subalgebra $\mfn\subset \mfb$ is called the nilpotent radical because it contains exactly nilpotent elements in $\mfb$, i.e., those elements $x$ such that $(\ad_x)^{\circ n}=0$ for $n>>0$.

	Note that we have $\mft \simeq \mfb/\mbn$.
\end{rem}

\begin{exe}
	\red{This is not a homework!} For $\mfg=\sl_n$ and its standard Cartan subalgebra (Example \ref{exam-Cartan-sln}).
	\begin{itemize}
	 	\item[(1)]
	 		Find an explicit description of $\Phi$ and $\mfg_\alpha$.
	 	\item[(2)]
	 		Show there is a unique choice of $\Phi^+$ such that the corresponding $\mfb$ is the subspace of upper triangulated matrices.
	 	\item[(3)] 
	 		For the choice of $\Phi^+$ in (2), find all the simple roots and write each root as a linear combination of these simple roots.
	 \end{itemize}  
\end{exe}

\section{Root system}

\begin{defn}
	Let $E$ be a finite-dimensional Euclidean space and $\Phi\subset E$ be a finite subset such that $0\notin \Phi$. We say $(E,\Phi)$ is a \textbf{root system} if the following is satisfied:
	\begin{itemize}
		\item 
			The subset $\Phi$ spans $E$;
		\item 
			For any $\alpha\in \Phi$, $\mathbb{R}\alpha\cap \Phi = \{\pm \alpha\}$;
		\item
			For $\alpha,\beta\in \Phi$, the number $2\frac{ (\beta,\alpha) }{(\alpha,\alpha)}$ is an integer;
		\item
			The subset $\Phi$ is closed under reflection along any $\alpha\in \Phi$, i.e., for $\alpha,\beta\in \Phi$, the element $\beta-2\frac{ (\beta,\alpha) }{(\alpha,\alpha)} \alpha$ is contained in $\Phi$.
	\end{itemize}
\end{defn}

\begin{defn}
	Let $(E,\Phi)$ be a root system. The \textbf{dual root system} is defined to be $(E^*,\check \Phi)$, where $E^*$ is the dual Euclidean space of $E$ and $\check\Phi$ consists of those $\check \alpha$ for $\alpha\in \Phi$ defined by $\check\alpha(-)=2\frac{ (-,\alpha) }{(\alpha,\alpha)}$.

\end{defn}

\begin{exe}
	\red{This is not a homework!} Show the double-dual of a root system is itself.
\end{exe}

Let us return to the notations in the last section. Let $E_\mBQ:= \mBQ \Phi$ be the $\mBQ$-vector space spaned by $\Phi$ (such that we have $E_\mBQ \otimes_\mBQ k \simeq \mft^*$). Write $E:= E_\mBQ \otimes_{\mBQ} \mBR$\footnote{
	In the lecture, I wrote $E \otimes_\mBR k \simeq \mft^*$ which is only valid if we are given a homomorphism $\mBR \to k$.
}We are going to show $(E,\Phi)$ is a root system. For this purpose, we need to define an inner product on $E$.

\begin{defn}
	Let $\mfg$ be any finite-dimensional Lie algebra. The \textbf{Killing form} on $\mfg$ is the bilinear form $\Kil: \mfg \times \mfg \to k$, $\Kil(x,y):= \tr( \ad(x)\circ \ad(y) )$.
\end{defn}

\begin{prop}
	The Killing form is symmetric and ($\ad$-)invariant, i.e., 
	\begin{itemize}
		\item 
			For $x,y\in \mfg$, $\Kil(x,y) = \Kil(y,x)$;
		\item
			For $x,y,z\in \mfg$, $\Kil(\ad_z(x),y) + \Kil(x,\ad_z(y))=0$.
	\end{itemize}
\end{prop}

\begin{prop}
	If $\mfg$ is simple, then any symmetric invariant bilinear form on $\mfg$ is of the form $c\Kil$ for $c\in k$.
\end{prop}

\begin{warn}
	The similar claim is false if $k$ is not algebraically closed.
\end{warn}


\begin{thm}[Cartan--Killing Criterion]
	The Lie algebra $\mfg$ is semisimple iff its Killing form is non-degenerate. Moreover, in this case, the restriction of $\Kil$ on $\mft$ is also non-degenerate.
\end{thm}

\begin{constr}
	Since $\Kil|_{\mft}$ is non-degenerate, it induces an isomorphism $\mft \xrightarrow{\sim} \mft^*$ sending $x$ to the unique element $x^*$ such that $x^*(-)=\Kil(x,-)$. Consider the inverse $\mft^* \xrightarrow{\sim} \mft$ of this isomorphism, which also corresponds to a non-degenerate bilinear form on $\mft^*$.
\end{constr}

\begin{lem}
	The restriction of the above bilinear form on $E_\mBQ\subset \mft^*$ is $\mBQ$-valued and positive-definite. In particular, it induced a inner product on $E=E_\mBQ\otimes_{\mBQ}\mBR$.
\end{lem}

\begin{convn}
	From now on, we always view $E$ as an Eucilidean space via the above inner product.
\end{convn}

\begin{thm}
	The pair $(E,\Phi)$ defined above is a root system.
\end{thm}

Note that for any $\check\alpha \in \check \Phi$, viewed as an element in $E^*=\mathsf{Hom}(E,\mBR)$, its restriction on $E_\mBQ \subset E$ is $\mBQ$-valued. It follows that $\check\Phi$ is contained in $E_\mBQ^*$ (via the identification $E_\mBQ^* \otimes_\mBQ \mBR \simeq E^*$). Hence we can also view $\check \Phi$ as a subset of $\mft\simeq E_\mBQ^* \otimes_\mBQ k$.

\begin{defn}
	For any root $\alpha\in \Phi$, define the correponding \textbf{coroot} to be $\check \alpha \in \check \Phi \subset \mft$.
\end{defn}

\begin{rem}
	There is a (unique if stated properly) semisimple Lie algebra corresponding to the dual root system $(E^*,\check\Phi)$, known as the \emph{Langlands dual} Lie algebra $\check \mfg$ of $\mfg$.

\end{rem}

\begin{thebibliography}{Yau}


\bibitem[Hum]{Hum} Humphreys, James E. Representations of Semisimple Lie Algebras in the BGG Category $\mathcal{O} $. Vol. 94. American Mathematical Soc., 2008.

\bibitem[Ser]{Ser} Serre, Jean-Pierre. Complex semisimple Lie algebras. Springer Science \& Business Media, 2000.


\end{thebibliography}

\end{document} 


