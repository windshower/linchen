
%!TEX root = main.tex
\documentclass{amsart}
\textwidth=14.5cm \oddsidemargin=1cm
\evensidemargin=1cm
\usepackage{amsmath}
\usepackage{amsxtra}
\usepackage{amscd}
\usepackage{amsthm}
\usepackage{amsfonts}
\usepackage{amssymb}
\usepackage[foot]{amsaddr}
\usepackage{cite}
\usepackage{url}
\usepackage{rotating}
\usepackage{eucal}
\usepackage{tikz-cd}
\usepackage[all,2cell,color]{xy}
\UseAllTwocells
\UseCrayolaColors
\usepackage{graphicx}
\usepackage{pifont}
\usepackage{comment}
\usepackage{verbatim}
\usepackage{xcolor}
\usepackage{hyperref}
\usepackage{xparse}
\usepackage{upgreek}
\usepackage{MnSymbol}
\sloppy


%%%%%%%%%%%%%%%%%%%%Theorem%%%%%%%%%%%%%%%%%%%%
\newcounter{theorem}
\setcounter{theorem}{0}

\newtheorem{cor}[subsection]{Corollary}
\newtheorem{lem}[subsection]{Lemma}
\newtheorem{goal}[subsection]{Goal}
\newtheorem{lemdefn}[subsection]{Lemma-Definition}
\newtheorem{prop}[subsection]{Proposition}
\newtheorem{propdefn}[subsection]{Proposition-Definition}
\newtheorem{cordefn}[subsection]{Corollary-Definition}
\newtheorem{variant}[subsection]{Variant}
\newtheorem{warn}[subsection]{Warning}
\newtheorem{sugg}[subsection]{Suggestion}
\newtheorem{facts}[subsection]{Fact}
\newtheorem{ques}{Question}
\newtheorem{guess}{Guess}
\newtheorem{claim}{Claim}
\newtheorem{propconstr}[subsection]{Proposition-Construction}
\newtheorem{lemconstr}[subsection]{Lemma-Construction}
\newtheorem{ax}{Axiom}
\newtheorem{conje}[subsection]{Conjecture}
\newtheorem{mainthm}[subsection]{Main-Theorem}
\newtheorem{summ}[subsection]{Summary}
\newtheorem{thm}[subsection]{Theorem}
\newtheorem{thmdefn}[subsection]{Theorem-Definition}
\newtheorem{notn}[subsection]{Notation}
\newtheorem{convn}[subsection]{Convention}
\newtheorem{constr}[subsection]{Construction}


\theoremstyle{definition}

\newtheorem{defn}[subsection]{Definition}
\newtheorem{exam}[subsection]{Example}
\newtheorem{assum}[subsection]{Assumption}

\theoremstyle{remark}
\newtheorem{rem}[subsection]{Remark}
\newtheorem{exe}[subsection]{Exercise}


\numberwithin{equation}{section}


%%%%%%%%%%%%%%%%%%%%Commands%%%%%%%%%%%%%%%%%%%%

\newcommand{\nc}{\newcommand}
\nc\on{\operatorname}
\nc\renc{\renewcommand}


%%%%%%%%%%%%%%%%%%%%Sections%%%%%%%%%%%%%%%%%%%%

\nc\ssec{\subsection}
\nc\sssec{\subsubsection}

%%%%%%%%%%%%%%%%%%%%Environment%%%%%%%%%%%%%%%%%
\nc\blongeqn{\[ \begin{aligned}}
\nc\elongeqn{\end{aligned} \]}



%%%%%%%%%%%%%%%%%%%%Mathfont%%%%%%%%%%%%%%%%%%%%

\nc\mBA{{\mathbb A}}
\nc\mBB{{\mathbb B}}
\nc\mBC{{\mathbb C}}
\nc\mBD{{\mathbb D}}
\nc\mBE{{\mathbb E}}
\nc\mBF{{\mathbb F}}
\nc\mBG{{\mathbb G}}
\nc\mBH{{\mathbb H}}
\nc\mBI{{\mathbb I}}
\nc\mBJ{{\mathbb J}}
\nc\mBK{{\mathbb K}}
\nc\mBL{{\mathbb L}}
\nc\mBM{{\mathbb M}}
\nc\mBN{{\mathbb N}}
\nc\mBO{{\mathbb O}}
\nc\mBP{{\mathbb P}}
\nc\mBQ{{\mathbb Q}}
\nc\mBR{{\mathbb R}}
\nc\mBS{{\mathbb S}}
\nc\mBT{{\mathbb T}}
\nc\mBU{{\mathbb U}}
\nc\mBV{{\mathbb V}}
\nc\mBW{{\mathbb W}}
\nc\mBX{{\mathbb X}}
\nc\mBY{{\mathbb Y}}
\nc\mBZ{{\mathbb Z}}


\nc\mCA{{\mathcal A}}
\nc\mCB{{\mathcal B}}
\nc\mCC{{\mathcal C}}
\nc\mCD{{\mathcal D}}
\nc\mCE{{\mathcal E}}
\nc\mCF{{\mathcal F}}
\nc\mCG{{\mathcal G}}
\nc\mCH{{\mathcal H}}
\nc\mCI{{\mathcal I}}
\nc\mCJ{{\mathcal J}}
\nc\mCK{{\mathcal K}}
\nc\mCL{{\mathcal L}}
\nc\mCM{{\mathcal M}}
\nc\mCN{{\mathcal N}}
\nc\mCO{{\mathcal O}}
\nc\mCP{{\mathcal P}}
\nc\mCQ{{\mathcal Q}}
\nc\mCR{{\mathcal R}}
\nc\mCS{{\mathcal S}}
\nc\mCT{{\mathcal T}}
\nc\mCU{{\mathcal U}}
\nc\mCV{{\mathcal V}}
\nc\mCW{{\mathcal W}}
\nc\mCX{{\mathcal X}}
\nc\mCY{{\mathcal Y}}
\nc\mCZ{{\mathcal Z}}


\nc\mbA{{\mathsf A}}
\nc\mbB{{\mathsf B}}
\nc\mbC{{\mathsf C}}
\nc\mbD{{\mathsf D}}
\nc\mbE{{\mathsf E}}
\nc\mbF{{\mathsf F}}
\nc\mbG{{\mathsf G}}
\nc\mbH{{\mathsf H}}
\nc\mbI{{\mathsf I}}
\nc\mbJ{{\mathsf J}}
\nc\mbK{{\mathsf K}}
\nc\mbL{{\mathsf L}}
\nc\mbM{{\mathsf M}}
\nc\mbN{{\mathsf N}}
\nc\mbO{{\mathsf O}}
\nc\mbP{{\mathsf P}}
\nc\mbQ{{\mathsf Q}}
\nc\mbR{{\mathsf R}}
\nc\mbS{{\mathsf S}}
\nc\mbT{{\mathsf T}}
\nc\mbU{{\mathsf U}}
\nc\mbV{{\mathsf V}}
\nc\mbW{{\mathsf W}}
\nc\mbX{{\mathsf X}}
\nc\mbY{{\mathsf Y}}
\nc\mbZ{{\mathsf Z}}

\nc\mba{{\mathsf a}}
\nc\mbb{{\mathsf b}}
\nc\mbc{{\mathsf c}}
\nc\mbd{{\mathsf d}}
\nc\mbe{{\mathsf e}}
\nc\mbf{{\mathsf f}}
\nc\mbg{{\mathsf g}}
\nc\mbh{{\mathsf h}}
\nc\mbi{{\mathsf i}}
\nc\mbj{{\mathsf j}}
\nc\mbk{{\mathsf k}}
\nc\mbl{{\mathsf l}}
\nc\mbm{{\mathsf m}}
\nc\mbn{{\mathsf n}}
\nc\mbo{{\mathsf o}}
\nc\mbp{{\mathsf p}}
\nc\mbq{{\mathsf q}}
\nc\mbr{{\mathsf r}}
\nc\mbs{{\mathsf s}}
\nc\mbt{{\mathsf t}}
\nc\mbu{{\mathsf u}}
\nc\mbv{{\mathsf v}}
\nc\mbw{{\mathsf w}}
\nc\mbx{{\mathsf x}}
\nc\mby{{\mathsf y}}
\nc\mbz{{\mathsf z}}




\nc\mbfA{{\mathbf A}}
\nc\mbfB{{\mathbf B}}
\nc\mbfC{{\mathbf C}}
\nc\mbfD{{\mathbf D}}
\nc\mbfE{{\mathbf E}}
\nc\mbfF{{\mathbf F}}
\nc\mbfG{{\mathbf G}}
\nc\mbfH{{\mathbf H}}
\nc\mbfI{{\mathbf I}}
\nc\mbfJ{{\mathbf J}}
\nc\mbfK{{\mathbf K}}
\nc\mbfL{{\mathbf L}}
\nc\mbfM{{\mathbf M}}
\nc\mbfN{{\mathbf N}}
\nc\mbfO{{\mathbf O}}
\nc\mbfP{{\mathbf P}}
\nc\mbfQ{{\mathbf Q}}
\nc\mbfR{{\mathbf R}}
\nc\mbfS{{\mathbf S}}
\nc\mbfT{{\mathbf T}}
\nc\mbfU{{\mathbf U}}
\nc\mbfV{{\mathbf V}}
\nc\mbfW{{\mathbf W}}
\nc\mbfX{{\mathbf X}}
\nc\mbfY{{\mathbf Y}}
\nc\mbfZ{{\mathbf Z}}

\nc\mbfa{{\mathbf a}}
\nc\mbfb{{\mathbf b}}
\nc\mbfc{{\mathbf c}}
\nc\mbfd{{\mathbf d}}
\nc\mbfe{{\mathbf e}}
\nc\mbff{{\mathbf f}}
\nc\mbfg{{\mathbf g}}
\nc\mbfh{{\mathbf h}}
\nc\mbfi{{\mathbf i}}
\nc\mbfj{{\mathbf j}}
\nc\mbfk{{\mathbf k}}
\nc\mbfl{{\mathbf l}}
\nc\mbfm{{\mathbf m}}
\nc\mbfn{{\mathbf n}}
\nc\mbfo{{\mathbf o}}
\nc\mbfp{{\mathbf p}}
\nc\mbfq{{\mathbf q}}
\nc\mbfr{{\mathbf r}}
\nc\mbfs{{\mathbf s}}
\nc\mbft{{\mathbf t}}
\nc\mbfu{{\mathbf u}}
\nc\mbfv{{\mathbf v}}
\nc\mbfw{{\mathbf w}}
\nc\mbfx{{\mathbf x}}
\nc\mbfy{{\mathbf y}}
\nc\mbfz{{\mathbf z}}

\nc\mfa{{\mathfrak a}}
\nc\mfb{{\mathfrak b}}
\nc\mfc{{\mathfrak c}}
\nc\mfd{{\mathfrak d}}
\nc\mfe{{\mathfrak e}}
\nc\mff{{\mathfrak f}}
\nc\mfg{{\mathfrak g}}
\nc\mfh{{\mathfrak h}}
\nc\mfi{{\mathfrak i}}
\nc\mfj{{\mathfrak j}}
\nc\mfk{{\mathfrak k}}
\nc\mfl{{\mathfrak l}}
\nc\mfm{{\mathfrak m}}
\nc\mfn{{\mathfrak n}}
\nc\mfo{{\mathfrak o}}
\nc\mfp{{\mathfrak p}}
\nc\mfq{{\mathfrak q}}
\nc\mfr{{\mathfrak r}}
\nc\mfs{{\mathfrak s}}
\nc\mft{{\mathfrak t}}
\nc\mfu{{\mathfrak u}}
\nc\mfv{{\mathfrak v}}
\nc\mfw{{\mathfrak w}}
\nc\mfx{{\mathfrak x}}
\nc\mfy{{\mathfrak y}}
\nc\mfz{{\mathfrak z}}

\nc{\one}{{\mathsf{1}}}


\nc\clambda{ {\check{\lambda} }}
\nc\cmu{ {\check{\mu} }}

\nc\bDelta{\mathsf{\Delta}}
\nc\bGamma{\mathsf{\Gamma}}
\nc\bLambda{\mathsf{\Lambda}}


\nc\loccit{\emph{loc.cit.}}



%%%%%%%%%%%%%%%%%%%%Operations-limit%%%%%%%%%%%%%%%%%%%%

\NewDocumentCommand{\ot}{e{_^}}{
  \mathbin{\mathop{\otimes}\displaylimits
    \IfValueT{#1}{_{#1}}
    \IfValueT{#2}{^{#2}}
  }
}
\NewDocumentCommand{\boxt}{e{_^}}{
  \mathbin{\mathop{\boxtimes}\displaylimits
    \IfValueT{#1}{_{#1}}
    \IfValueT{#2}{^{#2}}
  }
}
\NewDocumentCommand{\mt}{e{_^}}{
  \mathbin{\mathop{\times}\displaylimits
    \IfValueT{#1}{_{#1}}
    \IfValueT{#2}{^{#2}}
  }
}
\NewDocumentCommand{\convolve}{e{_^}}{
  \mathbin{\mathop{\star}\displaylimits
    \IfValueT{#1}{_{#1}}
    \IfValueT{#2}{^{#2}}
  }
}
\NewDocumentCommand{\colim}{e{_^}}{
  \mathbin{\mathop{\operatorname{colim}}\displaylimits
    \IfValueT{#1}{_{#1}\,}
    \IfValueT{#2}{^{#2}\,}
  }
}
\NewDocumentCommand{\laxlim}{e{_^}}{
  \mathbin{\mathop{\operatorname{laxlim}}\displaylimits
    \IfValueT{#1}{_{#1}\,}
    \IfValueT{#2}{^{#2}\,}
  }
}
\NewDocumentCommand{\oplaxlim}{e{_^}}{
  \mathbin{\mathop\operatorname{oplax-lim}\displaylimits
    \IfValueT{#1}{_{#1}\,}
    \IfValueT{#2}{^{#2}\,}
  }
}


%%%%%%%%%%%%%%%%%%%%Arrows%%%%%%%%%%%%%%%%%%%%


\makeatletter
\newcommand{\laxto}{\dashedrightarrow}
\newcommand{\xrightleftarrows}[1]{\mathrel{\substack{\xrightarrow{#1} \\[-.9ex] \xleftarrow{#1}}}}
\newcommand{\adj}{\xrightleftarrows{\rule{0.5cm}{0cm}}}

\newcommand*{\da@rightarrow}{\mathchar"0\hexnumber@\symAMSa 4B }
\newcommand*{\da@leftarrow}{\mathchar"0\hexnumber@\symAMSa 4C }
\newcommand*{\xlaxto}[2][]{%
  \mathrel{%
    \mathpalette{\da@xarrow{#1}{#2}{}\da@rightarrow{\,}{}}{}%
  }%
}
\newcommand{\xlaxgets}[2][]{%
  \mathrel{%
    \mathpalette{\da@xarrow{#1}{#2}\da@leftarrow{}{}{\,}}{}%
  }%
}
\newcommand*{\da@xarrow}[7]{%
  % #1: below
  % #2: above
  % #3: arrow left
  % #4: arrow right
  % #5: space left 
  % #6: space right
  % #7: math style 
  \sbox0{$\ifx#7\scriptstyle\scriptscriptstyle\else\scriptstyle\fi#5#1#6\m@th$}%
  \sbox2{$\ifx#7\scriptstyle\scriptscriptstyle\else\scriptstyle\fi#5#2#6\m@th$}%
  \sbox4{$#7\dabar@\m@th$}%
  \dimen@=\wd0 %
  \ifdim\wd2 >\dimen@
    \dimen@=\wd2 %   
  \fi
  \count@=2 %
  \def\da@bars{\dabar@\dabar@}%
  \@whiledim\count@\wd4<\dimen@\do{%
    \advance\count@\@ne
    \expandafter\def\expandafter\da@bars\expandafter{%
      \da@bars
      \dabar@ 
    }%
  }%  
  \mathrel{#3}%
  \mathrel{%   
    \mathop{\da@bars}\limits
    \ifx\\#1\\%
    \else
      _{\copy0}%
    \fi
    \ifx\\#2\\%
    \else
      ^{\copy2}%
    \fi
  }%   
  \mathrel{#4}%
}
\makeatother

%%%%%%%%%%%%%%%%%%%%Decorations%%%%%%%%%%%%%%%%%%%%
\nc{\wt}{\widetilde}
\nc{\ol}{\overline}

\nc{\red}{\textcolor{red}}
\nc{\blue}{\textcolor{blue}}
\nc{\purple}{\textcolor{violet}}

\nc{\simorlax}{{\red\simeq/\blue\lax}}

%%%%%%%%%%%%%%%%%%%%All%%%%%%%%%%%%%%%%%%%%

\nc{\Id}{\mathsf{Id}}
\nc{\gl}{\mathfrak{gl}}
\renc{\sl}{\mathfrak{sl}}
\nc{\GL}{\mathsf{GL}}
\nc{\SL}{\mathsf{SL}}
\nc{\PGL}{\mathsf{PGL}}
\nc{\hmod}{\mathsf{-mod}}
\nc{\Vect}{\mathsf{Vect}}
\nc{\tr}{\mathsf{tr}}
\nc{\Kil}{\mathsf{Kil}}
\nc{\ad}{{\mathsf{ad}}}
\nc{\Ad}{\mathsf{Ad}}
\nc{\oblv}{\mathsf{oblv}}
\nc{\gr}{\mathsf{gr}}
\nc{\Sym}{\mathsf{Sym}}
\nc{\QCoh}{\mathsf{QCoh}}
\nc{\ind}{\mathsf{ind}}
\nc{\Spec}{\mathsf{Spec}}
\nc{\Hom}{\mathsf{Hom}}
\nc{\Ext}{\mathsf{Ext}}
\nc{\Grp}{\mathsf{Grp}}
\nc{\pt}{\mathsf{pt}}
\nc{\Lie}{\mathsf{Lie}}
\nc{\CAlg}{\mathsf{CAlg}}
\nc{\Der}{\mathsf{Der}}
\nc{\Rep}{\mathsf{Rep}}
\renc{\sc}{{\mathsf{sc}}}
\nc{\Fl}{\mathsf{Fl}}
\nc{\Fun}{\mathsf{Fun}}
\nc{\ev}{\mathsf{ev}}
\nc{\surj}{\twoheadrightarrow}
\nc{\inj}{\hookrightarrow}
\nc{\HC}{\mathsf{HC}}
\nc{\cl}{\mathsf{cl}}
\renc{\Im}{\mathsf{Im}}
\renc{\ker}{\mathsf{ker}}
\nc{\coker}{\mathsf{coker}}
\nc{\Tor}{\mathsf{Tor}}
\nc{\op}{\mathsf{op}}
\nc{\length}{\mathsf{length}}
\nc{\fd}{{\mathsf{fd}}}
\nc{\weight}{\mathsf{wt}}
\nc{\semis}{{\mathsf{ss}}}
\nc{\qc}{{\mathsf{qc}}}
\nc{\pr}{\mathsf{pr}}
\nc{\act}{\mathsf{act}}
\nc{\dR}{{\mathsf{dR}}}
\nc{\hol}{{\mathsf{hol}}}
\nc{\Pic}{{\mathsf{Pic}}}
\nc{\Loc}{\mathsf{Loc}}
\nc{\IC}{\mathsf{IC}}

\begin{document}


\title{Lecture 13}

\date{May 20, 2024}

\maketitle

	In this lecture, we prove the first part of the localization theorem. Throughout this lecture, we write $X:=\Fl_G$.

\section{Fibers of the localization functor}

	In this section, we prove the following result.

		\begin{prop}
		\label{prop-fiber-localization}
		For $M \in U(\mfg)\hmod$ and any closed point $x\in X$, we have
		\[
			\Loc(M)|_x \simeq M_{\stab_\mfg(x)},
		\]
		where $\stab_\mfg(x)$ is the stabilizer of $\mfg$ at $x$, i.e., $\stab_\mfg(x):=\ker( \mfg \to \mCT(X) \to \mCT_{X,x}  )$.
	\end{prop}

	\begin{rem}
		It is easy to see $\stab_\mfg(x)$ is the Borel subalgebra of $\mfg$ corresponding to the closed point $x\in X$ (see [Lecture 12, Construction 1.7]).
	\end{rem}

	\proof
		By definition,
		\[
			\Loc(M)|_x \simeq \Gamma(X,k_x \ot_{\mCO_{X}} \mCD_{X} \ot_{\ul{U(\mfg)}} \ul{M}) \simeq \Gamma(X, \delta_x \ot_{ \ul{U(\mfg)} } \ul{M}) \simeq \Gamma(X,\delta_x) \ot_{ U(\mfg) } M,
		\]
		where $\delta_x$ is the Delta right $\mCD$-module in [Lecture 11, Exercise 6.8]. By \emph{loc.cit.}, $\delta_x$ has a unique global section $\Dirac_x\in  \Gamma(X,\delta_x)$ such that $\Dirac_x \cdot f = f(x)\Dirac_x$ for any local section $f$ of $\mCO_{X}$. It follows for any vector field $\partial$ with $\partial|_x =0$, we have $\Dirac_x \cdot \partial = 0$ because locally we can write $\partial=\sum f_k \partial_k$ with $f_k(x)=0$. In particular, the right $U(\mfg)$-action on $\Dirac_x$ annihilates $\stab_\mfg(x) \subset \mfg \subset U(\mfg)$. In other words, we have a right $U(\mfg)$-linear map
		\[
			k\ot_{U(\stab_\mfg(x))} U(\mfg) \to \Gamma(X, \delta_x ),\; 1\ot u \mapsto \Dirac_x \cdot u. 
		\]
		It is easy to see both sides have natural filtrations induced respectively by the PBW filtrations on $U(\mfg)$ and $\mCD_{X}$, and the above map is compatible with the filtrations. Taking associated graded spaces, we only need to show the following obtained map is an isomorphism
		\[
			\Sym^\bullet( \mfg/\stab_\mfg(x) ) \to \Sym^\bullet(\mCT_{X,x}).
		\]
		Unwinding the definitions, this map is induced by the isomorphism $\mfg/\stab_\mfg(x) \simeq \mCT_{X,x}$.

	\qed

	\begin{rem}
		As can be seen from the proof, Proposition \ref{prop-fiber-localization} remains true if $X$ is replaced by any homogenous space under $G$.
	\end{rem}

	\begin{rem}
		As can be seen from the proof, Proposition \ref{prop-fiber-localization} remains true for derived categories and derived functors. In other words, the derived fiber of $\Loc(M)$ at $x$ can be identified with the derived coinvariance of $M$ for $\stab_\mfg(x)$.

	\end{rem}

	Let $e\in X \simeq G/B$ be the closed point corresponding to the chosen Borel subgroup $B$. In the above proof, we have shown
	\[
		 \Gamma(X, \delta_e ) \simeq k\ot_{U(\mfb)} U(\mfg)
	\]
	as right $U(\mfg)$-modules. Note that the RHS is the ``\emph{right} Verma module'' with highest weight $0$. As stated in the localization theorem, we can produce the (left) Verma module $M_{-2\rho}$ with highest weight $-2\rho$ if using the left $\mCD$-module corresponding to $\delta_e$. The following exercise gives a direct proof to this fact.

	\begin{exe}
		\label{exe-delta-Verma-2rho}
		This is \red{Homework 6, Problem 4}. 
		In above, let $\delta_e^l\simeq \delta_e\ot \omega_{X}^{-1}$ be the left $\mCD$-module corresponding to $\delta_e$. Consider the left $U(\mfg)$-module $V:=\Gamma(X, \delta_e^l)$.
		
		\begin{itemize}
			\item[(1)]
				Prove: there is a \emph{canonical} isomorphism
				\[
					\delta_e^l \simeq \mCD_{X} \ot_{\mCO_{X}} \ell,
				\]
				where $\ell$ is the fiber of $\omega_{X}^{-1}$ at $e$, viewed as a skyscrapter sheaf.
			\item[(2)]
				Let $ \ell \inj V$ be the injection induced by taking global sections for the embedding $\mCO_{X} \ot_{\mCO_{X}} \ell \inj \mCD_{X} \ot_{\mCO_{X}}\ell$. Prove: this line in $V$ is a weight subspace of weight $-2\rho$\footnote{Hint: $\ell \simeq \wedge^d \mCT_{X,e} $ and $\mCT_{X,e}\simeq \mfn^-$.}.
			\item[(3)]
				Prove: the subalgebra $\mfb \subset \mfg$ stabilizes the line $\ell\subset V$\footnote{Hint: consider the PBW filtration of $\mCD_X$ and the induced filtration on $V$. Show that $\mfb \ot \ell \to V$ factors through $\mbF^{\le 1} V$ and the composition $b\ot \ell \to \mbF^{\le 1} V \to \gr^1 V$ is zero.}.
			\item[(4)]
				Construct a $U(\mfg)$-linear map 
				\[
					M_{-2\rho} \to V
				\]
				and prove it is an isomorphism.
		\end{itemize}

	\end{exe}



\section{The ring \texorpdfstring{$\mCD(X)$}{D(Fl)}}
	

	\begin{prop}
		The homomorphism $a:U(\mfg) \to \mCD(X)$ factors through $U(\mfg)_{\chi_0}$.
	\end{prop}

	\proof
		We only need to show $a(z)=0$ for any $z\in \ker(\chi_0)\subset Z(\mfg)$. We only need to show for any closed point $x\in X$, the composition
		\[
			\ker(\chi_0) \to \mCD(X) \to \Gamma(X, k_x\ot_{\mCO_X} \mCD_X)
		\]
		is zero. By the proof of Proposition \ref{prop-fiber-localization}, this map can be identified with
		\[
			\ker(\chi_0) \to k\ot_{U(\mfb_x)} U(\mfg),
		\]
		where $\mfb_x = \stab_\mfg(x)$ is the Borel subalgebra corresponding to $x$. Note that the ideal $\ker(\chi_0)\subset Z(\mfg)$ does not depond on the choice of any Borel subalgebra: it is the character for the trivial representation. Hence we only need to show $\ker(\chi_0) \to k\ot_{U(\mfb^-)} U(\mfg)$ is the zero map. But this follows from the Harish-Chandra embedding
		\[
			Z(\mfg) \to U(\mfg) \to k\ot_{U(\mfn^-)} U(\mfg) \ot_{U(\mfn)} k\simeq U(\mft).
		\]

	\qed

	\begin{rem}
		Alternatively, we can use left $\mCD$-modules and reduce to show 
		\[
			\ker(\chi_0) \to \mCD(X) \to \Gamma(X,  \mCD_X \ot_{\mCO_X} k_e)
		\]
		is zero. By Exercise \ref{exe-delta-Verma-2rho}, the RHS is non-canonically isomorphic to $M_{-2\rho}$\footnote{Such an isomorphism depends on a trivialization of the line $\ell$, i.e., a choice of vector in it.} and the above map can be identified with the action map on a highest weight vector. Then the claim follows from $\varpi(-2\rho) = \chi_0$.
	\end{rem}

	To prove the obtained homomorphism
	\[
		U(\mfg)_{\chi_0} \to \mCD(X)
	\]
	is an isomorphism, we consider filtrations on both sides.

	\begin{constr}
		The PBW filtration on $U(\mfg)$ induces a filtration on $U(\mfg)_{\chi_0}$. The surjection $U(\mfg) \to U(\mfg)_{\chi_0}$ induces a surjection $\Sym^\bullet(\mfg) \to \gr^\bullet (U(\mfg)_{\chi_0})$ which sends $\ker(\gr^\bullet(Z(\mfg))\to k)$ to $0$. Recall $\gr^\bullet (Z(\mfg)) \simeq \Sym^\bullet(\mfg)^\mfg$ ([Lecture 5, Lemma 3.2]). Hence we obtain a surjection
		\[
			\mCO( \mfg^*\mt_{\mfg^*/\!/G} 0 ) \simeq \Sym^\bullet(\mfg) \ot_{  \Sym^\bullet(\mfg)^\mfg } k \surj \gr^\bullet (U(\mfg)_{\chi_0}).
		\] 
		Note that \emph{a priori} we do not know this is an isomorphism.

	\end{constr}
	
	\begin{constr}
		On the other hand, the short exact sequences $0 \to \mbF^{\le k-1} \mCD_X \to  \mbF^{\le k} \mCD_X \to \Sym_{\mCO_X}^k \mCT_X \to 0$ induce 
		\[
			0 \to \Gamma(X, \mbF^{\le k-1} \mCD_X) \to \Gamma(X, \mbF^{\le k} \mCD_X) \to \Gamma(X,  \Sym_{\mCO_X}^k \mCT_X )
		\]
		and therefore an injection
		\[
			\gr^\bullet \mCD(X) \inj \Gamma(X,  \Sym_{\mCO_X}^\bullet \mCT_X ) \simeq \mCO(T^*X),
		\]
		where $T^*X \simeq \Spec_X( \Sym_{\mCO_X}^\bullet \mCT_X   )$ is the cotangent bundle on $X$. Note that \emph{a priori} we do not know this is an isomorphism.

	\end{constr}

	
	Combining the above constructions, we obtain homomorphisms
	\[
		\mCO( \mfg^*\mt_{\mfg^*/\!/G} 0 ) \surj \gr^\bullet (U(\mfg)_{\chi_0}) \to \gr^\bullet \mCD(X) \inj\mCO(T^*X).
	\]
	We only need to show this composition is an isomorphism. This composition corresponds to a map
	\begin{equation}
		\label{eqn-Springer-resolution}
		T^*X \to  \mfg^*\mt_{\mfg^*/\!/G} 0
	\end{equation}
	which will be studied in the next section.

	\begin{rem}
		The map $T^*X\to \mfg^*$, which is the (algebro-geometric) dual of $\mfg \to \mCT(X)$ is called the \textbf{moment map}.

	\end{rem}

\section{Nilpotent cone and the Springer resolution}
	
	In this and the next sections, we study the map \eqref{eqn-Springer-resolution}. I recommend \cite[Section 3]{CG} for these contents.


	Recall we have an identification $\mfg \simeq \mfg^*$ provided by the Killing form. Also recall $\mfg^*/\!/G\simeq \mfg/\!/G \simeq \mft/\!/W$ are isomorphic to an affine space of dimension equal to $\dim(\mft)$ (see [Lecture 6]).

	We first describe the target of \eqref{eqn-Springer-resolution}.

	\begin{defn}
		Define $\mCN$ to be the fiber product
		\[
			\xymatrix{
				\mCN \ar[r] \ar[d] & \mfg \ar[d] \\
				0 \ar[r] & \mfg/\!/G,
			}
		\]
		and call it the \textbf{nilpotent cone} of $\mfg$.
	\end{defn}

	\begin{rem}
		\label{rem-Nilp-CM}
		By Kostant's theorem ([Lecture 6, Corollary 1.15]), the projection map $\mfg \to \mfg/\!/G$ is flat. Recall regular immersions are closed under flat base-changes. Hence $\mCN \to \mfg$ is a regular immersion. In particular, $\mCN$ is Cohen--Macaulay.
	\end{rem}

	\begin{rem}
		We have $\dim(\mCN) = \dim(\mfg)-\dim(\mft)$.
	\end{rem}

	\begin{warn}
		The nilpotent cone $\mCN$ is always singular.
	\end{warn}

	The name ``nilpotent cone'' is justified by the following result:

	\begin{prop}
		A closed point of $\mfg$ is contained in $\mCN$ iff it is an nilpotent element.
	\end{prop}

	\proof
		Recall for any Borel pair $(\mfb,\mft)$, we have a commutative diagram (see [Lecture 5, (4.2)])
		\[
			\xymatrix{
				\mfb \ar[r] \ar[d] & \mfg/\!/G \\
				\mft \ar[r] & \mft/\!/W. \ar[u]_-\simeq
			}
		\]
		Also, $W$ acts transitively on the fibers of the map $\mft\to \mft/\!/W$ at the closed points ([Lecture 6, Proposition 1.1]). It follows that a closed point $v\in \mfb$ is sent to $0\in \mfg/\!/G$ iff it is sent to $0\in \mft$. The latter condition is equivalent to $v$ being nilpotent. Now the claim follows from the fact that any element of $\mfg$ is contained in some Borel subalgebra.

	\qed

	\begin{rem}
		We will see $\mCN$ is reduced (and even normal) and therefore it can be characterized by the above proposition.
	\end{rem}

	Now we describe the source of \eqref{eqn-Springer-resolution}. Note that $T^*X$ is smooth because $X$ is so.

	\begin{prop}
		Consider the obvious projection $T^* X \to X$ and the moment map $T^*X \to \mfg^*$. The obtained map
		\begin{equation}
			\label{eqn-cotagent-as-closed-sub}
			T^*X \to X\mt \mfg^*
		\end{equation}
		is a closed embedding. Moreover, via the identification $\mfg\simeq \mfg^*$, a closed points $(x,v)\in X\mt \mfg$ is contained in $T^*X$ iff $v\in \mfn_x:=[\mfb_x,\mfb_x]$, where $\mfb_x$ is the Borel subalgebra corresponding to $x$.
	\end{prop}

	\proof
		Let $x\in X$ be a closed point. We have a ``realizing'' map $X\simeq G/B_x$, $x\mapsto B_x/B_x$. It follows that $\mCT_{X,x}\simeq \mfg/\mfb_x $ and therefore $\mCT^*_{X,x}\simeq (\mfg/\mfb_x)^* $. By definition, the fiber of \eqref{eqn-cotagent-as-closed-sub} at $x\in X$ is the obvious map $(\mfg/\mfb_x )^* \to \mfg^*$ which is a closed embedding.

		In general, a linear map between two vector bundles on $X$ is a closed embedding iff its fiber at any closed point $x\in X$ is a closed embedding. Therefore \eqref{eqn-cotagent-as-closed-sub} is a closed embedding.

		Now the second claim follows from the isomorphism $(\mfg/\mfb_x)^*\simeq \mfn_x$.

	\qed

	\begin{rem}
		\label{rem-Springer-coordinate}
		One can find local trivialization of the vector bundle $T^*X \to \Fl_G$ as follows. Let $x^-\in X$ be any closed point and consider the big Bruhat cell of $X$ with respect to the Borel subgroup $B_{x^-}$, i.e., the unique open $B_{x^-}$-orbit in $X$. Denote this orbit by $U_{x^-}$. It follows that the commposition
		\[
			T^* X \to X\mt \mfg^* \to X \mt \mfn_{x^-}^*
		\]
		is an isomorphism when restricted to $U_{x^-} \subset X$. Indeed, for any $x\in U_{x^-}$, $B_x$ and $B_{x^-}$ are in generic position and therefore $\mfn_{x^-} \to \mfg \to \mfg/\mfb_x$ is an isomorphism.

		Also note that for any chosen $x\in U_{x^-}$, we have $\mfn_{x^-}^* \simeq (\mfg/\mfb_x)^* \simeq \mfn_x$, where the last isomorphism uses the Killing form. Hence we can also trivialize $T^*X|_{U_{x^-}}$ as $X\mt \mfn_x$ after choosing a point $x\in U_{x^-}.$
	\end{rem}

	\begin{defn}
		We write $\wt{\mCN}:=T^* X $ can call the map \eqref{eqn-Springer-resolution}
		\[
			\mfp: \wt\mCN \to \mCN.
		\]	
		the \textbf{Springer resolution of the nilpotent cone}.
	\end{defn}

	\begin{lem}
		\label{lem-Springer-proper-surj}
		The map $\mfp:\wt\mCN \to \mCN$ is proper and surjective.
	\end{lem}

	\proof
		We have a commutative diagram
		\[
			\xymatrix{
				\wt\mCN \ar[r]\ar[d] & X\mt \mfg \ar[d] \\
				\mCN \ar[r] & \mfg.
			}
		\]
		The top horizontal map is proper because it is a closed embedding. The right vertical map is proper because $X$ is complete. Hence the composition $\wt\mCN\to \mfg$ is proper. Since $\mCN \to \mfg$ is separated, the map $\wt\mCN \to \mCN$ is also proper.

		It remains to show $\mfp$ is surjective on closed points. This follows from the fact that any (nilpotent) element in $\mfg$ is contained in some Borel subalgebras.

	\qed

	\begin{cor}
		\label{cor-Nilp-irre}
		The scheme $\mCN$ is irreducible.
	\end{cor}

	We will see $\mfp: \wt\mCN \to \mCN$ is a resolution of singularities. For example, in the case of $\SL_2$, we have:

	\begin{exe}
		This is \red{Homework 6, Problem 5}. For $G=\SL_2$, prove $\mfp: \wt\mCN \to \mCN$ is the blow-up of $\mCN$ at the point $0\in \mCN$.
	\end{exe}

	\begin{rem}
		The Springer resolution plays a central role in geometric representation theory.
	\end{rem}

\section{Kostant's theorem}

	Our goal is to prove the following result.

	\begin{thm}[Kostant]
		\label{thm-Kostant-Springer}
		The map $\wt\mCN \to \mCN$ induces an isomorphism $\mCO(\mCN) \xrightarrow{\sim} \mCO(\wt\mCN)$.
	\end{thm}

	\begin{rem}
		In fact, one can show the \emph{derived} direct image functor $\mfp_*:D(\mCO_{\wt\mCN}\hmod_\qc) \to D(\mCO_{\mCN}\hmod_\qc)$ sends $\mCO_{\wt\mCN}$ to $\mCO_{\mCN}$. The proof of this stronger result is an elaboration of the proof of Theorem \ref{thm-Kostant-Springer} displayed below, with the help of the (derived non-flat) base-change isomorphisms. See \cite[Section 7]{G} for more details.

	\end{rem}

	Note that the above theorem implies the first part of the localization theorem.

	\begin{cor}
		\label{cor-U-vs-D}
		The homomorphism $U(\mfg)_{\chi_0} \to \mCD(X)$ is an isomorphism.
	\end{cor}

	\proof
		By the discussion in previous sections, we only need to show $\mCO(\mCN) \to \mCO(\wt\mCN)$ is an isomorphism, which is Kostant's theorem.

	\qed

	To prove Kostant's theorem, we need more geometric inputs.

	\begin{propdefn}
		\label{propdefn-Groth-alter}
		There is a unique reduced closed subscheme $\wt\mfg$ of $X\mt \mfg$, called the \textbf{Grothendieck's alteration}, whose closed points are those $(x,v)$ satisfying $v\in \mfb_x$. Moreover, $\wt\mfg$ is smooth.
	\end{propdefn}

	\proof[Sketch]
		It is easy to show $v\in \mfb_x$ is a closed condition and therefore defines a reduced closed subscheme $\wt\mfg$. Also, as in Remark \ref{rem-Springer-coordinate}, the composition 
		\[
			\wt{\mfg} \to X\mt \mfg \to X \mt \mfg/\mfn_{x^-}
		\] 
		is an isomorphism when restricted to the open Bruhat cell $U_{x^-}\subset X$. This implies $\wt{\mfg}$ is smooth.

	\qed



	\begin{lem}
		\label{lem-Groth-alter-Cartan}
		There exists a Cartesian square
		\[
			\xymatrix{
				\wt{\mCN} \ar[r] \ar[d] & \wt{\mfg} \ar[d] \\
				0 \ar[r] & \mbft,
			}
		\]
		where $\mbft$ is the \emph{abstract} Cartan Lie algebra (see Appendix \ref{app-abs}). Moreover, the vertical maps are smooth.

	\end{lem}

	\proof[Sketch]
		We have an obvious injective map $\wt{\mCN} \to \wt{\mfg}$ between vector bundles on $X$. By Remark \ref{rem-Springer-coordinate} and the proof of Proposition-Definition \ref{propdefn-Groth-alter}, this map can be identified with $X\mt \mfn_x \to X\mt \mfb_x $ when restricted to the open Bruhat cell $U_{x^-} \subset X$ with a chosen point $x\in U_{x^-}$. Hence the quotient bundle can be identified with $X \mt \mft_x \simeq X\mt \mbft$ over $U_{x^{-}}$, where we used the realizing isomorphism $\mbft \to \mft_x$.

		One can show these identifications do not depend on $x$, and can be glued into a short exact sequence of vector bundles over $X$:
		\[
			0 \to \wt\mCN \to \wt\mfg \to X \mt \mbft \to 0,
		\]
		which makes the desired claim manifest.

	\qed

	\begin{notn}
		Let $\mfg_\rss \subset \mfg_\reg \subset \mfg$ be the open subschemes whose closed points are regular semisimple (resp. regular\footnote{Recall an element $v\in \mfg$ is regular if its centralizer is of minimal dimension, which is $\dim(\mbft)$.}) elements in $\mfg$. Let $\wt{\mfg}_\rss\subset \wt{\mfg}_\reg \subset \wt\mfg$ be their preimages.

		Let $\mCN_\reg:=\mfg_\reg\cap \mCN$ be the open subscheme of $\mCN$. Its closed points are regular nilpotent elements.

		Let $\mbft_\reg\subset \mbft$ be the open subscheme whose closed points are regular elements\footnote{This means the realizations in any/all Cartan subalgebras are regular. Equivalently, this means $\mbfW$ acts freely at these points.}.
	\end{notn}

	We have the following basic results. See e.g. \cite[Section 3.1]{CG} for a proof.

	\begin{prop}
		\label{prop-Groth-alt}
		Consider the map $\mfg \to \mfg/\!/G \to \mbft/\!/\mbfW$ given by the abstract Chevalley isomorphism (see Appendix \ref{app-abs}). We have:
		\begin{itemize}
			\item[(1)]
				The following diagram commutes:
				\[
					\xymatrix{
						\wt\mfg \ar[r] \ar[d] & \mbft \ar[d] \\
						\mfg \ar[r] & \mbft/\!/\mbfW
					}
				\]
			\item[(2)]
				When restricted to the regular locus $\wt\mfg_\reg$, the above diagram is Cartesian. In other words, the following diagram is Cartesian:
				\[
					\xymatrix{
						\wt\mfg_\reg \ar[r] \ar[d] & \mbft \ar[d] \\
						\mfg_\reg \ar[r] & \mbft/\!/\mbfW.
					}
				\]
			\item[(3)]
				When restricted to the regular semisimple locus $\wt\mfg_\rss$, the following diagram is Cartesian, and the Vertical maps are finite étale covers with Galois group $\mbfW$:
				\[
					\xymatrix{
						\wt\mfg_\rss \ar[r] \ar[d] & \mbft_\reg \ar[d] \\
						\mfg_\rss \ar[r] & \mbft_\reg/\mbfW.
					}
				\]
		\end{itemize}

	\end{prop}

	\begin{warn}
		The map $\wt\mfg \to \mbft$ does \emph{not} send regular elements to regular elements. Indeed, it sends $\mCN_\reg$ to $0$.
	\end{warn}


	
	\begin{prop}
		The scheme $\mCN$ is normal.
	\end{prop}

	\proof[Sketch]
		We have seen $\mCN$ is Cohen--Macaulay (Remark \ref{rem-Nilp-CM}). Hence by Serre's criterion, we only need to show $\mCN$ is regular in codimension $1$. 

		By Lemma \ref{lem-Groth-alter-Cartan}, the map $\wt{\mfg} \to \mbft$ is smooth, hence so is $\wt{\mfg}_\reg \to \mbft$. Recall $\mbft\to \mbft/\!/W$ is faithfully flat ([Lecture 6, Proposition 1.1 and Corollary 1.5]). Hence by Proposition \ref{prop-Groth-alt}(2), the map $\mfg_\reg \to \mbft/\!/\mbfW$ is smooth (by flat descent of smooth maps). By definition, we have a Cartesian diagram
		\[
			\xymatrix{
				\mCN_\reg \ar[r] \ar[d] & \mfg_\reg \ar[d] \\
				0 \ar[r] & \mfg/\!/G \simeq \mbft/\!/\mbfW.
			}
		\]
		Hence $\mCN_\reg$ is smooth.

		It remains to show the closed subset $\mCN-\mCN_\reg$ of $\mCN$ is of codimension $\ge 2$. Since $\mCN$ is irreducible (Corollary \ref{cor-Nilp-irre}), $\mCN-\mCN_\reg$ is of codimension $\ge 1$. We need to use the following two basic facts:
		\begin{itemize}
			\item[(i)]
				The adjoint action of $G$ on $\mCN$ has only finitely many orbits\footnote{This can be viewed as a generalization of the theory of Jordan blocks.} (see \cite[Proposition 3.2.9]{CG});
			\item[(ii)]
				Each $G$-orbit on $\mfg$ has a symplectic structure (see \cite[Proposition 1.1.5]{CG}).
		\end{itemize}
		By (ii), each $G$-orbit has an even dimension. Hence each $G$-orbit in $\mCN-\mCN_\reg$ has even codimension. By (i), $\mCN-\mCN_\reg$ has codimension $\ge 2$ as desired.

	\qed

	\begin{cor}
		The map $\mfp:\wt\mCN \to \mCN$ is a resolution of singularities, i.e., it is birational proper and surjective.
	\end{cor}

	\proof
		We have already proved $\mfp$ is proper and surjective (Lemma \ref{lem-Springer-proper-surj}). It remains to show $\mfp$ is birational. We claim its restriction on $\mCN_\reg\subset \mCN$ is an isomorphism. Since $\mCN$ is reduced, we only need to show any closed point of $\mCN_\reg$ has a unique preimage in $\wt\mCN_\reg$. Now the claim follows from Proposition \ref{prop-Groth-alt}(2) because $0\in \mbft/\!/\mbfW$ has a unique preimage in $\mbft$.

	\qed

	\begin{rem}
		In fact, $\mfp:\wt\mCN \to \mCN$ is a \emph{semismall} resolution, i.e., $\dim(\wt\mCN\mt_\mCN\wt\mCN) = \dim(\mCN)$. This fact is crucial in the Springer theory. The fiber product $\wt\mCN\mt_\mCN\wt\mCN$ is known as the \textbf{Steinberg variety}, which also plays a central role in geometric representation theory. For more information, see \cite{CG}.
	\end{rem}

	\proof[Proof of Theorem \ref{thm-Kostant-Springer}]

		Follows by applying Zariski's main theorem to the projection $\mfp: \wt\mCN \to \mCN$. Direct proof: $\mfp_* \mCO_{\wt\mCN}$ is coherent because $\mfp$ is proper. It is generically of rank $1$ because $\mfp$ is birational. Both the source and target of $\mCO_\mCN\to \mfp_* \mCO_{\wt\mCN}$ are sheaves of integral domains, hence they have isomorphic sheaves of fractional fields. Then we win because $\mCO_\mCN$ is integrally closed.

	\qed


	\appendix

	\section{Abstract Cartan group and abstract Weyl group}
	\label{app-abs}

	\begin{constr}
		Let $B_x$ and $B_y$ be two Borel subgroups of $G$. Let $T_x:=B_x/[B_x,B_x]$ and $T_y:=B_y/[B_y,B_y]$ be their abelianizations. Recall there exists $g\in G(k)$ such that $\Ad_g$ induces an isomorphism $\Ad_g: B_x \xrightarrow{\simeq} B_y$. Hence we obtain an isomorphism between the abelianizations $\ol{\Ad}_g: T_x \xrightarrow{\simeq} T_y$.
		The isomorphism $\ol{\Ad}_g$ does not depend on the choice of $g$ because any other choice $g'$ satisfies $g'\in g B_x(k)$ and the adjoint action of $B_x$ on $T_x$ is trivial. For this reason, we write the above isomorphism as
		\[
			\phi_{x,y}: T_x \xrightarrow{\sim} T_y.
		\]

		It is easy to check $\phi_{x,x}=\Id$ and $\phi_{y,z}\circ \phi_{x,y}=\phi_{x,z}$. Hence there exists an algebraic group $\mbfT$, equipped with isomorphisms
		\[
			r_x: \mbfT \xrightarrow{\sim} T_x,
		\]
		such that $r_y = \phi_{x,y}\circ r_x$. The data $(\mbfT, r_x)$ are \emph{unique up to an unique isomorphism}\footnote{This means for $(\mbfT, r_x)$ and $((\mbfT)', r_x')$, there is a unique isomorphism $\alpha:\mbfT \xrightarrow{\sim} (\mbfT)'$ such that $r_x= r_x'\circ \alpha$.}.

		We call $\mbfT$ the \textbf{abstract Cartan group} for $G$, and call $r_x$ the \textbf{realizing isomorphisms}.

		Similarly, the Lie algebra of $\mbfT$ is denoted by $\mbft$ and is called the \textbf{abstract Cartan algebra} for $\mfg$.
	\end{constr}

	\begin{warn}
		The algebraic group $\mbfT$ is \emph{not} a subgroup of $G$, at least not in a canonical way.
	\end{warn}

	\begin{rem}
		A Cartan subgroup $T_1\inj G$ of $G$ does \emph{not} give a realizing isomorphism $\mbfT \to T_1$, at least not in a canonical way. Instead, if we further choose a Borel subgroup $B_x$ that contains $T_1$, i.e., if we have a \textbf{Borel pair} $(B_x,T_1)$, then there is a realizing isomorphism $\mbfT \to T_1$ defined to be the composition
		\[
			r_{(B_x,T_1)}:\mbfT \xrightarrow{r_x} T_x \xleftarrow{\sim} T_1,
		\]
		where the second isomorphism is given by $T_1 \inj B_x \surj T_x$.
	\end{rem}

	\begin{warn}
		One cannot define the abstract Borel group for $G$.
	\end{warn}

	\begin{constr}
		Let $(B_x,T_1)$ and $(B_y,T_2)$ be two Borel pairs. Recall there is a unique element $g\in G(k)$ such that $\Ad_g: B_x\xrightarrow{\sim} B_y$ and $\Ad_g: T_1 \xrightarrow{\sim} T_2$. Hence we obtain an isomorphism between the normalizers $\Ad_g: N_G(T_1) \xrightarrow{\sim} N_G(T_2)$ and therefore an isomorphism between the corresponding Weyl groups. We denote this isomorphism by
		\[
			\varphi_{(B_x,T_1),(B_y,T_2)}: W_{T_1} \to W_{T_2}.
		\]

		It is easy to check $\varphi_{(B_x,T_1),(B_x,T_1)}=\Id$ and $ \varphi_{(B_y,T_2),(B_z,T_3)}\circ \varphi_{(B_x,T_1),(B_y,T_2)} = \varphi_{(B_x,T_1),(B_z,T_3)} $. Hence there exists a group $\mbfW$ equipped with isomorphisms
		\[
			r_{(B_x,T_1)}: \mbfW \xrightarrow{\sim} W_{T_1}
		\]
		such that $r_{(B_y,T_2)} = \varphi_{(B_x,T_1),(B_y,T_2)}\circ r_{(B_x,T_1)}$. The data $(\mbfW, r_{(B_x,T_1)})$ are unique up to an unique isomorphism.

		We call $\mbfW$ the \textbf{abstract Weyl group} for $G$, and call $r_{(B_x,T_1)}$ the \textbf{realizing isomorphisms}.

	\end{constr}

	\begin{warn}
		The isomorphism $\varphi_{(B_x,T_1),(B_y,T_2)}$ depends on $B_x$ and $B_y$.
	\end{warn}

	\begin{warn}
		The group $\mbfW$ is \emph{not} a subgroup of $G$, at least not in a canonical way.
	\end{warn}

	\begin{rem}
		One can also define the abstract Weyl group by providing a group structure on $|G\backslash (X\mt X)|$. This construction was introduced by Deligne--Lusztig when developing the theory named by them.

	\end{rem}

	\begin{constr}
		Let $(B_x,T_1)$ be any Borel pair. The action of $W_{T_1}$ on $T_1$ defines an action of $\mbfW$ on $\mbfT$ via the realizing isomorphisms $r_{(B_x,T_1)}: \mbfT \xrightarrow{\sim} T_1$ and $r_{(B_x,T_1)}: \mbfW \xrightarrow{\sim} W_{T_1}$. Unwinding the definitions, one can show this action does not depend on the choice of the Borel pair. Hence we obtain a \emph{canonical} action of $\mbfW$ on $\mbfT$, which is called \textbf{the (abstract) action} of $\mbfW$ on $\mbfT$.
	\end{constr}

	\begin{constr}
		Recall for any Borel pair $(B,T)$, we have the Chevalley isomorphism $\mft/\!/W \xrightarrow{\sim} g/\!/G$ characterized by the following commutative diagram (see [Lecture 5, (4.2)])
		\[
			\xymatrix{
				\mfb \ar[r] \ar[d] & \mfg/\!/G \\
				\mft \ar[r] & \mft/\!/W. \ar[u]_-\simeq
			}
		\]
		Via the realizing isomorphism $r_{(B,T)}: \mbft/\!/\mbfW \xrightarrow{\sim} \mft/\!/W$, we obtain an isomorphism 
		\[
			\mbft/\!/\mbfW \xrightarrow{\sim} \mfg/\!/G
		\]
		which can be shown to do not depend on the choice of the Borel pair. We call this isomorphism the \textbf{abstract Chevalley isomorphism}.
	\end{constr}



	
	
	

\begin{thebibliography}{Yau}

	\bibitem[CG]{CG} Chriss, Neil, and Victor Ginzburg. Representation theory and complex geometry. Vol. 42. Boston: Birkhäuser, 1997.
	

	\bibitem[G]{G} Gaitsgory, Dennis. Course Notes for Geometric Representation Theory, 2005, available at \url{https://people.mpim-bonn.mpg.de/gaitsgde/267y/catO.pdf}.

\end{thebibliography}


\end{document} 


