
%!TEX root = main.tex
\documentclass{amsart}
\textwidth=14.5cm \oddsidemargin=1cm
\evensidemargin=1cm
\usepackage{amsmath}
\usepackage{amsxtra}
\usepackage{amscd}
\usepackage{amsthm}
\usepackage{amsfonts}
\usepackage{amssymb}
\usepackage[foot]{amsaddr}
\usepackage{cite}
\usepackage{url}
\usepackage{rotating}
\usepackage{eucal}
\usepackage{tikz-cd}
\usepackage[all,2cell,color]{xy}
\UseAllTwocells
\UseCrayolaColors
\usepackage{graphicx}
\usepackage{pifont}
\usepackage{comment}
\usepackage{verbatim}
\usepackage{xcolor}
\usepackage{hyperref}
\usepackage{xparse}
\usepackage{upgreek}
\usepackage{MnSymbol}
\sloppy


%%%%%%%%%%%%%%%%%%%%Theorem%%%%%%%%%%%%%%%%%%%%
\newcounter{theorem}
\setcounter{theorem}{0}

\newtheorem{cor}[subsection]{Corollary}
\newtheorem{lem}[subsection]{Lemma}
\newtheorem{goal}[subsection]{Goal}
\newtheorem{lemdefn}[subsection]{Lemma-Definition}
\newtheorem{prop}[subsection]{Proposition}
\newtheorem{propdefn}[subsection]{Proposition-Definition}
\newtheorem{cordefn}[subsection]{Corollary-Definition}
\newtheorem{variant}[subsection]{Variant}
\newtheorem{warn}[subsection]{Warning}
\newtheorem{sugg}[subsection]{Suggestion}
\newtheorem{facts}[subsection]{Fact}
\newtheorem{ques}{Question}
\newtheorem{guess}{Guess}
\newtheorem{claim}{Claim}
\newtheorem{propconstr}[subsection]{Proposition-Construction}
\newtheorem{lemconstr}[subsection]{Lemma-Construction}
\newtheorem{ax}{Axiom}
\newtheorem{conje}[subsection]{Conjecture}
\newtheorem{mainthm}[subsection]{Main-Theorem}
\newtheorem{summ}[subsection]{Summary}
\newtheorem{thm}[subsection]{Theorem}
\newtheorem{thmdefn}[subsection]{Theorem-Definition}
\newtheorem{notn}[subsection]{Notation}
\newtheorem{convn}[subsection]{Convention}
\newtheorem{constr}[subsection]{Construction}


\theoremstyle{definition}

\newtheorem{defn}[subsection]{Definition}
\newtheorem{exam}[subsection]{Example}
\newtheorem{assum}[subsection]{Assumption}

\theoremstyle{remark}
\newtheorem{rem}[subsection]{Remark}
\newtheorem{exe}[subsection]{Exercise}


\numberwithin{equation}{section}


%%%%%%%%%%%%%%%%%%%%Commands%%%%%%%%%%%%%%%%%%%%

\newcommand{\nc}{\newcommand}
\nc\on{\operatorname}
\nc\renc{\renewcommand}


%%%%%%%%%%%%%%%%%%%%Sections%%%%%%%%%%%%%%%%%%%%

\nc\ssec{\subsection}
\nc\sssec{\subsubsection}

%%%%%%%%%%%%%%%%%%%%Environment%%%%%%%%%%%%%%%%%
\nc\blongeqn{\[ \begin{aligned}}
\nc\elongeqn{\end{aligned} \]}



%%%%%%%%%%%%%%%%%%%%Mathfont%%%%%%%%%%%%%%%%%%%%

\nc\mBA{{\mathbb A}}
\nc\mBB{{\mathbb B}}
\nc\mBC{{\mathbb C}}
\nc\mBD{{\mathbb D}}
\nc\mBE{{\mathbb E}}
\nc\mBF{{\mathbb F}}
\nc\mBG{{\mathbb G}}
\nc\mBH{{\mathbb H}}
\nc\mBI{{\mathbb I}}
\nc\mBJ{{\mathbb J}}
\nc\mBK{{\mathbb K}}
\nc\mBL{{\mathbb L}}
\nc\mBM{{\mathbb M}}
\nc\mBN{{\mathbb N}}
\nc\mBO{{\mathbb O}}
\nc\mBP{{\mathbb P}}
\nc\mBQ{{\mathbb Q}}
\nc\mBR{{\mathbb R}}
\nc\mBS{{\mathbb S}}
\nc\mBT{{\mathbb T}}
\nc\mBU{{\mathbb U}}
\nc\mBV{{\mathbb V}}
\nc\mBW{{\mathbb W}}
\nc\mBX{{\mathbb X}}
\nc\mBY{{\mathbb Y}}
\nc\mBZ{{\mathbb Z}}


\nc\mCA{{\mathcal A}}
\nc\mCB{{\mathcal B}}
\nc\mCC{{\mathcal C}}
\nc\mCD{{\mathcal D}}
\nc\mCE{{\mathcal E}}
\nc\mCF{{\mathcal F}}
\nc\mCG{{\mathcal G}}
\nc\mCH{{\mathcal H}}
\nc\mCI{{\mathcal I}}
\nc\mCJ{{\mathcal J}}
\nc\mCK{{\mathcal K}}
\nc\mCL{{\mathcal L}}
\nc\mCM{{\mathcal M}}
\nc\mCN{{\mathcal N}}
\nc\mCO{{\mathcal O}}
\nc\mCP{{\mathcal P}}
\nc\mCQ{{\mathcal Q}}
\nc\mCR{{\mathcal R}}
\nc\mCS{{\mathcal S}}
\nc\mCT{{\mathcal T}}
\nc\mCU{{\mathcal U}}
\nc\mCV{{\mathcal V}}
\nc\mCW{{\mathcal W}}
\nc\mCX{{\mathcal X}}
\nc\mCY{{\mathcal Y}}
\nc\mCZ{{\mathcal Z}}


\nc\mbA{{\mathsf A}}
\nc\mbB{{\mathsf B}}
\nc\mbC{{\mathsf C}}
\nc\mbD{{\mathsf D}}
\nc\mbE{{\mathsf E}}
\nc\mbF{{\mathsf F}}
\nc\mbG{{\mathsf G}}
\nc\mbH{{\mathsf H}}
\nc\mbI{{\mathsf I}}
\nc\mbJ{{\mathsf J}}
\nc\mbK{{\mathsf K}}
\nc\mbL{{\mathsf L}}
\nc\mbM{{\mathsf M}}
\nc\mbN{{\mathsf N}}
\nc\mbO{{\mathsf O}}
\nc\mbP{{\mathsf P}}
\nc\mbQ{{\mathsf Q}}
\nc\mbR{{\mathsf R}}
\nc\mbS{{\mathsf S}}
\nc\mbT{{\mathsf T}}
\nc\mbU{{\mathsf U}}
\nc\mbV{{\mathsf V}}
\nc\mbW{{\mathsf W}}
\nc\mbX{{\mathsf X}}
\nc\mbY{{\mathsf Y}}
\nc\mbZ{{\mathsf Z}}

\nc\mba{{\mathsf a}}
\nc\mbb{{\mathsf b}}
\nc\mbc{{\mathsf c}}
\nc\mbd{{\mathsf d}}
\nc\mbe{{\mathsf e}}
\nc\mbf{{\mathsf f}}
\nc\mbg{{\mathsf g}}
\nc\mbh{{\mathsf h}}
\nc\mbi{{\mathsf i}}
\nc\mbj{{\mathsf j}}
\nc\mbk{{\mathsf k}}
\nc\mbl{{\mathsf l}}
\nc\mbm{{\mathsf m}}
\nc\mbn{{\mathsf n}}
\nc\mbo{{\mathsf o}}
\nc\mbp{{\mathsf p}}
\nc\mbq{{\mathsf q}}
\nc\mbr{{\mathsf r}}
\nc\mbs{{\mathsf s}}
\nc\mbt{{\mathsf t}}
\nc\mbu{{\mathsf u}}
\nc\mbv{{\mathsf v}}
\nc\mbw{{\mathsf w}}
\nc\mbx{{\mathsf x}}
\nc\mby{{\mathsf y}}
\nc\mbz{{\mathsf z}}




\nc\mbfA{{\mathbf A}}
\nc\mbfB{{\mathbf B}}
\nc\mbfC{{\mathbf C}}
\nc\mbfD{{\mathbf D}}
\nc\mbfE{{\mathbf E}}
\nc\mbfF{{\mathbf F}}
\nc\mbfG{{\mathbf G}}
\nc\mbfH{{\mathbf H}}
\nc\mbfI{{\mathbf I}}
\nc\mbfJ{{\mathbf J}}
\nc\mbfK{{\mathbf K}}
\nc\mbfL{{\mathbf L}}
\nc\mbfM{{\mathbf M}}
\nc\mbfN{{\mathbf N}}
\nc\mbfO{{\mathbf O}}
\nc\mbfP{{\mathbf P}}
\nc\mbfQ{{\mathbf Q}}
\nc\mbfR{{\mathbf R}}
\nc\mbfS{{\mathbf S}}
\nc\mbfT{{\mathbf T}}
\nc\mbfU{{\mathbf U}}
\nc\mbfV{{\mathbf V}}
\nc\mbfW{{\mathbf W}}
\nc\mbfX{{\mathbf X}}
\nc\mbfY{{\mathbf Y}}
\nc\mbfZ{{\mathbf Z}}

\nc\mbfa{{\mathbf a}}
\nc\mbfb{{\mathbf b}}
\nc\mbfc{{\mathbf c}}
\nc\mbfd{{\mathbf d}}
\nc\mbfe{{\mathbf e}}
\nc\mbff{{\mathbf f}}
\nc\mbfg{{\mathbf g}}
\nc\mbfh{{\mathbf h}}
\nc\mbfi{{\mathbf i}}
\nc\mbfj{{\mathbf j}}
\nc\mbfk{{\mathbf k}}
\nc\mbfl{{\mathbf l}}
\nc\mbfm{{\mathbf m}}
\nc\mbfn{{\mathbf n}}
\nc\mbfo{{\mathbf o}}
\nc\mbfp{{\mathbf p}}
\nc\mbfq{{\mathbf q}}
\nc\mbfr{{\mathbf r}}
\nc\mbfs{{\mathbf s}}
\nc\mbft{{\mathbf t}}
\nc\mbfu{{\mathbf u}}
\nc\mbfv{{\mathbf v}}
\nc\mbfw{{\mathbf w}}
\nc\mbfx{{\mathbf x}}
\nc\mbfy{{\mathbf y}}
\nc\mbfz{{\mathbf z}}

\nc\mfa{{\mathfrak a}}
\nc\mfb{{\mathfrak b}}
\nc\mfc{{\mathfrak c}}
\nc\mfd{{\mathfrak d}}
\nc\mfe{{\mathfrak e}}
\nc\mff{{\mathfrak f}}
\nc\mfg{{\mathfrak g}}
\nc\mfh{{\mathfrak h}}
\nc\mfi{{\mathfrak i}}
\nc\mfj{{\mathfrak j}}
\nc\mfk{{\mathfrak k}}
\nc\mfl{{\mathfrak l}}
\nc\mfm{{\mathfrak m}}
\nc\mfn{{\mathfrak n}}
\nc\mfo{{\mathfrak o}}
\nc\mfp{{\mathfrak p}}
\nc\mfq{{\mathfrak q}}
\nc\mfr{{\mathfrak r}}
\nc\mfs{{\mathfrak s}}
\nc\mft{{\mathfrak t}}
\nc\mfu{{\mathfrak u}}
\nc\mfv{{\mathfrak v}}
\nc\mfw{{\mathfrak w}}
\nc\mfx{{\mathfrak x}}
\nc\mfy{{\mathfrak y}}
\nc\mfz{{\mathfrak z}}

\nc{\one}{{\mathsf{1}}}


\nc\clambda{ {\check{\lambda} }}
\nc\cmu{ {\check{\mu} }}

\nc\bDelta{\mathsf{\Delta}}
\nc\bGamma{\mathsf{\Gamma}}
\nc\bLambda{\mathsf{\Lambda}}


\nc\loccit{\emph{loc.cit.}}



%%%%%%%%%%%%%%%%%%%%Operations-limit%%%%%%%%%%%%%%%%%%%%

\NewDocumentCommand{\ot}{e{_^}}{
  \mathbin{\mathop{\otimes}\displaylimits
    \IfValueT{#1}{_{#1}}
    \IfValueT{#2}{^{#2}}
  }
}
\NewDocumentCommand{\boxt}{e{_^}}{
  \mathbin{\mathop{\boxtimes}\displaylimits
    \IfValueT{#1}{_{#1}}
    \IfValueT{#2}{^{#2}}
  }
}
\NewDocumentCommand{\mt}{e{_^}}{
  \mathbin{\mathop{\times}\displaylimits
    \IfValueT{#1}{_{#1}}
    \IfValueT{#2}{^{#2}}
  }
}
\NewDocumentCommand{\convolve}{e{_^}}{
  \mathbin{\mathop{\star}\displaylimits
    \IfValueT{#1}{_{#1}}
    \IfValueT{#2}{^{#2}}
  }
}
\NewDocumentCommand{\colim}{e{_^}}{
  \mathbin{\mathop{\operatorname{colim}}\displaylimits
    \IfValueT{#1}{_{#1}\,}
    \IfValueT{#2}{^{#2}\,}
  }
}
\NewDocumentCommand{\laxlim}{e{_^}}{
  \mathbin{\mathop{\operatorname{laxlim}}\displaylimits
    \IfValueT{#1}{_{#1}\,}
    \IfValueT{#2}{^{#2}\,}
  }
}
\NewDocumentCommand{\oplaxlim}{e{_^}}{
  \mathbin{\mathop\operatorname{oplax-lim}\displaylimits
    \IfValueT{#1}{_{#1}\,}
    \IfValueT{#2}{^{#2}\,}
  }
}


%%%%%%%%%%%%%%%%%%%%Arrows%%%%%%%%%%%%%%%%%%%%


\makeatletter
\newcommand{\laxto}{\dashedrightarrow}
\newcommand{\xrightleftarrows}[1]{\mathrel{\substack{\xrightarrow{#1} \\[-.9ex] \xleftarrow{#1}}}}
\newcommand{\adj}{\xrightleftarrows{\rule{0.5cm}{0cm}}}

\newcommand*{\da@rightarrow}{\mathchar"0\hexnumber@\symAMSa 4B }
\newcommand*{\da@leftarrow}{\mathchar"0\hexnumber@\symAMSa 4C }
\newcommand*{\xlaxto}[2][]{%
  \mathrel{%
    \mathpalette{\da@xarrow{#1}{#2}{}\da@rightarrow{\,}{}}{}%
  }%
}
\newcommand{\xlaxgets}[2][]{%
  \mathrel{%
    \mathpalette{\da@xarrow{#1}{#2}\da@leftarrow{}{}{\,}}{}%
  }%
}
\newcommand*{\da@xarrow}[7]{%
  % #1: below
  % #2: above
  % #3: arrow left
  % #4: arrow right
  % #5: space left 
  % #6: space right
  % #7: math style 
  \sbox0{$\ifx#7\scriptstyle\scriptscriptstyle\else\scriptstyle\fi#5#1#6\m@th$}%
  \sbox2{$\ifx#7\scriptstyle\scriptscriptstyle\else\scriptstyle\fi#5#2#6\m@th$}%
  \sbox4{$#7\dabar@\m@th$}%
  \dimen@=\wd0 %
  \ifdim\wd2 >\dimen@
    \dimen@=\wd2 %   
  \fi
  \count@=2 %
  \def\da@bars{\dabar@\dabar@}%
  \@whiledim\count@\wd4<\dimen@\do{%
    \advance\count@\@ne
    \expandafter\def\expandafter\da@bars\expandafter{%
      \da@bars
      \dabar@ 
    }%
  }%  
  \mathrel{#3}%
  \mathrel{%   
    \mathop{\da@bars}\limits
    \ifx\\#1\\%
    \else
      _{\copy0}%
    \fi
    \ifx\\#2\\%
    \else
      ^{\copy2}%
    \fi
  }%   
  \mathrel{#4}%
}
\makeatother

%%%%%%%%%%%%%%%%%%%%Decorations%%%%%%%%%%%%%%%%%%%%
\nc{\wt}{\widetilde}
\nc{\ol}{\overline}

\nc{\red}{\textcolor{red}}
\nc{\blue}{\textcolor{blue}}
\nc{\purple}{\textcolor{violet}}

\nc{\simorlax}{{\red\simeq/\blue\lax}}

%%%%%%%%%%%%%%%%%%%%All%%%%%%%%%%%%%%%%%%%%

\nc{\Id}{\mathsf{Id}}
\nc{\gl}{\mathfrak{gl}}
\renc{\sl}{\mathfrak{sl}}
\nc{\GL}{\mathsf{GL}}
\nc{\SL}{\mathsf{SL}}
\nc{\PGL}{\mathsf{PGL}}
\nc{\hmod}{\mathsf{-mod}}
\nc{\Vect}{\mathsf{Vect}}
\nc{\tr}{\mathsf{tr}}
\nc{\Kil}{\mathsf{Kil}}
\nc{\ad}{{\mathsf{ad}}}
\nc{\Ad}{\mathsf{Ad}}
\nc{\oblv}{\mathsf{oblv}}
\nc{\gr}{\mathsf{gr}}
\nc{\Sym}{\mathsf{Sym}}
\nc{\QCoh}{\mathsf{QCoh}}
\nc{\ind}{\mathsf{ind}}
\nc{\Spec}{\mathsf{Spec}}
\nc{\Hom}{\mathsf{Hom}}
\nc{\Ext}{\mathsf{Ext}}
\nc{\Grp}{\mathsf{Grp}}
\nc{\pt}{\mathsf{pt}}
\nc{\Lie}{\mathsf{Lie}}
\nc{\CAlg}{\mathsf{CAlg}}
\nc{\Der}{\mathsf{Der}}
\nc{\Rep}{\mathsf{Rep}}
\renc{\sc}{{\mathsf{sc}}}
\nc{\Fl}{\mathsf{Fl}}
\nc{\Fun}{\mathsf{Fun}}
\nc{\ev}{\mathsf{ev}}
\nc{\surj}{\twoheadrightarrow}
\nc{\inj}{\hookrightarrow}
\nc{\HC}{\mathsf{HC}}
\nc{\cl}{\mathsf{cl}}
\renc{\Im}{\mathsf{Im}}
\renc{\ker}{\mathsf{ker}}
\nc{\coker}{\mathsf{coker}}
\nc{\Tor}{\mathsf{Tor}}
\nc{\op}{\mathsf{op}}
\nc{\length}{\mathsf{length}}
\nc{\fd}{{\mathsf{fd}}}
\nc{\weight}{\mathsf{wt}}
\nc{\semis}{{\mathsf{ss}}}
\nc{\qc}{{\mathsf{qc}}}
\nc{\pr}{\mathsf{pr}}
\nc{\act}{\mathsf{act}}
\nc{\dR}{{\mathsf{dR}}}
\nc{\hol}{{\mathsf{hol}}}
\nc{\Pic}{{\mathsf{Pic}}}
\nc{\Loc}{\mathsf{Loc}}
\nc{\IC}{\mathsf{IC}}

\begin{document}


\title{Lecture 4}

\date{Mar 18, 2024}

\maketitle


	After the preparations of the first three lectures, we can finally say a few words about what this course is actually about.

	From now on, we fix the following notations. Let $G$ be a connected semisimple algebraic group with Lie algebra $\mfg$. We fixed $\mfg \supset \mfb \supset \mft$ and let $G\supset B \supset T$ be the corresponding closed connected algebraic subgroups\footnote{In general, for an algebraic group $G$ and $\mfg=\Lie(G)$, there is a 1-1 correspondence between the set of closed connected algebraic subgroups of $G$ and the set of Lie subalgebras of $\mfg$. Warning: This is only true in characteristic $0$.
	}\footnote{As in the study of semisimple Lie algebras, $B$ is a \textbf{Borel subgroup} of $G$, which is defined to be a maximal connected solvable subgroup of $G$; $T$ is called \textbf{Cantan subgroup} of $G$, which is defined to be a maximal connected torus inside $G$.} of $G$.

	We can consider the right multiplication action of $B$ on $G$. This is a free action and thereby the quotient $G/B$ exists\footnote{In future lectures, we will revisit its definition.}. This quotient is a projective smooth $k$-scheme, and actually does not depend on the choice of $B$. It is known as the \textbf{flag variety}\footnote{Example: for $G=\SL_2$, $\Fl_G$ is isomorphic to the projective line $\mBP^1$.} of $G$, usually denoted by $\Fl_G$.

	Now the main goal of this course is to exhibit the following phenomenon: representations of the Lie algebra $\mfg$ is closely related to the geometry of $\Fl_G$. In particular, we will see that the structure of the category $\mCO$ is related to the orbits of the \emph{left} $B$-action on $\Fl_G\simeq G/B$.


\section{Blocks}

	Let's be back to the study of the category $\mCO$. The main goal of this and the next lectures is to split the category $\mCO$ into blocks 
\[
	\mCO \simeq \bigoplus_{\chi\in \Spec(Z(\mfg))} \mCO_\chi
\]
and study each block. Here $Z(\mfg) \subset U(\mfg)$ is the center of the associative algebra $U(\mfg)$. 

Note that any $M\in \mfg\hmod$ can be viewed as a $Z(\mfg)$-module, which is the same as a quasi-coherent sheaf on $\Spec(Z(\mfg))$. We have:

\begin{thm}
	\label{thm-block}
	Let $M\in \mCO$, then the corresponding quasi-coherent sheaf on $\Spec(Z(\mfg))$ is supported on a 0-dimensional subvariety.
\end{thm}

\begin{rem}
	What we really means here is the annihilator $I:=\{ f\in Z(\mfg)\,\vert\, f\cdot M=0 \}$ of $M$ has a finite codimension, i.e., $\dim(Z(\mfg)/I)<\infty$\footnote{Since $M$ is \emph{not} a finite type $Z(\mfg)$-module, there is no consensus definition of its schematic support. One may extend the finite type case and \emph{define} the schematic support as $\Spec ( Z(\mfg)/I )$, which is what I did in the class. One can show the underlying topological space is indeed the set-theoretic support, i.e., containing those prime ideals $\mfp\subset Z(\mfg)$ such that $M_\mfp\neq 0$. Indeed, as in the proof of the theorem, we reduce to the case when $M$ is a Verma module, where the claim is manifest.}. 
\end{rem}

Before we prove the above theorem, we make the following definition and state a corollary of it.

\begin{defn}
	For any closed point $\chi\in \Spec(Z(\mfg))$, let $\mCO_\chi \subset \mCO$ be the full subcategory containing those $M\in \mCO$ that is set-theoretically supported on $\chi$. We call $\mCO_\chi$ a \textbf{block} of $\mCO$.
\end{defn}

\begin{rem}
	In a more representation-theoretic language: a closed point $\Spec(Z(\mfg))$ is the same as a character $\chi: Z(\mfg) \to k$. Then for $M\in \mCO$, it is set-theoretically supported on $\chi$ iff $\ker(\chi)^n \cdot M = 0$ for $n>>0$.

\end{rem}

\begin{cor}
	We have
	\[
		\mCO \simeq \bigoplus_{\chi\in \Spec(Z(\mfg))} \mCO_\chi.
	\]
	In other words, any $M\in \mCO$ admits a unique finite decomposition $M = \oplus_\chi M_\chi$ such that $M_\chi \in \mCO_\chi$. Also, for $M = \oplus_\chi M_\chi$ and $N = \oplus_\chi N_\chi$, we have
	\[
		\Hom_\mCO( M,N ) \simeq \oplus_\chi \Hom_{\mCO_\chi} ( M_\chi, N_\chi ).
	\]
\end{cor}

\proof
	Let us first construct the decomposition. By the theorem, we have a \emph{finite} decomposition of $Z(\mfg)$-modules $M = \oplus_\chi M_\chi$, where
	\[
		M_\chi \simeq M/(\ker(\chi)^n\cdot M)
	\]
	for $n>>0$. Since $\ker(\chi)$ is contained in the center of $U(\mfg)$, the subspace $\ker(\chi)^n\cdot M \subset M$ is preserved by the $U(\mfg)$-action and in particular is a sub-$\mfg$-module of $M$. It follows that both $\ker(\chi)^n\cdot M $ and $M_\chi$ are objects in $\mCO$. Since $\ker(\chi)^n$ annihilates $M_\chi$, the latter is contained in $\mCO_\chi$. Moreover, the projection $M\to M_\chi$ is clearly $\mfg$-linear. This implies $M = \oplus_\chi M_\chi$ is a decomposition in $\mCO$.

	For $\chi\neq \chi'$, there is no nonzero $\mfg$-linear map $M_\chi \to N_{\chi'}$ because there is no nonzero $Z(\mfg)$-linear map between them. 
\qed

\begin{rem}
	In the future, we will see the decomposition $\mCO \simeq \bigoplus_{\chi\in \Spec(Z(\mfg))} \mCO_\chi$ also holds for the derived categories. In other words, we can replace $\Hom_\bullet(-,-)$ in above by $\Ext^i_\bullet(-,-)$.
\end{rem}



\proof[Proof of Theorem \ref{thm-block}]
	Note that if an ideal $I\subset Z(\mfg)$ annihilates two $\mfg$-modules, then $I^2$ annihilates any extension of them. By [Prop. 31, Lec. 2], any $M\in \mCO$ is a quotient of a finite successive extension of Verma modules. It follows that we can assume $M=M_\lambda$ is a Verma module. 

	We will use the following lemma.

\begin{lem}
	\label{lem-central-characer-Verma}
	For any element $z\in Z(\mfg)$ and Verma module $M_\lambda$, the action of $z$ on $M_\lambda$ is given by a scaler. In other words, there exists a unique scaler $\xi_{z,\lambda}\in k$ such that
	\[
		z\cdot v = \xi_{z,\lambda}v
	\]
	for any $v\in M_\lambda$.
\end{lem}
	
	Let us first finish the proof using the lemma. By the definition of actions, the map 
	\[
		\chi:=\xi_{-,\lambda}:Z(\mfg) \to k
	\]
	has to be a homomorphism between $k$-algebras. In other words, $\chi$ is a character of $Z$ and the lemma says $M_\lambda \in \mCO_\chi$, i.e., $M_\lambda$ is annilated by the 1-codimensional ideal $\ker(\chi)$.

\proof[Lemma \ref{lem-central-characer-Verma}]
	Since $z$ is an element in the center, the map $z\cdot - :M_\lambda \to M_\lambda$ is a $U(\mfg)$-linear map. In particular, it is $U(\mft)$-linear and thereby preserves the weight subspaces of $M_\lambda$. Consider the higheset weight subspace, which is spanned by the vector $v_\lambda\in M_\lambda$. This subspace is preserved by the action of $z$, hence there exists a unique scaler $\xi_{z,\lambda}\in k$ such that
	\[
		z\cdot v_\lambda = \xi_{z,\lambda}v_\lambda.
	\]
	Recall $M_\lambda$ is generated by $v_\lambda$. Hence any $v\in M_\lambda$ can be written as $u\cdot v_\lambda$ for some $u\in U(\mfg)$. Then we have
	\[
		z\cdot v = z \cdot (u \cdot v_\lambda) = u\cdot (z\cdot v_\lambda) = \xi_{z,\lambda} u\cdot v_\lambda = \xi_{z,\lambda}v
	\]
	as desired. Here we used the assumption that $z$ is contained in the center.


\qed[Lemma \ref{lem-central-characer-Verma}]

\qed[Theorem \ref{thm-block}]

\section{The Harish--Chandra isomorphism: preparation}

The scaler $\xi_{z,\lambda}$ in Lemma \ref{lem-central-characer-Verma} can be calculated as follows.

\begin{constr}
	The trivial modules define homomorphisms between associative algebras $U(\mfn^\pm) \surj k$. Consider the $k$-linear map
	\[
		U(\mfg) \simeq U(\mfn^-) \ot_{U(\mfn^-)} U(\mfg) \ot_{U(\mfn)} U(\mfn) \surj k \ot_{U(\mfn^-)} U(\mfg) \ot_{U(\mfn)} k
	\]
	induced by these homomorphisms. By the PBW theorem, the RHS can be identified as $U(\mft)= \Sym(\mft)$. In other words, we obtain a $k$-linear surjection 
	\begin{equation} \label{eqn-Ug-to-Ut}
		U(\mfg) \surj \Sym(\mft).
	\end{equation}
\end{constr}

To have a better feeling about this projection, please do the following exercise.

\begin{exe} 
	This is \red{Homework 2, Problem 1}. Let $\mfg=\sl_2$ and $e,h,f$ be the standard basis. Consider $\Omega:=ef+fe+\frac{1}{2}h^2\in U(\sl_2)$. Calculate its image in $\Sym(\mft)=k[h]$ under the map (\ref{eqn-Ug-to-Ut}).
\end{exe}

\begin{lem}
	\label{lem-HC-map}
	The composition 
	\[
		\phi:Z(\mfg) \inj U(\mfg) \surj \Sym(\mft)
	\]
	is a homomorphism and makes the following diagram commute
	\[
		\xymatrix{
			Z(\mfg) 
				\ar[r]^-\phi
				\ar[rd]_-{\chi=\xi_{-,\lambda}}
			& \Sym(\mft) \ar[d]^-{\ev_\lambda} \\
			& k_\lambda,
		}
	\]
	where $\ev_\lambda$ is taking the value of a function $f\in \Sym(\mft)\simeq \Fun(\mft^*)$ at $\lambda\in \mft^*$.
\end{lem}

\proof
	Note that $\chi$ and $\ev_\lambda$ are homomorphisms. Also, the map $\prod_{\lambda\in \mft^*} \ev_\lambda: \Sym(\mft) \to \prod_{\lambda\in \mft^*} k_\lambda$ is injective. Hence we only need to show the above diagram commutes as \emph{vector spaces}.

	Note that the PBW theorem also implies
	\[
		U(\mfb)\ot_{U(\mfn)} k \simeq \Sym(\mft)
	\]
	as $U(\mfb)$-modules, where the $U(\mfb)$-module on the RHS is provided by the homomorphism $U(\mfb) \to U(\mft)\simeq \Sym(\mft)$. It follows that the map (\ref{eqn-Ug-to-Ut}) is equal to
	\[
		U(\mfg) \to k \ot_{U(\mfn^-)} U(\mfg) \ot_{U(\mfb)} \Sym(\mft) \simeq \Sym(\mft).
	\]
	Then $\ev_\lambda\circ \phi$ is given by
	\[
		Z(\mfg)\inj U(\mfg) \surj k \ot_{U(\mfn^-)} U(\mfg) \ot_{U(\mfb)} k_\lambda \simeq k_\lambda,
	\]
	where as before we view $k_\lambda$ as a $U(\mfb)$-module via $U(\mfb)\surj \Sym(\mft)$.

	On the other hand, by definition, $\chi$ is the composition
	\[
		Z(\mfg) \inj U(\mfg) \xrightarrow{-\cdot v_\lambda} M_\lambda \xrightarrow{\mathsf{pr}} k_\lambda
	\]
	where $\mathsf{pr}$ is the projection to the highest weight subspace. Recall
	\[
		M_\lambda \simeq U(\mfg) \ot_{U(\mfb)} k_\lambda
	\]
	and via this identification, the projection $\mathsf{pr}$ is
	\[
		U(\mfg) \ot_{U(\mfb)} k_\lambda \surj k \ot_{U(\mfn^-)} U(\mfg) \ot_{U(\mfb)} k_\lambda \simeq k_\lambda.
	\]
	It follows $\chi$ is also given by
	\[
		Z(\mfg) \inj U(\mfg) \surj U(\mfg) \ot_{U(\mfb)} k_\lambda \surj k \ot_{U(\mfn^-)} U(\mfg) \ot_{U(\mfb)} k_\lambda \simeq k_\lambda.
	\]
	
\qed


\begin{notn}
	The homomorphism $\phi$ corresponds to a morphism between $k$-schemes, which we denote by
	\[
		\varpi: \mft^* \to \Spec(Z(\mfg)).
	\]
	Using this notation, Lemma \ref{lem-central-characer-Verma} says
	\[
		M_\lambda \in \mCO_{\varpi(\lambda)}.
	\]
	In fact, it implies the \emph{schematic support} of $M_\lambda$ is the point $\varpi(\lambda)\in \mft^*$.
\end{notn}

\begin{rem}
	The map $\varpi: \mft^* \to \Spec(Z(\mfg))$ is not an isomorphism because two Verma modules can belong to the same block.
\end{rem}

\begin{warn}
	The embedding $\mft\to \mfg$ also induces an injection $\Sym(\mft) \inj U(\mfg)$, but the center $Z(\mfg)$ is \emph{not} contained in the image of it. This can be seen e.g. from the following exercise.
\end{warn}

\begin{exe}
	This is \red{Homework 2, Problem 2}. Let $\kappa:\mfg\times\mfg \to k$ be any nondegenerate symmetric invariant bilinear form on $\mfg$. For any basis $x_1,\cdots,x_n$ of $\mfg$ and its dual basis $x_1^*,\cdots,x_n^*$ with respect to the form $\kappa$\footnote{By definition, this means $\kappa(x_i,x_j^*)_{i,j}$ is the unit matrix.}, consider the \textbf{Casimir element}
	\[
		\Omega_\kappa = \sum_{i=1}^n x_i \cdot x_i^* \in U(\mfg).
	\]
	\begin{itemize}
		\item[(1)]
			Prove: the Casimir element $\Omega_\kappa$ does not depend on the choice of the basis, and is contained in the center $Z(\mfg)$.
		\item[(2)]
			For $\mfg=\sl_2$, $\kappa=\Kil$ and the canonical basis $e,h,f$, find $\Omega_\Kil$ and prove it is not contained in $\Sym(\mft) \subset U(\mfg)$.
	\end{itemize}
\end{exe}

\begin{exam}
	\label{exam-rho-shift}
	On the other hand, recall there is a nontrivial map $M_{-l-2} \to M_l$ for $l\in \mBZ^{\ge 0}$. It follows that $\varpi(-l-2)=\varpi(l)$ where we identify $\mft^*$ with $\Spec(k[h])$. Since $ \mBZ^{\ge 0} \subset \Spec(k[h])$ is a dense subset\footnote{Note that we are using the Zariski topology on $\Spec(k[h])$. Equivalently, this means any polynomial in $k[h]$ that vanishes at $h\in \mBZ^{\ge 0}$ is $0$.}, we obtain $\varpi(-l-2)=\varpi(l)$ for any $l\in \mft^*$. In other words, the image of $\phi:Z(\sl_2) \to \Sym(\mft)\simeq k[h]$ is invariant under the change of variable $h\mapsto -h-2$.

	Note that this can serve as a sanity-check for your calculation of $\Omega_\Kil$ in the above exercise.
\end{exam}

\section{Recollection: the Weyl group}

In the above example, $Z(\sl_2)$ is actually isomorphic to the subalgebra of $k[h]$ containing exactly those polynomials that are invariant under $h\mapsto -h-2$. We will prove this for general $\mfg$. Namely $Z(\mfg)$ is isomorphic to a certain invariance of $\Sym(\mft)$. Let us first define the desired symmetry on $\Sym(\mft)$.

Recall the following definition.

\begin{defn}
	Let $(E,\Phi)$ be a root system. The \textbf{Weyl group} $W\subset \GL_E$ is the subgroup generated by the reflections $s_\alpha$, where $s_\alpha(\beta) = \beta - \frac{2(\alpha,\beta)}{(\alpha,\alpha)} \alpha$.

	If $(E,\Phi)$ corresponds to a semisimple Lie algebra $\mfg$, we also say $W$ is the Weyl group of $\mfg$ (and even of the algebraic group $G$).
\end{defn}

\begin{rem}
	By definition, $W$ preserves $\Phi$ (this is an axiom of root systems) and any element $w\in W$ is determined by its action on $\Phi$ (because $\Phi$ spans $E$). It follows that $W$ is a finite subgroup because $\Phi$ is finite.
\end{rem}



\begin{rem}
	It also follows that $W$ preserves $E_\mBQ\subset E$ where recall $E_\mBQ$ is the $\mBQ$-vector space spanned by $\Phi$. Hence we obtain a linear $W$-action on $\mft^* \simeq k\ot_\mBQ E_\mBQ$. This action is often denoted as
	\[
		W\times \mft^* \to \mft^*,\; (w,\lambda)\mapsto w(\lambda).
	\]
	Passing to dual vector spaces, we obtained a linear $W$-action on $\mft$.
\end{rem}

\begin{rem}
	The Weyl groups of $(E,\Phi)$ and $(E^*,\check\Phi)$ are isomorphic such that $s_\alpha$ corresponds to $s_{\check \alpha}$.
\end{rem}

\begin{prop}
	Let $N_G(T)$ be the normalizer of $T$ inside $G$. Then we have a canonical isomorphism
	\[
		W\simeq N_G(T)/T
	\]
	such that the $W$-action on $\mft$ is identified with the adjoint action of $N_G(T)/T$ on $\Lie(T)$\footnote{We haven't defined normalizers for algebraic groups, nor the quotient group $N_G(T)/T$ and its action on $\Lie(T)$. But nothing surprising happens here.}.
\end{prop}

\begin{rem}
	Weyl groups are \emph{(crystallographic) Coxeter groups}. Namely, if we choose positive roots $\Phi^+\subset \Phi$ and simple roots $\mathsf{\Delta}\subset \Phi^+$. Then
	\[
		W  = \langle s_\alpha, \alpha\in \mathsf{\Delta} \,\vert\, s_\alpha^2=1, (s_\alpha s_\beta)^{m_{\alpha\beta}}=1, \alpha\neq \beta \rangle
	\]
	where $m_{\alpha\beta}\in \{2,3,4,6\}$. Elements $s_\alpha$, $\alpha\in \Delta$ are called \textbf{simple reflections}.
\end{rem}

\begin{exam}
	For $\mfg=\sl_n$, $W$ is isomorphic to the symmetric group $\Sigma_n$ and the simple reflections are $(i,i+1)$ for $i=1,\cdots,n-1$. The $\Sigma_n$-action on $\mft$ is the standard one.
\end{exam}

\begin{exam}
	For $\mfg=\sl_2$, the nontrivial element $s\in W=\Sigma_2$ acts on $\mft^* = \Spec(k[h])$ as $h\mapsto -h$. Note that this is not the reflection in Example \ref{exam-rho-shift}.
\end{exam}

\begin{defn}
	Let 
	\[
		\rho = \frac{1}{2}\sum_{\alpha\in \Phi^+} \alpha
	\]
	be the half-sum of all positive roots. The \textbf{dotted action} of $W$ on $\mft^*$, denoted by
	\[
		W\times \mft^* \to \mft^*,\; (w,\lambda)\mapsto w\cdot \lambda,
	\]
	is defined to be $w\cdot \lambda := w(\lambda+\rho)-\rho$.
\end{defn}

\begin{constr}
	The dotted $W$-action on $\mft^*$ is not $k$-linear (e.g. the fixed point is $-\rho\in \mft^*$). Instead, each $w\cdot -: \mft^* \to \mft^*$ is an affine transformation. In particular, it is a morphism between $k$-schemes and corresponds to a homomorphism $\Sym(\mft)\gets \Sym(\mft)$. This defines a \emph{right} action of $W$ on $\Sym(\mft)$. The standard \textbf{dotted} $W$-action on $\Sym(\mft)$ is obtained by taking inverse.
\end{constr}

\section{The Harish-Chandra isomorphism: statement}


\begin{thm}[Harish-Chandra]
	\label{thm-HC}
	The homomorphism $\phi: Z(\mfg) \to \Sym(\mft)$ induces an isomorphism
	\[
		\phi_\HC: Z(\mfg)  \xrightarrow{\sim} \Sym(\mft)^{W_\bullet}
	\]
	from $Z(\mfg)$ to the invariance of $\Sym(\mft)$ with respect to the dotted $W$-action.

\end{thm}
 
	We will prove this theorem next time. For now, let us state some corollaries of it.

\begin{cor}
	\label{cor-HC}
	The map $\varpi:\mft^* \to \Spec(Z(\mfg))$ induces an isomorphism
	\[
		\mft^*/\!/W_\bullet \xrightarrow{\sim} \Spec(Z(\mfg))
	\]
	from the (GIT) quotient of $\mft^*$ by $W_\bullet$ to $\Spec(Z(\mfg))$.
\end{cor}

\begin{rem}
	In algebraic geometry, the quotient of a scheme $X$ by an algebraic group $H$, defined to be the coequalizer $H\mt X \rightrightarrows X$, does not always exist as a scheme. Mostly for this reason, people defined the notion of \emph{stacks} such that quotients always exist as stacks.

	However, the quotients inside the category of schemes and the category of stacks are different whenever the action is not free. Hence people call the former \emph{categorical quotient} and the latter \emph{quotient stack}. 

	On the other hand, the \emph{GIT quotient}, or \emph{geometric invariant theory quotient} is a variant of the categorical quotient when we restrict to the categories of \emph{affine schemes}. For $X=\Spec(A)$ is affine, the GIT quotient is defined to be $X/\!/H:=\Spec(A^H)$. Therefore the above corollary is just a rephrasing of the theorem.

	Note however in general GIT quotient is \emph{not} the categorical quotient \footnote{Warning: Wikipedia falsely claims this. Maybe they mean the categorical quotient considered in the category of \emph{affine} schemes. But this is not the standard terminology.}. For example, the categorical quotient $G/B$ is projective rather than affine.

	It is known that $\mft^*/\!/W_\bullet$ is also the categorical quotient\footnote{If you know some algebraic geometry, try proving this by yourself.}. But we keep this notation because $\mft^*/W_\bullet$ is reserved for the quotient stack (at least by modern mathematicians).
\end{rem}

\begin{prop}[Linkage principle]
	Verma modules $M_\lambda$ and $M_\mu$ belong to the same block iff $\mu = w\cdot \lambda$ for some $w\in W$. In particular, there are only finitely many Verma modules and irreducible objects\footnote{Recall $M_\lambda$ has a unique irreducible quotient $L_\lambda$, and all irreducible objects of $\mCO$ are of this form. See [Thm. 25, Prop. 31, Lec. 2].} in each block.
\end{prop}

\begin{rem}
	In fact, the proposition follows from Corollary \ref{cor-HC} once we know $\mft^*/\!/ W_\bullet$ is a ``well-behaved'' quotient. Namely, we need:
	\begin{itemize}
		\item $W$ acts transitively along each fiber of the map $\mft^* \to \mft^*/\!/ W_\bullet$.
	\end{itemize}
	Note however that this is not true for general GIT quotients (e.g. $G/\!/B = \pt$)\footnote{Even worse $G\to G/\!/N$ is not surjective because $G/N$ is not affine. This counterexample actually plays a role in the geometric study of Eisenstein series.}. But we will see next time that this is indeed true for the $W$-action on $\mft^*$. And we will prove $\mft^* \to \mft^*/\!/ W_\bullet$ is indeed surjective.

	Also, next time, we will actually first prove the ``if'' part of the corollary and use it to show $\phi$ factors through the $W_\bullet$-invariance.
\end{rem}


\begin{exe}
	This is \red{Homework 2, Problem 3}. Let $\mfg=\sl_2$. Prove $Z(\sl_2)\simeq k[\Omega_\Kil]$ where $\Omega_\Kil$ is the Casimir element. You can use Theorem \ref{thm-HC} for this exercise.
\end{exe}



\end{document} 


