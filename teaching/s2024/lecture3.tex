
%!TEX root = main.tex
\documentclass{amsart}
\textwidth=14.5cm \oddsidemargin=1cm
\evensidemargin=1cm
\usepackage{amsmath}
\usepackage{amsxtra}
\usepackage{amscd}
\usepackage{amsthm}
\usepackage{amsfonts}
\usepackage{amssymb}
\usepackage[foot]{amsaddr}
\usepackage{cite}
\usepackage{url}
\usepackage{rotating}
\usepackage{eucal}
\usepackage{tikz-cd}
\usepackage[all,2cell,color]{xy}
\UseAllTwocells
\UseCrayolaColors
\usepackage{graphicx}
\usepackage{pifont}
\usepackage{comment}
\usepackage{verbatim}
\usepackage{xcolor}
\usepackage{hyperref}
\usepackage{xparse}
\usepackage{upgreek}
\usepackage{MnSymbol}
\sloppy


%%%%%%%%%%%%%%%%%%%%Theorem%%%%%%%%%%%%%%%%%%%%
\newcounter{theorem}
\setcounter{theorem}{0}

\newtheorem{cor}[subsection]{Corollary}
\newtheorem{lem}[subsection]{Lemma}
\newtheorem{goal}[subsection]{Goal}
\newtheorem{lemdefn}[subsection]{Lemma-Definition}
\newtheorem{prop}[subsection]{Proposition}
\newtheorem{propdefn}[subsection]{Proposition-Definition}
\newtheorem{cordefn}[subsection]{Corollary-Definition}
\newtheorem{variant}[subsection]{Variant}
\newtheorem{warn}[subsection]{Warning}
\newtheorem{sugg}[subsection]{Suggestion}
\newtheorem{facts}[subsection]{Fact}
\newtheorem{ques}{Question}
\newtheorem{guess}{Guess}
\newtheorem{claim}{Claim}
\newtheorem{propconstr}[subsection]{Proposition-Construction}
\newtheorem{lemconstr}[subsection]{Lemma-Construction}
\newtheorem{ax}{Axiom}
\newtheorem{conje}[subsection]{Conjecture}
\newtheorem{mainthm}[subsection]{Main-Theorem}
\newtheorem{summ}[subsection]{Summary}
\newtheorem{thm}[subsection]{Theorem}
\newtheorem{thmdefn}[subsection]{Theorem-Definition}
\newtheorem{notn}[subsection]{Notation}
\newtheorem{convn}[subsection]{Convention}
\newtheorem{constr}[subsection]{Construction}


\theoremstyle{definition}

\newtheorem{defn}[subsection]{Definition}
\newtheorem{exam}[subsection]{Example}
\newtheorem{assum}[subsection]{Assumption}

\theoremstyle{remark}
\newtheorem{rem}[subsection]{Remark}
\newtheorem{exe}[subsection]{Exercise}


\numberwithin{equation}{section}


%%%%%%%%%%%%%%%%%%%%Commands%%%%%%%%%%%%%%%%%%%%

\newcommand{\nc}{\newcommand}
\nc\on{\operatorname}
\nc\renc{\renewcommand}


%%%%%%%%%%%%%%%%%%%%Sections%%%%%%%%%%%%%%%%%%%%

\nc\ssec{\subsection}
\nc\sssec{\subsubsection}

%%%%%%%%%%%%%%%%%%%%Environment%%%%%%%%%%%%%%%%%
\nc\blongeqn{\[ \begin{aligned}}
\nc\elongeqn{\end{aligned} \]}



%%%%%%%%%%%%%%%%%%%%Mathfont%%%%%%%%%%%%%%%%%%%%

\nc\mBA{{\mathbb A}}
\nc\mBB{{\mathbb B}}
\nc\mBC{{\mathbb C}}
\nc\mBD{{\mathbb D}}
\nc\mBE{{\mathbb E}}
\nc\mBF{{\mathbb F}}
\nc\mBG{{\mathbb G}}
\nc\mBH{{\mathbb H}}
\nc\mBI{{\mathbb I}}
\nc\mBJ{{\mathbb J}}
\nc\mBK{{\mathbb K}}
\nc\mBL{{\mathbb L}}
\nc\mBM{{\mathbb M}}
\nc\mBN{{\mathbb N}}
\nc\mBO{{\mathbb O}}
\nc\mBP{{\mathbb P}}
\nc\mBQ{{\mathbb Q}}
\nc\mBR{{\mathbb R}}
\nc\mBS{{\mathbb S}}
\nc\mBT{{\mathbb T}}
\nc\mBU{{\mathbb U}}
\nc\mBV{{\mathbb V}}
\nc\mBW{{\mathbb W}}
\nc\mBX{{\mathbb X}}
\nc\mBY{{\mathbb Y}}
\nc\mBZ{{\mathbb Z}}


\nc\mCA{{\mathcal A}}
\nc\mCB{{\mathcal B}}
\nc\mCC{{\mathcal C}}
\nc\mCD{{\mathcal D}}
\nc\mCE{{\mathcal E}}
\nc\mCF{{\mathcal F}}
\nc\mCG{{\mathcal G}}
\nc\mCH{{\mathcal H}}
\nc\mCI{{\mathcal I}}
\nc\mCJ{{\mathcal J}}
\nc\mCK{{\mathcal K}}
\nc\mCL{{\mathcal L}}
\nc\mCM{{\mathcal M}}
\nc\mCN{{\mathcal N}}
\nc\mCO{{\mathcal O}}
\nc\mCP{{\mathcal P}}
\nc\mCQ{{\mathcal Q}}
\nc\mCR{{\mathcal R}}
\nc\mCS{{\mathcal S}}
\nc\mCT{{\mathcal T}}
\nc\mCU{{\mathcal U}}
\nc\mCV{{\mathcal V}}
\nc\mCW{{\mathcal W}}
\nc\mCX{{\mathcal X}}
\nc\mCY{{\mathcal Y}}
\nc\mCZ{{\mathcal Z}}


\nc\mbA{{\mathsf A}}
\nc\mbB{{\mathsf B}}
\nc\mbC{{\mathsf C}}
\nc\mbD{{\mathsf D}}
\nc\mbE{{\mathsf E}}
\nc\mbF{{\mathsf F}}
\nc\mbG{{\mathsf G}}
\nc\mbH{{\mathsf H}}
\nc\mbI{{\mathsf I}}
\nc\mbJ{{\mathsf J}}
\nc\mbK{{\mathsf K}}
\nc\mbL{{\mathsf L}}
\nc\mbM{{\mathsf M}}
\nc\mbN{{\mathsf N}}
\nc\mbO{{\mathsf O}}
\nc\mbP{{\mathsf P}}
\nc\mbQ{{\mathsf Q}}
\nc\mbR{{\mathsf R}}
\nc\mbS{{\mathsf S}}
\nc\mbT{{\mathsf T}}
\nc\mbU{{\mathsf U}}
\nc\mbV{{\mathsf V}}
\nc\mbW{{\mathsf W}}
\nc\mbX{{\mathsf X}}
\nc\mbY{{\mathsf Y}}
\nc\mbZ{{\mathsf Z}}

\nc\mba{{\mathsf a}}
\nc\mbb{{\mathsf b}}
\nc\mbc{{\mathsf c}}
\nc\mbd{{\mathsf d}}
\nc\mbe{{\mathsf e}}
\nc\mbf{{\mathsf f}}
\nc\mbg{{\mathsf g}}
\nc\mbh{{\mathsf h}}
\nc\mbi{{\mathsf i}}
\nc\mbj{{\mathsf j}}
\nc\mbk{{\mathsf k}}
\nc\mbl{{\mathsf l}}
\nc\mbm{{\mathsf m}}
\nc\mbn{{\mathsf n}}
\nc\mbo{{\mathsf o}}
\nc\mbp{{\mathsf p}}
\nc\mbq{{\mathsf q}}
\nc\mbr{{\mathsf r}}
\nc\mbs{{\mathsf s}}
\nc\mbt{{\mathsf t}}
\nc\mbu{{\mathsf u}}
\nc\mbv{{\mathsf v}}
\nc\mbw{{\mathsf w}}
\nc\mbx{{\mathsf x}}
\nc\mby{{\mathsf y}}
\nc\mbz{{\mathsf z}}




\nc\mbfA{{\mathbf A}}
\nc\mbfB{{\mathbf B}}
\nc\mbfC{{\mathbf C}}
\nc\mbfD{{\mathbf D}}
\nc\mbfE{{\mathbf E}}
\nc\mbfF{{\mathbf F}}
\nc\mbfG{{\mathbf G}}
\nc\mbfH{{\mathbf H}}
\nc\mbfI{{\mathbf I}}
\nc\mbfJ{{\mathbf J}}
\nc\mbfK{{\mathbf K}}
\nc\mbfL{{\mathbf L}}
\nc\mbfM{{\mathbf M}}
\nc\mbfN{{\mathbf N}}
\nc\mbfO{{\mathbf O}}
\nc\mbfP{{\mathbf P}}
\nc\mbfQ{{\mathbf Q}}
\nc\mbfR{{\mathbf R}}
\nc\mbfS{{\mathbf S}}
\nc\mbfT{{\mathbf T}}
\nc\mbfU{{\mathbf U}}
\nc\mbfV{{\mathbf V}}
\nc\mbfW{{\mathbf W}}
\nc\mbfX{{\mathbf X}}
\nc\mbfY{{\mathbf Y}}
\nc\mbfZ{{\mathbf Z}}

\nc\mbfa{{\mathbf a}}
\nc\mbfb{{\mathbf b}}
\nc\mbfc{{\mathbf c}}
\nc\mbfd{{\mathbf d}}
\nc\mbfe{{\mathbf e}}
\nc\mbff{{\mathbf f}}
\nc\mbfg{{\mathbf g}}
\nc\mbfh{{\mathbf h}}
\nc\mbfi{{\mathbf i}}
\nc\mbfj{{\mathbf j}}
\nc\mbfk{{\mathbf k}}
\nc\mbfl{{\mathbf l}}
\nc\mbfm{{\mathbf m}}
\nc\mbfn{{\mathbf n}}
\nc\mbfo{{\mathbf o}}
\nc\mbfp{{\mathbf p}}
\nc\mbfq{{\mathbf q}}
\nc\mbfr{{\mathbf r}}
\nc\mbfs{{\mathbf s}}
\nc\mbft{{\mathbf t}}
\nc\mbfu{{\mathbf u}}
\nc\mbfv{{\mathbf v}}
\nc\mbfw{{\mathbf w}}
\nc\mbfx{{\mathbf x}}
\nc\mbfy{{\mathbf y}}
\nc\mbfz{{\mathbf z}}

\nc\mfa{{\mathfrak a}}
\nc\mfb{{\mathfrak b}}
\nc\mfc{{\mathfrak c}}
\nc\mfd{{\mathfrak d}}
\nc\mfe{{\mathfrak e}}
\nc\mff{{\mathfrak f}}
\nc\mfg{{\mathfrak g}}
\nc\mfh{{\mathfrak h}}
\nc\mfi{{\mathfrak i}}
\nc\mfj{{\mathfrak j}}
\nc\mfk{{\mathfrak k}}
\nc\mfl{{\mathfrak l}}
\nc\mfm{{\mathfrak m}}
\nc\mfn{{\mathfrak n}}
\nc\mfo{{\mathfrak o}}
\nc\mfp{{\mathfrak p}}
\nc\mfq{{\mathfrak q}}
\nc\mfr{{\mathfrak r}}
\nc\mfs{{\mathfrak s}}
\nc\mft{{\mathfrak t}}
\nc\mfu{{\mathfrak u}}
\nc\mfv{{\mathfrak v}}
\nc\mfw{{\mathfrak w}}
\nc\mfx{{\mathfrak x}}
\nc\mfy{{\mathfrak y}}
\nc\mfz{{\mathfrak z}}

\nc{\one}{{\mathsf{1}}}


\nc\clambda{ {\check{\lambda} }}
\nc\cmu{ {\check{\mu} }}

\nc\bDelta{\mathsf{\Delta}}
\nc\bGamma{\mathsf{\Gamma}}
\nc\bLambda{\mathsf{\Lambda}}


\nc\loccit{\emph{loc.cit.}}



%%%%%%%%%%%%%%%%%%%%Operations-limit%%%%%%%%%%%%%%%%%%%%

\NewDocumentCommand{\ot}{e{_^}}{
  \mathbin{\mathop{\otimes}\displaylimits
    \IfValueT{#1}{_{#1}}
    \IfValueT{#2}{^{#2}}
  }
}
\NewDocumentCommand{\boxt}{e{_^}}{
  \mathbin{\mathop{\boxtimes}\displaylimits
    \IfValueT{#1}{_{#1}}
    \IfValueT{#2}{^{#2}}
  }
}
\NewDocumentCommand{\mt}{e{_^}}{
  \mathbin{\mathop{\times}\displaylimits
    \IfValueT{#1}{_{#1}}
    \IfValueT{#2}{^{#2}}
  }
}
\NewDocumentCommand{\convolve}{e{_^}}{
  \mathbin{\mathop{\star}\displaylimits
    \IfValueT{#1}{_{#1}}
    \IfValueT{#2}{^{#2}}
  }
}
\NewDocumentCommand{\colim}{e{_^}}{
  \mathbin{\mathop{\operatorname{colim}}\displaylimits
    \IfValueT{#1}{_{#1}\,}
    \IfValueT{#2}{^{#2}\,}
  }
}
\NewDocumentCommand{\laxlim}{e{_^}}{
  \mathbin{\mathop{\operatorname{laxlim}}\displaylimits
    \IfValueT{#1}{_{#1}\,}
    \IfValueT{#2}{^{#2}\,}
  }
}
\NewDocumentCommand{\oplaxlim}{e{_^}}{
  \mathbin{\mathop\operatorname{oplax-lim}\displaylimits
    \IfValueT{#1}{_{#1}\,}
    \IfValueT{#2}{^{#2}\,}
  }
}


%%%%%%%%%%%%%%%%%%%%Arrows%%%%%%%%%%%%%%%%%%%%


\makeatletter
\newcommand{\laxto}{\dashedrightarrow}
\newcommand{\xrightleftarrows}[1]{\mathrel{\substack{\xrightarrow{#1} \\[-.9ex] \xleftarrow{#1}}}}
\newcommand{\adj}{\xrightleftarrows{\rule{0.5cm}{0cm}}}

\newcommand*{\da@rightarrow}{\mathchar"0\hexnumber@\symAMSa 4B }
\newcommand*{\da@leftarrow}{\mathchar"0\hexnumber@\symAMSa 4C }
\newcommand*{\xlaxto}[2][]{%
  \mathrel{%
    \mathpalette{\da@xarrow{#1}{#2}{}\da@rightarrow{\,}{}}{}%
  }%
}
\newcommand{\xlaxgets}[2][]{%
  \mathrel{%
    \mathpalette{\da@xarrow{#1}{#2}\da@leftarrow{}{}{\,}}{}%
  }%
}
\newcommand*{\da@xarrow}[7]{%
  % #1: below
  % #2: above
  % #3: arrow left
  % #4: arrow right
  % #5: space left 
  % #6: space right
  % #7: math style 
  \sbox0{$\ifx#7\scriptstyle\scriptscriptstyle\else\scriptstyle\fi#5#1#6\m@th$}%
  \sbox2{$\ifx#7\scriptstyle\scriptscriptstyle\else\scriptstyle\fi#5#2#6\m@th$}%
  \sbox4{$#7\dabar@\m@th$}%
  \dimen@=\wd0 %
  \ifdim\wd2 >\dimen@
    \dimen@=\wd2 %   
  \fi
  \count@=2 %
  \def\da@bars{\dabar@\dabar@}%
  \@whiledim\count@\wd4<\dimen@\do{%
    \advance\count@\@ne
    \expandafter\def\expandafter\da@bars\expandafter{%
      \da@bars
      \dabar@ 
    }%
  }%  
  \mathrel{#3}%
  \mathrel{%   
    \mathop{\da@bars}\limits
    \ifx\\#1\\%
    \else
      _{\copy0}%
    \fi
    \ifx\\#2\\%
    \else
      ^{\copy2}%
    \fi
  }%   
  \mathrel{#4}%
}
\makeatother

%%%%%%%%%%%%%%%%%%%%Decorations%%%%%%%%%%%%%%%%%%%%
\nc{\wt}{\widetilde}
\nc{\ol}{\overline}

\nc{\red}{\textcolor{red}}
\nc{\blue}{\textcolor{blue}}
\nc{\purple}{\textcolor{violet}}

\nc{\simorlax}{{\red\simeq/\blue\lax}}

%%%%%%%%%%%%%%%%%%%%All%%%%%%%%%%%%%%%%%%%%

\nc{\Id}{\mathsf{Id}}
\nc{\gl}{\mathfrak{gl}}
\renc{\sl}{\mathfrak{sl}}
\nc{\GL}{\mathsf{GL}}
\nc{\SL}{\mathsf{SL}}
\nc{\PGL}{\mathsf{PGL}}
\nc{\hmod}{\mathsf{-mod}}
\nc{\Vect}{\mathsf{Vect}}
\nc{\tr}{\mathsf{tr}}
\nc{\Kil}{\mathsf{Kil}}
\nc{\ad}{{\mathsf{ad}}}
\nc{\Ad}{\mathsf{Ad}}
\nc{\oblv}{\mathsf{oblv}}
\nc{\gr}{\mathsf{gr}}
\nc{\Sym}{\mathsf{Sym}}
\nc{\QCoh}{\mathsf{QCoh}}
\nc{\ind}{\mathsf{ind}}
\nc{\Spec}{\mathsf{Spec}}
\nc{\Hom}{\mathsf{Hom}}
\nc{\Ext}{\mathsf{Ext}}
\nc{\Grp}{\mathsf{Grp}}
\nc{\pt}{\mathsf{pt}}
\nc{\Lie}{\mathsf{Lie}}
\nc{\CAlg}{\mathsf{CAlg}}
\nc{\Der}{\mathsf{Der}}
\nc{\Rep}{\mathsf{Rep}}
\renc{\sc}{{\mathsf{sc}}}
\nc{\Fl}{\mathsf{Fl}}
\nc{\Fun}{\mathsf{Fun}}
\nc{\ev}{\mathsf{ev}}
\nc{\surj}{\twoheadrightarrow}
\nc{\inj}{\hookrightarrow}
\nc{\HC}{\mathsf{HC}}
\nc{\cl}{\mathsf{cl}}
\renc{\Im}{\mathsf{Im}}
\renc{\ker}{\mathsf{ker}}
\nc{\coker}{\mathsf{coker}}
\nc{\Tor}{\mathsf{Tor}}
\nc{\op}{\mathsf{op}}
\nc{\length}{\mathsf{length}}
\nc{\fd}{{\mathsf{fd}}}
\nc{\weight}{\mathsf{wt}}
\nc{\semis}{{\mathsf{ss}}}
\nc{\qc}{{\mathsf{qc}}}
\nc{\pr}{\mathsf{pr}}
\nc{\act}{\mathsf{act}}
\nc{\dR}{{\mathsf{dR}}}
\nc{\hol}{{\mathsf{hol}}}
\nc{\Pic}{{\mathsf{Pic}}}
\nc{\Loc}{\mathsf{Loc}}
\nc{\IC}{\mathsf{IC}}

\begin{document}


\title{Lecture 3}

\date{Mar 11, 2024}

\maketitle

In this lecture, we give a quick review of the theory of algebraic groups. This theory is analogous to that of complex Lie groups, but the techniques are more algebraic and some proofs are subtler\footnote{Especially if one allows positive-characteristic or non-algebraic-closed base field $k$.}. Standard textbooks include: \cite{B}, \cite{H} and \cite{Sp}. See \cite{M} for a modern treatment of this theory, which is also my favourite.



\section{Algebraic Groups}

\begin{defn}
	An \textbf{algebraic group} over $k$ is a finite type $k$-scheme $G$ equipped with a group structure, i.e., a multiplication map $m:G\mt G\to G$ subject to axioms similar to those for an abstract group.

	Homomorphisms between algebraic groups are defined in the obvious way. Let $\Grp_k$ be the category of algebraic groups.
\end{defn}

\begin{rem}
	As in the study of abstract groups, the unit and the inversion maps are determined by the multiplication map. We denote them respectively by:
	\[
		e: \pt \to G,\; \sigma: G\to G,
	\]
	where $\pt:=\Spec(k)$.
\end{rem}


\begin{constr}
	Let $G$ be an algebraic group. For any commutative $k$-algebra $A$, write
	\[
		G(A):= \Hom( \Spec(A),G )
	\]
	be the set of maps $\Spec(A)\to G$ between $k$-schemes. The group structure on $G$ induces a group structure on $G(A)$.

	Note that for $A\to B$, we have a homomorphism $G(A) \to G(B)$. Hence we obtain a functor
	\[
		G(-): \CAlg_k \to \Grp
	\]
	from the category of commutative $k$-algebras to the category of (abstract) groups. By the Yoneda lemma, the algebraic group $G$ is determined by this functor.
\end{constr}

\begin{exam}
	The \textbf{additive group} $\mBG_\mba$ is defined such that $\mBG_\mba(A) = A$, viewed as a commutative group under addition. The underlying $k$-scheme is the affine line $\mBA^1$.
\end{exam}

\begin{exam}
	The \textbf{multplicative group} $\GL_1 = \mBG_\mbm$ is defined such that $\mBG_\mbm(A) = A^\times$, i.e. the subset of unit elements in $A$, viewed as a commutative group under multiplication. The underlying $k$-scheme is $\mBA^1 \setminus 0$, i.e., the affine line with the origin removed.
\end{exam}

\begin{exam}
	One can define algebraic groups $G:=\GL_n$, $\SL_n$, $\mathsf{SO}_n$, etc. such that $G(A)$ is the group of matrices of the corresponding type with coefficients in $A$. 
\end{exam}

\begin{exam}
	One can define the algebraic group $\mathsf{PGL}_n$ such that $\mCO_{ \mathsf{PGL}_n }$ is a subring of $\mCO_{\GL_n}$\footnote{Warning: $\mathsf{PGL}_n(A)\neq \GL_n(A)/\GL_1(A)$ for general $A$. In fact, viewed as functors in $A$, the LHS is the sheafification of the RHS in the fpqc topology.}.
\end{exam}

\begin{rem}
	Not every functor $\CAlg_k \to \Grp$ comes from an algebraic group. For example, any $k$-vector space $V$ defines a functor
	\[
		\GL_V(-): \CAlg_k \to \Grp
	\]
	that sends $A$ to the group of $A$-linear automorphisms of $A\ot V$. This functor is not represented by an algebraic group unless $V$ is finite-dimensional.
\end{rem}

\section{Hopf algebras}
	From now on, we assume $G$ is affine\footnote{Any affine algebraic group over field of characteristic $0$ is smooth.}\footnote{Projective algebraic groups, a.k.a., abelian varieties, are also important and play a central role in modern mathematics.}.

\begin{constr}
	An affine algebraic group $G$ is determined by its ring of functions $\mCO_G$. The maps $m: G\mt G \to G$ and $e: \pt \to G$ correspond to homomorphisms between algebras
	\[
		\Delta:\mCO_G \to \mCO_G\ot \mCO_G, \; \epsilon: \mCO_G \to k,
	\]
	which are called the \textbf{comultiplication} and \textbf{counit} maps of $\mCO_G$. Together with the usual multiplication and unit maps of $\mCO_G$, we obtain a \textbf{bialgebra} $(\mCO_G, \cdot,\Delta)$. 

	Note that this bialgebra is commutative but not cocommutative unless $G$ is so.

	The inverse map $\sigma:G\to G$ corresponds to a homomorphism $S: \mCO_G\to \mCO_G$, which is called the \textbf{antipode} of $\mCO_G$. This makes $\mCO_G$ into a commutative \textbf{Hopf algebra}. 
\end{constr}

\begin{rem}
	A Hopf algebra is a bialgebra $A$ equipped with an antipode map $S:A\to A$ subject to a certain axiom. For our purposes, it is less useful to memorize this axiom than to imagine it amounts to say ``$\Spec(A)$''\footnote{Note however that this does not make sense if $A$ is not commutative.} has an inversion map.
\end{rem}

\begin{exam}
	For $G=\mBG_\mba$, we have $\mCO_G=k[t]$ and
	\[
		\Delta(t) = t\otimes 1+1\otimes t ,\; \epsilon(f) = f(0),\; S(f)(t) = f(-t).
	\]
\end{exam}

\begin{exam}
	For $G=\mBG_\mbm$, we have $\mCO_G=k[t,t^{-1}]$ and
	\[
		\Delta(t) = t\otimes t ,\; \epsilon(f) = f(1),\; S(f)(t) = f(t^{-1}).
	\]
\end{exam}


\begin{exam}
	The universal enveloping algebra $U(\mfg)$ of any Lie algebra is a Hopf algebra. The comultiplication $\Delta:U(\mfg) \to U(\mfg)\ot U(\mfg)$ is determined by $\Delta(x) = x\otimes 1 + 1\otimes x$, $x\in \mfg \subset U(\mfg)$ and its compatibility with the multiplication. Similarly, the antipode is determined by $S(x)=-x$, $x\in \mfg$.

	The Hopf algebra $U(\mfg)$ is cocommutative but not commutative unless $\mfg$ is abelian.
\end{exam}

\begin{rem}
	Using the Hopf algebra structure on $U(\mfg)$, the tensor product structure in $\mfg\hmod$ can be defined as follows. Let $V_1$ and $V_2$ be left $U(\mfg)$-modules. Their tensor product $V_1\ot V_2$ is naturally a $U(\mfg)\ot U(\mfg)$-module. Restricting along $\Delta$, we can view $V_1\ot V_2$ as a $U(\mfg)$-module.
\end{rem}



\begin{rem}
	There are also interesting Hopf algebras that are neither commutative nor cocommutative. For example, \emph{quantum algebras} are such gadgets. See \cite{L} for a standard textbook.
\end{rem}


\section{Tangent spaces}

	As in the theory of Lie groups, the tangent space of an algebraic group at its unit is a Lie algebra. To desribe this, let us review the definition of tangent spaces in algebraic geometry.

\begin{defn}
	Let $X$ be any $k$-scheme and $x\in X$ be a $k$-point, i.e., a map $x: \Spec(k) \to X$. The \textbf{tangent space} of $X$ at $x$, denoted by $T_xX$, is the \emph{set} of dotted arrows making the following diagram commute:
	\[
		\xymatrix{
			\Spec(k) \ar[r]^-x \ar[d]
			& X \\
			\Spec(k[\epsilon]/\epsilon^2). \ar@{.>}[ru]
		}
	\]
	Here the vertical map is given by the homomorphism $k[\epsilon]/\epsilon^2 \to k$, $\epsilon\mapsto 0$.

	Elements in $T_xX$ are called \textbf{tangent vectors} of $X$ at $x$.
\end{defn}


	Tangent vectors are related to \emph{derivations}. Let us review its definition in the algebraic setting.

\begin{defn}
	Let $A$ be a $k$-algebra and $M$ be an $A$-module. A \textbf{$k$-derivation} of $A$ into $M$ is a $k$-linear map $D:A\to M$ satisfying the \textbf{Lebniz rule}
	\[
		D(f\cdot g) = f\cdot D (g) + g\cdot D (f).
	\]
	Let $\Der_k(A,M)$ be the set of such $k$-derivations. This is naturally a $k$-vector space.
\end{defn}

\begin{constr}
	Let $X=\Spec(A)$ be an affine scheme and $x:\Spec(k) \to X$ be given by a homomorphism $\phi:A\to k$. View $k$ as an $A$-module via this homomorphism and denote it by $k_x$. For any $D\in \Der_k(A,k_x)$, the map 
	\[
		A\to k[\epsilon]/\epsilon^2,\; f\mapsto \phi(f)+D(f)\epsilon
	\]
	is a homomorphism and thereby gives a map $\Spec(k[\epsilon]/\epsilon^2) \to X$, which is an element in $T_x X$. It is easy to see this gives a bijection
	\[
		\Der_k(A,k_x) \simeq T_x X.
	\]
	In particular, we obtain a $k$-vector space structure on $T_x X$.

\end{constr}


\begin{constr}
	Let $f:X\to Y$ be a morphism between $k$-schemes. Let $x\in X$ and $y:=f(x)\in Y$ be $k$-points. There is an obvious $k$-linear map
	\[
		df: T_x X \to T_y Y
	\]
	given by composing with $f$. We call it the \textbf{differential} of $f$.
\end{constr}

	We have the following obvious result:

\begin{lem}
	\label{lem-tangent-product}
 	Let $X$ and $Y$ be $k$-schemes. Let $x\in X$ and $y\in Y$ be $k$-points. For $\partial_1\in T_x X$ and $\partial_2\in T_yY$, write $\partial_1\oplus\partial_2\in T_{(x,y)}(X\mt Y)$ for the map
 	\[
 		\Spec(k[\epsilon]/\epsilon^2) \xrightarrow{(\partial_1,\partial_2)} X\mt  Y.
 	\]
 	Then the map
 	\begin{equation}
 		\label{eqn-sum-tangent}
 		T_x X \times T_y Y \to T_{(x,y)}(X\mt Y),\; (\partial_1,\partial_2) \mapsto \partial_1\oplus\partial_2
 	\end{equation}
 	induces an isomorphism $T_x X\oplus T_y Y = T_{(x,y)}(X\mt Y)$.

\end{lem}


\section{Lie Algebras and Algebraic Groups}

\begin{notn}
	Let $G$ be an algebraic group. Define
	\[
		\Lie(G):= T_e G.
	\]
\end{notn}

\begin{exam}
	For $G=\GL_n$, we have $\Lie(\GL_n) \simeq \gl_n$. In particular, $\Lie(\GL_n)$ is naturally a Lie algebra. We have similar results for other classical subgroups of $\GL_n$.
\end{exam}

We state the following result without proof:

\begin{thm}	
	There is a canonical functor $\Grp_k \to \Lie_k$ sending $G$ to $\Lie(G)$ equipped with a natural Lie bracket, such that for $G=\GL_n$, the Lie bracket on $\Lie(\GL_n)$ is given by that on $\gl_n$.
\end{thm}

\begin{rem}
	The above functor $\Grp_k \to \Lie_k$ is unique if stated properly. See \cite[Theorem 10.23]{M}.
\end{rem}

\begin{warn}
	It is not true that every Lie algebra can be obtained from algebraic groups. See \cite[I, \S 5, Exercise 6]{Bou} for a counterexample.
\end{warn}

\begin{rem}
	Consider the multiplication map $m: G\mt G\to G$ and its differential $dm: \Lie(G\mt G)\to \Lie (G)$. One can show that the composition
	\[
		\Lie(G)\oplus \Lie(G) \simeq \Lie(G\mt G) \xrightarrow{dm} \Lie(G)
	\]
	sends $(\partial_1,\partial_2)$ to $\partial_1+\partial_2$.	This is the algebraic analogue of the formula $\exp(t u)\cdot \exp(t v) = \exp(t (u+v)) + O(t^2)$, $u,v\in \Lie(G)$ that appears in the study of Lie groups.

\end{rem}

\begin{rem}
	Let $G$ be an affine algebraic group and $\mfg:=\Lie(G)$ be its Lie algebra. The Hopf algebras $\mCO_G$ and $U(\mfg)$ are related as follows.

	Consider the dual $U(\mfg)^*:=\Hom_k(U(\mfg),k)$ as a topological vector space\footnote{For a vector space $V$ equipped with the discrete topology, the dual $V^*$ is equipped with the weakest topology such that for any finite-dimensional subspace $V_0\subset V$, the map $V^*\to V_0^*$ is continuous. Here $V_0^*$ is equipped with the discrete topology. Equivalently, we can define $V^*$ as an object in the pro-category $\mathsf{Pro}(\Vect_{k,\mathsf{fd}})$ of finite dimensional vector spaces.}. The cocommutative Hopf algebra structure on $U(\mfg)$ induces a commutative topological Hopf algebra structure on it\footnote{Here we must use the \emph{complete tensor product} $U(\mfg)^* \hat\otimes U(\mfg)^*$ instead of the usual tensor product. By design, this is the dual of $U(\mfg)\ot U(\mfg)$.
	}.

	On the other hand, consider the maximal ideal $\mfm\subset \mCO_G$ corresponding to the unit point $e:\pt \to G$. Let $\hat\mCO_G$ be the $\mfm$-adic completion of $\mCO_G$. The commutative Hopf algebra structure on $\mCO_G$ induces a commutative topological Hopf algebra structure on it.

	We have
	\[
		U(\mfg)^* \simeq \hat\mCO_G
	\]
	as commutative topological Hopf algebras.
\end{rem}

\begin{exam}
	Note that the Lie algebras of $\mBG_\mba$ and $\mBG_\mbm$ are isomorphic, it follows that $\hat\mCO_{\mBG_\mba} \simeq k[\![t]\!]$ and $\hat\mCO_{\mBG_\mbm}=k[\![t-1]\!]$ are isomorphic. Up to a scaler, this isomorphism is given by
	\[
		k[\![t-1]\!]\to k[\![t]\!],\; f\mapsto f(\exp(t)).
	\]
	Note that given a power series $a_0+a_1(t-1)+a_2(t-1)^2+\cdots$, the series $a_0+a_1(\exp(t)-1)+a_2(\exp(t)-1)^2+\cdots$ indeed converges in the $t$-adic topology.
	
\end{exam}

\section{Representations of Algebraic Groups}

In this section, $G$ is an affine algebraic group.

\begin{defn}
	A \textbf{representation of $G$}, or equivalently a \textbf{$G$-module} is a $k$-vector space $V$ equipped with a natural transformation $G(-)\to \GL_V(-)$ as functors $\CAlg_k \to \Grp$.

	Let $V$ and $W$ be $G$-modules, a \textbf{$G$-linear map} $\phi:V\to W$ is a $k$-linear map such that for any $A\in \CAlg_k$, the following diagram commutes
	\[
		\xymatrix{
			G(A) \ar[r] \ar[d] &
			\GL_V(A) \ar[d]^-{\phi\circ-} \\
			\GL_W(A) \ar[r]^-{-\circ \phi} &
			\Hom_A( A\ot V,A\ot W ).
		}
	\]

	Let $\Rep(G)$ be the category of $G$-modules.
\end{defn}

\begin{exam}
	Any $V$ can be equipped with a trivial $G$-module structure such that the homomorphisms $G(A) \to \GL_V(A)$ are trivial.
\end{exam}

\begin{prop}
	The category $\Rep(G)$ is an abelian category and the forgetful functor $\Rep(G)\to \Vect_k$ is exact. 
\end{prop}

\begin{rem}
	If $V$ is finite-dimensional, then $\GL_V$ is represented by an algebraic group. A $G$-module structure on $V$ is just a homomorphism $G\to \GL_V$ between algebraic groups.
\end{rem}

\begin{warn}
	Evaluate at $k\in \CAlg_k$, we obtain a group homomorphism $G(k) \to \GL_V(k)=\GL(V)$. But a $G$-module structure contains more information than such a homomorphism. This can be seen from the following exercise. (Note that for $k=\mBC$, the \emph{abstract} groups $\mBG_\mba(\mBC)=\mBC$ and $\mBG_\mbm(\mBC)=\mBC^\times$ are isomorphic via the exponential map.)
\end{warn}


\begin{exe}
	This is \red{Homework 1, Problem 4}.
	\begin{itemize}
		\item[(1)]
			Find all maps between $k$-schemes $\mBA^1\to \mBA^1\setminus 0$.
		\item[(2)]
			Find all 1-dimensional representations of the additive group $\mBG_\mba$.
		\item[(3)]
			Find all maps between $k$-schemes $\mBA^1\setminus 0\to \mBA^1\setminus 0$.
		\item[(4)]
			Find all 1-dimensional representations of the multiplicative group $\mBG_\mbm$.
	\end{itemize}

\end{exe}

\begin{prop}
	Any $G$-action on a vector space $V$ is locally finite, i.e., $V$ is the union of its finite-dimensional subrepresentations.
\end{prop}


\begin{prop}
	There is a canonical equivalence
	\[
		\Rep(G) \simeq \mCO_G\mathsf{-comod}
	\]
	from the category of $G$-modules to the category of $\mCO_G$-comodules. This equivalence is compatible with the forgetful functors to $\Vect_k$.
\end{prop}

\begin{rem}
	The functor $\Rep(G) \simeq \mCO_G\mathsf{-comod}$ is constructed as follows. Let $V\in \Rep(G)$. Consider $A:=\mCO_G$ and the homomorphism $G(A) \to \GL_V(A)$. The identity map $G\to G$ can be written as $\Spec(A) \to G$ which corresponds to an element in $G(A)$\footnote{Warning: this is \emph{not} the unit element of this group.}. Consider the image of this element in $\GL_V(A)$, which is a $A$-linear map $A\ot V\to A\ot V$. This is the same as a $k$-linear map $V\to A\ot V$, i.e., a $k$-linear map
	\[
		V\to \mCO_G \ot V.
	\]
	One can verify this defines a $\mCO_G$-comodule structure on $V$.
\end{rem}

\begin{constr}
	There is a forgetful functor
	\[
		\Rep(G) \to \Lie(G)\hmod
	\]
	that can be constructed in the following equivalent ways:

	\begin{itemize}
		\item 
			Let we first suppose $V\in \Rep(G)$ is finite-dimensional. Then we have a homomorphism between algebraic groups $G\to \GL_V$ which induces a homomorphism between their Lie algebras $\Lie(G) \to \Lie(\GL_V) =\gl(V)$, i.e., a $\Lie(G)$-module structure on $V$.

			When $V$ is infinite-dimensional, the functor $\GL_V(-)$ is no longer represented by an algebraic group, but there is a formal method to define its Lie algebra and make the above construction work.
		\item
			Any $V\in \Rep(G)$ has an $\mCO_G$-comodule structure. By co-restricting along the map $\mCO_G \to \hat\mCO_G$, we obtain a (continuous) comodule structure for $\hat\mCO_G\simeq U(\mfg)^*$. Passing to duality, we obtain a module structure for $U(\mfg)$.

	\end{itemize}
\end{constr}

\begin{lem}
	The adjoint representation $\Lie(G)\in \Lie(G)\hmod$ has a canonical lift to an object in $\Rep(G)$. We call it the \textbf{adjoint action of $G$ on $\Lie(G)$}.
\end{lem}

\begin{rem}
	On the level of $k$-points, the conjugate action of $G(k)$ on $G$ induces an (abstract) action of $G(k)$ on $\Lie(G)$. This can be generalized to an action of $G(A)$ on $A\ot\Lie(G)$ for any $A\in \CAlg_k$ and thereby obtain the desired $G$-module structure on $\Lie(G)$.
\end{rem}

\begin{prop}
	If $G$ is connected, then the functor $\Rep(G) \to \Lie(G)\hmod$ is fully faithful. In particular, the $G$-invariance and $\Lie(G)$-invariance for a $G$-module are the same. 
\end{prop}

\begin{defn}
	If $G$ is connected, we say an object $V\in \Lie(G)\hmod$ is \textbf{$G$-integrable} if it is contained in the essential image of the above functor.
\end{defn}

\begin{warn}
	The similar claim for derived categories is false. In other words, extensions of $G$-integrable modules are not necessarily $G$-integrable. The multiplicative group $\mBG_\mbm$ is a counterexample.
\end{warn}

\section{Semisimple Algebraic Groups}

\begin{thm}
	Any semisimple Lie algebra $\mfg$ can be realized as the Lie algebra of an algebraic group. In the category of connected algebraic groups $G$ with $\Lie(G)\simeq \mfg$, there is a final object $G_\ad$ and an initial object $G_\sc$. Moreover, the homomorphisms
	\[
		G_\sc \to G \to G_\ad
	\]
	are \emph{isogenies}, i.e., are surjective and have finite kernels.
\end{thm}

\begin{exam}
	For $\mfg=\sl_n$, we have $G_\sc=\SL_n$ and $G_\ad = \mathsf{PGL}_n$.
\end{exam}


\begin{defn}
	We say $G$ is \textbf{semisimple}\footnote{This is an ad hoc definition that only is only correct under our assumptions on $k$.} if it is connected and its Lie algebra is semisimple. For a semisimple algebraic group $G$, we say it is \textbf{of adjoint type} (resp. \textbf{simply connected}) if it is of the form $G_\ad$ (resp. $G_\sc$).
\end{defn}

\begin{thm}
	If $G_\sc$ is a simply-connected semisimple algebraic group, then any finite-dimensional $\mfg$-module is $G_\sc$-integrable, i.e.,
	\[
		\Rep(G_\sc)_\mathsf{fd} = \mfg\hmod_\mathsf{fd}.
	\]
\end{thm}

\begin{warn}
	The similar claim for infinite-dimensional representation is false. This can be seen from the following exercise.
\end{warn}

\begin{exe}
	This is \red{Homework 1, Problem 5}. Let $G$ be any semisimple algebraic group with Lie algbra $\mfg$. Prove: any Verma module of $\mfg$ is not $G$-integrable.
\end{exe}


\begin{thm}
	Let $G$ be any semisimple algebraic group with Lie algebra $\mfg$. Then the abelian categories $\Rep(G)$ and $\mfg\hmod_\mathsf{fd}$ are semisimple\footnote{I.e., any object can be written as a direct sum of simple objects.}. Simple objects in $\Rep(G)$ are finite-dimensional, and an object $V\in \Rep(G)$ is simple iff it is an simple object in $\mfg\hmod$.

\end{thm}

\begin{comment}
\section{Finite dimensional representations}

This section is actually a continuation of the last lecture. Recall the following theorem:

\begin{thm}[Weyl]
	The abelian category $\mfg\hmod_\mathsf{fd}$ of finite-dimensional $\mfg$-modules is semisimple\footnote{Recall this means any object $M\in \mfg\hmod_\mathsf{fd}$ is a direct sum of simple objects.}.
\end{thm}

The following lemma is also well-known:

\begin{lem}
	Finite-dimensional $\mfg$-modules are weight modules.
\end{lem}

\proof
	By the preservation of Jordan decomposition (see \cite[Sect. 6.4]{H1}), for any finite-dimensional representation $\mfg\to \gl(V)$, semisimple (resp. nilpotent) elements of $\mfg$ are sent to semisimple (resp. nilpotent) elements in $\gl(V)$.
\qed

\begin{cor}
	Finite-dimensional $\mfg$-modules are contained in $\mCO$.
\end{cor}

\begin{rem}
	The dimension of weight spaces of finite-dimensional $\mfg$-modules can be calculated using \textbf{Weyl character formula}. We do not need this formula for now. In fact, there is a geometric proof of it that (hopefully) we will discuss later in this semester.
\end{rem}



\section{Blocks}

The main goal of this and the next lectures is to split the category $\mCO$ into blocks 
\[
	\mCO \simeq \bigoplus_{\chi\in \Spec(Z(\mfg))} \mCO_\chi
\]
and study each block. Here $Z(\mfg) \subset U(\mfg)$ is the center of the associative algebra $U(\mfg)$. 

Note that any $M\in \mfg\hmod$ can be viewed as a $Z(\mfg)$-module, which is the same as a quasi-coherent sheaf on $\Spec(Z(\mfg))$. We will prove:

\begin{thm}
	\label{thm-block}
	Let $M\in \mCO$, then the corresponding quasi-coherent sheaf on $\Spec(Z(\mfg))$ is supported on a 0-dimensional subvariety.
\end{thm}

Once we prove the above theorem, we make the following definition:

\begin{defn}
	For any closed point $\chi\in \Spec(Z(\mfg))$, let $\mCO_\chi \subset \mCO$ be the full subcategory containing those $M\in \mCO$ that is set-theoretically supported on $\chi$. We call $\mCO_\chi$ a \textbf{block} of $\mCO$.
\end{defn}


\begin{rem}
	In a more representation-theoretic language: a closed point $\Spec(Z(\mfg))$ is the same as a character $\chi: Z(\mfg) \to k$. Then for $M\in \mCO$, it is set-theoretically supported on $\chi$ iff $\ker(\chi)^n \cdot M = 0$ for $n>>0$.

\end{rem}

\begin{cor}
	We have
	\[
		\mCO \simeq \bigoplus_{\chi\in \Spec(Z(\mfg))} \mCO_\chi.
	\]
	In other words, any $M\in \mCO$ admits a unique finite decomposition $M = \oplus_\chi M_\chi$ such that $M_\chi \in \mCO_\chi$. Also, for $M = \oplus_\chi M_\chi$ and $N = \oplus_\chi N_\chi$, we have
	\[
		\Hom_\mCO( M,N ) \simeq \oplus_\chi \Hom_{\mCO_\chi} ( M_\chi, N_\chi ).
	\]
\end{cor}

\proof
	The decomposition $M = \oplus_\chi M_\chi$ follows obviously from the theorem. For $\chi\neq \chi'$, there is no nonzero $\mfg$-linear map $M_\chi \to N_{\chi'}$ because there is no nonzero $Z(\mfg)$-linear map between them. 
\qed

\begin{rem}
	In the future, we will see the decomposition $\mCO \simeq \bigoplus_{\chi\in \Spec(Z(\mfg))} \mCO_\chi$ also holds in the derived setting. This implies we can replace $\Hom_\bullet(-,-)$ in above by $\Ext^i_\bullet(-,-)$.
\end{rem}

To pvove Theorem \ref{thm-block}, we need (a lot of) preparations.

\section{The algebra \texorpdfstring{$\gr^\bullet Z(\mfg)$}{gr Z(g)}}

\begin{constr}
	The PBW filtration on $U(\mfg)$ induces a filtration on $Z(\mfg)$ with $\mbF^{\le i} Z(\mfg):= Z(\mfg)\cap \mbF^{\le i}U(\mfg)$. Note that $\gr^\bullet Z(\mfg)$ is a subalgebra of $\gr^\bullet U(\mfg) \simeq \Sym^\bullet(\mfg)$.
\end{constr}

We want to understand the commutative algebra $Z(\mfg)$. At a first approximation, let us study $\gr^\bullet Z(\mfg)$.

\begin{constr}
	\label{constr-adj-action-sym}
	Recall $\Sym^\bullet(\mfg)$ has a natural $\mfg$-module structure constructed as follows. For any $V\in \mfg$, there is a natural $\mfg$-module structure on $V^{\otimes n}$ given by 
	\[
		\mfg \mt V^{\otimes n} \to V^{\otimes n},\; (x, \ot_i v_i)\mapsto \sum_i ( v_1\ot \cdots\ot v_{i-1}\ot (x\cdot v_i) \ot v_{i+1}\cdots \ot v_n ).
	\]
	This action is compatible with the symmetric group $\Sigma_n$-action on $V^{\otimes n}$ and thereby induces a $\mfg$-module structure on $\Sym^n(\mfg)$. Taking direct sum, we obtain a $\mfg$-module structure on $\Sym^\bullet(V)$. 

	In the case $V=\mfg$, to distinguish with the multiplication structure on $\Sym^\bullet(\mfg)$, we denote this action by
	\[
		\mfg \times \Sym^\bullet(\mfg) \to \Sym^\bullet(\mfg) ,\; (x,u)\mapsto \ad_x(u),
	\]
	and call it the \textbf{adjoint action}.

	Note that by definition, for $x\in \mfg$ and $u,v\in \Sym^\bullet(\mfg)$, we have
	\[
		\ad_x(u\cdot v) = \ad_x(u)\cdot v + u\cdot \ad_x(v).
	\]
	In particular, the \emph{$\mfg$-invariance} 
	\[
		\Sym^\bullet(\mfg)^\mfg:= \{ u\in \Sym^\bullet(\mfg)\,\vert\, \ad_x(u)=0\textrm{ for any }x\in \mfg \}
	\]
	is a subalgebra of $\Sym^\bullet(\mfg)$.
\end{constr}

\begin{lem}
	\label{lem-Zg-fil}
	There is a unique dotted isomorphism making the following diagram commute
	\[
		\xymatrix{
			\gr^\bullet Z(\mfg) \ar@{.>}[r]^-\simeq \ar[d]^-\subset
			& \Sym^\bullet(\mfg)^\mfg \ar[d]^-\subset \\
			\gr^\bullet U(\mfg) \ar[r]^-\simeq 
			& \Sym^\bullet(\mfg),
		}
	\]
	where the bottom isomorphism is given by the PBW theorem.
\end{lem}

To prove this lemma, we use the following exercise:

\begin{exe}
	This is \red{Homework 1, Problem 4}. Prove: the adjoint $\mfg$-action on $U(\mfg)$, i.e.,
	\[
		\mfg \times U(\mfg) \to U(\mfg),\; (x,u)\mapsto \ad_x(u)=[x,u],
	\]
	preserves each $\mbF^{\le n}U(\mfg)$, and the induced $\mfg$-action on $\gr^\bullet(U(\mfg)) \simeq \Sym^\bullet(\mfg)$ is the adjoint action in Construction \ref{constr-adj-action-sym}.


\end{exe}


\proof[Proof of Lemma \ref{lem-Zg-fil}]
	By the above exercise, we have a short exact sequence of \emph{finite-dimensional} $\mfg$-modules:
	\[
		0 \to \mbF^{\le n-1} U(\mfg) \to \mbF^{\le n} U(\mfg)  \to \Sym^n(\mfg) \to 0.
	\]
	Since $\mfg\hmod_\mathsf{fd}$ is semisimple, this short exact sequence splits. Hence taking $\mfg$-invariance, we obtain\footnote{
		Warning: in general, taking invariance is only \emph{left exact}. (Memory method: it is given by $\Hom_{\mfg}(k,-)$.) Hence we need the existence of a splitting.
	}
	\[
		0 \to \mbF^{\le n-1} Z(\mfg) \to \mbF^{\le n} Z(\mfg) \to \Sym^n(\mfg)^\mfg \to 0
	\]
	This gives the desired isomorphism $\gr^n Z(\mfg) \simeq \Sym^n(\mfg)^\mfg$.


\qed[Lemma \ref{lem-Zg-fil}]

	We want to further explore the structure of $\Sym^\bullet(\mfg)^\mfg$, e.g., giving a pure combinatorial (=root system) description of it. This is known as the \textbf{Chevalley isomorphism}

\begin{thm}[Chevalley]
		There is a canonical isomorphism between commutative algebras
		\[
			\Sym^\bullet(\mfg)^\mfg \simeq \Sym^\bullet(\mft)^W.
		\]
\end{thm}

	In above, $W$ is the \emph{Weyl group} of $\mfg$. We need review its properties before we can prove the above theorem.

\section{Recollection: Weyl group}


	Recall (see Lecture 1) $\mfg$ corresponds to a root system $(E,\Phi)$, where $E=E_\mBQ \ot_\mBQ \mBR$ is an Euclidean space and $E_\mBQ:=\mBQ \Phi$ is the $\mBQ$-vector subspace of $\mft^*$ spanned by the roots $\Phi$.

	For each $\alpha\in \Phi$, we defined a reflection of $E$:
	\[
		s_\alpha(\beta):= \beta - 2\frac{(\alpha,\beta)}{(\alpha,\alpha)} \alpha.
	\]

\begin{defn}
	The \textbf{Weyl group} of a root system $(E,\Phi)$ is the subgroup $W\subset \GL(E)$ generated by the reflections $s_\alpha$, $\alpha\in \Phi$.

	For a semisimple Lie algebra $\mfg$, its \textbf{Weyl group} $W$ is the Weyl group of its root system.
\end{defn}

\begin{rem}
	The action of $W$ on $E$ preserves $E_\mBQ$ and therefore induces actions on $\mft^*$ and $\mft$.
\end{rem}

\begin{exam}
	The Weyl group of $\sl_n$ is the symmetric group $\Sigma_n$. It acts on the standard Cartan in the obvious way.
\end{exam}

	We state several well-known results about the Weyl group:

\begin{prop}
	Let $(E,\Phi)$ and $(E^*,\check \Phi)$ be root systems dual to each other. Then their Weyl groups are isomorphic, with $s_\alpha$ corresponding to $s_{\check \alpha}$.
\end{prop}

\begin{rem}
	Weyl groups are \emph{(crystallographic) Coxeter groups}. Namely, if we choose positive roots $\Phi^+\subset \Phi$ and simple roots $\mathsf{\Delta}\subset \Phi^+$. Then
	\[
		W  = \langle s_\alpha, \alpha\in \mathsf{\Delta} \,\vert\, s_\alpha^2=1, (s_\alpha s_\beta)^{m_\alpha\beta}=1, \alpha\neq \beta \rangle
	\]
	where $m_{\alpha\beta}\in \{2,3,4,6\}$.
\end{rem}
\end{comment}

	




\begin{thebibliography}{Yau}

\bibitem[B]{B} Borel, Armand. Linear algebraic groups. Vol. 126. Springer Science \& Business Media, 2012.

\bibitem[Bou]{Bou} Bourbaki, Nicolas. Groupes et Algebres de Lie: Elements de Mathematique. Hermann, 1968.

\bibitem[H]{H} Humphreys, James E. Linear algebraic groups. Vol. 21. Springer Science \& Business Media, 2012.

\bibitem[L]{L} Lusztig, George. Introduction to quantum groups. Springer Science \& Business Media, 2010.

\bibitem[M]{M} Milne, James S. Algebraic groups: the theory of group schemes of finite type over a field. Vol. 170. Cambridge University Press, 2017.


\bibitem[Sp]{Sp} Springer, Tonny Albert. "Linear algebraic groups." In Algebraic Geometry IV: Linear Algebraic Groups Invariant Theory, pp. 1-121. Berlin, Heidelberg: Springer Berlin Heidelberg, 1998.

\end{thebibliography}

\end{document} 


