
%!TEX root = main.tex
\documentclass{amsart}
\textwidth=14.5cm \oddsidemargin=1cm
\evensidemargin=1cm
\usepackage{amsmath}
\usepackage{amsxtra}
\usepackage{amscd}
\usepackage{amsthm}
\usepackage{amsfonts}
\usepackage{amssymb}
\usepackage[foot]{amsaddr}
\usepackage{cite}
\usepackage{url}
\usepackage{rotating}
\usepackage{eucal}
\usepackage{tikz-cd}
\usepackage[all,2cell,color]{xy}
\UseAllTwocells
\UseCrayolaColors
\usepackage{graphicx}
\usepackage{pifont}
\usepackage{comment}
\usepackage{verbatim}
\usepackage{xcolor}
\usepackage{hyperref}
\usepackage{xparse}
\usepackage{upgreek}
\usepackage{MnSymbol}
\sloppy


%%%%%%%%%%%%%%%%%%%%Theorem%%%%%%%%%%%%%%%%%%%%
\newcounter{theorem}
\setcounter{theorem}{0}

\newtheorem{cor}[subsection]{Corollary}
\newtheorem{lem}[subsection]{Lemma}
\newtheorem{goal}[subsection]{Goal}
\newtheorem{lemdefn}[subsection]{Lemma-Definition}
\newtheorem{prop}[subsection]{Proposition}
\newtheorem{propdefn}[subsection]{Proposition-Definition}
\newtheorem{cordefn}[subsection]{Corollary-Definition}
\newtheorem{variant}[subsection]{Variant}
\newtheorem{warn}[subsection]{Warning}
\newtheorem{sugg}[subsection]{Suggestion}
\newtheorem{facts}[subsection]{Fact}
\newtheorem{ques}{Question}
\newtheorem{guess}{Guess}
\newtheorem{claim}{Claim}
\newtheorem{propconstr}[subsection]{Proposition-Construction}
\newtheorem{lemconstr}[subsection]{Lemma-Construction}
\newtheorem{ax}{Axiom}
\newtheorem{conje}[subsection]{Conjecture}
\newtheorem{mainthm}[subsection]{Main-Theorem}
\newtheorem{summ}[subsection]{Summary}
\newtheorem{thm}[subsection]{Theorem}
\newtheorem{thmdefn}[subsection]{Theorem-Definition}
\newtheorem{notn}[subsection]{Notation}
\newtheorem{convn}[subsection]{Convention}
\newtheorem{constr}[subsection]{Construction}


\theoremstyle{definition}

\newtheorem{defn}[subsection]{Definition}
\newtheorem{exam}[subsection]{Example}
\newtheorem{assum}[subsection]{Assumption}

\theoremstyle{remark}
\newtheorem{rem}[subsection]{Remark}
\newtheorem{exe}[subsection]{Exercise}


\numberwithin{equation}{section}


%%%%%%%%%%%%%%%%%%%%Commands%%%%%%%%%%%%%%%%%%%%

\newcommand{\nc}{\newcommand}
\nc\on{\operatorname}
\nc\renc{\renewcommand}


%%%%%%%%%%%%%%%%%%%%Sections%%%%%%%%%%%%%%%%%%%%

\nc\ssec{\subsection}
\nc\sssec{\subsubsection}

%%%%%%%%%%%%%%%%%%%%Environment%%%%%%%%%%%%%%%%%
\nc\blongeqn{\[ \begin{aligned}}
\nc\elongeqn{\end{aligned} \]}



%%%%%%%%%%%%%%%%%%%%Mathfont%%%%%%%%%%%%%%%%%%%%

\nc\mBA{{\mathbb A}}
\nc\mBB{{\mathbb B}}
\nc\mBC{{\mathbb C}}
\nc\mBD{{\mathbb D}}
\nc\mBE{{\mathbb E}}
\nc\mBF{{\mathbb F}}
\nc\mBG{{\mathbb G}}
\nc\mBH{{\mathbb H}}
\nc\mBI{{\mathbb I}}
\nc\mBJ{{\mathbb J}}
\nc\mBK{{\mathbb K}}
\nc\mBL{{\mathbb L}}
\nc\mBM{{\mathbb M}}
\nc\mBN{{\mathbb N}}
\nc\mBO{{\mathbb O}}
\nc\mBP{{\mathbb P}}
\nc\mBQ{{\mathbb Q}}
\nc\mBR{{\mathbb R}}
\nc\mBS{{\mathbb S}}
\nc\mBT{{\mathbb T}}
\nc\mBU{{\mathbb U}}
\nc\mBV{{\mathbb V}}
\nc\mBW{{\mathbb W}}
\nc\mBX{{\mathbb X}}
\nc\mBY{{\mathbb Y}}
\nc\mBZ{{\mathbb Z}}


\nc\mCA{{\mathcal A}}
\nc\mCB{{\mathcal B}}
\nc\mCC{{\mathcal C}}
\nc\mCD{{\mathcal D}}
\nc\mCE{{\mathcal E}}
\nc\mCF{{\mathcal F}}
\nc\mCG{{\mathcal G}}
\nc\mCH{{\mathcal H}}
\nc\mCI{{\mathcal I}}
\nc\mCJ{{\mathcal J}}
\nc\mCK{{\mathcal K}}
\nc\mCL{{\mathcal L}}
\nc\mCM{{\mathcal M}}
\nc\mCN{{\mathcal N}}
\nc\mCO{{\mathcal O}}
\nc\mCP{{\mathcal P}}
\nc\mCQ{{\mathcal Q}}
\nc\mCR{{\mathcal R}}
\nc\mCS{{\mathcal S}}
\nc\mCT{{\mathcal T}}
\nc\mCU{{\mathcal U}}
\nc\mCV{{\mathcal V}}
\nc\mCW{{\mathcal W}}
\nc\mCX{{\mathcal X}}
\nc\mCY{{\mathcal Y}}
\nc\mCZ{{\mathcal Z}}


\nc\mbA{{\mathsf A}}
\nc\mbB{{\mathsf B}}
\nc\mbC{{\mathsf C}}
\nc\mbD{{\mathsf D}}
\nc\mbE{{\mathsf E}}
\nc\mbF{{\mathsf F}}
\nc\mbG{{\mathsf G}}
\nc\mbH{{\mathsf H}}
\nc\mbI{{\mathsf I}}
\nc\mbJ{{\mathsf J}}
\nc\mbK{{\mathsf K}}
\nc\mbL{{\mathsf L}}
\nc\mbM{{\mathsf M}}
\nc\mbN{{\mathsf N}}
\nc\mbO{{\mathsf O}}
\nc\mbP{{\mathsf P}}
\nc\mbQ{{\mathsf Q}}
\nc\mbR{{\mathsf R}}
\nc\mbS{{\mathsf S}}
\nc\mbT{{\mathsf T}}
\nc\mbU{{\mathsf U}}
\nc\mbV{{\mathsf V}}
\nc\mbW{{\mathsf W}}
\nc\mbX{{\mathsf X}}
\nc\mbY{{\mathsf Y}}
\nc\mbZ{{\mathsf Z}}

\nc\mba{{\mathsf a}}
\nc\mbb{{\mathsf b}}
\nc\mbc{{\mathsf c}}
\nc\mbd{{\mathsf d}}
\nc\mbe{{\mathsf e}}
\nc\mbf{{\mathsf f}}
\nc\mbg{{\mathsf g}}
\nc\mbh{{\mathsf h}}
\nc\mbi{{\mathsf i}}
\nc\mbj{{\mathsf j}}
\nc\mbk{{\mathsf k}}
\nc\mbl{{\mathsf l}}
\nc\mbm{{\mathsf m}}
\nc\mbn{{\mathsf n}}
\nc\mbo{{\mathsf o}}
\nc\mbp{{\mathsf p}}
\nc\mbq{{\mathsf q}}
\nc\mbr{{\mathsf r}}
\nc\mbs{{\mathsf s}}
\nc\mbt{{\mathsf t}}
\nc\mbu{{\mathsf u}}
\nc\mbv{{\mathsf v}}
\nc\mbw{{\mathsf w}}
\nc\mbx{{\mathsf x}}
\nc\mby{{\mathsf y}}
\nc\mbz{{\mathsf z}}




\nc\mbfA{{\mathbf A}}
\nc\mbfB{{\mathbf B}}
\nc\mbfC{{\mathbf C}}
\nc\mbfD{{\mathbf D}}
\nc\mbfE{{\mathbf E}}
\nc\mbfF{{\mathbf F}}
\nc\mbfG{{\mathbf G}}
\nc\mbfH{{\mathbf H}}
\nc\mbfI{{\mathbf I}}
\nc\mbfJ{{\mathbf J}}
\nc\mbfK{{\mathbf K}}
\nc\mbfL{{\mathbf L}}
\nc\mbfM{{\mathbf M}}
\nc\mbfN{{\mathbf N}}
\nc\mbfO{{\mathbf O}}
\nc\mbfP{{\mathbf P}}
\nc\mbfQ{{\mathbf Q}}
\nc\mbfR{{\mathbf R}}
\nc\mbfS{{\mathbf S}}
\nc\mbfT{{\mathbf T}}
\nc\mbfU{{\mathbf U}}
\nc\mbfV{{\mathbf V}}
\nc\mbfW{{\mathbf W}}
\nc\mbfX{{\mathbf X}}
\nc\mbfY{{\mathbf Y}}
\nc\mbfZ{{\mathbf Z}}

\nc\mbfa{{\mathbf a}}
\nc\mbfb{{\mathbf b}}
\nc\mbfc{{\mathbf c}}
\nc\mbfd{{\mathbf d}}
\nc\mbfe{{\mathbf e}}
\nc\mbff{{\mathbf f}}
\nc\mbfg{{\mathbf g}}
\nc\mbfh{{\mathbf h}}
\nc\mbfi{{\mathbf i}}
\nc\mbfj{{\mathbf j}}
\nc\mbfk{{\mathbf k}}
\nc\mbfl{{\mathbf l}}
\nc\mbfm{{\mathbf m}}
\nc\mbfn{{\mathbf n}}
\nc\mbfo{{\mathbf o}}
\nc\mbfp{{\mathbf p}}
\nc\mbfq{{\mathbf q}}
\nc\mbfr{{\mathbf r}}
\nc\mbfs{{\mathbf s}}
\nc\mbft{{\mathbf t}}
\nc\mbfu{{\mathbf u}}
\nc\mbfv{{\mathbf v}}
\nc\mbfw{{\mathbf w}}
\nc\mbfx{{\mathbf x}}
\nc\mbfy{{\mathbf y}}
\nc\mbfz{{\mathbf z}}

\nc\mfa{{\mathfrak a}}
\nc\mfb{{\mathfrak b}}
\nc\mfc{{\mathfrak c}}
\nc\mfd{{\mathfrak d}}
\nc\mfe{{\mathfrak e}}
\nc\mff{{\mathfrak f}}
\nc\mfg{{\mathfrak g}}
\nc\mfh{{\mathfrak h}}
\nc\mfi{{\mathfrak i}}
\nc\mfj{{\mathfrak j}}
\nc\mfk{{\mathfrak k}}
\nc\mfl{{\mathfrak l}}
\nc\mfm{{\mathfrak m}}
\nc\mfn{{\mathfrak n}}
\nc\mfo{{\mathfrak o}}
\nc\mfp{{\mathfrak p}}
\nc\mfq{{\mathfrak q}}
\nc\mfr{{\mathfrak r}}
\nc\mfs{{\mathfrak s}}
\nc\mft{{\mathfrak t}}
\nc\mfu{{\mathfrak u}}
\nc\mfv{{\mathfrak v}}
\nc\mfw{{\mathfrak w}}
\nc\mfx{{\mathfrak x}}
\nc\mfy{{\mathfrak y}}
\nc\mfz{{\mathfrak z}}

\nc{\one}{{\mathsf{1}}}


\nc\clambda{ {\check{\lambda} }}
\nc\cmu{ {\check{\mu} }}

\nc\bDelta{\mathsf{\Delta}}
\nc\bGamma{\mathsf{\Gamma}}
\nc\bLambda{\mathsf{\Lambda}}


\nc\loccit{\emph{loc.cit.}}



%%%%%%%%%%%%%%%%%%%%Operations-limit%%%%%%%%%%%%%%%%%%%%

\NewDocumentCommand{\ot}{e{_^}}{
  \mathbin{\mathop{\otimes}\displaylimits
    \IfValueT{#1}{_{#1}}
    \IfValueT{#2}{^{#2}}
  }
}
\NewDocumentCommand{\boxt}{e{_^}}{
  \mathbin{\mathop{\boxtimes}\displaylimits
    \IfValueT{#1}{_{#1}}
    \IfValueT{#2}{^{#2}}
  }
}
\NewDocumentCommand{\mt}{e{_^}}{
  \mathbin{\mathop{\times}\displaylimits
    \IfValueT{#1}{_{#1}}
    \IfValueT{#2}{^{#2}}
  }
}
\NewDocumentCommand{\convolve}{e{_^}}{
  \mathbin{\mathop{\star}\displaylimits
    \IfValueT{#1}{_{#1}}
    \IfValueT{#2}{^{#2}}
  }
}
\NewDocumentCommand{\colim}{e{_^}}{
  \mathbin{\mathop{\operatorname{colim}}\displaylimits
    \IfValueT{#1}{_{#1}\,}
    \IfValueT{#2}{^{#2}\,}
  }
}
\NewDocumentCommand{\laxlim}{e{_^}}{
  \mathbin{\mathop{\operatorname{laxlim}}\displaylimits
    \IfValueT{#1}{_{#1}\,}
    \IfValueT{#2}{^{#2}\,}
  }
}
\NewDocumentCommand{\oplaxlim}{e{_^}}{
  \mathbin{\mathop\operatorname{oplax-lim}\displaylimits
    \IfValueT{#1}{_{#1}\,}
    \IfValueT{#2}{^{#2}\,}
  }
}


%%%%%%%%%%%%%%%%%%%%Arrows%%%%%%%%%%%%%%%%%%%%


\makeatletter
\newcommand{\laxto}{\dashedrightarrow}
\newcommand{\xrightleftarrows}[1]{\mathrel{\substack{\xrightarrow{#1} \\[-.9ex] \xleftarrow{#1}}}}
\newcommand{\adj}{\xrightleftarrows{\rule{0.5cm}{0cm}}}

\newcommand*{\da@rightarrow}{\mathchar"0\hexnumber@\symAMSa 4B }
\newcommand*{\da@leftarrow}{\mathchar"0\hexnumber@\symAMSa 4C }
\newcommand*{\xlaxto}[2][]{%
  \mathrel{%
    \mathpalette{\da@xarrow{#1}{#2}{}\da@rightarrow{\,}{}}{}%
  }%
}
\newcommand{\xlaxgets}[2][]{%
  \mathrel{%
    \mathpalette{\da@xarrow{#1}{#2}\da@leftarrow{}{}{\,}}{}%
  }%
}
\newcommand*{\da@xarrow}[7]{%
  % #1: below
  % #2: above
  % #3: arrow left
  % #4: arrow right
  % #5: space left 
  % #6: space right
  % #7: math style 
  \sbox0{$\ifx#7\scriptstyle\scriptscriptstyle\else\scriptstyle\fi#5#1#6\m@th$}%
  \sbox2{$\ifx#7\scriptstyle\scriptscriptstyle\else\scriptstyle\fi#5#2#6\m@th$}%
  \sbox4{$#7\dabar@\m@th$}%
  \dimen@=\wd0 %
  \ifdim\wd2 >\dimen@
    \dimen@=\wd2 %   
  \fi
  \count@=2 %
  \def\da@bars{\dabar@\dabar@}%
  \@whiledim\count@\wd4<\dimen@\do{%
    \advance\count@\@ne
    \expandafter\def\expandafter\da@bars\expandafter{%
      \da@bars
      \dabar@ 
    }%
  }%  
  \mathrel{#3}%
  \mathrel{%   
    \mathop{\da@bars}\limits
    \ifx\\#1\\%
    \else
      _{\copy0}%
    \fi
    \ifx\\#2\\%
    \else
      ^{\copy2}%
    \fi
  }%   
  \mathrel{#4}%
}
\makeatother

%%%%%%%%%%%%%%%%%%%%Decorations%%%%%%%%%%%%%%%%%%%%
\nc{\wt}{\widetilde}
\nc{\ol}{\overline}

\nc{\red}{\textcolor{red}}
\nc{\blue}{\textcolor{blue}}
\nc{\purple}{\textcolor{violet}}

\nc{\simorlax}{{\red\simeq/\blue\lax}}

%%%%%%%%%%%%%%%%%%%%All%%%%%%%%%%%%%%%%%%%%

\nc{\Id}{\mathsf{Id}}
\nc{\gl}{\mathfrak{gl}}
\renc{\sl}{\mathfrak{sl}}
\nc{\GL}{\mathsf{GL}}
\nc{\SL}{\mathsf{SL}}
\nc{\PGL}{\mathsf{PGL}}
\nc{\hmod}{\mathsf{-mod}}
\nc{\Vect}{\mathsf{Vect}}
\nc{\tr}{\mathsf{tr}}
\nc{\Kil}{\mathsf{Kil}}
\nc{\ad}{{\mathsf{ad}}}
\nc{\Ad}{\mathsf{Ad}}
\nc{\oblv}{\mathsf{oblv}}
\nc{\gr}{\mathsf{gr}}
\nc{\Sym}{\mathsf{Sym}}
\nc{\QCoh}{\mathsf{QCoh}}
\nc{\ind}{\mathsf{ind}}
\nc{\Spec}{\mathsf{Spec}}
\nc{\Hom}{\mathsf{Hom}}
\nc{\Ext}{\mathsf{Ext}}
\nc{\Grp}{\mathsf{Grp}}
\nc{\pt}{\mathsf{pt}}
\nc{\Lie}{\mathsf{Lie}}
\nc{\CAlg}{\mathsf{CAlg}}
\nc{\Der}{\mathsf{Der}}
\nc{\Rep}{\mathsf{Rep}}
\renc{\sc}{{\mathsf{sc}}}
\nc{\Fl}{\mathsf{Fl}}
\nc{\Fun}{\mathsf{Fun}}
\nc{\ev}{\mathsf{ev}}
\nc{\surj}{\twoheadrightarrow}
\nc{\inj}{\hookrightarrow}
\nc{\HC}{\mathsf{HC}}
\nc{\cl}{\mathsf{cl}}
\renc{\Im}{\mathsf{Im}}
\renc{\ker}{\mathsf{ker}}
\nc{\coker}{\mathsf{coker}}
\nc{\Tor}{\mathsf{Tor}}
\nc{\op}{\mathsf{op}}
\nc{\length}{\mathsf{length}}
\nc{\fd}{{\mathsf{fd}}}
\nc{\weight}{\mathsf{wt}}
\nc{\semis}{{\mathsf{ss}}}
\nc{\qc}{{\mathsf{qc}}}
\nc{\pr}{\mathsf{pr}}
\nc{\act}{\mathsf{act}}
\nc{\dR}{{\mathsf{dR}}}
\nc{\hol}{{\mathsf{hol}}}
\nc{\Pic}{{\mathsf{Pic}}}
\nc{\Loc}{\mathsf{Loc}}
\nc{\IC}{\mathsf{IC}}

\begin{document}


\title{Lecture 2}

\date{Mar 4, 2024}

\maketitle


\section{Universal enveloping algebra}

\begin{constr}
	Recall we have a forgetful functor $\mathsf{oblv}:\mathsf{Alg}_k \to \mathsf{Lie}_k$ from the category of associative algebras to that of Lie algebras. This functor admits a left adjoint
	\[
		U: \mathsf{Lie}_k \to \mathsf{Alg}_k
	\]
	that sends a Lie algebra $\mfg$ to the associative algebra
	\[
		U(\mfg) = T(\mfg)/\langle xy-yx-[x,y],\; x,y\in \mfg \rangle.
	\]
	Here 
	\[
		T(\mfg) := \bigoplus_{n\ge 0} \mfg^{\otimes n}
	\]
	is the tensor algebra of the underlying vector space of $\mfg$, and $\langle xy-yx-[x,y],\; x,y\in \mfg \rangle$ is the two-sided ideal generated by elements of the form $xy-yx-[x,y]$.

	The associative algebra $U(\mfg)$ is called the \textbf{universal enveloping algebra} of $\mfg$.
\end{constr}

Let $U(\mfg) \hmod$ be the abelian category of left modules for $U(\mfg)$.

\begin{lem}
	There is an equivalence
	\[
		\mfg\hmod \simeq U(\mfg) \hmod
	\]
	that commutes with forgetful functors to $\Vect_k$.
\end{lem}

\proof
	For a given vector space $V$, a $\mfg$-module structure on $V$ is a Lie algebra homomorphism $\mfg \to \oblv(\gl(V))$. By adjunction, this is the same as a homomorphism $U(\mfg) \to \gl(V)$, i.e., a left $U(\mfg)$-module structure on $V$.

\qed

\begin{constr}
	The tensor algebra $T(\mfg)$ is naturally \emph{graded}. But this grading does \emph{not} descent to $U(\mfg)$ because $xy-yx-[x,y]$ is not a homogenous element. Instead, $U(\mfg)$ has an exhausted filtration
	\[
		\mbF^{\le n} U(\mfg) := \mathsf{im}( \mbF^{\le n} T(\mfg) \to U(\mfg) )
	\]
	that is compatible with the algebra structure, i.e.,
	\[
		\mbF^{\le m} U(\mfg) \ot_k \mbF^{\le n} U(\mfg) \xrightarrow{\mathsf{mult}} \mbF^{\le {m+n}} U(\mfg) .
 	\]
 	Taking associated graded pieces, we obtain a graded algebra 
 	\[
 		\gr^\bullet U(\mfg) := \bigoplus_{n\ge 0} \mbF^{\le n} U(\mfg) / \mbF^{<n} U(\mfg).
 	\]

 	By the universal property of the tensor algebra, we have a unique homomorphism $T(\mfg) \to \gr^\bullet U(\mfg)$ whose restriction on $\mfg\subset T(\mfg)$ is the composition $\mfg \to \mbF^{\le 1}U(\mfg) \to \gr^1 U(\mfg) \subset \gr^\bullet U(\mfg)$. Denote this composition by $x\mapsto \bar{x}$. Note that we have $\bar{x}\bar{y} = \bar{y} \bar{x}$ as elements in $\gr^2U(\mfg)$ because the term $[x,y]$ is killed by the surjection $\mbF^{\le 2}U(\mfg) \to \gr^2 U(\mfg)$. It follows that we have a commutative diagram of surjective maps:
 	\[
 		\xymatrix{
 			T(\mfg) \ar@{->>}[r]^-{x\mapsto \bar{x}} \ar@{->>}[d] &
 			\gr^\bullet U(\mfg) \\
 			\Sym(\mfg), \ar@{.>>}[ru]_-\phi
 		}
 	\]
 	where $\Sym(\mfg):= T(\mfg)/\langle xy-yx \rangle $ is the \emph{symmetric algebra} of $\mfg$. In particular, $\gr^\bullet U(\mfg)$ is a commutative algebra.

\end{constr}

\begin{rem}
	Note that $\gr^\bullet U(\mfg)$ being commutative is equivalent to $[\mbF^i U(\mfg), \mbF^j U(\mfg)] \subset \mbF^{i+j-1} U(\mfg)$, where we write $\mbF^{-n}U(\mfg)=0$ for $n>0$.

\end{rem}

\begin{thm}[Poincaré--Birkhoff--Witt, a.k.a. PBW]
 	For any Lie algebra $\mfg$, the above homomorphism $\phi:\Sym(\mfg)\to  \gr^\bullet U(\mfg)$ is an isomorphism.
\end{thm}

\begin{cor}
	\label{cor-basis-Ug}
	Let $\{x_i\}_{i\in I}$ be a basis of $\mfg$ as a vector space. Choose a total order on the set $I$. Then the set 
	$\{ x_{i_1}^{m_1} x_{i_2}^{m_2} \cdots x_{i_n}^{m_n} \,\vert\, n\ge 0, i_1<i_2<\cdots <i_n, m_1,m_2,\dots,m_n\in \mBZ^{>0} \}$ is a basis of the vector space $U(\mfg)$.
\end{cor}

\begin{cor}
	If $\mfg$ is a finite-dimensional algebra, then $U(\mfg)$ is left and right Noetherian. 
\end{cor}

\proof
	A filtered ring $A$ is left (resp. right) Noetherian if its assoicated graded ring $\gr^\bullet A$ is so. See \cite[Chapter 1, Theorem 6.9]{MR}\footnote{Sketch: a left ideal $I\subset A$ defines a left ideal $\gr^\bullet I\subset \gr^\bullet A$ with $\gr^n I= ((I + \mbF^{n-1}A) \cap \mbF^nA)/\mbF^{n-1}A$. This assignment is injective.}.
\qed

\section{Verma modules}

From now on, we fix a finite-dimensional semisimple Lie algebra $\mfg$ and choose $\mft\subset\mfb \subset \mfg$, i.e., a Cartain subalgebra and a Borel subalgebra of it. Recall we have $\mfg = \mfn^- \oplus \mft \oplus \mfn$, $\mfb = \mft\oplus \mfn$, $\mft\simeq \mfb/\mfn$ and $\mfn = [\mfb,\mfb]$\footnote{We didn't mention the last one in the last lecture, but it follows easily from the root decomposition.}.

\begin{constr}
	The projection $\mfb \to \mft$ induces a restriction functor $\mft\hmod \to \mfb\hmod$. Note that we have 
	\begin{equation} \label{eqn-Fourier}
		\mft\hmod \simeq U(\mft)\hmod \simeq \Sym(\mft)\hmod \simeq \QCoh(\mft^*).
	\end{equation}
	Hence for any $\lambda\in \mft^*$, the skyscrapter sheaf at $\lambda$ gives a 1-dimensional representation
	\[
		k_\lambda \in \mft\hmod.
	\]
	In other words, for $x\in \mft$, its action on $k_\lambda$ is given by the scaler $\lambda(x)$.

	We abuse notation and write $k_\lambda$ for the corresponding object in $\mfb\hmod$.
\end{constr}

\begin{rem}
	Note that any 1-dimensional $\mfb$-module $V$ is of the form $k_\lambda$. Indeed, the Lie homomorphism $\mfb \to \gl(V)$ must kill $\mfn=[\mfb,\mfb]$ because $\gl(V)$ is abelian.

\end{rem}

\begin{defn}
	Consider the restriction functor $\mfg\hmod \to \mfb\hmod$ and its left adjoint
	\[
		\ind_\mfb^\mfg: \mfb\hmod\to \mfg\hmod.
	\]
	For any weight $\lambda\in \mft^*$, we define the \textbf{Verma module} to be
	\[
		M_\lambda := \ind_\mfb^\mfg( k_\lambda ) \in \mfg\hmod.
	\]

\end{defn}

\begin{rem}
Explicitly, we have
\[
	M_\lambda \simeq U(\mfg)\ot_{U(\mfb)} k_\lambda.
\]
In particular, $M_\lambda$ is infinite-dimensional.
\end{rem}

\begin{defn}
	By adjunction, there is a $\mfb$-linear map $k_\lambda\to M_\lambda$ corresponding to the identity morphism $M_\lambda\to M_\lambda$ in $\mfg\hmod$. After fixing a nonzero vector $1_\lambda$ of $k_\lambda$, we obtain a vector $v_\lambda \in M_\lambda$. We call it a \textbf{highest weight vector} of $M_\lambda$.
\end{defn}

The meaning of this name will be explained shortly. Note that by definition, $\mfn\cdot v_\lambda=0$ and $v_\lambda$ is a $\lambda$-eigenvector for the $\mft$-action.

\begin{exe}\label{exe-Verma}
	This is \red{Homework 1, Problem 1}. Prove:
	\begin{itemize}
		\item[(1)]
			The map
			\[
				U(\mfn^-) \ot_k U (\mfb) \xrightarrow{\mathsf{mult}} U(\mfg)
			\]
			is an isomorphism between $(U(\mfn^-) , U (\mfb))$-bimodules.
		\item[(2)]
			As an $\mfn^-$-module, $M_\lambda$ is freely generated by $v_\lambda$, i.e.,
			\[
				U(\mfn^-) \to M_\lambda,\; x \mapsto x\cdot v_\lambda
			\]
			is an isomorphism.
	\end{itemize}
\end{exe}

As a contrary, we have:

\begin{lem}
	The $\mfn$-action on $M_\lambda$ is locally finite.
\end{lem}

\proof
	By the above exercise, we have $M_\lambda = \bigcup_i \mbF^i U(\mfg)\cdot v_\lambda$, where $F^\bullet U(\mfg)$ is the PBW filtration on $U(\mfg)$. Each $ \mbF^i U(\mfg)\cdot v_\lambda$ is finite dimensional. Hence we only need to show these subspaces are $\mfn$-stable. For $u\in \mbF^i U(\mfg)$ and $x\in \mfn$ we have
	\[
		x\cdot (u\cdot v_\lambda) = u\cdot (x\cdot v_\lambda) + [x,u]\cdot v_\lambda. 
	\]
	By definition $x\cdot v_\lambda=0$. Then we win because $[x,u]\in [\mfg, \mbF^i U(\mfg)] \subset \mbF^{i-1} U(\mfg)$.

\qed


We are going to describe the $\mft$-action on $M_\lambda$. We need some definitions.

\begin{defn}
	Let $V\in \mft\hmod$. We say $V$ is a \textbf{weight module} if $V = \oplus_{\lambda\in \mft^*} V_\lambda$, where $V_\lambda\subset V$ is the $\lambda$-eigenspace. We say $\lambda$ is a \textbf{weight} of $V$ if $V_\lambda\neq 0$. Vectors in $V_\lambda$ are called \textbf{$\lambda$-weight vectors}.
\end{defn}

\begin{rem}
	A $\mft$-module $V$ is a weight module iff the action is locally finite and semisimple. This means for any $v\in V$, the subspace $\mft\cdot v$ is finite-dimensional and any $x\in \mft$ is sent to a diagonalizable endomorphism in $\gl(\mft\cdot v)$.
\end{rem}

\begin{rem}
	\label{rem-weight-module-qcoh}
	A $\mft$-module is a weight module iff the corresponding quasi-coherent sheaf on $\mft^*$ is a direct sum of 1-dimensional skyscrapters at closed points.
\end{rem}

\begin{exam}
	By the root decomposition, $\mfg$ is a weight module when viewed as a $\mft$-module via the adjoint action. Nonzero weights are roots.
\end{exam}

\begin{exam}
	The object $U(\mft)\in \mft\hmod$ is not a weight module. Indeed, it corresponds to the structure sheaf of $\mft^*$.
\end{exam}

\begin{rem}
	\label{rem-weight-module-subq}
	Weight modules in $\mft\hmod$ are closed under taking subquotients (e.g. by Remark \ref{rem-weight-module-qcoh}), but not closed under extensions.
\end{rem}

\begin{prop}
	\label{prop-Verma-weight}
	The Verma module $M_\lambda$ is a weight module, and the weights are given exactly by
	\[
		\lambda - \sum_{\alpha\in \Phi^+} n_\alpha \alpha,\; n_\alpha \in \mBZ^{\ge 0}.
	\]
	Moreover, each weight space is finite-dimensional.
\end{prop}

\proof
	First, note that $v_\lambda \in M_\lambda$ is a $\lambda$-weight vector because it is the image of $1_\lambda \in k_\lambda$.

	By the PBW theorem (Corollary \ref{cor-basis-Ug}), $U(\mfn^-)$ has a basis consists of weight vectors whose weights are $- \sum_{\alpha\in \Phi^+} n_\alpha \alpha,\; n_\alpha \in \mBZ^{\ge 0}$. Also, each weight space is finite dimensional.

	Let $x\in U(\mfn^-)$ be such a weight vector and $\mu$ be its weight. By the following equation, $x\cdot v_\lambda\in M_\lambda$ is a $(\lambda+\mu)$-weight vector:
	\[
		t\cdot (x\cdot v_\lambda) = x\cdot (t\cdot v_\lambda) + [t,x]\cdot v_\lambda,\; t\in \mft.
	\]
	Then we win by Exercise \ref{exe-Verma}.

\qed

\begin{defn}
	We define a partial order $\preceq$ on $\mft^*$ such that $\mu_1\preceq \mu_2$ iff $\mu_2-\mu_1 \in \mBZ^{\ge 0}\Phi^+$.
\end{defn}

Note that under the above partial order, the weight of $v_\lambda\in M_\lambda$ is indeed the highest one.


\begin{exam}
	Consider the case $\mfg=\sl_2$ equipped with its standard Cartan and Borel subalgebras. A weight $\lambda\in \mft^*$ is the same as a scaler $l:= \langle \lambda,\check \alpha\rangle $, where $ \check \alpha := h:= \big(\begin{smallmatrix} 1 & 0\\ 0 & -1\end{smallmatrix}\big)\in \mft$ is the coroot. 

	Since $\langle \alpha,\check \alpha\rangle =2$, the weights of the Verma module $M_l$ are of the form $l-2n$, $n\ge 0$. For each such $l':=l-2n$, since $\mfn^-$ is 1-dimensional, the $l'$-weight space of $M_l$ is also 1-dimensional. Namely, it is spaned by $f^n\cdot v_l\in M_l$, where $f:=  \big(\begin{smallmatrix} 0 & 0\\ 1 & 0\end{smallmatrix}\big)$ generates $\mfn^-$.

\end{exam}

\begin{exe}
	This is \red{Homework 1, Problem 2}\footnote{Warning: the solution in Gaitsgory's notes contains a critical typo and the last paragraph there should be justified. Also, don't forget to show $L_l$ is irreducible.
	}. 
	In the case $\mfg=\sl_2$, show the Verma module $M_l$ is irreducible unless $l\in \mBZ^{\ge 0}$. In the latter case, show there is a non-split short exact sequence 
	\begin{equation} \label{eqn-sl2}
		0 \to M_{-l-2} \to M_l \to L_l \to 0
	\end{equation}
	such that $L_l$ is a finite-dimensional irreducible $\sl_2$-module with highest weight $l$.
\end{exe}

\begin{comment}
\proof
	We only need to classify all submodules of $M_l$. Suppose $M_l$ contains a proper nonzero submodule $N$. Then $N$ is a weight module. Let $l'$ be the highest weight of $N$. By the previous discussion, $l'=l-2n$, $n>0$, and $v_{l'}:=f^n\cdot v_l$ is a weight vector.

	A direct calculation shows $e\cdot v_{l'}$ is a $(l'+2)$-weight vector, where $e:= \big(\begin{smallmatrix} 0 & 1\\ 0 & 0\end{smallmatrix}\big)$ generates $\mfn$. Hence we must have $e\cdot v_{l'}=0$. We have
	\[
		0 = e \cdot f^n\cdot v_l 
		= \sum_{1\le i\le n} f^{n-i} \cdot [e,f] \cdot f^{i-1} \cdot v_l + f^n \cdot e\cdot v_l.
	\]
	Note that $e\cdot v_l=0$. Also, $[e,f]=h$ and $h\cdot f^j = f^j \cdot h -2j f^{j}.$ We obtain
	\[
		0 = \sum_{1\le i\le n} ( f^{n-1} \cdot h - 2(i-1) f^{n-1} )\cdot v_l = n(l-n+1) f^{n-1}\cdot v_l.
	\]
	Note that $f^{n-1}\cdot v_l\neq 0$ because $f^n\cdot v_l\neq 0$. Hence we must have $l=n-1$ and $l'=-l-2$.

	This shows $M_l$ is irreducible unless $l\in \mBZ^{\ge 0}$. In the latter case, $M_l$ has a unique proper nonzero submodule $N$, and the highest weight of $N$ is $-l-2$. Since $e\cdot v_{-l-2}=0$, we have a nonzero $\mfb$-linear map $k^{-l-2} \to N$ given by the vector $v_{-l-2}$. By adjunction, we obtain a nonzero $\mfg$-linear map $M_{-l-2} \to N$. We have already shown $M_{-l-2}$ is irreducible, hence this map is injective. Then it is an isomorphism by the uniqueness of the submodule $N$. Now the quotient $L_l:=M_l/M_{-l-2}$ has weights $-l,-l+2,\cdots,l$ and each weight space is 1-dimensional. Hence $V_l$ is finite-dimensional. It is irreducible because the uniqueness of the submodule $N$.




\qed
\end{comment}

We return to the study of general semisimple Lie algebra $\mfg$.

\begin{thm}
	The Verma module $M_\lambda$ admits a unique irreducible quotient module $L_\lambda$, and the highest weight of $L_\lambda$ is $\lambda$. In particular, $L_\lambda$ and $L_\lambda'$ are non-isomorphic for $\lambda\neq \lambda'$.
\end{thm}

\proof
	Any proper submodule $N\subset M_\lambda$ is a weight module whose weights do not contain $\lambda$. It follows that the union of all the proper submodules satisfies the same property. By construction, this is the maximal proper submodule of $M_\lambda$. Then $L_\lambda$ is the corresponding quotient.

\qed

\section{Category \texorpdfstring{$\mCO$}{O}}

Roughly speaking, the Bernstein--Gelfand--Gelfand (a.k.a. BGG) category $\mCO$ is the full subcategory of $\mfg\hmod$ consisting of objects similar to Verma modules. Let us first give the traditional definition:

\begin{defn}
	We define the \textbf{category $\mCO$} to be the full subcategory of $\mfg\hmod$ consisting of objects $M$ satisfying the following properties:
	\begin{itemize}
		\item[(O1)] $M$ is finitely generated as a $\mfg$-module;
		\item[(O2)] $M$ is a weight module;
		\item[(O3)] The action of $\mfn$ on $M$ is locally finite.
	\end{itemize}

\end{defn}

\begin{exam}
	We have already seen that the Verma modules $M_\lambda\in \mCO$.
\end{exam}

\begin{lem}
	The subcategory $\mCO$ of $\mfg\hmod$ is closed under taking sub-quotients and finite direct sums. In particular, $\mCO$ is an abelian category.
\end{lem}

\proof
	For (O1), $U(\mfg)$ is Noetherian. For (O2), Remark \ref{rem-weight-module-subq}. The claim for (O3) is obvious. 
\qed



\begin{warn}
	The subcategory $\mCO$ is \emph{not} closed under extensions. This can be seen by considering $\ind_\mfb^\mfg(N)$ where $N$ is a finite dimensional $\mft$-module that does not have a weight decomposition.
\end{warn}

\begin{lem}
	Any object $M\in \mCO$ is Noetherian, i.e., satisfies the ascending chain condition for subobjects.
\end{lem}

\proof
	Follows from the fact that $U(\mfg)$ is Noetherian.
\qed


\begin{prop} \label{prop-Verma-generate}
	Any object $M\in \mCO$ is a quotient of a finite successive extension of Verma modules. In particular, $M$ is finitely generated as an $\mfn^-$-module.
\end{prop}

\proof
	The last claim follows from the first one because of Exercise \ref{exe-Verma}.

	By (O1), $M$ is generated by a finite-dimensional subspace $M_0$ as a $\mfg$-module. By (O2), we can enlarge $M_0$ and assumme it is a finite direct sum of weight spaces. By (O3), $U(\mfb)\cdot M_0 = U(\mfn)\cdot M_0$ is finite-dimensional. Hence we may assume $M_0$ is stable under the $\mfb$-action. By adjunction, we have a $\mfg$-linear map
	\[
		\ind_\mfb^\mfg( M_0 ) \to M,
	\]
	which is surjective because $M_0$ generates $M$ as a $\mfg$-module. It remains to show $M_0$ is a successive extension of 1-dimensional $\mfb$-modules. We state this as the following lemma.

\begin{lem}
	Let $M\in \mCO$ and $M_0\subset M$ be a finite-dimensional subspace stable under the $\mfb$-action. Then the $\mfn$-action on $M_0$ is nilpotent and $M_0$ is a successive extension of 1-dimensional $\mfb$-modules.
\end{lem}

\proof
	Note that the second claim follows from the first one. Namely, let $N_0\subset M_0$ be the subspace annihilated by $\mfn$. This is a sub-$\mfb$-representation because $\mfn$ is an ideal of $\mfb$. The first claim implies $N_0\neq 0$. Since $N_0$ is annihilated by $\mfn$, it is in the image of the restriction functor $\mft\hmod \to \mfb\hmod$. It follows that $N_0$ is a direct sum of 1-dimensional $\mfb$-representations because it is a weight module. Replacing $M_0$ by $M_0/N_0$, we win by induction.

	It remains to prove the first claim. We only need to show $\mfn$ acts nilpotently on any weight vector $v\in M_0$. Let $x\in \mfn$ be a weight vector. A direct calculation shows $x\cdot v$ is a weight vector whose weight is the sum of those of $v$ and $x$. In particular, the weight of $x\cdot v$ is strictly greater than that of $v$ with respect to the partial order $\prec$. Since the set of weights of $M_0$ is finite, we see $\mfn$ acts nilpotently on $v$.

\qed

\qed[Proposition \ref{prop-Verma-generate}]

\begin{cor}
	Let $M\in \mCO$. Then each weight space of $M$ is finite-dimensional.
\end{cor}

\proof
	Follows from Proposition \ref{prop-Verma-weight} and Proposition \ref{prop-Verma-generate}.

\qed

\begin{exe}
	This is \red{Homework 1, Problem 3}. Recall for any $V_1,V_2\in \mfg\hmod$, the tensor product $V_1\ot V_2$ of the underlying vector spaces has a natural $\mfg$-module structure defined by $x\cdot (v_1\ot v_2):= (x\cdot v_1)\ot v_2 + v_1\ot(x\cdot v_2)$. 

	\begin{itemize}
		\item[(1)]
			Prove: if $V_1$ and $V_2$ are weight modules, so is $V_1\ot V_2$. Determine the weights and weight spaces of $V_1\ot V_2$ in term of those for $V_1$ and $V_2$.
		\item[(2)]
			Consider the case $\mfg=\sl_2$. Prove: the tensor product of two Verma modules is not contained in $\mCO$.
	\end{itemize}

\end{exe}




\begin{thebibliography}{Yau}



\bibitem[MR]{MR} McConnell, John C., James Christopher Robson, and Lance W. Small. Noncommutative noetherian rings. Vol. 30. American Mathematical Soc., 2001.

\end{thebibliography}

\end{document} 


