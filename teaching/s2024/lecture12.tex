
%!TEX root = main.tex
\documentclass{amsart}
\textwidth=14.5cm \oddsidemargin=1cm
\evensidemargin=1cm
\usepackage{amsmath}
\usepackage{amsxtra}
\usepackage{amscd}
\usepackage{amsthm}
\usepackage{amsfonts}
\usepackage{amssymb}
\usepackage[foot]{amsaddr}
\usepackage{cite}
\usepackage{url}
\usepackage{rotating}
\usepackage{eucal}
\usepackage{tikz-cd}
\usepackage[all,2cell,color]{xy}
\UseAllTwocells
\UseCrayolaColors
\usepackage{graphicx}
\usepackage{pifont}
\usepackage{comment}
\usepackage{verbatim}
\usepackage{xcolor}
\usepackage{hyperref}
\usepackage{xparse}
\usepackage{upgreek}
\usepackage{MnSymbol}
\sloppy


%%%%%%%%%%%%%%%%%%%%Theorem%%%%%%%%%%%%%%%%%%%%
\newcounter{theorem}
\setcounter{theorem}{0}

\newtheorem{cor}[subsection]{Corollary}
\newtheorem{lem}[subsection]{Lemma}
\newtheorem{goal}[subsection]{Goal}
\newtheorem{lemdefn}[subsection]{Lemma-Definition}
\newtheorem{prop}[subsection]{Proposition}
\newtheorem{propdefn}[subsection]{Proposition-Definition}
\newtheorem{cordefn}[subsection]{Corollary-Definition}
\newtheorem{variant}[subsection]{Variant}
\newtheorem{warn}[subsection]{Warning}
\newtheorem{sugg}[subsection]{Suggestion}
\newtheorem{facts}[subsection]{Fact}
\newtheorem{ques}{Question}
\newtheorem{guess}{Guess}
\newtheorem{claim}{Claim}
\newtheorem{propconstr}[subsection]{Proposition-Construction}
\newtheorem{lemconstr}[subsection]{Lemma-Construction}
\newtheorem{ax}{Axiom}
\newtheorem{conje}[subsection]{Conjecture}
\newtheorem{mainthm}[subsection]{Main-Theorem}
\newtheorem{summ}[subsection]{Summary}
\newtheorem{thm}[subsection]{Theorem}
\newtheorem{thmdefn}[subsection]{Theorem-Definition}
\newtheorem{notn}[subsection]{Notation}
\newtheorem{convn}[subsection]{Convention}
\newtheorem{constr}[subsection]{Construction}


\theoremstyle{definition}

\newtheorem{defn}[subsection]{Definition}
\newtheorem{exam}[subsection]{Example}
\newtheorem{assum}[subsection]{Assumption}

\theoremstyle{remark}
\newtheorem{rem}[subsection]{Remark}
\newtheorem{exe}[subsection]{Exercise}


\numberwithin{equation}{section}


%%%%%%%%%%%%%%%%%%%%Commands%%%%%%%%%%%%%%%%%%%%

\newcommand{\nc}{\newcommand}
\nc\on{\operatorname}
\nc\renc{\renewcommand}


%%%%%%%%%%%%%%%%%%%%Sections%%%%%%%%%%%%%%%%%%%%

\nc\ssec{\subsection}
\nc\sssec{\subsubsection}

%%%%%%%%%%%%%%%%%%%%Environment%%%%%%%%%%%%%%%%%
\nc\blongeqn{\[ \begin{aligned}}
\nc\elongeqn{\end{aligned} \]}



%%%%%%%%%%%%%%%%%%%%Mathfont%%%%%%%%%%%%%%%%%%%%

\nc\mBA{{\mathbb A}}
\nc\mBB{{\mathbb B}}
\nc\mBC{{\mathbb C}}
\nc\mBD{{\mathbb D}}
\nc\mBE{{\mathbb E}}
\nc\mBF{{\mathbb F}}
\nc\mBG{{\mathbb G}}
\nc\mBH{{\mathbb H}}
\nc\mBI{{\mathbb I}}
\nc\mBJ{{\mathbb J}}
\nc\mBK{{\mathbb K}}
\nc\mBL{{\mathbb L}}
\nc\mBM{{\mathbb M}}
\nc\mBN{{\mathbb N}}
\nc\mBO{{\mathbb O}}
\nc\mBP{{\mathbb P}}
\nc\mBQ{{\mathbb Q}}
\nc\mBR{{\mathbb R}}
\nc\mBS{{\mathbb S}}
\nc\mBT{{\mathbb T}}
\nc\mBU{{\mathbb U}}
\nc\mBV{{\mathbb V}}
\nc\mBW{{\mathbb W}}
\nc\mBX{{\mathbb X}}
\nc\mBY{{\mathbb Y}}
\nc\mBZ{{\mathbb Z}}


\nc\mCA{{\mathcal A}}
\nc\mCB{{\mathcal B}}
\nc\mCC{{\mathcal C}}
\nc\mCD{{\mathcal D}}
\nc\mCE{{\mathcal E}}
\nc\mCF{{\mathcal F}}
\nc\mCG{{\mathcal G}}
\nc\mCH{{\mathcal H}}
\nc\mCI{{\mathcal I}}
\nc\mCJ{{\mathcal J}}
\nc\mCK{{\mathcal K}}
\nc\mCL{{\mathcal L}}
\nc\mCM{{\mathcal M}}
\nc\mCN{{\mathcal N}}
\nc\mCO{{\mathcal O}}
\nc\mCP{{\mathcal P}}
\nc\mCQ{{\mathcal Q}}
\nc\mCR{{\mathcal R}}
\nc\mCS{{\mathcal S}}
\nc\mCT{{\mathcal T}}
\nc\mCU{{\mathcal U}}
\nc\mCV{{\mathcal V}}
\nc\mCW{{\mathcal W}}
\nc\mCX{{\mathcal X}}
\nc\mCY{{\mathcal Y}}
\nc\mCZ{{\mathcal Z}}


\nc\mbA{{\mathsf A}}
\nc\mbB{{\mathsf B}}
\nc\mbC{{\mathsf C}}
\nc\mbD{{\mathsf D}}
\nc\mbE{{\mathsf E}}
\nc\mbF{{\mathsf F}}
\nc\mbG{{\mathsf G}}
\nc\mbH{{\mathsf H}}
\nc\mbI{{\mathsf I}}
\nc\mbJ{{\mathsf J}}
\nc\mbK{{\mathsf K}}
\nc\mbL{{\mathsf L}}
\nc\mbM{{\mathsf M}}
\nc\mbN{{\mathsf N}}
\nc\mbO{{\mathsf O}}
\nc\mbP{{\mathsf P}}
\nc\mbQ{{\mathsf Q}}
\nc\mbR{{\mathsf R}}
\nc\mbS{{\mathsf S}}
\nc\mbT{{\mathsf T}}
\nc\mbU{{\mathsf U}}
\nc\mbV{{\mathsf V}}
\nc\mbW{{\mathsf W}}
\nc\mbX{{\mathsf X}}
\nc\mbY{{\mathsf Y}}
\nc\mbZ{{\mathsf Z}}

\nc\mba{{\mathsf a}}
\nc\mbb{{\mathsf b}}
\nc\mbc{{\mathsf c}}
\nc\mbd{{\mathsf d}}
\nc\mbe{{\mathsf e}}
\nc\mbf{{\mathsf f}}
\nc\mbg{{\mathsf g}}
\nc\mbh{{\mathsf h}}
\nc\mbi{{\mathsf i}}
\nc\mbj{{\mathsf j}}
\nc\mbk{{\mathsf k}}
\nc\mbl{{\mathsf l}}
\nc\mbm{{\mathsf m}}
\nc\mbn{{\mathsf n}}
\nc\mbo{{\mathsf o}}
\nc\mbp{{\mathsf p}}
\nc\mbq{{\mathsf q}}
\nc\mbr{{\mathsf r}}
\nc\mbs{{\mathsf s}}
\nc\mbt{{\mathsf t}}
\nc\mbu{{\mathsf u}}
\nc\mbv{{\mathsf v}}
\nc\mbw{{\mathsf w}}
\nc\mbx{{\mathsf x}}
\nc\mby{{\mathsf y}}
\nc\mbz{{\mathsf z}}




\nc\mbfA{{\mathbf A}}
\nc\mbfB{{\mathbf B}}
\nc\mbfC{{\mathbf C}}
\nc\mbfD{{\mathbf D}}
\nc\mbfE{{\mathbf E}}
\nc\mbfF{{\mathbf F}}
\nc\mbfG{{\mathbf G}}
\nc\mbfH{{\mathbf H}}
\nc\mbfI{{\mathbf I}}
\nc\mbfJ{{\mathbf J}}
\nc\mbfK{{\mathbf K}}
\nc\mbfL{{\mathbf L}}
\nc\mbfM{{\mathbf M}}
\nc\mbfN{{\mathbf N}}
\nc\mbfO{{\mathbf O}}
\nc\mbfP{{\mathbf P}}
\nc\mbfQ{{\mathbf Q}}
\nc\mbfR{{\mathbf R}}
\nc\mbfS{{\mathbf S}}
\nc\mbfT{{\mathbf T}}
\nc\mbfU{{\mathbf U}}
\nc\mbfV{{\mathbf V}}
\nc\mbfW{{\mathbf W}}
\nc\mbfX{{\mathbf X}}
\nc\mbfY{{\mathbf Y}}
\nc\mbfZ{{\mathbf Z}}

\nc\mbfa{{\mathbf a}}
\nc\mbfb{{\mathbf b}}
\nc\mbfc{{\mathbf c}}
\nc\mbfd{{\mathbf d}}
\nc\mbfe{{\mathbf e}}
\nc\mbff{{\mathbf f}}
\nc\mbfg{{\mathbf g}}
\nc\mbfh{{\mathbf h}}
\nc\mbfi{{\mathbf i}}
\nc\mbfj{{\mathbf j}}
\nc\mbfk{{\mathbf k}}
\nc\mbfl{{\mathbf l}}
\nc\mbfm{{\mathbf m}}
\nc\mbfn{{\mathbf n}}
\nc\mbfo{{\mathbf o}}
\nc\mbfp{{\mathbf p}}
\nc\mbfq{{\mathbf q}}
\nc\mbfr{{\mathbf r}}
\nc\mbfs{{\mathbf s}}
\nc\mbft{{\mathbf t}}
\nc\mbfu{{\mathbf u}}
\nc\mbfv{{\mathbf v}}
\nc\mbfw{{\mathbf w}}
\nc\mbfx{{\mathbf x}}
\nc\mbfy{{\mathbf y}}
\nc\mbfz{{\mathbf z}}

\nc\mfa{{\mathfrak a}}
\nc\mfb{{\mathfrak b}}
\nc\mfc{{\mathfrak c}}
\nc\mfd{{\mathfrak d}}
\nc\mfe{{\mathfrak e}}
\nc\mff{{\mathfrak f}}
\nc\mfg{{\mathfrak g}}
\nc\mfh{{\mathfrak h}}
\nc\mfi{{\mathfrak i}}
\nc\mfj{{\mathfrak j}}
\nc\mfk{{\mathfrak k}}
\nc\mfl{{\mathfrak l}}
\nc\mfm{{\mathfrak m}}
\nc\mfn{{\mathfrak n}}
\nc\mfo{{\mathfrak o}}
\nc\mfp{{\mathfrak p}}
\nc\mfq{{\mathfrak q}}
\nc\mfr{{\mathfrak r}}
\nc\mfs{{\mathfrak s}}
\nc\mft{{\mathfrak t}}
\nc\mfu{{\mathfrak u}}
\nc\mfv{{\mathfrak v}}
\nc\mfw{{\mathfrak w}}
\nc\mfx{{\mathfrak x}}
\nc\mfy{{\mathfrak y}}
\nc\mfz{{\mathfrak z}}

\nc{\one}{{\mathsf{1}}}


\nc\clambda{ {\check{\lambda} }}
\nc\cmu{ {\check{\mu} }}

\nc\bDelta{\mathsf{\Delta}}
\nc\bGamma{\mathsf{\Gamma}}
\nc\bLambda{\mathsf{\Lambda}}


\nc\loccit{\emph{loc.cit.}}



%%%%%%%%%%%%%%%%%%%%Operations-limit%%%%%%%%%%%%%%%%%%%%

\NewDocumentCommand{\ot}{e{_^}}{
  \mathbin{\mathop{\otimes}\displaylimits
    \IfValueT{#1}{_{#1}}
    \IfValueT{#2}{^{#2}}
  }
}
\NewDocumentCommand{\boxt}{e{_^}}{
  \mathbin{\mathop{\boxtimes}\displaylimits
    \IfValueT{#1}{_{#1}}
    \IfValueT{#2}{^{#2}}
  }
}
\NewDocumentCommand{\mt}{e{_^}}{
  \mathbin{\mathop{\times}\displaylimits
    \IfValueT{#1}{_{#1}}
    \IfValueT{#2}{^{#2}}
  }
}
\NewDocumentCommand{\convolve}{e{_^}}{
  \mathbin{\mathop{\star}\displaylimits
    \IfValueT{#1}{_{#1}}
    \IfValueT{#2}{^{#2}}
  }
}
\NewDocumentCommand{\colim}{e{_^}}{
  \mathbin{\mathop{\operatorname{colim}}\displaylimits
    \IfValueT{#1}{_{#1}\,}
    \IfValueT{#2}{^{#2}\,}
  }
}
\NewDocumentCommand{\laxlim}{e{_^}}{
  \mathbin{\mathop{\operatorname{laxlim}}\displaylimits
    \IfValueT{#1}{_{#1}\,}
    \IfValueT{#2}{^{#2}\,}
  }
}
\NewDocumentCommand{\oplaxlim}{e{_^}}{
  \mathbin{\mathop\operatorname{oplax-lim}\displaylimits
    \IfValueT{#1}{_{#1}\,}
    \IfValueT{#2}{^{#2}\,}
  }
}


%%%%%%%%%%%%%%%%%%%%Arrows%%%%%%%%%%%%%%%%%%%%


\makeatletter
\newcommand{\laxto}{\dashedrightarrow}
\newcommand{\xrightleftarrows}[1]{\mathrel{\substack{\xrightarrow{#1} \\[-.9ex] \xleftarrow{#1}}}}
\newcommand{\adj}{\xrightleftarrows{\rule{0.5cm}{0cm}}}

\newcommand*{\da@rightarrow}{\mathchar"0\hexnumber@\symAMSa 4B }
\newcommand*{\da@leftarrow}{\mathchar"0\hexnumber@\symAMSa 4C }
\newcommand*{\xlaxto}[2][]{%
  \mathrel{%
    \mathpalette{\da@xarrow{#1}{#2}{}\da@rightarrow{\,}{}}{}%
  }%
}
\newcommand{\xlaxgets}[2][]{%
  \mathrel{%
    \mathpalette{\da@xarrow{#1}{#2}\da@leftarrow{}{}{\,}}{}%
  }%
}
\newcommand*{\da@xarrow}[7]{%
  % #1: below
  % #2: above
  % #3: arrow left
  % #4: arrow right
  % #5: space left 
  % #6: space right
  % #7: math style 
  \sbox0{$\ifx#7\scriptstyle\scriptscriptstyle\else\scriptstyle\fi#5#1#6\m@th$}%
  \sbox2{$\ifx#7\scriptstyle\scriptscriptstyle\else\scriptstyle\fi#5#2#6\m@th$}%
  \sbox4{$#7\dabar@\m@th$}%
  \dimen@=\wd0 %
  \ifdim\wd2 >\dimen@
    \dimen@=\wd2 %   
  \fi
  \count@=2 %
  \def\da@bars{\dabar@\dabar@}%
  \@whiledim\count@\wd4<\dimen@\do{%
    \advance\count@\@ne
    \expandafter\def\expandafter\da@bars\expandafter{%
      \da@bars
      \dabar@ 
    }%
  }%  
  \mathrel{#3}%
  \mathrel{%   
    \mathop{\da@bars}\limits
    \ifx\\#1\\%
    \else
      _{\copy0}%
    \fi
    \ifx\\#2\\%
    \else
      ^{\copy2}%
    \fi
  }%   
  \mathrel{#4}%
}
\makeatother

%%%%%%%%%%%%%%%%%%%%Decorations%%%%%%%%%%%%%%%%%%%%
\nc{\wt}{\widetilde}
\nc{\ol}{\overline}

\nc{\red}{\textcolor{red}}
\nc{\blue}{\textcolor{blue}}
\nc{\purple}{\textcolor{violet}}

\nc{\simorlax}{{\red\simeq/\blue\lax}}

%%%%%%%%%%%%%%%%%%%%All%%%%%%%%%%%%%%%%%%%%

\nc{\Id}{\mathsf{Id}}
\nc{\gl}{\mathfrak{gl}}
\renc{\sl}{\mathfrak{sl}}
\nc{\GL}{\mathsf{GL}}
\nc{\SL}{\mathsf{SL}}
\nc{\PGL}{\mathsf{PGL}}
\nc{\hmod}{\mathsf{-mod}}
\nc{\Vect}{\mathsf{Vect}}
\nc{\tr}{\mathsf{tr}}
\nc{\Kil}{\mathsf{Kil}}
\nc{\ad}{{\mathsf{ad}}}
\nc{\Ad}{\mathsf{Ad}}
\nc{\oblv}{\mathsf{oblv}}
\nc{\gr}{\mathsf{gr}}
\nc{\Sym}{\mathsf{Sym}}
\nc{\QCoh}{\mathsf{QCoh}}
\nc{\ind}{\mathsf{ind}}
\nc{\Spec}{\mathsf{Spec}}
\nc{\Hom}{\mathsf{Hom}}
\nc{\Ext}{\mathsf{Ext}}
\nc{\Grp}{\mathsf{Grp}}
\nc{\pt}{\mathsf{pt}}
\nc{\Lie}{\mathsf{Lie}}
\nc{\CAlg}{\mathsf{CAlg}}
\nc{\Der}{\mathsf{Der}}
\nc{\Rep}{\mathsf{Rep}}
\renc{\sc}{{\mathsf{sc}}}
\nc{\Fl}{\mathsf{Fl}}
\nc{\Fun}{\mathsf{Fun}}
\nc{\ev}{\mathsf{ev}}
\nc{\surj}{\twoheadrightarrow}
\nc{\inj}{\hookrightarrow}
\nc{\HC}{\mathsf{HC}}
\nc{\cl}{\mathsf{cl}}
\renc{\Im}{\mathsf{Im}}
\renc{\ker}{\mathsf{ker}}
\nc{\coker}{\mathsf{coker}}
\nc{\Tor}{\mathsf{Tor}}
\nc{\op}{\mathsf{op}}
\nc{\length}{\mathsf{length}}
\nc{\fd}{{\mathsf{fd}}}
\nc{\weight}{\mathsf{wt}}
\nc{\semis}{{\mathsf{ss}}}
\nc{\qc}{{\mathsf{qc}}}
\nc{\pr}{\mathsf{pr}}
\nc{\act}{\mathsf{act}}
\nc{\dR}{{\mathsf{dR}}}
\nc{\hol}{{\mathsf{hol}}}
\nc{\Pic}{{\mathsf{Pic}}}
\nc{\Loc}{\mathsf{Loc}}
\nc{\IC}{\mathsf{IC}}

\begin{document}


\title{Lecture 12}

\date{May 13, 2024}

\maketitle

	In this lecture, we discuss about the localization theorem.

\section{Flag variety}

	In this section, $G$ can be any connected affine algebraic group over an algebraic closed field $k$. Let $B$ be a Borel subgroup of $G$, which is defined to be a maximal connected solvable subgroup. Consider the right multiplication action of $B$ on $G$ on itself. We will review some basic facts about the flag variety $G/B$.

	\begin{thm}
		\label{thm-flag-projective}
		The quotient $G/B$ exists and is a projective $k$-scheme.
	\end{thm}

	\begin{rem}
		The proof of this theorem in my prefered reference \cite{M} is more involved than those in the other literatures because Milne uses scheme theory and allows nilpotency in the definition of algebraic groups. Below is a guideline to this proof.

		\begin{itemize}
			\item
				(Existence of quotient) In general, for any subgroup $K$ of an algebraic group $H$ over any field $k$, the quotient $H/K$ exists (see \cite[Section 5.c]{M}). Such $k$-scheme $H/K$ is called a \textbf{homogeneous space} under $H$. More precisely, the fppf sheafification of the naïve functor $\CAlg_k \to \mathsf{Set}$, $R\mapsto H(R)/K(R)$ is represented by a $k$-scheme $H/K$, and the map $H \to H/K$ exhibits $H$ as a fppf $K$-torsor on $H/K$. When $\mathsf{char}(k)=0$, ``fppf'' can be replaced by ``étale''\footnote{The contents of the above words are: for any $R\in \CAlg_k$, $H(R) \to (H/K)(R)$ factors as $H(R) \surj H(R)/K(R) \inj (H/K)(R)$. And for any element in $x\in (H/K)(R)$, there exists a finite presented faithfully flat $R$-algebra $R'$ such that the image $x'$ of $x$ under $(H/K)(R)\to (H/K)(R')$ is contained in the image of $H(R') \to (H/K)(R')$. In zero characteristic, we can require $R'$ to be an étale $R$-algebra.}.

			\item
				(Quasi-projective) In general, any homogeneous space $H/K$ under $H$ is quasi-projective (see \cite[Section 8.k]{M}).

			\item 
				The above two parts are more about scheme theory rather than representation theory.

			\item
				(Completeness) The quotient $G/B$ is shown to be complete in \cite[Section 17]{M}. There are several important representation theoretic ingredients in this proof: Chevalley's theorem\footnote{Every subgroup $K$ of an affine algebraic group $H$ can be realized as the stablizer of a 1-dimensional subspace $L$ in a finite-dimensional representation $V$. See \cite[Section 4.h]{M}.}, Lie--Kolchin theorem\footnote{Irreducible representations of a smooth connected solvable algebraic group $B$ are 1-dimensional. See \cite[Section 16.d]{M}.}
		\end{itemize}

	\end{rem}

	

	\begin{exam}
		\label{exam-SL2-flag}
		For $G=\SL_2$ or $\GL_2$ and the standard Borel subgroup $B$, $G/B$ is isomorphic to $\mBP^1$ such that the $k$-point $\pt\simeq B/B \to G/B$ corresponds to the $k$-point $\infty\in \mBP^1$.

		Indeed, consider the standard 2-dimensional representation $V$ of $G$. We view $V$ as an affine $k$-scheme and $\mBP(V)$ as a projective $k$-scheme. We obtain a transitive action of $G$ on $\mBP(V)\simeq \mBP^1$. Consider the $k$-point $\infty\in \mBP(V)$ given by the vector $\big(\begin{smallmatrix} 1\\ 0 \end{smallmatrix}\big)\in V$. The stablizer subgroup at this point is the standard Borel subgroup $B$. It follows that we have an isomorphism $G/B \simeq \mBP(V)$ that sends $gB$ to $g\cdot \infty$.
	\end{exam}

	\begin{exam}
		For $G=\SL_n$ or $\GL_n$ and the standard Borel subgroup $B$, $G/B$ classifes complete flags in $k^{\oplus n}$, i.e., subspaces $0=V_0 \subset V_1\subset \cdots \subset V_n = k^{\oplus n}$ such that $\dim(V_k)=k$. For this reason, $G/B$ is called the \textbf{flag variety} of $G$.

		In this case, it is easy to write down homogenous coordinates to show $G/B$ is projective. See \cite[Section 7.g]{M}.
	\end{exam}


	Theorem \ref{thm-flag-projective}, together with the Borel fixed point theorem\footnote{Any action of a smooth connected solvable algebraic group $B$ on a finite type separated $k$-scheme $X$ has a fixed $k$-point.}, imply the following results, which were used in our previous lectures. See \cite[Section 17]{M}.

	\begin{thm}
		\label{thm-Borel-conjugate}
		Any two Borel subgroups of $G$ are conjugate by an element of $G(k)$.
	\end{thm}

	\begin{thm}
		\label{thm-Borel-normalizer}
		The normalizer subgroup of $B$ in $G$ is equal to $B$, i.e., $N_G(B) = B$.
	\end{thm}



	\begin{constr}
		We have $(G/B)(k) \simeq G(k)/B(k)$. Then Theorem \ref{thm-Borel-conjugate} and Theorem \ref{thm-Borel-normalizer} imply there is an isomorphism
		\[
			(G/B)(k) \xrightarrow{\sim} \{\textrm{Borel subgroups of }G\},\; gB \mapsto \Ad_g(B)
		\]
		and therefore
		\[
			(G/B)(k) \xrightarrow{\sim} \{\textrm{Borel subalgebras of }\mfg\},\; gB \mapsto \Ad_g(\mfb).
		\]
	\end{constr}

	\begin{cor}
		For two Borel subgroups $B$ and $B'$ of $G$, there is a unique $G$-equivariant isomorphism $G/B \to G/B'$.
	\end{cor}

	\begin{notn}
		The above corollary says that as a $k$-scheme equipped with a $G$-action, $G/B$ does not depend on the choice of $G$. We write $\Fl_G$ for it.
	\end{notn}


	\begin{rem}
		In fact, one can \emph{define} the moduli problem classifying Borel subalgebras of $\mfg$ and show that it is representable by a $k$-scheme $\Fl_G$ which is isomorphic to $G/B$ for any Borel subgroup $B$. Namely, let $n=\dim(G)$ and $d$ be the common dimension of all the Borel subgroups. Then one can show the above moduli problem is a closed subfunctor of the Grassmannian $\mathsf{Gr}(d,\mfg)$ that classifies $d$-dimensional subspaces of $\mfg$. 

		It is a well-known fact that the obvious map $\mathsf{Gr}(d,V) \to \mBP(\wedge^d V)$ is a closed embedding, known as the \textbf{Plücker embedding}. Hence the composition $\Fl_G \to \mathsf{Gr}(d,\mfg) \to \mBP(\wedge^d \mfg)$ gives a closed embedding of $\Fl_G$ into a projective space.
	\end{rem}



\section{Line bundles on flag variety}

	From now on, $G$ is assumed to be connected and \emph{reductive}. Recall this means the unipotent radical of $G$ is trivial. We can classify the line bundles on the flag variety $G/B$ using the characters of $B$ (or equivalently, of $T\simeq B/[B,B]$). To describe this classfication, we need the following construction.

	\begin{defn}
		Let $K$ be an algebraic group and $Y$ be a $k$-scheme equipped with an action of $H$. Consider the maps $\act,\pr: K \mt Y \rightrightarrows Y$. A \textbf{$K$-equivariant quasi-coherent $\mCO_Y$-module} is an object $\mCF \in \mCO_Y \hmod_\qc$ euipped with isomorphisms $\act^*\mCF \to \pr^*\mCF$ that satisfies the cocycle condition over $K\mt K\mt Y$.

		Let $\mCO_Y \hmod_\qc^K$ be the category of $K$-equivariant quasi-coherent $\mCO_Y$-modules.
	\end{defn}

	\begin{rem}
		The category $\mCO_Y \hmod_\qc^K$ is an abelian category and the forgetful functor $\mCO_Y \hmod_\qc^K\to \mCO_Y \hmod_\qc$ is exact. However, this functor is \emph{not} fully faithful. \emph{Being equivariant is a structure rather than a property}.
	\end{rem}

	\begin{exam}
		The structure sheaf $\mCO_Y$ has an obvious $K$-equivariant structure.
	\end{exam}

	\begin{constr}
		Suppose the quotient $Y/K$ exsits\footnote{More precisely, we mean the fppf sheafification of the functor $R \mapsto Y(R)/K(R)$ is represented by a $k$-scheme.}. Let $\pi:Y\to Y/K$ be the projection map. Note that $\pi\circ \act = \pi\circ \pr$. Hence we have a functor
		\begin{equation}
			\label{eqn-quotient-vs-equivariant}
			\mCO_{Y/K}\hmod_\qc \to \mCO_Y \hmod_\qc^K,\; \mCM \mapsto \pi^*\mCM
		\end{equation}
		that sends $\mCM$ to $\pi^*\mCM$ equipped with the obvious equivariant structure.

	\end{constr}

	By the flat descent of quasi-coherent sheaves, we have:

	\begin{prop}
		The functor \eqref{eqn-quotient-vs-equivariant} is an equivalence.
	\end{prop}

	At least when $K$ is affine, the inverse of \eqref{eqn-quotient-vs-equivariant} can be constructed as follows.

	\begin{constr}
		\label{constr-inverse-quotient-vs-equivariant}
	
		Let $\mCF\in \mCO_Y \hmod_\qc^K$. The $K$-equivariant structure induces a morphism $\mCF \to \act_* \circ \pr^* \mCF$. Taking global sections, we obtain a map
		\[
			\mCF(Y) \to \pr^*\mCF(K\mt Y) \simeq \mCO(K) \ot \mCF(Y).
		\]
		One can show the cocycle condition for the $K$-equivariant structure on $\mCF$ is translated to the associative law for a \emph{right} $K$-action on $\mCF(Y)$. By taking inverse, we obtain a functor
		\[
			\Gamma(Y,-): \mCO_Y \hmod_\qc^K \to \Rep(K).
		\]
		One can check the multplication map $\mCO(Y)\ot \mCO(Y)\to \mCO(Y)$ and the action map $\mCO(Y)\ot \mCF(Y) \to \mCF(Y)$ are $K$-linear.
	\end{constr}

	\begin{rem}
		Let $g\in K$ be a closed point and consider the map $g: Y \to Y$ given by its action. The $K$-equivariant structure on $\mCF$ provides an isomorphism $g^*\mCF \to \mCF$ and therefore $\mCF \to g_*\mCF$. Taking global sections, we obtain an automorphism $\Gamma(Y,\mCF) \to \Gamma(Y,\mCF) $, which gives the right action of the point $g\in K$ on $\Gamma(Y,\mCF)$. When $\mCF=\mCO$ is the structure sheaf, this is the usual formula $(\phi\cdot g)(y) = \phi(gy)$.
	\end{rem}	

	\begin{constr}
		\label{constr-infinitesimal-action-on-equivariant}
		The following construction is an infinitesimal variant of Construction \ref{constr-inverse-quotient-vs-equivariant}. Consider $\mfk:=\Lie(K)$. Then for $\mCF\in \mCO_Y \hmod_\qc^K$, the underlying sheaf $\mCF$ has a natural action by $\mfk$ such that the induced $\mfk$-module structure on $\mCF(Y)$ coincides with that induced by the $K$-module structure in Construction \ref{constr-inverse-quotient-vs-equivariant}.

		Namely, let $K^{(1)}:= \Spec( \mCO_{K,e}/\mfm_{K,e}^2 )$ be the first neightborhood of the unit element $e$ insider $K$. For any open subscheme, the maps $\act,\pr$ induce maps $\act_U^{(1)} ,\pr_U^{(1)}: K^{(1)} \mt U \to U$. The $K$-equivariant structure provides a morphism $\mCF|_U \to (\act_U^{(1)})_* \circ (\pr_U^{(1)})^* (\mCF|_U)$. Taking global sections, we obtain a map
		\[
			\mCF(U) \to (\pr_U^{(1)})^* (\mCF|_U)( K^{(1)} \mt U) \simeq \mCO_{K,e}/\mfm_{K,e}^2 \ot \mCF(U).
		\]
		Note that the short exact sequence
		\[
			0 \to \mfm_{K,e}/\mfm_{K,e}^2 \to \mCO_{K,e}/\mfm_{K,e}^2 \to k \to 0
		\]
		has a canonical spliting. Hence we obtain a map $\mCF(U) \to \mfm_{K,e}/\mfm_{K,e}^2\ot \mCF(U)$. By definition, we have $\mfk^* \simeq \mfm_{K,e}/\mfm_{K,e}^2$. Hence we obtain a map
		\[
			\mfk\ot \mCF(U) \to \mCF(U).
		\]
		One can show\footnote{Note: I do not have time to check if there should be a negative sign. I will return to this problem after the class.} this defines a $\mfk$-module structure on $\mCF(U)$ for any $U$ and therefore on $\mCF$. 
		
	\end{constr}


	\begin{constr}
		Now for any open subset $U\subset Y/K$, its inverse image $\pi^{-1}(U)\subset Y$ is preserved by the $K$-action. Hence we obtain a right $K$-representation structure on $\mCF(\pi^{-1}(U))$ compatible with the $\mCO(\pi^{-1}(U))$-module structure. One can show $\mCO(U) \simeq \mCO(\pi^{-1}(U))^K$ and $U \mapsto \mCF(\pi^{-1}(U))^K$ defines a quasi-coherent $\mCO_{Y/K}$-module, which we denote by $\mCF^K$. We have a functor
		\[
			 \mCO_Y \hmod_\qc^K\to \mCO_{Y/K}\hmod_\qc ,\; \mCF \mapsto \mCF^K
		\]
		inverse to the functor \eqref{eqn-quotient-vs-equivariant}.
	\end{constr}

	

	

	\begin{constr}
		Let $V\in \Rep(K)$ be a $K$-representation. The quasi-coherent $\mCO_Y$-module $\mCO_Y \ot V$ has a natural $K$-equivariant structure such that the $K$-representation structure on
		\[
			(\mCO_Y \ot V)(\pi^{-1}(U))  \simeq \mCO_Y(\pi^{-1}(U)) \ot V
		\]
		is given by the (anti)diagonal action. Hence we obtain a funnctor
		\[
			\Rep(K) \to \mCO_{Y/K} \hmod_\qc,\; V\mapsto (\mCO_Y \ot V)^K.
		\]
	\end{constr}

	\begin{rem}
		In fact, we have $\Rep(K)\simeq \QCoh(\pt/K)$ where $\pt/K$ is the classifying stack of $K$. Via this equivalence, the above functor corresponds to the pullback functor along the map $Y/K \to \pt/K$.
	\end{rem}

	\begin{constr}
		For any character $\lambda$ of $T$ and the corresponding 1-dimensional $B$-representation $k_{-\lambda}$, we obtain a line bundle\footnote{Note the negative sign!}
		\[
			\mCL^\lambda:= (\mCO_G\ot k_{-\lambda})^B \in \mCO_{G/B}\hmod_\qc
		\]
		on the flag variety $G/B$. It is easy to see
		\begin{equation}
			\label{eqn-character-to-line-bundle}
			\mBX(T)\simeq \mBX(B) \to \Pic(G/B),\; \lambda \mapsto \mCL^\lambda
		\end{equation}
		is a homomorphism, which is called the \textbf{characteristic map} for $G$.
	\end{constr}

	\begin{rem}
		In fact, for any connected algebraic group $G$ and its Borel subgroup $B$, we have an exact sequence
		\[
			0 \to \mBX(G) \to \mBX(B) \to  \Pic(G/B) \to \Pic(G) \to 0.
		\]
		When $G$ is semisimple, $\mBX(G)=0$. When $G$ is further simply connected, $\Pic(G) = 0$. See \cite[Section 18]{M}.

	\end{rem}

	

	\begin{exam}
		For $G=\SL_2$ equipped with its standard Borel and Cartan subgroups. Any character of $B$ is of the form $\big(\begin{smallmatrix} t & *\\ 0 & t^{-1} \end{smallmatrix}\big)\mapsto t^n$ for some $n\in \mBZ$. Unwinding the definitions, the corresponding line bundle on $G/B\simeq \mBP^1$ is $\mCO(n)$. Indeed, its global section is the space of functions $\phi$ on $\SL_2$ such that $\phi(g\big(\begin{smallmatrix} t & *\\ 0 & t^{-1} \end{smallmatrix}\big))=t^n\phi(g)$.
	\end{exam}

	\begin{warn}
		Other authors might choose different conventions and obtain the line bundle $\mCO(-n)$.
	\end{warn}

	\begin{exam}
		We have $\mCL^{-2\rho} \simeq \omega_{G/B}$.
	\end{exam}

	\begin{constr}
		Note that $\mCL^\lambda$ is naturally equivariant with respect to the left multplication action of $G$ on $G/B$. By Construction \ref{constr-inverse-quotient-vs-equivariant}, we obtain a $G$-module structure on $\mbH^0(G/B,\mCL^\lambda)$. This representation is finite-dimensional because $G/B$ is proper.
	\end{constr}

	Recall the following well-known result:

	\begin{thm}[Borel-Weil-Bott]
		When $\mathsf{char}(k)=0$ and $G$ is semisimple, if $\lambda$ is dominant and integral, then $\mbH^0(G/B,\mCL^\lambda) \simeq L_\lambda$ and $\mbH^i(G/B,\mCL^\lambda)=0$ for $i>0$.
	\end{thm}

	\begin{rem}
		The statement $\mbH^i(G/B,\mCL^\lambda)=0$ for $i>0$ remains true in positive characteristic, known as \emph{Kempf's vanishing theorem}. See \cite[Chapter 4]{J} for a proof. However, $\mbH^0(G/B,\mCL^\lambda)$ is not always irreducible.
	\end{rem}

	We also record the following result. For a proof, see \cite[Chapter 4]{J}.

	\begin{thm}
		Let $G$ be semisimple. The following conditions are equivalent:
		\begin{itemize}
			\item[(i)]
				When viewed as an integral weight of $\mft$, $\lambda$ is regular and dominant, i.e., $\langle \lambda,\check \alpha \rangle \in \mBZ^{>0}$ for any $\alpha\in \Phi^+$;
			\item[(ii)]
				The line bundle $\mCL^\lambda$ is very ample;
			\item[(iii)]
				The line bundle $\mCL^\lambda$ is ample.
		\end{itemize}
	\end{thm}

	\begin{rem}
		In fact, the characteristic map $\mBX(T) \to \Pic(\Fl_G)$ does not depend on the choice of $B$, as long as we replace $T$ by the \emph{abstract Cartan group} $T_{\mathsf{abs}}$ for $G$. More precisely, for any choice of $B$, there is a corresponding realization isomorphism $T_{\mathsf{abs}} \to T$, then the composition $\mBX(T_{\mathsf{abs}}) \simeq \mBX(T)\to \Pic(\Fl_G)$ does not depend on $B$.
	\end{rem}


\section{Bruhat decomposition}

	Recall $G$ is assumed to be reductive. References for this section include \cite[Section 21.h]{M} and \cite[Chapter 13]{J}. 

	\begin{thm}[Bruhat decomposition for {$G(k)$}]
		We have
		\[
			G(k) = \bigsqcup_{w\in W} B(k)wB(k),
		\]
		where in the RHS we lift $w\in W$ to an element in $N_G(T)(k)$.
	\end{thm}

	

	\begin{exam}
		For $G=\GL_n$, this decomposition follows from Gaussian elimination.
	\end{exam}

	We also have the version for schemes. 

	\begin{thm}[Bruhat decomposition for {$G$}]
		For any $w\in W$, there is a unique smooth locally closed subscheme of $G$, denoted by $BwB$, such that $BwB(k)=B(k)wB(k)$. We have a disjoint union decomposition of underlying topological spaces:
		\[
			G = \bigsqcup_{w\in W} BwB.
		\]
		Each subscheme $BwB$ is stablized by the $(B,B)$-action on $G$ and is equal to the $(B,B)$-orbit that contains $w$.
	\end{thm}

	\begin{rem}
		Using $B=NT$, we have $B(k)wB(k) = N(k)wB(k) = B(k)wN(k)$ and similarly $BwB = NwB = BwN$.
	\end{rem}

	Taking quotient for the right $B$-action, we also have:

	\begin{thmdefn}[Bruhat decomposition for {$\Fl_G$}]
		For any $w\in W$, there is a unique smooth locally closed subscheme of $\Fl_G$, denoted by $\Fl_G^{=w}$, such that $\Fl_G^{=w}=BwB/B$. We have a disjoint union decomposition of underlying topological spaces:
		\[
			\Fl_G = \bigsqcup_{w\in W} \Fl_G^{=w}.
		\]
		Each subscheme $\Fl_G^{=w}$ is stablized by the $B$-action on $\Fl_G$ and is equal to the $B$-orbit/$N$-orbit that contains $wB/B$. The subschemes $\Fl_G^{=w}$ are called the \textbf{Bruhat cells} of the flag variety $\Fl_G$.
	\end{thmdefn}

	\begin{exam}
		For $G=\SL_2$ and the isomorphism $\Fl_G\simeq \mBP^1$ in Example \ref{exam-SL2-flag}. We hvae $\Fl_G^{=1} \simeq \{\infty\}$ and $\Fl_G^{=s} \simeq  \mBA^1$.
	\end{exam}

	The following lemma is easy on the level of $k$-points. A careful argument shows it is also true on the level of schemes.

	\begin{lem}
		We have $\Fl_G^{=w} \simeq N/(\Ad_{w}(N)\cap N) \simeq \Ad_{w}(N)\cap N^-$.
	\end{lem}

	\begin{rem}
		When $\mathsf{char}(k)=0$, for any unipotent algebraic group $U$, there is a well-defined isomorphism between \emph{$k$-schemes} $\exp: \Lie(U) \to U$, where $\Lie(U)$ means the affine space scheme corresponding to the same-named vector space (see \cite[Section 14.d]{M}). It follows that $\Fl_G^{=w}$ is isomorphic to an affine space. This explains the word ``cell''.
	\end{rem}

	\begin{exe}
		This is \red{Homework 6, Problem 1}. Prove: $\dim(\Fl_G^{=w})\simeq \ell(w)$\footnote{If you do not know the basics about reductive groups, prove this for semisimple $G$.}.

	\end{exe}

	\begin{defn}
		The (reduced) closures $\overline{\Fl_G^{=w}}$ of $\Fl_G^{=w}$ inside $\Fl_G$ are called the \textbf{Schubert varieties} for $G$.
	\end{defn}

	\begin{rem}
		Fortunately, the Schubert varieties are in general singular.
	\end{rem}

	\begin{thm}
		We have a disjoint union decomposition of underlying topological spaces
		\[
			\overline{\Fl_G^{=w}} = \bigsqcup_{w'\le w} \Fl_G^{=w'}.
		\]
	\end{thm}

	\begin{notn}
		Because of the above theorem, we also write $\Fl_G^{\le w}:= \overline{\Fl_G^{=w}}$.
	\end{notn}

\section{Statement of localization theorem}
	
	In this section, we state the main theorems of this course. Next time, we prove them.
	
	Consider the $G$-action on the flag variety $\Fl_G$. As in [Lecture 10, Construction 8.1], we have a homomorphism
	\begin{equation}
		\label{eqn-U-to-D}
		a: U(\mfg) \to \mCD(\Fl_G)
	\end{equation}
	induced from the Lie algebra homomorphism $\mfg \to \mCT(\Fl_G)$.

	\begin{constr}
		We have adjoint functors
		\begin{equation}
			\label{eqn-loc-gamma}
			\Loc: U(\mfg)\hmod \adj \mCD_{\Fl_G}\hmod_\qc^l : \Gamma
		\end{equation}
		constructed as follows:
		\begin{itemize}
			\item 
				$\Loc(M):=  \mCD_{\Fl_G}\ot_{\underline{U(\mfg)}} \underline{M}$, where $\underline{M}$ denotes the sheaf on $\Fl_G$ with constant values $M$.
			\item
				$\Gamma(\mCF)$ is the space of global section of $\mCF$, equipped with the $U(\mfg)$-module structure obtained by restricting along \eqref{eqn-U-to-D}.
		\end{itemize}
	\end{constr}

	\begin{rem}
		The above adjoint pair can not be equivalences because the homomorphism \eqref{eqn-U-to-D} is not an isormophism. Below is an imprecise but motivating explanation. The ring $U(\mfg)$ has the same size as $\Sym(\mfg)$, which is a commutative algebra of dimension $\dim(\mfg)$. On the other hand, the ring $\mCD(\Fl_G)$ has the same size as $\Sym_{\mCO(\Fl_G)}(\mCT(\Fl_G))$ (lying!), which is a commutative algebra of dimension $2\dim \Fl_G = \dim(\mfg)-\dim(\mft)$.
	\end{rem}

	However, the differences are eliminated once we kill the kernel of the character of $Z(\mfg)$. Recall the latter is a commutative algebra of dimension $\dim(\mft)$.

	\begin{constr}
		Let $\chi_0$ be the central character of the trivial $\mfg$-module and write $U(\mfg)_{\chi_0}:= U(\mfg) \ot_{Z(\mfg)} k_{\chi_0}$. In other words, $U(\mfg)_{\chi_0}$ is the quotient of $U(\mfg)$ by its double-sided ideal generated by $\ker(\chi_0)$. Note that 
		\[
			U(\mfg)_{\chi_0}\hmod  \subset U(\mfg)\hmod
		\]
		is the full subcategory of $U(\mfg)$-modules such that $Z(\mfg)$ acts via the character $\chi_0$. Recall for any $w\in W$ we have
		\[
			M_{w\bullet (-2\rho)}, \,M_{w\bullet (-2\rho)}^\vee, \,L_{w\bullet (-2\rho)} \in U(\mfg)_{\chi_0}\hmod.
		\]
	\end{constr}

	\begin{constr}
		For any $w\in W$, let $i_w: \Fl_G^{=w}\to \Fl_G$ be the locally closed embedding. Consider the left $\mCD_{\Fl_G}$-modules:
		\begin{eqnarray*}
			\Delta_w &:=&  i_{w,!}(\mCO_{\Fl_G^{=w}}),\\
			\nabla_w &:=& i_{w,*,\dR}(\mCO_{\Fl_G^{=w}}),\\
			\IC_w &:=& \Im( \Delta_w \to \nabla_w ).
		\end{eqnarray*}
	\end{constr}

	\begin{rem}
		Let us clarify the definitions of these objects in this remark.

		We first translate $\mCO_{\Fl_G^{=w}}$ to the right $\mCD$-module 
		\[
			\omega_{\Fl_G^{=w}} \in \mCD_{\Fl_G^{=w}}\hmod_\qc^r,
		\]
		then apply the functors $i_{w,!}$ and $i_{w,*,\dR}$ defined in Lecture 11 to obtain complices of right $\mCD$-modules on $\Fl_G$. However, by Kashiwara's lemma, $i_{w,*,\dR}(\omega_{\Fl_G^{=w}})$ is a genuine right $\mCD$-module, i.e., is contained in the abelian category. Moreover, it is \emph{holonomic} by [Lecture 11, Fact 10.5]. Hence by [Lecture 11, Fact 10.1 and Example 9.3], 
		\[
			i_{w,!}(\omega_{\Fl_G^{=w}}) \simeq \mBD \circ i_{w,*,\dR} \circ \mBD (\omega_{\Fl_G^{=w}}) \simeq \mBD(i_{w,*,\dR}(\omega_{\Fl_G^{=w}}))
		\]
		is also a holonomic right $\mCD$-module. Now $\Delta_w $ and $\nabla_w $ are defined to be the holonomic left $\mCD$-modules corresponding to them.

		As for $\IC_w$, by the base-change isomorphism ([Lecture 11, Fact 8.1]), we have an equivalence $\Id \simeq i_{w}^! \circ i_{w,*,\dR}$. Via adjunction, it induces a natural transformation $i_{w,!} \to i_{w,*,\dR}$, which then induces a morphism $\Delta_w \to \nabla_w$.
	\end{rem}

	\begin{rem}
		The symbol $\IC$ stands for \emph{intersection cohomology} because $\IC_w$ corresponds to the intersection cohomology sheaf on the Schubert variety $\Fl_G^{\le w}$ via the Riemann--Hilbert correspondence.
	\end{rem}


	\begin{thm}[Localization theorem]
		\label{thm-localization}
		We have
		\begin{itemize}
			\item[(1)]
				The homomorphism \eqref{eqn-U-to-D} induces an isomorphism
				\[
					U(\mfg)_{\chi_0} \simeq \mCD(\Fl_G).
				\]
			\item[(2)]
				The isomorphism in (1) induces functors inverse to each other:
				\[
					\Loc: U(\mfg)_{\chi_0}\hmod \adj \mCD_{\Fl_G}\hmod_\qc^l : \Gamma.
				\]
			\item[(3)]
				Via the equivalences in (2), we have
				\[
					M_{w\bullet (-2\rho)} \longleftrightarrow \Delta_w,\; M^\vee_{w\bullet (-2\rho)} \longleftrightarrow \nabla_w,\; L_{w\bullet (-2\rho)} \longleftrightarrow \IC_w.
				\]
		\end{itemize}
	\end{thm}

	\begin{rem}
		The adjoint functors in (2) are compatible with \eqref{eqn-loc-gamma} in the obvious way.
	\end{rem}

	\begin{exe}
		This is \red{Homework 6, Problem 2}. Deduce the BGG theorem from the localization theorem\footnote{HintL prove any subquotient of $\Delta_w$ is set-theoretically support on the Schubert variety $\Fl_G^{\le w}$.}.
	\end{exe}

	\begin{exe}
		This is \red{Homework 6, Problem 3}. Prove the isomorphism in (1) for the special case $G=\SL_2$.
	\end{exe}

\section{Strongly equivariant \texorpdfstring{$\mCD$}{D}-modules}

	Note that the block $\mCO_{\chi_0}$ is \emph{not} contained in $U(\mfg)_{\chi_0}\hmod $ because each module in $\mCO_{\chi_0}$ is annihilated by $\ker(\chi_0)^N$, $N>>0$ rather than just by $\ker(\chi_0)$. In this section, we give a description of the category
	\[
		\mCO_{\chi_0}' := \mCO_{\chi_0} \cap U(\mfg)_{\chi_0}\hmod
	\]
	via localization theory.

	\begin{defn}
		Let $Y$ be a smooth $k$-scheme acted by an algebraic group $K$. We say a left $\mCD$-module $\mCF\in \mCO_Y\hmod_\qc^l$ is equipped with a \textbf{weakly $K$-equivariant} structure if
		\begin{itemize}
			\item[(i)]
				It is equipped with a $K$-equivariant structure as a quasi-coherent $\mCO_Y$-module;
			\item[(ii)]
				The action morphism $\mCD_Y \ot_{\mCO_Y} \mCF \to \mCF$ is compatible with the $K$-equivariant structures.
		\end{itemize}
		We say $\mCF\in \mCO_Y\hmod_\qc^l$ is equipped with a \textbf{strongly $K$-equivariant} structure if we further have:
		\begin{itemize}
			\item[(iii)]
				The $\mfk$-module structure on $\mCF$ obtained in Construction \ref{constr-infinitesimal-action-on-equivariant} is the same as the $\mfk$-module structure obtained by restricting along the map $\underline{U(\mfk)} \to \mCD_Y$.
		\end{itemize}

		We also translate the above definitions to right $\mCD$-modules.
	\end{defn}

	\begin{exam}
		The structure sheaf $\mCO_Y$, viewed as a left $\mCD$-module, has an obvious weakly $K$-equivariant structure, and this structure is strong.
	\end{exam}

	\begin{rem}
		Being weakly equivariant is a structure rather than a property. However, one can show being strongly equivariant is actually a property when $K$ is connected.

		One the other hand, in the derived category of $\mCD$-modules, being strongly equivariant is also a structure rather than a property. Note that strongly equivariant complices of $\mCD$-modules are not the same as complices of strongly equivariant $\mCD$-modules.
	\end{rem}

	I do not recommend to learn the traditional proof of the following result. Modern techniques about $\mCD$-modules would provide a more illuminating proof.

	\begin{prop}
		If the quotient $Y/K$ exists (which is automatically smooth), then the functor $\pi^*$ induces an equivalence
		\[
			\pi^*: \mCD_{Y/K}\hmod_\qc \to \mCD_Y\hmod_\qc^{K-\mathsf{strong}}.
		\]

	\end{prop}

	\begin{exam}
		The ``six functors'' introduced last time can be updated to functors between strongly $K$-equivariant $\mCD$-modules, at least after taking cohomologies.
	\end{exam}

	\begin{thm}[Localization thoerem, continued]
		We have
		\begin{itemize}
			\item[(4)]
				The equivalences in Theorem \ref{thm-localization} restricts to equivalences
				\[
					\Loc: \mCO_{\chi_0}' \adj \mCD_{\Fl_G}\hmod_\mbc^{l,B-\mathsf{strong}} : \Gamma.
				\]
		\end{itemize}
		
	\end{thm}

	\begin{rem}
		Note that on the geometric side, we use \emph{coherent} $\mCD$-modules. This is because objects in $\mCO_{\chi_0}' $ are finitely generated. Equivalently, we can use $\mCD_{\Fl_G}\hmod_\hol^{l,B-\mathsf{strong}}$ becuse any strongly $B$-equivariant coherent $\mCD$-module on $\Fl_G$ is holonomic. This is essentially because there are finitely many $B$-orbits on $\Fl_G$.
	\end{rem}

	\begin{warn}
		Although the $B$-orbits and $N$-orbits on $\Fl_G$ are the same, we must use strongly equivariant condition for $B$ rather than for $N$. Otherwise the global section would not be weight modules. Namely, being strongly $K$-equivariant implies the $\mfg$-module structure on the global section is $K$-integrable, and any object in $\mCO$ is required to be $B$-integrable (rather than just $N$-integrable).
	\end{warn}
	
	\begin{rem}
		If we want an equivalence about the actual block $\mCO_{\chi_0}$, we need:
		\begin{itemize}
			\item 
				Replace $\Fl_G$ by the \emph{basic affine space}\footnote{This is a bad name: it is not affine!} $G/U$;
			\item
				Impose weakly equivariant structure with respect to the \emph{right} $T$-action on $G/U$;
			\item
				Restrict to the full subcategory generated by subquotients and extensions of strongly $T$-equivariant objects. These objects are called $(T,0)$-monodromic objects.
		\end{itemize}
		This is beyond the scope of our course.

	\end{rem}


\section{Twisted \texorpdfstring{$\mCD$}{D}-modules}

	In this section, we sketch the story for other blocks $\mCO_{\varpi(\lambda)}$.

	For any weight $\lambda$, there is a variant $\mCD_{\Fl_G}^\lambda$ of $\mCD_{\Fl_G}$, called the sheaf of \textbf{$\lambda$-twisted differential operators}. Recall $\mCD_{\Fl_G}$ is the universal enveloping algebra of the split Picard algebroid $\wt\mCT_{\Fl_G}:=\mCO_{\Fl_G} \oplus \mCT_{\Fl_G}$ (see [Lecture 10, Remark 5.4]). For any $\lambda$, one can construct a Picard algebroid $\wt\mCT_{\Fl_G}^\lambda$, and $\mCD_{\Fl_G}^\lambda$ is defined to be its universal enveloping algebra\footnote{In particular, we have a PBW filtration such that $\gr^\bullet \mCD_{\Fl_G}^\lambda \simeq \Sym^\bullet_{\mCO_{\Fl_G}}\mCT_{\Fl_G}$. Moreover, $\mbF^{\le 1} \mCD_{\Fl_G}^\lambda \simeq \wt\mCT_{\Fl_G}^\lambda$}. When $\lambda$ is integral, $\mCD_{\Fl_G}^\lambda$ is just the sheaf of differential operators on the line bundle $\mCL^\lambda$. For more details, see \cite{BB} and \cite[Section 9]{G}. Now:
	\begin{itemize}
		\item 
			For any $\lambda$ and $\chi:=\varpi(\lambda)$, we have $U(\mfg)_{\chi} \simeq \mCD^\lambda(\Fl_G)$.
		\item 
			When $\lambda$ is dot-dominant, the functor $\Gamma$ is exact.
		\item
			When $\lambda$ is dominant, the functor $\Gamma$ is an equivalence. Also, Verma, dual Verma and irreducible modules correspond respectively to $!$/$*$/$\IC$ objects.
	\end{itemize} 
	

	

\begin{thebibliography}{Yau}

	\bibitem[BB]{BB} Beilinson, Alexander, and Joseph Bernstein. "A proof of Jantzen conjectures." ADVSOV (1993): 1-50.

	\bibitem[G]{G} Gaitsgory, Dennis. Course Notes for Geometric Representation Theory, 2005, available at \url{https://people.mpim-bonn.mpg.de/gaitsgde/267y/catO.pdf}.

	\bibitem[J]{J} Jantzen, Jens Carsten. Representations of algebraic groups. Vol. 107. American Mathematical Soc., 2003.

	\bibitem[M]{M} Milne, James S. Algebraic groups: the theory of group schemes of finite type over a field. Vol. 170. Cambridge University Press, 2017.
\end{thebibliography}


\end{document} 


