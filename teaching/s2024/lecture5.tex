
%!TEX root = main.tex
\documentclass{amsart}
\textwidth=14.5cm \oddsidemargin=1cm
\evensidemargin=1cm
\usepackage{amsmath}
\usepackage{amsxtra}
\usepackage{amscd}
\usepackage{amsthm}
\usepackage{amsfonts}
\usepackage{amssymb}
\usepackage[foot]{amsaddr}
\usepackage{cite}
\usepackage{url}
\usepackage{rotating}
\usepackage{eucal}
\usepackage{tikz-cd}
\usepackage[all,2cell,color]{xy}
\UseAllTwocells
\UseCrayolaColors
\usepackage{graphicx}
\usepackage{pifont}
\usepackage{comment}
\usepackage{verbatim}
\usepackage{xcolor}
\usepackage{hyperref}
\usepackage{xparse}
\usepackage{upgreek}
\usepackage{MnSymbol}
\sloppy


%%%%%%%%%%%%%%%%%%%%Theorem%%%%%%%%%%%%%%%%%%%%
\newcounter{theorem}
\setcounter{theorem}{0}

\newtheorem{cor}[subsection]{Corollary}
\newtheorem{lem}[subsection]{Lemma}
\newtheorem{goal}[subsection]{Goal}
\newtheorem{lemdefn}[subsection]{Lemma-Definition}
\newtheorem{prop}[subsection]{Proposition}
\newtheorem{propdefn}[subsection]{Proposition-Definition}
\newtheorem{cordefn}[subsection]{Corollary-Definition}
\newtheorem{variant}[subsection]{Variant}
\newtheorem{warn}[subsection]{Warning}
\newtheorem{sugg}[subsection]{Suggestion}
\newtheorem{facts}[subsection]{Fact}
\newtheorem{ques}{Question}
\newtheorem{guess}{Guess}
\newtheorem{claim}{Claim}
\newtheorem{propconstr}[subsection]{Proposition-Construction}
\newtheorem{lemconstr}[subsection]{Lemma-Construction}
\newtheorem{ax}{Axiom}
\newtheorem{conje}[subsection]{Conjecture}
\newtheorem{mainthm}[subsection]{Main-Theorem}
\newtheorem{summ}[subsection]{Summary}
\newtheorem{thm}[subsection]{Theorem}
\newtheorem{thmdefn}[subsection]{Theorem-Definition}
\newtheorem{notn}[subsection]{Notation}
\newtheorem{convn}[subsection]{Convention}
\newtheorem{constr}[subsection]{Construction}


\theoremstyle{definition}

\newtheorem{defn}[subsection]{Definition}
\newtheorem{exam}[subsection]{Example}
\newtheorem{assum}[subsection]{Assumption}

\theoremstyle{remark}
\newtheorem{rem}[subsection]{Remark}
\newtheorem{exe}[subsection]{Exercise}


\numberwithin{equation}{section}


%%%%%%%%%%%%%%%%%%%%Commands%%%%%%%%%%%%%%%%%%%%

\newcommand{\nc}{\newcommand}
\nc\on{\operatorname}
\nc\renc{\renewcommand}


%%%%%%%%%%%%%%%%%%%%Sections%%%%%%%%%%%%%%%%%%%%

\nc\ssec{\subsection}
\nc\sssec{\subsubsection}

%%%%%%%%%%%%%%%%%%%%Environment%%%%%%%%%%%%%%%%%
\nc\blongeqn{\[ \begin{aligned}}
\nc\elongeqn{\end{aligned} \]}



%%%%%%%%%%%%%%%%%%%%Mathfont%%%%%%%%%%%%%%%%%%%%

\nc\mBA{{\mathbb A}}
\nc\mBB{{\mathbb B}}
\nc\mBC{{\mathbb C}}
\nc\mBD{{\mathbb D}}
\nc\mBE{{\mathbb E}}
\nc\mBF{{\mathbb F}}
\nc\mBG{{\mathbb G}}
\nc\mBH{{\mathbb H}}
\nc\mBI{{\mathbb I}}
\nc\mBJ{{\mathbb J}}
\nc\mBK{{\mathbb K}}
\nc\mBL{{\mathbb L}}
\nc\mBM{{\mathbb M}}
\nc\mBN{{\mathbb N}}
\nc\mBO{{\mathbb O}}
\nc\mBP{{\mathbb P}}
\nc\mBQ{{\mathbb Q}}
\nc\mBR{{\mathbb R}}
\nc\mBS{{\mathbb S}}
\nc\mBT{{\mathbb T}}
\nc\mBU{{\mathbb U}}
\nc\mBV{{\mathbb V}}
\nc\mBW{{\mathbb W}}
\nc\mBX{{\mathbb X}}
\nc\mBY{{\mathbb Y}}
\nc\mBZ{{\mathbb Z}}


\nc\mCA{{\mathcal A}}
\nc\mCB{{\mathcal B}}
\nc\mCC{{\mathcal C}}
\nc\mCD{{\mathcal D}}
\nc\mCE{{\mathcal E}}
\nc\mCF{{\mathcal F}}
\nc\mCG{{\mathcal G}}
\nc\mCH{{\mathcal H}}
\nc\mCI{{\mathcal I}}
\nc\mCJ{{\mathcal J}}
\nc\mCK{{\mathcal K}}
\nc\mCL{{\mathcal L}}
\nc\mCM{{\mathcal M}}
\nc\mCN{{\mathcal N}}
\nc\mCO{{\mathcal O}}
\nc\mCP{{\mathcal P}}
\nc\mCQ{{\mathcal Q}}
\nc\mCR{{\mathcal R}}
\nc\mCS{{\mathcal S}}
\nc\mCT{{\mathcal T}}
\nc\mCU{{\mathcal U}}
\nc\mCV{{\mathcal V}}
\nc\mCW{{\mathcal W}}
\nc\mCX{{\mathcal X}}
\nc\mCY{{\mathcal Y}}
\nc\mCZ{{\mathcal Z}}


\nc\mbA{{\mathsf A}}
\nc\mbB{{\mathsf B}}
\nc\mbC{{\mathsf C}}
\nc\mbD{{\mathsf D}}
\nc\mbE{{\mathsf E}}
\nc\mbF{{\mathsf F}}
\nc\mbG{{\mathsf G}}
\nc\mbH{{\mathsf H}}
\nc\mbI{{\mathsf I}}
\nc\mbJ{{\mathsf J}}
\nc\mbK{{\mathsf K}}
\nc\mbL{{\mathsf L}}
\nc\mbM{{\mathsf M}}
\nc\mbN{{\mathsf N}}
\nc\mbO{{\mathsf O}}
\nc\mbP{{\mathsf P}}
\nc\mbQ{{\mathsf Q}}
\nc\mbR{{\mathsf R}}
\nc\mbS{{\mathsf S}}
\nc\mbT{{\mathsf T}}
\nc\mbU{{\mathsf U}}
\nc\mbV{{\mathsf V}}
\nc\mbW{{\mathsf W}}
\nc\mbX{{\mathsf X}}
\nc\mbY{{\mathsf Y}}
\nc\mbZ{{\mathsf Z}}

\nc\mba{{\mathsf a}}
\nc\mbb{{\mathsf b}}
\nc\mbc{{\mathsf c}}
\nc\mbd{{\mathsf d}}
\nc\mbe{{\mathsf e}}
\nc\mbf{{\mathsf f}}
\nc\mbg{{\mathsf g}}
\nc\mbh{{\mathsf h}}
\nc\mbi{{\mathsf i}}
\nc\mbj{{\mathsf j}}
\nc\mbk{{\mathsf k}}
\nc\mbl{{\mathsf l}}
\nc\mbm{{\mathsf m}}
\nc\mbn{{\mathsf n}}
\nc\mbo{{\mathsf o}}
\nc\mbp{{\mathsf p}}
\nc\mbq{{\mathsf q}}
\nc\mbr{{\mathsf r}}
\nc\mbs{{\mathsf s}}
\nc\mbt{{\mathsf t}}
\nc\mbu{{\mathsf u}}
\nc\mbv{{\mathsf v}}
\nc\mbw{{\mathsf w}}
\nc\mbx{{\mathsf x}}
\nc\mby{{\mathsf y}}
\nc\mbz{{\mathsf z}}




\nc\mbfA{{\mathbf A}}
\nc\mbfB{{\mathbf B}}
\nc\mbfC{{\mathbf C}}
\nc\mbfD{{\mathbf D}}
\nc\mbfE{{\mathbf E}}
\nc\mbfF{{\mathbf F}}
\nc\mbfG{{\mathbf G}}
\nc\mbfH{{\mathbf H}}
\nc\mbfI{{\mathbf I}}
\nc\mbfJ{{\mathbf J}}
\nc\mbfK{{\mathbf K}}
\nc\mbfL{{\mathbf L}}
\nc\mbfM{{\mathbf M}}
\nc\mbfN{{\mathbf N}}
\nc\mbfO{{\mathbf O}}
\nc\mbfP{{\mathbf P}}
\nc\mbfQ{{\mathbf Q}}
\nc\mbfR{{\mathbf R}}
\nc\mbfS{{\mathbf S}}
\nc\mbfT{{\mathbf T}}
\nc\mbfU{{\mathbf U}}
\nc\mbfV{{\mathbf V}}
\nc\mbfW{{\mathbf W}}
\nc\mbfX{{\mathbf X}}
\nc\mbfY{{\mathbf Y}}
\nc\mbfZ{{\mathbf Z}}

\nc\mbfa{{\mathbf a}}
\nc\mbfb{{\mathbf b}}
\nc\mbfc{{\mathbf c}}
\nc\mbfd{{\mathbf d}}
\nc\mbfe{{\mathbf e}}
\nc\mbff{{\mathbf f}}
\nc\mbfg{{\mathbf g}}
\nc\mbfh{{\mathbf h}}
\nc\mbfi{{\mathbf i}}
\nc\mbfj{{\mathbf j}}
\nc\mbfk{{\mathbf k}}
\nc\mbfl{{\mathbf l}}
\nc\mbfm{{\mathbf m}}
\nc\mbfn{{\mathbf n}}
\nc\mbfo{{\mathbf o}}
\nc\mbfp{{\mathbf p}}
\nc\mbfq{{\mathbf q}}
\nc\mbfr{{\mathbf r}}
\nc\mbfs{{\mathbf s}}
\nc\mbft{{\mathbf t}}
\nc\mbfu{{\mathbf u}}
\nc\mbfv{{\mathbf v}}
\nc\mbfw{{\mathbf w}}
\nc\mbfx{{\mathbf x}}
\nc\mbfy{{\mathbf y}}
\nc\mbfz{{\mathbf z}}

\nc\mfa{{\mathfrak a}}
\nc\mfb{{\mathfrak b}}
\nc\mfc{{\mathfrak c}}
\nc\mfd{{\mathfrak d}}
\nc\mfe{{\mathfrak e}}
\nc\mff{{\mathfrak f}}
\nc\mfg{{\mathfrak g}}
\nc\mfh{{\mathfrak h}}
\nc\mfi{{\mathfrak i}}
\nc\mfj{{\mathfrak j}}
\nc\mfk{{\mathfrak k}}
\nc\mfl{{\mathfrak l}}
\nc\mfm{{\mathfrak m}}
\nc\mfn{{\mathfrak n}}
\nc\mfo{{\mathfrak o}}
\nc\mfp{{\mathfrak p}}
\nc\mfq{{\mathfrak q}}
\nc\mfr{{\mathfrak r}}
\nc\mfs{{\mathfrak s}}
\nc\mft{{\mathfrak t}}
\nc\mfu{{\mathfrak u}}
\nc\mfv{{\mathfrak v}}
\nc\mfw{{\mathfrak w}}
\nc\mfx{{\mathfrak x}}
\nc\mfy{{\mathfrak y}}
\nc\mfz{{\mathfrak z}}

\nc{\one}{{\mathsf{1}}}


\nc\clambda{ {\check{\lambda} }}
\nc\cmu{ {\check{\mu} }}

\nc\bDelta{\mathsf{\Delta}}
\nc\bGamma{\mathsf{\Gamma}}
\nc\bLambda{\mathsf{\Lambda}}


\nc\loccit{\emph{loc.cit.}}



%%%%%%%%%%%%%%%%%%%%Operations-limit%%%%%%%%%%%%%%%%%%%%

\NewDocumentCommand{\ot}{e{_^}}{
  \mathbin{\mathop{\otimes}\displaylimits
    \IfValueT{#1}{_{#1}}
    \IfValueT{#2}{^{#2}}
  }
}
\NewDocumentCommand{\boxt}{e{_^}}{
  \mathbin{\mathop{\boxtimes}\displaylimits
    \IfValueT{#1}{_{#1}}
    \IfValueT{#2}{^{#2}}
  }
}
\NewDocumentCommand{\mt}{e{_^}}{
  \mathbin{\mathop{\times}\displaylimits
    \IfValueT{#1}{_{#1}}
    \IfValueT{#2}{^{#2}}
  }
}
\NewDocumentCommand{\convolve}{e{_^}}{
  \mathbin{\mathop{\star}\displaylimits
    \IfValueT{#1}{_{#1}}
    \IfValueT{#2}{^{#2}}
  }
}
\NewDocumentCommand{\colim}{e{_^}}{
  \mathbin{\mathop{\operatorname{colim}}\displaylimits
    \IfValueT{#1}{_{#1}\,}
    \IfValueT{#2}{^{#2}\,}
  }
}
\NewDocumentCommand{\laxlim}{e{_^}}{
  \mathbin{\mathop{\operatorname{laxlim}}\displaylimits
    \IfValueT{#1}{_{#1}\,}
    \IfValueT{#2}{^{#2}\,}
  }
}
\NewDocumentCommand{\oplaxlim}{e{_^}}{
  \mathbin{\mathop\operatorname{oplax-lim}\displaylimits
    \IfValueT{#1}{_{#1}\,}
    \IfValueT{#2}{^{#2}\,}
  }
}


%%%%%%%%%%%%%%%%%%%%Arrows%%%%%%%%%%%%%%%%%%%%


\makeatletter
\newcommand{\laxto}{\dashedrightarrow}
\newcommand{\xrightleftarrows}[1]{\mathrel{\substack{\xrightarrow{#1} \\[-.9ex] \xleftarrow{#1}}}}
\newcommand{\adj}{\xrightleftarrows{\rule{0.5cm}{0cm}}}

\newcommand*{\da@rightarrow}{\mathchar"0\hexnumber@\symAMSa 4B }
\newcommand*{\da@leftarrow}{\mathchar"0\hexnumber@\symAMSa 4C }
\newcommand*{\xlaxto}[2][]{%
  \mathrel{%
    \mathpalette{\da@xarrow{#1}{#2}{}\da@rightarrow{\,}{}}{}%
  }%
}
\newcommand{\xlaxgets}[2][]{%
  \mathrel{%
    \mathpalette{\da@xarrow{#1}{#2}\da@leftarrow{}{}{\,}}{}%
  }%
}
\newcommand*{\da@xarrow}[7]{%
  % #1: below
  % #2: above
  % #3: arrow left
  % #4: arrow right
  % #5: space left 
  % #6: space right
  % #7: math style 
  \sbox0{$\ifx#7\scriptstyle\scriptscriptstyle\else\scriptstyle\fi#5#1#6\m@th$}%
  \sbox2{$\ifx#7\scriptstyle\scriptscriptstyle\else\scriptstyle\fi#5#2#6\m@th$}%
  \sbox4{$#7\dabar@\m@th$}%
  \dimen@=\wd0 %
  \ifdim\wd2 >\dimen@
    \dimen@=\wd2 %   
  \fi
  \count@=2 %
  \def\da@bars{\dabar@\dabar@}%
  \@whiledim\count@\wd4<\dimen@\do{%
    \advance\count@\@ne
    \expandafter\def\expandafter\da@bars\expandafter{%
      \da@bars
      \dabar@ 
    }%
  }%  
  \mathrel{#3}%
  \mathrel{%   
    \mathop{\da@bars}\limits
    \ifx\\#1\\%
    \else
      _{\copy0}%
    \fi
    \ifx\\#2\\%
    \else
      ^{\copy2}%
    \fi
  }%   
  \mathrel{#4}%
}
\makeatother

%%%%%%%%%%%%%%%%%%%%Decorations%%%%%%%%%%%%%%%%%%%%
\nc{\wt}{\widetilde}
\nc{\ol}{\overline}

\nc{\red}{\textcolor{red}}
\nc{\blue}{\textcolor{blue}}
\nc{\purple}{\textcolor{violet}}

\nc{\simorlax}{{\red\simeq/\blue\lax}}

%%%%%%%%%%%%%%%%%%%%All%%%%%%%%%%%%%%%%%%%%

\nc{\Id}{\mathsf{Id}}
\nc{\gl}{\mathfrak{gl}}
\renc{\sl}{\mathfrak{sl}}
\nc{\GL}{\mathsf{GL}}
\nc{\SL}{\mathsf{SL}}
\nc{\PGL}{\mathsf{PGL}}
\nc{\hmod}{\mathsf{-mod}}
\nc{\Vect}{\mathsf{Vect}}
\nc{\tr}{\mathsf{tr}}
\nc{\Kil}{\mathsf{Kil}}
\nc{\ad}{{\mathsf{ad}}}
\nc{\Ad}{\mathsf{Ad}}
\nc{\oblv}{\mathsf{oblv}}
\nc{\gr}{\mathsf{gr}}
\nc{\Sym}{\mathsf{Sym}}
\nc{\QCoh}{\mathsf{QCoh}}
\nc{\ind}{\mathsf{ind}}
\nc{\Spec}{\mathsf{Spec}}
\nc{\Hom}{\mathsf{Hom}}
\nc{\Ext}{\mathsf{Ext}}
\nc{\Grp}{\mathsf{Grp}}
\nc{\pt}{\mathsf{pt}}
\nc{\Lie}{\mathsf{Lie}}
\nc{\CAlg}{\mathsf{CAlg}}
\nc{\Der}{\mathsf{Der}}
\nc{\Rep}{\mathsf{Rep}}
\renc{\sc}{{\mathsf{sc}}}
\nc{\Fl}{\mathsf{Fl}}
\nc{\Fun}{\mathsf{Fun}}
\nc{\ev}{\mathsf{ev}}
\nc{\surj}{\twoheadrightarrow}
\nc{\inj}{\hookrightarrow}
\nc{\HC}{\mathsf{HC}}
\nc{\cl}{\mathsf{cl}}
\renc{\Im}{\mathsf{Im}}
\renc{\ker}{\mathsf{ker}}
\nc{\coker}{\mathsf{coker}}
\nc{\Tor}{\mathsf{Tor}}
\nc{\op}{\mathsf{op}}
\nc{\length}{\mathsf{length}}
\nc{\fd}{{\mathsf{fd}}}
\nc{\weight}{\mathsf{wt}}
\nc{\semis}{{\mathsf{ss}}}
\nc{\qc}{{\mathsf{qc}}}
\nc{\pr}{\mathsf{pr}}
\nc{\act}{\mathsf{act}}
\nc{\dR}{{\mathsf{dR}}}
\nc{\hol}{{\mathsf{hol}}}
\nc{\Pic}{{\mathsf{Pic}}}
\nc{\Loc}{\mathsf{Loc}}
\nc{\IC}{\mathsf{IC}}

\begin{document}


\title{Lecture 5}

\date{Mar 25, 2024}

\maketitle


	Last time we constructed a homomorphism $\phi: Z(\mfg) \to \Sym(\mft)$ fitting into the following diagram
	\begin{equation}
		\label{eqn-phi}
		\xymatrix{
			Z(\mfg) 
				\ar[r]^-\phi
				\ar[rd]_-{\chi_\lambda=\xi_{-,\lambda}}
			& \Sym(\mft) \ar[d]^-{\ev_\lambda} \\
			& k_\lambda,
		}
	\end{equation}
	where for any $\lambda\in \mft^*$, $\chi_\lambda=\xi_{-,\lambda}$ is the central character of the Verma module $M_\lambda$. We stated the following result, which will be proved today.

	\begin{thm}[Harish-Chandra]
	\label{thm-HC}
		The homomorphism $\phi$ induces an isomorphism
		\[
			\phi_\HC: Z(\mfg)  \xrightarrow{\sim} \Sym(\mft)^{W_\bullet}
		\]
		from $Z(\mfg)$ to the invariance of $\Sym(\mft)$ with respect to the dotted $W$-action.

	\end{thm}

\section{Step 1: image is \texorpdfstring{$W_\bullet$}{W.}-invariant}

	Let us first prove the image of $\phi$ is indeed contained in $\Sym(\mft)^{W_\bullet}$. Since $W$ is generated by simple reflections $s_\alpha$, $\alpha\in \Delta$, we only need to show $\phi(z) = s_\alpha \cdot \phi(z)$ for any $z\in Z(\mfg)$. In other words, we need to show
	\begin{equation}
		\label{eqn-1}
		\ev_\lambda(\phi(z)) = \ev_\lambda(s_\alpha \cdot \phi(z))
	\end{equation}
	for any $\lambda\in \mft^*$. In fact, we only need to prove this for a Zariski dense subset of $\mft^*$. Let us first give this dense subset. The following result is obvious after drawing a picture.

	\begin{lem}
		Let $\alpha\in \Delta$ be a simple root. The subset $\{ \lambda\in \mft^*\,\vert\, \langle \lambda+\rho, \check\alpha \rangle \in \mBZ^{\ge 0} \}$ is a Zariski dense subset of $\mft^*$.
	\end{lem}

	\begin{exam}
		For $\mfg=\sl_2$, this subset is $\mBZ^{\ge -1}\subset \mBA^1$. In fact, in [Exam. 15, Lect. 4], we have used this subset to show the image of $\phi:Z(\sl_2) \to k[h]$ is invariant under $h\mapsto -h-2$. The argument below is an immediate generalization.
	\end{exam}

	\begin{lem}
		\label{lem-Verma-simple-reflection}
		Let $\alpha\in \Delta$ be a simple root and $\lambda\in \mft^*$. Suppose $\langle \lambda+\rho, \check\alpha \rangle \in \mBZ^{\ge 0} $. Then $M_\lambda$ contains $M_{s_\alpha\cdot \lambda}$ as a submodule.
	\end{lem}

	\begin{cor}
		The image of $\phi: Z(\mfg)\to \Sym(\mft)$ is indeed contained in $\Sym(\mft)^{W_\bullet}$. 
	\end{cor}

	\proof
		By previous discussion, we only need to prove \eqref{eqn-1} for $\alpha$ and $\lambda$ satisfying the assumption of Lemma \ref{lem-Verma-simple-reflection}. By \eqref{eqn-phi}, the LHS and RHS of \eqref{eqn-1} are exactly the central character of $M_\lambda$ and $M_{s_\alpha\cdot \lambda}$. Now Lemma \ref{lem-Verma-simple-reflection} implies they are equal because the central character of a module is equal to the central character of any nonzero submodule of it.

	\qed

	\proof[Proof of Lemma \ref{lem-Verma-simple-reflection}]
		Recall 
		\[
			s_\alpha(\lambda) = \lambda - 2\frac{(\lambda,\alpha)}{(\alpha,\alpha)}\alpha = \lambda - \langle \lambda,\check\alpha  \rangle \alpha.
		\]
		It follows that
		\[
			s_\alpha \cdot \lambda = \lambda - \langle \lambda+\rho, \check\alpha \rangle  \alpha  = \lambda - m\alpha
		\]
		for $m\in \mBZ^{\ge 0}$. The lemma is obvious for $m=0$. We assume $m>0$. Then we have $\langle \lambda,\check \alpha \rangle = m-1$ because $\langle \rho,\check\alpha \rangle = 1$ (see \cite[Cor. to Lem. 10.2(B)]{H1}).

		Let $f_\alpha\in \mfn^-$ be a nonzero vector of weight $-\alpha$. Consider the vector $f_\alpha^m \cdot v_\lambda$, which is of weight $\lambda-m\alpha=s_\alpha \cdot \lambda$. We only need to show
		\[
			\mfn \cdot (f_\alpha^m \cdot v_\lambda) =0.
		\]
		Indeed, if this is true, the map $k_{s_\alpha \cdot \lambda}\to M_\lambda,\; c\mapsto c(f_\alpha^m \cdot v_\lambda)$ is $\mfb$-linear, and thereby induces a $\mfg$-linear map $M_{s_\alpha\cdot \lambda}\to M_{\lambda}$. This map is injective because as a morphism between $U(\mfn^-)$-modules, it is given by $ -\cdot f_\alpha^m : U(\mfn^-) \to U(\mfn^-)$, which is injective by the PBW theorem.

		It remains to show $\mfn$ annihilates $f_\alpha^m \cdot v_\lambda$. For each simple root $\beta\in \Delta$, let $e_\beta\in \mfn$ be a nonzero vector of weight $\beta$. Note that the vectors $(e_\beta)_{\beta\in \Delta}$ generate $\mfn$ under Lie brackets\footnote{Because $[\mfg_\alpha,\mfg_\beta] = \mfg_{[\alpha,\beta]}$ whenever $\alpha,\beta,\alpha+\beta\in \Phi^+$.}. Hence we only need to show $e_\beta\cdot f_\alpha^m \cdot v_\lambda =0$. There are two cases:
		\begin{itemize}
			\item 
				If $\alpha\neq \beta$, then $[e_\beta,f_\alpha]=0$\footnote{Otherwise it is a nonzero vector with weight $\beta-\alpha$ but the latter is not a root because $\Phi = \Phi^+ \sqcup \Phi^-$.} and
				\[
					e_\beta\cdot f_\alpha^m \cdot v_\lambda = f_\alpha^m \cdot e_\beta \cdot v_\lambda = 0
				\]
				because $\mfn \cdot v_\lambda =0$.
			\item
				If $\alpha = \beta$, $[e_\alpha,f_\alpha]\in \mft$ is propotionate to $\check \alpha$ (see \cite[Prop. 8.3(d)]{H1}). Rescale $e_\alpha$, we may assume $[e_\alpha,f_\alpha ]= \check \alpha$. Then $[\check \alpha ,f_\alpha ]= \langle -\alpha,\check\alpha\rangle f_\alpha = -2f_\alpha$. Now the following calculation is essentially that in [Exe. 24, Lect. 2]. We have
				\[
					e_\alpha\cdot f_\alpha^m \cdot v_\lambda  = \sum_{1\le i\le m} f_\alpha^{m-i}\cdot [e_\alpha,f_\alpha] \cdot f_\alpha^{i-1}\cdot v_\lambda + f_\alpha^m \cdot e_\alpha \cdot v_\lambda.
				\]
				Recall we have $e_\alpha \cdot v_\lambda=0$. Also,
				\[
					\check \alpha \cdot f_\alpha^j = \sum_{1\le i\le j} f_\alpha^{j-i}\cdot [\check \alpha, f_\alpha] \cdot f_\alpha^{i-1} + f_\alpha^j  \cdot \check \alpha  = -2j f_\alpha^j +  f_\alpha^j  \cdot \check \alpha.
				\]
				Hence
				\[
					e_\alpha\cdot f_\alpha^m \cdot v_\lambda  = \sum_{1\le i\le m} (-2(i-1)f_\alpha^{m-1}+ f_\alpha^{m-1}\cdot \check\alpha ) v_\lambda = \big(-m(m-1) + m\langle \lambda,\check\alpha\rangle \big)v_\lambda = 0
				\]
				as desired.
		\end{itemize}
		
	\qed[Lemma \ref{lem-Verma-simple-reflection}]

\section{Step 2: Filtrations}
	
	By Step 1, we have a homomorphism
	\[
		\phi_\HC: Z(\mfg) \to \Sym(\mft)^{W_\bullet}.
	\]
	In this step, we equip both sides with filtrations, and show $\phi_\HC$ is compatible with them. The punchline is the following easy fact:

	\begin{facts}
		Let $V_1$ and $V_2$ be two vector spaces equipped with $\mBZ^{\ge 0}$-indexed (exhausted) filtrations. Suppose $\varphi:V_1\to V_2$ is a $k$-linear map compatible with the filtrations. Then $\varphi$ is an isomorphism iff $\gr^\bullet \varphi: \gr^\bullet V_1 \to \gr^\bullet V_2$ is so.
	\end{facts}

	\begin{constr}
		The PBW filtration on $U(\mfg)$ induces a filtration on $Z(\mfg)$ with $\mbF^{\le i} Z(\mfg):= Z(\mfg)\cap \mbF^{\le i}U(\mfg)$. Note that $\gr^\bullet Z(\mfg)$ is a subalgebra of $\gr^\bullet U(\mfg) \simeq \Sym(\mfg)$.
	\end{constr}

	\begin{constr}
		The PBW filtration on $U(\mft)\simeq \Sym(\mft)$ is preserved by the dotted $W$-action. Hence it induces a filtration on $U(\mft)^{W_\bullet}$\footnote{Note that $\gr^\bullet U(\mft)$ is also isomorphic to $\Sym(\mft)$. To distinguish them, we always use $U(\mft)$ to denote the filtered commutative ring while use $\Sym(\mft)$ to denote the graded commutative ring.}.
	\end{constr}

	\begin{warn}
		The \emph{dotted} $W$-action on $\Sym(\mft)$ does not preserve the grading. But the usual (linear) $W$-action does.
	\end{warn}

	\begin{lem}
		The homomorphism $\phi_\HC: Z(\mfg) \to U(\mft)^{W_\bullet}$ is compatible with the above filtrations.
	\end{lem}

	\proof
		This is obvious from the description of $\phi$ as
		\[
			Z(\mfg) \inj U(\mfg) \surj k\ot_{U(\mfn^-)} U(\mfg) \ot_{U(\mfn)} k \simeq U(\mft).
		\]
	\qed

\section{Step 3: Calculating the graded pieces}
	
	\begin{constr}
	\label{constr-adj-action-sym}
		Recall $\Sym(\mfg)$ has a natural $\mfg$-module structure constructed as follows. For any $V\in \mfg$, there is a natural $\mfg$-module structure on $V^{\otimes n}$ given by 
		\[
			\mfg \mt V^{\otimes n} \to V^{\otimes n},\; (x, \ot_i v_i)\mapsto \sum_i ( v_1\ot \cdots\ot v_{i-1}\ot (x\cdot v_i) \ot v_{i+1}\cdots \ot v_n ).
		\]
		This action is compatible with the symmetric group $\Sigma_n$-action on $V^{\otimes n}$ and thereby induces a $\mfg$-module structure on $\Sym^n(V)$. Taking direct sum, we obtain a $\mfg$-module structure on $\Sym(V)$. 

		In the case $V=\mfg$, to distinguish with the multiplication structure on $\Sym(\mfg)$, we denote this action by
		\[
			\mfg \times \Sym(\mfg) \to \Sym(\mfg) ,\; (x,u)\mapsto \ad_x(u),
		\]
		and call it the \textbf{adjoint action}.

		Note that by definition, for $x\in \mfg$ and $u,v\in \Sym(\mfg)$, we have
		\[
			\ad_x(u\cdot v) = \ad_x(u)\cdot v + u\cdot \ad_x(v).
		\]
		In particular, the \emph{$\mfg$-invariance} 
		\[
			\Sym(\mfg)^\mfg:= \{ u\in \Sym(\mfg)\,\vert\, \ad_x(u)=0\textrm{ for any }x\in \mfg \}
		\]
		is a subalgebra of $\Sym(\mfg)$.
	\end{constr}

	\begin{lem}
	\label{lem-Zg-fil}
		There is a unique dotted graded isomorphism making the following diagram commute
		\[
			\xymatrix{
				\gr^\bullet Z(\mfg) \ar@{.>}[r]^-\simeq \ar[d]^-\subset
				& \Sym(\mfg)^\mfg \ar[d]^-\subset \\
				\gr^\bullet U(\mfg) \ar[r]^-\simeq 
				& \Sym(\mfg),
			}
		\]
		where the bottom isomorphism is given by the PBW theorem.
	\end{lem}

	To prove this lemma, we use the following exercise:

	\begin{exe}
		This is \red{Homework 2, Problem 4}. Prove: the adjoint $\mfg$-action on $U(\mfg)$, i.e.,
		\[
			\mfg \times U(\mfg) \to U(\mfg),\; (x,u)\mapsto \ad_x(u)=[x,u],
		\]
		preserves each $\mbF^{\le n}U(\mfg)$, and the induced $\mfg$-action on $\gr^\bullet(U(\mfg)) \simeq \Sym(\mfg)$ is the adjoint action in Construction \ref{constr-adj-action-sym}.
	\end{exe}


	\proof[Proof of Lemma \ref{lem-Zg-fil}]
		By the above exercise, we have a short exact sequence of \emph{finite-dimensional} $\mfg$-modules:
		\[
			0 \to \mbF^{\le n-1} U(\mfg) \to \mbF^{\le n} U(\mfg)  \to \Sym^n(\mfg) \to 0.
		\]
		Since $\mfg\hmod_\mathsf{fd}$ is semisimple, this short exact sequence splits. Hence taking $\mfg$-invariance, we obtain\footnote{
			Warning: in general, taking invariance is only \emph{left exact}. (Memory method: it is given by $\Hom_{\mfg}(k,-)$.) Hence we need the existence of a splitting.
		}
		\[
			0 \to \mbF^{\le n-1} Z(\mfg) \to \mbF^{\le n} Z(\mfg) \to \Sym^n(\mfg)^\mfg \to 0
		\]
		This gives the desired isomorphism $\gr^n Z(\mfg) \simeq \Sym^n(\mfg)^\mfg$.

	\qed[Lemma \ref{lem-Zg-fil}]
	
	A similar proof\footnote{Note that $\Rep(W)_\mathsf{fd}$ is also semisimple} gives:

	\begin{lem}
	\label{lem-UtW-fil}
		There is a unique dotted graded isomorphism making the following diagram commute
		\[
			\xymatrix{
				\gr^\bullet (U(\mft)^{W_\bullet}) \ar@{.>}[r]^-\simeq \ar[d]^-\subset
				& \Sym(\mft)^W \ar[d]^-\subset \\
				\gr^\bullet U(\mft) \ar[r]^-\simeq 
				& \Sym(\mft),
			}
		\]
		where the right-top corner is the invariance for the \emph{linear} $W$-action on $\Sym(\mft)$.
	\end{lem}

	Combining the above two lemmas, we obtain:

	\begin{cor}
		There is a unique dotted graded homomorphism making the following diagram commute
		\[
			\xymatrix{
				\gr^\bullet Z(\mfg) \ar[r]^-\simeq \ar[d]^-{\gr^\bullet \phi_\HC}
				& \Sym(\mfg)^\mfg \ar@{.>}[d]^-{\phi_\cl} \\
				\gr^\bullet (U(\mft)^{W_\bullet})  \ar[r]^-\simeq 
				& \Sym(\mft)^W.
			}
		\]
	\end{cor}

\section{Step 4: Chevalley isomorphism}
	It remains to show
	\[
		\phi_\cl: \Sym(\mfg)^\mfg \to \Sym(\mft)^W
	\]
	is an isomorphism. Let us first give an explicit construction of this homomorphism. We need the following characterization of $\phi$.

	\begin{constr}
		Consider the composition
		\[
			Z(\mfg) \inj U(\mfg) \surj  U(\mfg) \ot_{U(\mfn)} k.
		\]
		With respect to the adjoint $\mft$-action, the source has weight $0$. Hence the composition factors through the $0$-weight subspace of the target, which is exactly $U(\mfb) \ot_{U(\mfn)} k \simeq U(\mft)$. By definition, the obtained map
		\[
			Z(\mfg) \to U(\mft)
		\] 
		is just $\phi$.
	\end{constr}

	\begin{constr}
		It follows $\phi_\cl$ can be constructed as follows. Consider the composition
		\begin{equation}
			\label{eqn-3}
			\Sym(\mfg)^\mfg \inj \Sym(\mfg) \surj \Sym(\mfg/\mfn).
		\end{equation}
		It factors through $\Sym(\mfb/\mfn)\simeq \Sym(\mft)$. The obtained map
		\[
			\Sym(\mfg)^\mfg  \to  \Sym(\mft)
		\]
		can be identified with $\gr^\bullet \phi$. Since $\phi$ factors through $U(\mft)^{W_\bullet}$, the map $\gr^\bullet \phi$ factors through $\gr^\bullet(U(\mft)^{W_\bullet}) \simeq \Sym(\mft)^W$. The obtained map 
		\[
			\Sym(\mfg)^\mfg  \to  \Sym(\mft)^W
		\]
		is just $\phi_\cl$.
	\end{constr}
 
	\begin{rem}
		\label{rem-geometric-Chevalley}
		The geometric meaning of the above construction is as follows. 

		Note that $\Sym(\mfg)^\mfg = \Sym(\mfg)^G$ because $G$-invariance is equal to $\mfg$-invariance\footnote{Because $\Rep(G) \to \mfg\hmod$ is fully faithful when $G$ is connected.}. Hence \eqref{eqn-3} corresponds to the morphisms
		\[
			(\mfg/\mfn)^* \to \mfg^* \to \mfg^*/\!/G.
		\]
		Since $\gr^\bullet \phi$ factors through $\gr^\bullet(U(\mft)^{W_\bullet}) \simeq \Sym(\mft)^W$. The above composition factors through $(\mfg/\mfn)^* \to (\mfb/\mfn)^* \simeq \mft^* \to \mft^*/\!/W$. In other words, we have
		\[
			\xymatrix{
				(\mfg/\mfn)^* \ar[r] \ar[d] & \mfg^*/\!/G  \\
				\mft^* \ar[r] & \mft^*/\!/W \ar@{.>}[u]
			}
		\]
		such that the dotted arrow is given by $\Spec(\phi_\cl)$.

		It is convenient to get rid of the dual spaces using the Killing form. Namely, $\Kil$ induces an isomorphism $\mfg\simeq \mfg^*$ compatible with the $G$-actions, while $\Kil|_\mft$ induces an isomorphism $\mft\simeq \mft^*$ compatible with the $W$-actions\footnote{Both claims follow from the fact that $\Kil$ is invariant with respect to the adjoint $\mfg$-action and thereby to the adjoint $G$-action. Here we use $W\simeq N_G(T)/T$.}. Via the first isomorphism, the subspaces $\mfn\subset \mfb\subset \mfg$ corresponds to $(\mfg/\mfb)^*\subset (\mfg/\mfn)^* \subset \mfg$. Then the above commutative diagram is identified with
		\begin{equation}
			\label{eqn-square}
			\xymatrix{
				\mfb \ar[r] \ar[d] & \mfg/\!/G  \\
				\mft \ar[r] & \mft/\!/W. \ar@{.>}[u]
			}
		\end{equation}
		We abuse notations and also view the dotted arrow as $\Spec(\phi_\cl)$.

	\end{rem}

	\begin{rem}
		Since the projection $\mfb\to \mft$ has a splitting $\mft \to \mfb$. The above claim implies $\Spec(\phi_\cl)$ can be characterized as the following dotted arrow
		\[
			\xymatrix{
				\mfg \ar[r]  & \mfg/\!/G  \\
				\mft \ar[r] \ar[u] & \mft/\!/W. \ar@{.>}[u]
			}
		\]
 		We can also prove the existence of this map using group-theoretic method. Namely, consider the normalizer $N_G(T)$ of $T$ insider $G$. Recall we have $W\simeq N_G(T)/T$ such that the linear $W$-action on $\mft$ can be identified with the adjoint action of $N_G(T)/T$. Then the morphism $\mft \to \mfg\to  \mfg/\!/G$ factors through $\mft/\!/ W$ because $N_G(T)$ is a subgroup of $G$.
	\end{rem}

	\begin{warn}
		I do not know any group-theoretic proof of \eqref{eqn-square}. This is because not $\mfb \to \mfg/G$ (the quotient stack) does not factor through $\mft$: two elements in $\mfb$ that have the same image in $\mft$ are not necessarily conjugate to each other. In fact, the $0$-fiber of the map $\mfg \to \mfg/\!/G$ contains exactly the nilpotent elements in $\mfg$.
	\end{warn}

	\begin{thm}[Chevalley]
		The homomorphism 
		\[
			\phi_\cl: \Sym(\mfg)^\mfg  \to  \Sym(\mft)^W
		\]
		is an isomorphism. In other words, the natural morphism $\mft/\!/ W \to \mfg/\!/G$ is an isomorphism.
	\end{thm}

	\proof
		As in Remark \ref{rem-geometric-Chevalley}, we can identify $\phi_\cl$ with the restriction map $\Fun(\mfg)^G \to \Fun(\mft)^W$, which is also $\Sym(\mfg^*)^G \to \Sym(\mft^*)^W$.

		This map is injective because if an adjoint-invariant function on $\mfg$ vanishes on $\mft$, then it vanishes on each semisimple elements. But the latter are Zariski dense in $\mfg$.

		To prove the surjectivity, we first find generators of $\Sym(\mft^*)^W$ as follows. Recall the subset $P^+$ of \textbf{dominant integral weights}, i.e.,
		\[
			P^+ := \{ \lambda\in \mft^*\,\vert\, \langle\lambda,\check\alpha\rangle \in \mBZ^{\ge 0} \textrm{ for all }  \alpha\in \Delta \}.
		\]
		Note that $P^+$ spans $\mft^*$. In fact, for each $\alpha\in \Delta$, we can find \textbf{fundamental dominant weights} $\omega_\alpha$ such that
		\[
			\langle \omega_\alpha,\check \beta \rangle = \delta_{\alpha\beta},\; \alpha,\beta\in \Delta.
		\]
		Then $\{\omega_\alpha\}$ is a basis of $\mft^*$ and $P^+= \mBZ^{\ge 0}\{\omega_\alpha\}$. A direct calculation shows that $\{\lambda^n\,\vert\, \lambda\in P^+ \}$ span $\Sym^n(\mft^*)$. Hence the sums
		\[
			b_{\lambda,n}:= \sum_{w\in W} w( \lambda^n ),\; \lambda \in P^+, n\ge 0
		\]
		span $\Sym(\mft^*)^W$.
		
		It remains to show each $b_{\lambda,n}$ is contained in the image of $\phi_\cl$. We need the following well-known fact.

		\begin{thm}[Weyl]
			For any $\lambda\in P^+$, there is a unique finite-dimensional irreducible $\mfg$-module $L_\lambda$ with higheset weight $\lambda$.
		\end{thm}

		For $\lambda \in P^+, n\ge 0$, consider the function $a_{\lambda,n}\in \Fun(\mfg) $ defined by
		\[
			a_{\lambda,n}(x):=\tr(x^n; L_\lambda),
		\]
		i.e., its value at any $x\in \mfg$ is the trace of the action of $x^n$ on $L_\lambda$. It is easy to see $a_{\lambda,n}$ is $\mfg$-invariant\footnote{Recall $L_\lambda$ is $G_\mathsf{sc}$-integrable ([Thm. 47, Lect. 3]). Hence $a_{\lambda,n}$ is $G_\mathsf{sc}$-invariant, and thereby $\mfg$-invariant.
		} and thereby $G$-invariant. 

		Now the following exercise implies each $b_{\lambda,n}$ is contained in the image of $\phi_\cl$. Indeed, this follows from induction on $\lambda$ with respect to the partial ordering $\prec$\footnote{Recall $\lambda'\preceq\lambda$ iff $\lambda-\lambda' \in \mBZ^{\ge 0}\Phi^+$. See [Defn. 22, Lect. 2].}.

		\begin{exe}
			This is \red{Homework 2, Problem 5}. Let $\lambda \in P^+$ be a dominant integral weight and $n\ge 0$. Prove there exists scalars $c_{\lambda'} \in k$, $\lambda' \prec \lambda$ such that
			\[
				\phi_\cl(a_{\lambda,n}) = a_{\lambda,n}|_{\mft} = \frac{1}{\#\mathsf{Stab}_W(\lambda)}b_{\lambda,n} + \sum_{\lambda' \prec \lambda} c_{\lambda'}b_{\lambda',n},
			\]
			where $\mathsf{Stab}_W(\lambda)\subset W$ is the stablizer of the $W$-action at $\lambda$.

		\end{exe}

	\qed

	Combining all the previous discussion, we finish the proof of Theorem \ref{thm-HC}.

	
	
\begin{thebibliography}{Yau}

	\bibitem[H1]{H1} Humphreys, James E. Introduction to Lie algebras and representation theory. Vol. 9. Springer Science \& Business Media, 2012.

	\bibitem[H2]{H2} Humphreys, James E. Representations of Semisimple Lie Algebras in the BGG Category $\mathcal{O} $. Vol. 94. American Mathematical Soc., 2008.

\end{thebibliography}

\end{document} 


